\documentclass[a4paper,10pt]{article}
\usepackage[utf8]{inputenc}
\input{auxiliar/tex/encabezado.tex}
\input{auxiliar/tex/tikzlibrarybayesnet.code.tex}
\newif\ifen
\newif\ifes
\newcommand{\en}[1]{\ifen#1\fi}
\newcommand{\es}[1]{\ifes#1\fi}
\estrue

%opening
\title{Fundamentos de la complejidad de la vida}
\author{Gustavo Landfried}

\begin{document}

\maketitle

\begin{abstract}
La complejidad actual de la vida es consecuencia de una serie de transiciones evolutivas en las que entidades capaces de replicarse de forma independiente, cediendo algo de valor a las otras, pasan a formar unidades evolutivas de nivel superior.
Una condición preliminar crucial es la alineación de intereses y su persistencia en el tiempo.
En este artículo mostramos que estas condiciones se cumplen siempre debido a la no-ergodicidad de los procesos multiplicativos a los que está sujeto la vida: la secuencia de probabilidades de superviencia y reproducción.
En los procesos no-ergódicos, las fluctuaciones tienen un efecto negativo en la tasa de crecimiento individual a largo plazo, pero no en la tasa de crecimiento del valor esperado.
Al compartir recursos, las estrategias cooperadoras reducen sus fluctuaciones, aumentando sus tasas de crecimiento a largo plazo.
Si bien las estrategias que evitan compartir recursos pueden invadir por selección natural poblaciones enteramente cooperadoras, tal comportamiento aumenta al mismo tiempo sus fluctuaciones, afectando su propia tasa de crecimiento a largo plazo sin necesidad de introducir castigos.
Utilizando la equivalencia entre la evolución de una población jerárquica bajo selección multinivel y la inferencia en modelos jerárquicos bayesianos, mostramos que las estrategias incondicionalmente cooperadoras, sin ser estas evolutivamente estables al interior de los grupos, se ven favorecidas evolutivamente a través de la selección grupal.
La sinergia inherente a toda dinámica de reproducción y superviviencia junto con la selección multinivel de poblaciones cooperadoras, son el fundamento que explica la complejidad actual de la vida.
\end{abstract}

En el último tercio de la historia del Universo, en algún momento hace aproximadamente 4500 millones de años, apareció en la tierra una forma de organización de la materia capaz de auto-replicarse.
El crecimiento de estos linajes siguieron procesos multiplicativos y ruidosos: secuencias de probabilidades de supervivencia y reproducción.
Los errores producidos durante la replicación diversificaron las formas de organización de la materia, y las diferentes tasas de crecimiento favorecieron a aquellas mejor adaptadas al medio.
Desde aquel momento hasta entonces la vida adquirió una extraordinaria complejidad~\cite{barOn2018-biomass}, consecuencia de una serie de transiciones evolutivas~\cite{maynardSmith1995-majorTransitions} en las que las entidades capaces de replicarse de forma independiente, cediendo algo de valor a las otras, pasan a formar unidades evolutivas de nivel superior.
Algunos ejemplos paradigmáticos son la transición de las moléculas replicantes a las protocélulas, la endosimbiosis de las mitocondrias y los plastos por parte de las células eucariotas, la aparición de los organismos multicelulares, la eusocialidad.
Finalmente, la establidad de la biocenosis de los ecosistemas se sostiene por una red compleja de interdependencias entre especies.
¿Cómo se explica esta tendencia permanente de la vida en favor de la agregación cooperativa?

\section{Conocimiento empírico}

La ciencia es una institución humana que tiene pretención de verdad, esto es de formular proposiciones que valgan para todas las personas, tanto intercultural como intersubjetivamente.
Las ciencias formales validan sus proposiciones mediante teoremas, resultados derivados de aplicar las reglas internas a un sistema axiomático cerrado.
Las ciencias empíricas, en cambio, deben validar sus proposiciones dentro de sistemas abiertos, lo que introduce siempre un grado de incertidumbre asociada.
¿Cuál es entonces la fuente de validez universal del conocimiento empírico?

Existe un principio epistemológico con validez intercultural, conocido como el \emph{principio de indiferencia}, afirma que si ante un espacio de hipótesis común llegamos a un acuerdo respecto de que no tenemos información previa, entonces estaremos de acuerdo en distribuir la creencia (aka incertidumbre) en partes iguales.
Esta distribución de creencias, que permite el acuerdo intersubjetivo, la vamos a llamar creencia honesta.
La fuente de validez universal se fundamenta en la maximización de la incertidumbre.
¿Se puede extender este principio para casos en los cuales tenemos información?

Actualmente, la teoría de la probabilidad es el enfoque más utilizado para representar incertidumbre asociada al conocimiento empírico.
Sus reglas han sido derivadas formalmente a partir de una gran cantidad de sistemas axiomáticos conceptualmente distintos e independientes entre sí, lo cual es ya un punto fuerte a su favor.
%La física estadística desarrolló una metodología para determinar las propiedades termodinámica de los sistemas que actúa maximizando la incertidumbre respetando las restricciones impuestas por la evidencia empírica (datos) o formal (modelos).
En cualquier caso, la aplicación estricta de las reglas de la probabilidad grantiza maximizar la incertidumbre dentro de las restricciones impuestas por la información empírica (datos) y la información formal (modelos).
El concepto de honestidad, definido como maximización de la incertidumbre dada la evidencia empírica y formal, es un principio de validez intercultural que permite alcanzar los acuerdos intersubjetivos que las esto proposiciones de las ciencias empíricas valgan para todas las personas.
El tiempo dirá si se pueden se pueden desarrollar enfoques superadores, que incorporen otras dimensiones importantes para la vida.

\section{Evolución}

En 1973 Mynard Smith y Price proponen el concepto de EES.
Taylor 1978 introduce el replicator dynamic, una método para modelar la dinámica de juego continuos y discretos mediante sistemas de ecuaciones diferenciales de primer orden no lineales.
Los equilibrios estables para la dinámica continua incluye, pero es algo más general que la noción de ESS.

Traducción realizada con la versión gratuita del traductor www.DeepL.com/Translator





the replicator equation is a deterministic monotone non-linear and non-innovative game dynamic used in evolutionary game theory




Since the replicator equation is non-linear, an exact solution is difficult to obtain (even in simple versions of the continuous form) so the equation is usually analyzed in terms of stability



el cual se ha mostrado ser equivalente tanto a 

the already
known equivalence between univariate Bayesian update and single-level replicator dynamics [11,12]
and (ii) a possible correspondence between properties of a hierarchical population composition and
multivariate probability theory.




Una condición preliminar crucial para el desarrollo de las transiciones evolutivas es la alineación de intereses y su persistencia en el tiempo.
En este trabajo mostramos que el proceso multiplicativo al que está sujeto la vida, al ser no-ergódico, ofrece una ventaja física a favor de las estrategias cooperativas, las que se ven favorecidas evolutivamente a través de la selección multinivel.

Decimos que un proceso es ergódico si la media de los estados temporales del sistema (o de una transformación $f$ de ellos) es igual a la media de todos los posibles estados,
\begin{equation}
 \underbrace{\lim_{T \mapsto \infty} \int_0^T f(\omega(t)) \diff t}_{\text{Media temporal}}  = \underbrace{\int_{\Omega} f(\omega)p(\omega) \diff\omega}_{\text{Media de estados}}
\end{equation}
donde $\omega \in \Omega$ representa los estados del sistema, $\omega(t)$ el estado del sistema obtenido aleatoriamente en el tiempo $t$ y $p(\cdot)$ la distribución de probabilidad de los estados.
Cuando un proceso es ergódicos, las descripciones de su dinámica pueden remplazarse mediante resúmenes probabilístico sencillos, eliminado el tiempo de los modelos.
Sin embargo, las condiciones para que un sistema cumpla con la hipótesis de ergodicidad son muy restrictivas, más aún en el caso de los sistemas vivos que, regidos por procesos multiplicativos de reproducción y supervivencia, están fuera del equilibrio.
Esta distinción es importante porque cuando un sistema es no-ergódicos, lo que le ocurre a los agentes individuales en el tiempo no coincide con la esperado según los posibles estados del sistema~\cite{peters2019-ergodicityEconomics}.

Para discutir las consecuencia de los procesos multiplicativos, consideremos el siguiente ejemplo.
La naturaleza lanza una moneda: si sale cara la población crece un 50\%, si sale seca se reduce un 40\%.
\begin{equation}
\Delta x =
\begin{cases}
 +0.5x & \text{ \en{Head}\es{Cara} } \\
 -0.4x & \text{ \en{Tail}\es{Seca} }
\end{cases}
\end{equation}
Según el promedio de estados, el crecimiento esperado es de $\langle \Delta x \rangle = 0.05$. 
Sin embargo, lo que efectivamente le ocurre a los agentes en el tiempo es muy diferente: a largo plazo todos pierden a una tasa de cercana de $-0.05$.
El reconocimiento de la diferencia entre el promedio temporal y promedio de estados aparece en la literatura de evolución bajo los conceptos de media aritmética y geométrica~\cite{dempster1955-geometricMean}.
La media geométrica es una buena aproximación de lo que le ocurre a la agentes individuales en el tiempo porque, como el propio crecimiento de la población, es intrínsecamente multiplicativa en lugar de aditiva.
Por lo tanto, es muy sensible a los valores pequeños ocasionales.
Si hay alguna variación, la media geométrica será menor que la media aritmética.
Parafraseando a Den Boer~\cite{denBoer1968-spreadingRisk}, la  supervivencia de una población depende de la distribución del riesgo dentro de la población y entre las poblaciones de diferentes especies.
Dado que la varianza realmente importa, una forma eficaz de reducirla es compartir los riesgos~\cite{yaari2010-cooperationEvolution, peters2015-evolutionaryAdvantageOfCooperation}.

Ole Peters considera las consecuencias de la distribución de riesgo que tiene una estrategia cooperativa sencilla: cada agente tira su propia moneda, actualiza sus propios recursos, ofrece sus recursos a un fondo común que se divide en partes iguales.

\begin{figure}[H]
\centering
\tikz{

    \node[latent, minimum size=2cm ] (x1_0) {$x_1(t)$} ;
    \node[latent, below=of x1_0, minimum size=2cm ] (x2_0) {$x_2(t)$} ;

    \node[latent, right=of x1_0, minimum size=3cm ] (x1_0g) {$x_1(t)+\Delta x_1(t)$} ;
    \node[latent, right=of x2_0, minimum size=1.8cm, xshift=0.6cm , align=left] (x2_0g) {$x_2(t)+$\\$\Delta x_2(t)$} ;
    
    \node[latent, right=of x1_0g, minimum size=3.8cm, yshift=-1.33cm, align=right] (x_0) {$x_1(t)+\Delta x_1(t)$\\$+x_2(t)+\Delta x_2(t)$ } ;
    
    \node[const, above=of x_0] (nx_0) {$\overbrace{\text{Pool}\hspace{2.5cm}\text{Share}}^{\text{\normalsize Cooperaci\'on}}$} ;
    
    \node[latent, right=of x1_0g, minimum size=2.5cm,  xshift=4.5cm] (x1_1) {$x_1(t+1)$ } ;
    \node[latent, below=of x1_1, minimum size=2.5cm, yshift=0.7cm] (x2_1) {$x_2(t+1)$ } ;
    
    \edge {x1_0} {x1_0g};
    \edge {x2_0} {x2_0g};
    \edge {x1_0g,x2_0g} {x_0};
    \edge {x_0} {x1_1,x2_1};
    
}
\caption{Estrategia cooperativa. Dos agentes comienzan con los mismos recursos iniciales. Luego crecen independientemente de acuerdo con la ecuaci\'on \ref{}. Luego cooperan poniendo sus recursos en un fondo común, que finalmente es dividio en partes iguales.}
\label{fig:protocolo}
\end{figure}

Las poblaciones enteramente cooperadoras, mediante la distribución del riesgo, logran reducir sus fluctuaciones, lo que genera una aumento en la tasa de crecimiento de todos sus miembros.
En la figura \ref{fig:cpr_individual} mostramos el resultado de las trayectorias individuales en el tiempo, y en la figura \ref{fig:cpr_cooperation} mostramos la trayectoria de un agente cooperador.
\begin{figure}[ht!]
    \centering
    \begin{subfigure}[b]{0.45\textwidth}
    \includegraphics[width=\linewidth]{figures/cpr_individual.pdf}
    \caption{}
    \label{fig:cpr_individual}
    \end{subfigure}
    \begin{subfigure}[b]{0.45\textwidth}
    \includegraphics[width=\linewidth]{figures/cpr_cooperation.pdf}
    \caption{}
    \label{fig:cpr_cooperation}
    \end{subfigure}
    \caption{
    Tamaño logarítimico de los recursos en el tiempo.
    En la figura \ref{fig:cpr_individual} los agentes juegan individualmente.
    En la figura \ref{fig:cpr_cooperation} todos los agentes comparten su riqueza en cada iteración.
    La recta negra que baja representa el promedio temporal, y la que sube el promedio de estados.
    }
    \label{fig:gamble}
\end{figure}

Mediante la cooperación los agentes logran acceder a tasas de crecimiento equivalentes al promedio de estados del sistema, que en los sistemas no-ergódicos es siempre superior que el promedio  temporal.

Las estrategias cooperadoras no son evolutivamente estables.



{\footnotesize
\bibliographystyle{auxiliar/biblio/plos2015.bst}
\bibliography{auxiliar/biblio/biblio_notUrl.bib}
}

\end{document}
