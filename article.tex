\documentclass[a4paper,10pt]{article}
\usepackage[utf8]{inputenc}
\input{auxiliar/tex/encabezado.tex}
\input{auxiliar/tex/tikzlibrarybayesnet.code.tex}
\newif\ifen
\newif\ifes
\newcommand{\en}[1]{\ifen#1\fi}
\newcommand{\es}[1]{\ifes#1\fi}
\estrue

\newtheorem{conclution}{\en{Conclution}\es{Conclusión}}%[section]
\newtheorem{objective}{\en{Objective}\es{Objetivo}}%[section]



%opening
\title{Fundamentos de la complejidad de la vida}
\author{Gustavo Landfried}

\begin{document}

\maketitle

\begin{abstract}

La ventaja evolutiva de la cooperación es consecuencia de la no-ergodicidad de los procesos multiplicativos a los que está sujeto la vida: la secuencia de probabilidades de superviencia y reproducción.
En ellos, las fluctuaciones tienen un efecto negativo en la tasa de crecimiento individual.
Compartir recursos permite reducir las fluctuaciones, aumentando la tasas de crecimiento a largo plazo de todos los indiviudos (hasta alcanzar el valor esperado).
% Ole Peters considera que este aumento es suficiente para explicar la ventaja evolutiva de la cooperación.
% Sin embargo no analiza el problema de la deserción, quien dice ``our cooperators are unable to break the cooperative pact''.
% No parece ser un problema menor, teniendo en cuenta la tentación que podría significar dejar de aportar al fondo común mientras se siguen recibiendo los beneficios de este.
Si bien las estrategias que evitan compartir recursos pueden invadir por selección natural poblaciones enteramente cooperadoras, tal comportamiento aumenta al mismo tiempo sus propias fluctuaciones, afectando su propia tasa de crecimiento a largo plazo sin necesidad de introducir castigos.
Utilizando la equivalencia entre la evolución de una población jerárquica bajo selección multinivel y la inferencia en modelos jerárquicos bayesianos, mostramos que las estrategias incondicionalmente cooperadoras, sin ser evolutivamente estables al interior de los grupos, se ven favorecidas evolutivamente a través de la selección grupal.
La sinergia inherente a toda dinámica de reproducción y superviviencia junto con la selección multinivel de poblaciones cooperadoras, son el fundamento que explica no sólo la complejidad actual de la vida, sino también su tendencia a la especialización.
\end{abstract}

\section{Introducción}

En el último tercio de la historia del Universo, en algún momento hace aproximadamente 4000 millones de años, apareció en la tierra una forma de organización de la materia capaz de auto-replicarse.
El crecimiento de estos linajes siguieron procesos multiplicativos y ruidosos: secuencias de probabilidades de supervivencia y reproducción.
Los errores producidos durante la replicación diversificaron las formas de vida, y las tasas de crecimiento de las diferentes estrategias favorecieron a aquellas mejor adaptadas al ambiente.
Desde aquel momento hasta ahora la vida adquirió una extraordinaria complejidad.
%
\begin{figure}[H]
    \centering
    \begin{subfigure}[b]{0.65\textwidth}
    \includegraphics[width=\linewidth]{auxiliar/images/biomass.jpg}
    \end{subfigure}
    \caption{
	Distribución actual de la biomasa en la tierra estimada por Bar-On et al.~\cite{barOn2018-biomass}.
    }
    \label{fig:cpr_individual}
\end{figure}
%
Esta complejidad es consecuencia de una serie de transiciones evolutivas~\cite{maynardSmith1995-majorTransitions} en las que las entidades capaces de replicarse de forma independiente, cediendo algo de valor a las otras, pasan a formar unidades evolutivas de nivel superior.
Algunos ejemplos paradigmáticos son la transición de las moléculas replicantes a las protocélulas, la endosimbiosis de las mitocondrias y los plastos por parte de las células eucariotas, la aparición de los organismos multicelulares.
¿Cómo se explica esta tendencia permanente de la vida en favor de la agregación cooperativa?

En evolución, el crecimiento de una estrategia $e$ en el tiempo, $\omega_e(t)$, esta gobernado por una secuencias estocástica de tasas de supervivencia y reproducción $f_e(\cdot)$ dependientes de un ambiente aleatorio $a$,
%
\begin{equation} \label{eq:modelo_exponencial}
\omega_e(T) = \prod_t^T f_e(a(t))
\end{equation}
%
donde $a(t)$ representa el estado del ambiente en el tiempo $t$.
Por ejemplo, supongamos que la naturaleza lanza una moneda, si sale cara la población se reproduce 50\% y si sale seca sobrevive 60\%.
\begin{equation} \label{eq:estrategia_base}
f_e(a) =
\begin{cases}
 1.5 & a = \text{ Cara } \\
 0.6 & a = \text{ Seca }
\end{cases}
\end{equation}
Según el modelo estándar de evolución, conocido como \emph{replicator dynamic} \cite{taylor1978-replicatorDynamic, schuster1983-replicatorDynamics, hofbauer2003-evolutionaryGameDynamics}, el cambio de la proporción de una estrategia en la población, $x_e$, está determinado por su tasa de crecimiento caracterísitica $f_e$,
\begin{equation} \label{eq:replicator_dynamic}  \tag{Replicator dynamic}
x_e^\prime = \frac{x_e f_e}{X}
\end{equation}
donde $X=\sum_i x_i f_i$ es el tamaño total de la población y actúa como constante de normalización.
¿Pero cuál es la tasa de crecimiento característica, $f_e$?
Su estimación requiere considerar explícitamente la variablidad del ambiente, $a$, y el tipo de proceso que genera esa variabilidad.

% Parrafo

Buena parte de la literatura en evolución considera que la estimación correcta se obtiene mediante el valor esperado de los recursos en el tiempo, $\omega_e(t)$.
Sea $\Omega_t$ el valor de todas las posibles trayectorias en el tiempo $t$, el valor esperado se obtiene promediando los estados $\omega \in \Omega_t$ por la probabilidad de que ocurran $P(\omega)$
\begin{equation}
\langle \omega_e \rangle_t = \sum_{\omega \in \Omega_t} \omega \cdot  P(\omega)
\end{equation}

% Parrafo

En el ejemplo de la moneda, el valor esperado en los dos primeros pasos temporales es, 
%
\begin{equation}
\begin{split}
\langle \omega_e \rangle_1 & = 1.5 \cdot \frac{1}{2} + 0.6 \cdot  \frac{1}{2} = 1.05 \\ 
\langle \omega_e \rangle_2 &=  1.5^2 \cdot \frac{1}{4} + 2 (0.6 \cdot 1.5 \cdot \frac{1}{4} ) + 0.6^2 \cdot \frac{1}{4}= 1.05^2
\end{split}
\end{equation}
%
Es decir, la tasa de crecimiento estimada según el valor esperado es de $5\%$ por cada paso temporal, $\langle \omega \rangle_t = 1.05^t$.
Y efectivamente eso es lo que ocurre con el promedio de las trayectoria individuales, $\omega_e(t)$, cuando la población es suficientemente grande,
%
\begin{figure}[H]
    \centering
    \begin{subfigure}[b]{0.45\textwidth}
    \includegraphics[width=\linewidth]{figures/pdf/ergodicity_expectedValue.pdf}
    \end{subfigure}
    \caption{
    Promedio de los recursos individuales en el tiempo para diferentes tamaños de la población, en escala logarítimica.
    A medida que aumentamos el tamaño de la población, el promedio se acerca al valor esperado $\langle \omega \rangle_t = 1.05^t$.
    }
    \label{fig:cpr_individual}
\end{figure}
%
Sin embargo, el valor esperado no representa lo que lo le ocurre a los agentes en el tiempo. 
Individualmente, todas las trayectorias pierden a largo plazo a una tasa cercana al 5\%.
%
\begin{figure}[H]
    \centering
    \begin{subfigure}[b]{0.45\textwidth}
    \includegraphics[width=\linewidth]{figures/pdf/ergodicity_individual_trayectories.pdf}
    \caption{}
    \label{fig:ergodicity_individual_trayectories}
    \end{subfigure}
    \begin{subfigure}[b]{0.45\textwidth}
    \includegraphics[width=\linewidth]{figures/pdf/ergodicity_individual_trayectories_longrun.pdf}
    \caption{}
    \label{fig:ergodicity_individual_trayectories_longrun}
    \end{subfigure}
    \caption{
    La recta negra representan el valor esperado.
    Figura \ref{fig:ergodicity_individual_trayectories}: tamaño de los recursos individuales en el tiempo, $ \log(\omega(t))$.
    Figura \ref{fig:ergodicity_individual_trayectories_longrun}: con suficiente tiempo todas las trayectorias individuales se pegan a la recta azul. 
    }
    \label{fig:cpr_individual}
\end{figure}
%
%La relación entre el valor esperado y lo que le ocurre a los agentes individuales en el tiempo es un problema bien conocido en mecánica estadística.
Cuando lo que le ocurre a los agentes individuales en el tiempo puede describirse mediante el valor esperado de los estados del sistema se dice que el proceso es ergódico~\cite{peters2019-ergodicityEconomics}.
Sin embargo, las condiciones para que esto se cumple son muy restrictivas.
En particular, la dinámica evolutiva está gobernada por un procesos multiplicativos (eq.~\ref{eq:modelo_exponencial}).
En el ejemplo de la moneda puede expresarse como,
%
\begin{equation}
\omega_e(T) = \prod^T_{t=1} f_e(a(t)) = f_e(\text{cara})^{n_1} f_e(\text{seca})^{n_2}
\end{equation}
%
donde $n_1$ y $n_2$ representa la cantidad de ocurrencias de $f_e(\text{cara})$ y $f_e(\text{seca})$, con $n_1 + n_2 = T$.
Las trayectorias observadas en la figura \ref{fig:ergodicity_individual_trayectories} son variables, pero cuanto más tiempo observemos el sistema más suave se vuelven esas líneas (figura \ref{fig:ergodicity_individual_trayectories_longrun}).
En el límite, $T \rightarrow \infty$ todas las trayectorias individuales estarán determinadas por la misma tasa de crecimiento caracterísitica ${f_e}$, por el que se multiplican los recursos $\omega_e(t)$ en cada iteración.
\begin{equation} \label{eq:geometric_mean}
\begin{split}
\lim_{T \rightarrow \infty} \omega_e(T) & = {f_e}^T \\
\left( \lim_{T \rightarrow \infty} \omega_e(T) \right)^{1/T} & =  {f_e} \\
\lim_{T \rightarrow \infty} f_e(\text{cara})^{n_1/T} f_e(\text{seca})^{n_2/T} & 
 \end{split}
\end{equation}
Donde las frecuencias $\frac{n_1}{T}$ y $\frac{n_2}{T}$ en el límite $T \rightarrow \infty$ son iguales a las probabilidades de ocurrencia de los estados del sistema.
Por lo tanto, la mejor estimación de la tasa de crecimiento es,
\begin{equation}
{f_e} = (1.5 \cdot 0.6)^{1/2} \approx 0.95
\end{equation}
%
Esta fórmula, que permite computar la tasa de crecimiento a largo plazo de las trayectorias individuales, ha sido usada previamente en la literatura de evolución bajo el nombre de \emph{media geométrica}~\cite{dempster1955-geometricMean}.
Una propiedad importante de la media geométrica es que su valor siempre es menor al valor esperado (o media aritmética).
Esto se debe a que en los procesos multiplicativos los impactos físicos de las pérdidas suelen ser más fuertes que los de las ganancias.
En el extremo, un único cero en la productoria alcanza para generar su extinción.

% Decimos que un proceso es ergódico si se cumple que,
% \begin{equation}
%  \underbrace{\lim_{T \mapsto \infty} \frac{1}{T} \sum_{t=1}^T \omega(t)}_{\text{Media temporal}}  = \underbrace{\sum_{\omega} \omega \cdot p(\omega)}_{\text{Media de estados}}
% \end{equation}
% 

\subsection{Cooperacion}

%En los procesos no-ergódicos, las fluctuaciones tienen un efecto negativo en la tasa de crecimiento individual a largo plazo, pero no en la tasa de crecimiento del valor esperado.
Dado que en los procesos multiplicativos la varianza realmente importa, una forma eficaz de reducirla es compartir los riesgos~\cite{yaari2010-cooperationEvolution, peters2015-evolutionaryAdvantageOfCooperation}.
%Parafraseando a Den Boer~\cite{denBoer1968-spreadingRisk}, la  supervivencia de una población depende de la distribución del riesgo dentro de la población y entre las poblaciones de diferentes especies.
Ole Peters~\cite{peters2015-evolutionaryAdvantageOfCooperation} considera las consecuencias que la distribución del riesgo de una estrategia cooperativa sencilla tiene sobre la tasa de crecimiento de los agentes.
Los agentes cooperadores simplemente reparten sus recursos en partes iguales luego de cada iteración.

\begin{figure}[H]
\centering
\scalebox{0.75}{
\tikz{

    \node[latent, minimum size=2cm ] (x1_0) {$\omega_1(t)$} ;
    \node[latent, below=of x1_0, minimum size=2cm ] (x2_0) {$\omega_2(t)$} ;

    \node[latent, right=of x1_0, minimum size=3cm ] (x1_0g) {$ \omega_1(t)\cdot f_e(a_1(t))$} ;
    \node[latent, right=of x2_0, minimum size=1.8cm, xshift=0.6cm , align=left] (x2_0g) {$\omega_2(t)\cdot$\\$f_e(a_2(t))$} ;
    
    \node[latent, right=of x1_0g, minimum size=3.8cm, yshift=-1.33cm, align=right] (x_0) {$\omega_1(t)\cdot f_e(a_1(t))$\\$+\omega_2(t)\cdot f_e(a_2(t))$ } ;
    
    \node[const, above=of x_0] (nx_0) {$\overbrace{\text{Pool}\hspace{2.5cm}\text{Share}}^{\text{\normalsize Cooperaci\'on}}$} ;
    
    \node[latent, right=of x1_0g, minimum size=2.5cm,  xshift=4.5cm] (x1_1) {$\omega_1(t+1)$ } ;
    \node[latent, below=of x1_1, minimum size=2.5cm, yshift=0.7cm] (x2_1) {$\omega_2(t+1)$ } ;
    
    \edge {x1_0} {x1_0g};
    \edge {x2_0} {x2_0g};
    \edge {x1_0g,x2_0g} {x_0};
    \edge {x_0} {x1_1,x2_1};
    
}
}
\caption{Estrategia cooperativa. Los agentes comienzan con los mismos recursos iniciales. Luego crecen independientemente de acuerdo con la ecuaci\'on \ref{eq:estrategia_base}. Luego cooperan poniendo sus recursos en un fondo común, que finalmente es dividio en partes iguales.}
\label{fig:protocolo}
\end{figure}

Las poblaciones enteramente cooperadoras reducen sus fluctuaciones, lo que genera una aumento en la tasa de crecimiento de todos sus miembros.
En la figura \ref{fig:ergodicity_cooperation} mostramos la trayectoria de un agente cooperador en una población de tamaño 33.
\begin{figure}[H]
    \centering
    \begin{subfigure}[b]{0.45\textwidth}
    \includegraphics[width=\linewidth]{figures/pdf/ergodicity_cooperation.pdf}
    \end{subfigure}
    \caption{
    Los recursos de un individuo de una población con 33 agentes que comparten su riqueza luego de cada iteración (recta verde), se pega al valor esperado (recta negra).
    Dejamos la tasa de crecimiento temporal individual caracterísitico (recta azul) como referencia visual.
    }
    \label{fig:ergodicity_cooperation}
\end{figure}
%
\paragraph{Conclusión Ole Peters} (La ventaja de la cooperación)\textbf{.} Mediante la cooperación total, todos los individuos logran acceder a tasas de crecimiento equivalentes al promedio de estados del sistema, que en los sistemas no-ergódicos es siempre superior que el promedio temporal que obtienen las estrategias individuales. \\

% Parrafo

Ole Peters considera que este aumento es suficiente para demostrar la ventaja evolutiva de la cooperación, lo que propone como principal explicación de las transiciones evolutivas.
Sin embargo no considera el problema de la deserción, quien dice ``our cooperators are unable to break the cooperative pact''.
No parece ser un problema menor, teniendo en cuenta la tentación que podría significar dejar de aportar al fondo común mientras se siguen recibiendo los beneficios de este.

\subsection{Estabilidad evolutiva}

Para ejemplificar el problema de la estabilidad evolutiva de la cooperación en este sistema, analicemos las tasas de crecimiento de las estrategias cooperadora y desertora en poblaciones mixtas.
La población más chica posible consiste de dos agentes.
En este caso las tasas de crecimiento temporal caracterísitica para las diferentes estrategias son (demostraciones en el Anexo):
%
\begin{equation}
   f(\cdot,\cdot) = \bordermatrix{ & C & D \cr
      C & \approx 1.0 & \approx 0.47 \cr
      D & \approx 0.95 & \approx 0.95 } 
\end{equation}
%
El primer agente que ``decida'' desertar unilateralmente va a ver reducida su tasa de crecimiento de $f(C,C) = 1.0$ a $ f(D,C) = 0.95$.
% 
\begin{conclution}[La no necesidad de castigos]
Sin necesidad de introducir castigos, las estrategias desertoras afectan negativamente su tasa de crecimiento a largo plazo a causa de su propio comportamiento, pues al evitar compartir sus recursos generan un aumento de sus fluctuaciones.
\end{conclution}
Esto parecería apoyar la idea de Ole Peters de que la cooperación no es altruista sino que está impulsada por el interés personal.
Sin embargo, la frecuencia de los agentes evolutivos en la población depende no del valor absoluto de la tasa de crecimiento, sino de la diferencia respecto de las tasas de crecimiento de las otras estrategias en la población.
Por más que la mutua cooperación ofrezca una tasa de crecimiento mayor que la deserción unilateral ($f(C,C) = 1.00 > 0.95 = f(D,C)$), un mutante desertor invadirá evolutivamente la población debido a que su tasa de crecimiento será mayor a la del agente cooperador ($f(D,C) = 0.95 > 0.47 = f(C,D)$). 

\begin{conclution}[Estabilidad evolutiva nivel 1]
Esto quiere decir que las estrategias cooperadoras no son evolutivamentes estables al interior de las poblaciones como proponía Ole Peters \footnote{La estructura de pagos coinicide con la matriz de pagos del Stag-Hunt. Los análisis que llegan a la conclusión de que las poblaciones enteramente cooperadoras son estrategias evolutivamente estables utilizan tasas de crecimiento de dinámicas aditivas en poblaciones infinitas, apartándose del modelo estandar de crecimiento exponencial.}.
\end{conclution}

\subsection{Objetivos}

Ole Peters no logra dar una respuesta completa que explique por qué las estrategias cooperadoras están favorecidas evolutivamente.
La idea general de que las grandes transiciones evolutivas involucra selección tanto a nivel individual como a nivel grupal, está ampliamente aceptada.
Para verificar si efectivamente las estrategias cooperativas se ven favorecidas por la evolución, como propone Ole Peters, será necesario entonces hacer un análisis multinivel.

A su vez, el co-autor del concepto de transiciones evolutivas \cite{szathmary1995-evolutionaryTransitions, szathmary2015-evolutionaryTransitions}, recientemente propuso analizar la evolución de las poblaciones sujetas a selección multinivel mediante modelos jerárquicos bayesianos~\cite{czegel2019-bayesianEvolution}.
Más allá de la popuesta, Czegel, Zachar y Szathmary~\cite{czegel2019-bayesianEvolution} no logran ofrecer un modelo jerarquico bayesiano que represente selección multinivel, sus ejemplos son sólo pictóricos.

La utilización de inferencia Bayesiana para resolver la pregunta acerca de la estabilidad evolutiva de las estrategias cooperadoras bajo selección multinivel está justificada en tanto existe un isomorfimos entre la teoría de la probabilidad (teormea de Bayes) y la teoría de la evolución (replicator dynamic)~\cite{harper2009-replicatorAsInference,shalizi2009-replicatorAsInference}.

\paragraph{Objetivos:}
(1) Demostrar la estabilidad evolutiva de la cooperación bajo procesos multiplicativos y ruidosos (2) mediante un modelo jerárquico bayesiano que represente la evolución de las poblaciones y las estrategias bajo selección multinivel.

\section{Metodología}

En esta sección presentamos el ismorfismo entre la teoría de la probabilidad y la evolución.
En la sección resultados utilizamos modelos bayesianos para resolver los problemas evolutivos.

\subsection{Teoría de la probabilidad}

La teoría de la probabilidad es el enfoque más utilizado en la actualidad para tratar la incertidumbre.
Sus reglas han sido derivadas formalmente a partir de sistemas axiomáticos conceptualmente distintos e independientes entre sí~\cite{halpern2017-RAU2}, lo cual es uno de los punto fuertes a su favor.
Pero quizás más importante sea que su aplicación estricta maximiza la incertidumbre dada la información empírica (datos) y formal (modelos causales)~\cite{jaynes2003-bookProbabilityTheory}, fuente de validación de las proposiciones de las ciencias empíricas.

Toda la teoría de la probabilidad puede resumirse en dos reglas: la~\ref{eq:sum_rule} y la~\ref{eq:product_rule}.
La \ref{eq:sum_rule} afirma que cualquier distribuci\'on marginal se puede obtener integrando o sumando la distribuci\'on conjunta.
\begin{equation} \label{eq:sum_rule}
 \tag{\en{sum rule}\es{regla de la suma}}
 P(x) = \sum_{y} P(x,y) \ \ \ \ \ \text{or} \ \ \ \ \ p(x) = \int p(x,y) \, dy
\end{equation}
Donde $P(\cdot)$ y $p(\cdot)$ representan distribuciones de probabilidad discretas y continuas respectivamente.
Por su parte, la \ref{eq:product_rule} se\~nala que cualquier distribuci\'on conjunta puede ser expresada como el producto de distribuciones condicionales uni-dimensionles.
\begin{equation}\label{eq:product_rule}
\tag{\en{product rule}\es{regla del producto}}
 p(x,y) = p(x|y) p(y)
\end{equation}
De la regla del producto obtenemos inmediatamente el~\ref{eq:bayes_theorem},
\begin{equation}\label{eq:bayes_theorem}
\tag{\en{Bayes' theorem}\es{teorema de Bayes}}
 p(y|x) = \frac{p(x|y)p(y)}{p(x)}
\end{equation}
La creencia a posteriori no es más que la creencia a priori que continúa siendo compatible con los datos.
%El \ref{eq:bayes_theorem} actualiza las creencia maximizando la incertidumbre luego de haber incorporado la información provista por el modelo y los datos.

\subsection{Isomorfismo entre las teorías de la evolución y la probabilidad.}

Existe un isomorfismo entre las ecuaciones fundamentales de la teoría de la evolución (replicator dynamic) y la teoría de la probabilidad (teorema de bayes),

\begin{equation} 
 p(e|a) = \frac{p(a|e)\,p(e)}{p(a)}   \ \ \ \ \ \ \ \   x_e^\prime = \frac{f_e(a)\,x_e}{X}  
\end{equation}
%
donde $e$ es la estrategia, $a$ el resultado observable. 
El isomorfimo ente la inferecia bayesiana y la evolución nos permite traducir un problema en términos del otro.
%
\begin{align*}
\centering
 \begin{tabular}{l|l}
  Teorema de Bayes & Replicator dynamic  \\ \hline
  Prior $p(e)$ & Proporción previa $x_e$ \\ \hline
  Verosimilitud $p(a|e)$ & Fitness $f_e(a)$ \\ \hline
  Evidencia $p(a)$ & Población total $X$ \\ \hline
  Posterior $p(e|a)$ & Proproción posterior $x_e^\prime$ \\ \hline
 \end{tabular}
\end{align*}

Para traducir el fitness en términos de probabilidades necesitamos normalizar su valor para que sume 1.
Luego, la estrategia original propuesta por Ole Peters tiene el siguiente likelihood,
%
\begin{equation}
f_e(a) \propto  P(a|e) = \begin{cases}
 \frac{1.5}{1.5+0.6} & a = 1 \\
 \frac{0.6}{1.5+0.6} & a = 0
  \end{cases}
  \approx
\begin{cases}
 0.71 & a= 1 \\
 0.29 & a= 0
\end{cases}
\end{equation}
%
Otras estretgias se podrán descrbir mediante otras distribuciones de probabilidad.
Una forma general de representar todo el espacio de estrategias es mediane la distribución Bernoulli.
%
\begin{equation}
P(a|e) = \text{B}(a,e) = (1-e)^{(1-a)} \cdot e^a 
\end{equation}
%
con $a \in \{0,1\}$.
La distribución Binomial, por su parte, permite representar de forma compacta una secuencia de observaciones, donde $a$ representa ahora la cantidad total de éxitos obtenidas en $n$ eventos, $a \in \{0, \dots, n\}$.
%Para determinar cuál de todas las estrategia se ve favorecida por la evolución debemos definir la proproción inicial de las estrategias mediante una distribución de probabilidad a priori. 
Si comenzamos con un prior uniforme, el posterior tiene solución analítica (pertenecerá a la distribución Beta).
De esta forma extendemos el ejemplo propuesto por Ole Peters mediante el siguiente modelo bayesiano.

\begin{figure}[H]
\centering
\tikz{
    \node[latent] (e) {$e$};
    \node[const, right=of e] (en) {\ $p(e)=\text{Beta}(\alpha,\beta)$};
    \node[const, left=of e] (ne) {Estrategias: \ \ \ };
    
    
    \node[latent, below=of e] (r) {$a$};
    \node[const, right=of r] (rn) {$P(a|e) = \text{Binomial}(a,e)$};
    \node[const, left=of r] (nr) {Aptitud: \ \ \ };
    
    \edge {e} {r};
    }
\caption{Representación bayesiana de un sistema evolutivo compuesto únicamente por estrategias desertoras, basado en el juego propuesto por Ole Peters.}
\label{fig:modelo_beta_binomial}
\end{figure}
%
Este modelo nos permitirá analizar la estabilidad evolutiva del espacio de estrategias desertoras. 
Supongamos que la moneda está sesgada de modo que el $0.71$ de las veces sale cara y el $0.29$ sale seca.
En las siguientes figuras mostramos cómo cambia la proporción de las estrategias a medida que agregamos observaciones al modelo.
%
\begin{figure}[H]
    \centering
    \begin{subfigure}[b]{0.32\textwidth}
    \includegraphics[width=\linewidth]{figures/coin1.pdf}
    \caption{$T = 0$}
    \end{subfigure}
    \begin{subfigure}[b]{0.32\textwidth}
    \includegraphics[width=\linewidth]{figures/coin2.pdf}
    \caption{$T = 10$}
    \end{subfigure}
    \begin{subfigure}[b]{0.32\textwidth}
    \includegraphics[width=\linewidth]{figures/coin3.pdf}
    \caption{$T = 10^7$}
    \end{subfigure}
    \caption{Densidad (eje y) de las diferentes estrategias desertoras (eje x) a medida que avanza el juego ($T=0, \, T=10, \, T=10^7$).}
    \label{fig:estrategias_individuales}
\end{figure}

El proceso evolutivo selecciona las estrategias individuales que mejor adaptadas están al ambiente.
Cuando el ambientes genera los estados con una probabilidad de $p*=0.71$, entonces la estrategia individual mejor adaptada es la que en cada ronda apuesta $0.71$ de sus recursos al ambiente $a=1$, y $0.29$ de sus recursos al ambiente $a=0$.

% Parrafo

El ejemplo de Ole Peters supone que la moneda no está sesgada.
En este contexto, la estrategia utilizada por él para mostrar la ventaja de la cooperación no es la que está mejor adaptada en términos individuales.
¿Las estrategias individuales que mejor se adaptaron al ambiente tienen la misma tasa de crecimiento temporal que la de los grupos cooperativos?
¿Que ocurre cuando la probabilidad del ambiente estocástico varía? 
¿Será que si la probabilidad del ambiente no cambia, no hay ninguna ventaja a favor de la cooperación?
Responderemos estas y otras preguntas en la siguiente sección.
%Si ese fuera el caso, un pequeño costo asociado a la coordinación haría generaría una ventaja a favor de las estrategias individuales.
%Al menos en ambientes estocásticos estables, que mantinen 

\section{Resultados}

En esta sección analizaremos si efectivamente existe alguna ventaja evolutiva de la cooperación.
Esta pregunta permanece abierta por dos motivos.
Por un lado, hemos visto en la sección Introducción que las estrategias cooperativas propuesta por Ole Peters no son evolutivamente estable al interior de los grupos.
Por otro lado, hemos visto en la sección Metodología que Ole Peters sólo ha mostrado la conveniencia de la cooperación para una estrategia individual mal adaptada al ambiente.

% Parrafo

Para empezar, analicemos lo que ocurre con las poblaciones enteramente desertoras y enteramente cooperadoras (tamaño infinito).
La tasa de crecimiento característica de los agentes desertores la podemos calcular utilizando la media geomética (ecuación \ref{eq:geometric_mean}).
¿Cuál es la tasa de crecimiento caracterísitica de las estrategias cooperadoras?

% Parrafo

Supongamos que tenemos una población de tres cooperadores, $n=3$, que comparten sus recursos.
Compartir sus recursos significa hacer una promedio de su distribuciones luego de tirar la moneda.
Sea $a_i$ el resultado obtenido por el agente $i$.
En cada paso temporal, el posterior de todos los agentes cooperadores es,
\begin{equation}
\begin{split}
P(e|a) &\propto \frac{1}{3} \text{Prior}  \overbrace{P(a_1|e)}^{\hfrac{\text{Likelohood}}{\text{individual 1}}} + \frac{1}{3} \text{Prior}  \overbrace{P(a_2|e)}^{\hfrac{\text{Likelohood}}{\text{individual 2}}}  + \frac{1}{3} \text{Prior}  \overbrace{P(a_3|e)}^{\hfrac{\text{Likelohood}}{\text{individual 3}}}  \\
&= \text{Prior} \underbrace{(\frac{1}{3} P(a_1|e) + \frac{1}{3} P(a_2|e) + \frac{1}{3} P(a_3|e))}_{\text{likelihood cooperativo}}
\end{split}
\end{equation}
%
Es decir, el likelihood cooperativo es el promedio artimético de los likelihood individuales.
Los posibles likelihoods cooperativos para la población de tamaño 3 son,
%
\begin{equation}
P_c(a|e) \propto 
\begin{cases}
(1-e) & \text{ si } a(t) = 0 \\
\frac{1}{3} e + \frac{2}{3} (1-e)  & \text{ si } a(t) = 1 \\
\frac{2}{3} e + \frac{1}{3} (1-e)    & \text{ si } a(t) = 2 \\
e & \text{ si } a(t) = 3
\end{cases}
\end{equation}
%
Y en general, sea $n$ el tamaño de la población, $a$ la cantidad de éxitos, el likelihood cooperativo es
\begin{equation}
P_c(a|e,n) = \frac{a}{n} e + \frac{n-a}{n}(1-e)
\end{equation}
%
El likelihood conjunto, luego de observar $T$ estados del ambiente, es
%
\begin{equation}
P_c(a(1)\dots a(T)|e,n) = \prod^T_t P_c(a(t)|e,n)
\end{equation}
%
Del mismo modo que calculamos la tasa de crecimiento caracterísitica en la ecuación~\ref{eq:geometric_mean}, cuando la frecuencia de observaciones de los estados del ambiente tiende a la probabilidad real de generación de los estados $p^*$, el likelihood temporal caracterísitico es
%
\begin{equation}
\overline{f_c}(e,n) = \prod_{a=0}^n P_c(a|e,n)^{P(a|n,p^*)}
\end{equation}
%
donde $P(a|n,p^*)$ es la verdadera binomial que genera los estados $a$.
Cuando la población es muy grande, $n\rightarrow \infty$, en cada ronda habrá invariantemente una proproción $p*$ que obtuvo éxito, por lo que el likelihood caracterísitico se reduce a,
%
\begin{equation}
\lim_{n\rightarrow \infty} \overline{f_c}(e) = p^* e + (1-p^*)(1-e)
\end{equation}
%
El likelihood temporal caracterísitico de una estrategia $e$ en una población cooperativa infinita es la media aritmética, la cual es lineal respecto de la verdadera probabilidad de generación de los estados del ambiente.
En la siguiente figura graficamos el likelihood característico individual (líneas continuas) y cooperativos (líneas punteadas) de tres estrategias ($e \in \{0.5, 0.75, 0.99\}$) para diferentes probabilidades de generación del ambiente.
%
\begin{figure}[H]
    \centering
    \begin{subfigure}[b]{0.66\textwidth}
    \includegraphics[width=\linewidth]{figures/pdf/tasa-temporal-0.pdf}
    \end{subfigure}
    \caption{Likelihood característico bajo regímenes individuales (líneas continuas) y cooperativos infinitos (líneas punteadas) de tres estrategias ($e \in \{0.5, 0.75, 0.99\}$) en diferentes tipos de ambiente $p^*$.}
    \label{fig:fitness_temporal}
\end{figure}
%
La flecha representa la conclusión a la que llegamos en la introducción, que en un ambiente que genera los estados con una probabilidad de $0.5$, la estretgia $0.71$ aumenta la tasa de crecimiento caracterísitica de su media geométrica a su media aritmética.
El punto rojo representa la conclusión que sacamos en la sección metodología, que en un ambiente con $p^*=0.71$ la estrategia individual mejor adaptada es $e=0.71$.
Con esta imagen podemos sacar algunas conclusiones nuevas.
Arriba del punto rojo se encuentra el likelihood caracterísitica de la población cooperativa de la estrategia especialista $e=0.99$.

\begin{conclution}[La ventaja de la especialización]
La cooperación ofrece una ventaja a favor de las estrategias especialistas. Estrategias que individualmente están mal adaptadas al ambiente, cooperando superan incluso a la posblación cooperativa que individualmemte está bien adaptada.
\end{conclution}

Esta conclusión la hemos obtenido a partir de tasa de crecimiento caracterísitica de una población cooperativa infinita.
Para que sea una conclusión interesante este comportamiento debería ocurrir en poblaciones finitas, pparticularmente pequeñas.
En la siguiente figura graficamos las tasas de crecimiento caracterísitica de la estrategia especialista $e=0.99$ para poblaciones cooperativas de tamaño 1 a 5.
%
\begin{figure}[H]
    \centering
    \begin{subfigure}[b]{0.66\textwidth}
    \includegraphics[width=\linewidth]{figures/pdf/tasa-temporal-1.pdf}
    \end{subfigure}
    \caption{
    Tasa de superviencia temporal de la estrategia especialista ($e=0.99$) en función del ambiente, para diferentes tamaños de población cooperativa (de 1 a 5).
    La línea punteada negra represeta la tasa de crecimiento de una población cooperativa inifinitamente grande.
    Las rectas grises sse dejan como referecia visual de las estrtegias $e \in \{0.5, 0.71\}$ analizadas en la figura anterior.
    }
    \label{fig:multilevel-selection-1}
\end{figure}
%
Un población de tres agentes es suficiente para que la estrategia especialista mal adaptada individualmente, logre cooperativamente superar a la población cooperativa compuesta de estrategia individualmente bien adapatadas!
Es realmente extraordinario que un sistema tan simple como el que estamos analizando tenga conclusiones tan fudamentales para entender la complejidad de la vida.
El proceso multiplicativo al cuál está sujeto la vida favorece tanto la cooperación como la especialización.

% Parrafo

El nivel especialización depende del tamaño de la población.
Para analizarlo, fijamos el ambiente en $p^* = 0.71$ y analizamos como varía la tasa de crecimiento caracterísitica cooperativa para todo el rango de estrategias para diferentes tamaños de población (de 1 a 5).
%
\begin{figure}[H]
    \centering
    \begin{subfigure}[b]{0.66\textwidth}
    \includegraphics[width=\linewidth]{figures/pdf/tasa-temporal-2.pdf}
    \end{subfigure}
    \caption{
    La tasa de crecimiento temporal de las estretgias para diferentes tamaño de población enteramente cooperativa (de 1 a 5) en un ambiente $p^*=0.71$.
    }
    \label{fig:multilevel-selection-1}
\end{figure}
%
Ele eje x representa las diferentes estrategias.
Cada una de las curvas representa un tamaño de población cooperativa.
Los puntos indican la estretgia óptima para el tamaño de la población.
Cuando la población tiene tamaño 1, la estrategia mejor adaptada es la que apuesta con la misma probjabilidad que el ambiente.
Pero rapidamente, a medida que aumentamos el tamaño de la población, la cooperación favorece a las estrategias especialistas.

% 

Para que estas conclusiones impacten de lleno en la teoría de la evolución hace falta demostrar que la cooperación, no sólo puede resisitir la invasión de la desertores, sino que puede invadir población de desertores 



\begin{figure}[H]
    \centering
    \begin{subfigure}[b]{0.32\textwidth}
    \includegraphics[width=\linewidth]{figures/pdf/multilevel-selection-1.pdf}
    \caption{9/9}
    \label{fig:multilevel-selection-1}
    \end{subfigure}
    \begin{subfigure}[b]{0.32\textwidth}
    \includegraphics[width=\linewidth]{figures/pdf/multilevel-selection-2.pdf}
    \caption{8/9}
    \label{fig:multilevel-selection-2}
    \end{subfigure}
    \begin{subfigure}[b]{0.32\textwidth}
    \includegraphics[width=\linewidth]{figures/pdf/multilevel-selection-3.pdf}
    \caption{7/9}
    \label{fig:multilevel-selection-3}
    \end{subfigure}
    \caption{
    Posterior de las estrategias individuales para una población de 9 agentes con 0, 1 y 2 desertores (\ref{fig:multilevel-selection-1}, \ref{fig:multilevel-selection-2} y \ref{fig:multilevel-selection-3} respectivamente).
    Se observa un cambio de fase en la selección de los desertores: en la figura \ref{fig:multilevel-selection-2} se seleccion
    }
    \label{fig:multilevel-selection-1}
\end{figure}


\begin{figure}[H]
    \centering
    \begin{subfigure}[b]{0.66\textwidth}
    \includegraphics[width=\linewidth]{figures/pdf/multilevel-selection-4.pdf}
    \end{subfigure}
    \caption{
    }
    \label{fig:multilevel-selection-4}
\end{figure}

\section{Conclusiones}

Es realmente extraordinario que un sistema tan simple como el que estamos analizando tenga conclusiones tan fudamentales para entender la complejidad de la vida.
El proceso multiplicativo al cuál está sujeto la vida favorece tanto la cooperación como la especialización.


\textbf{Hasta donde sabemos, nuestro trabajo sería el primero que desarrolla un modelo jerárquico bayesiano para resolver un problema de evolución bajo selección multinivel.}

%Aquí mostramos, basándonos en resultados anteriores que conectan el replicator dynamic y la inferencia bayesiana \cite{}, que si bien las estrategias cooperadoras no son evolutivamente estables al interior de los grupos, estas se ven favorecidas evolutivamente a través de la selección multinivel.

%Con la idea de analizar los efetos de la selección multinivel, traduicimos el modelo de Ole Peters a una distirbución de probabilidad jerárquica, la cual describiremos en términos gráficos y analizaremos utilizando solamente las reglas de la probabilidad.

En este trabajo, resolviendo dos problemas presentes en la literatura.
Con la propuesta metodológia de \cite{czegel2019-bayesianEvolution} (la selección multinivel como inferencia bayesiana jerárquica) resolvimos la demostración que le faltaba al modelo de Ole Peters (procesos multiplicativos ruidosos) que verifica que la cooperación está favorecida por la evolución.
Y a su vez, con el modelo de Peters proveímos el ejemplo concreto que le faltaba a la propuesta metodológica de \cite{czegel2019-bayesianEvolution}.
Ambas soluciones combinadas ofrecen una solución nueva al problema de las transiciones evolutivas mayores, que es más sencilla que las anterios (procesos multiplicativos ruidosos), basada en principios matemáticos bien fundados (la aplicación estrica de las reglas de la probabilidad).

Según \cite{czegel2019-bayesianEvolution} el isomorfismo entre los procesos evolutivos y la inferencia bayesiana multinivel,  ``support a learning theory-oriented narrative of evolutionary complexification: the complexity and depth of the hierarchical structure of individuality mirror the amount and complexity of data that have been integrated about the environment through the course of evolutionary history.''
Esta especulación es rechazada por el modelo de Ole Peters, el cual muestra que un simple proceso multicativo ruidoso favorece la selección multinivel de grupos cooperativos por sobre grupos con desertores.

Por otra parte, según \cite{peters} su modelo ``paints a picture of cooperation driven by self-interest, not altruism, with cooperators outgrowing similar non-cooperators''.
Esta especulación es rechazada por la metodología de czegel \cite{czegel2019-bayesianEvolution}, la cual muestra que si bien las estrategias cooperativas no son evolutivamente estables al interior de los grupos, estos se ven favorecidos gracias a la selección multinivel.

Además \cite{czegel2019-bayesianEvolution} dice que, ``This isomorphism allows for a natural interpretation of evolutionary transitions in individuality as \emph{learning the structure}''.
En este trabajo mostramos que lo que se aprende no es la estructura, sino que \textbf{aprenden la dinámica}, en particular el efecto no-ergódico de los procesos multiplicativos.

%La posibilidad de supervivencia y reproducción de una población depende no sólo del sistema de reciprocidad para la propagación del riesgo dentro de las poblaciones y entre las poblaciones de diferentes especies.


% La ciencia es una institución humana que tiene pretención de verdad, esto es de formular proposiciones que valgan para todas las personas, tanto intercultural como intersubjetivamente.
% Las ciencias formales validan sus proposiciones mediante teoremas, resultados derivados de aplicar las reglas internas a un sistema axiomático cerrado.
% Las ciencias empíricas, en cambio, deben validar sus proposiciones dentro de sistemas abiertos, lo que introduce siempre un grado de incertidumbre asociada.
% ¿Cuál es entonces la fuente de validez universal del conocimiento empírico?
% 
% Existe un principio epistemológico con validez intercultural, conocido como el \emph{principio de indiferencia}, afirma que si ante un espacio de hipótesis común llegamos a un acuerdo respecto de que no tenemos información previa, entonces estaremos de acuerdo en distribuir la creencia (aka incertidumbre) en partes iguales.
% Esta distribución de creencias, que permite el acuerdo intersubjetivo, la vamos a llamar creencia honesta.
% La fuente de validez universal se fundamenta en la maximización de la incertidumbre.
% ¿Se puede extender este principio para casos en los cuales tenemos información?
{\footnotesize
\bibliographystyle{auxiliar/biblio/plos2015.bst}
\bibliography{auxiliar/biblio/biblio_notUrl.bib}
}

\section{Apéndice}



\subsection{Tasa de crecimiento en poblaciones desertoras}

Sea $r$ el resultado generado por el ambiente, y sea $e$ la estrategia individual, luego el fitness individual es
\begin{equation}
f(r|e) \propto 
\begin{cases}
(1-e) & \text{ si } r = 0 \\
e & \text{ si } r = 1
\end{cases}
\end{equation}



\subsection{Tasa de crecimiento en poblaciones cooperadoras}

En el modelo multinivel bayesiano vemos que ocurre con diferentes  poblaciones (Coop-Coop, Coop-Des, Des-Des) de una estrategia, $e = 0.71$, en un ambiente, $a = 0.5$.

Sería interesante extender ese modelo para que considere al mismo tiempo todas las posibes estrategias.

Cuando hacemos el análisis individual, la proproción de las estretgias la representamos con una distribución Beta y el fitness con una likelihood binomial.

Arrancamos con un Beta a priori uniforme

\begin{equation}
p(e_0) = \text{Beta}(1,1) \propto e^1 (1-e)^1 
\end{equation}

En el primer paso, cada individuo tira la moneda, y recibe un likelihood-fitness aleatorio.
Supongamos que tenemos una población de tres cooperadores, $N=3$, que comparten sus recursos.
En el primer paso, compartir sus recursos significa hacer una promedio de su distribuciones a priori Beta$(1,1)$ luego de obtener el primer resultado.

\begin{equation}
p(e|r) \propto 
\begin{cases}
e^1 (1-e)^2 & \text{ si } r = 0 \\
\frac{1}{3} e^1 (1-e)^2 + \frac{2}{3} e^2 (1-e)^1  & \text{ si } r = 1 \\
\frac{2}{3} e^2 (1-e)^1 + \frac{1}{3} e^1 (1-e)^2    & \text{ si } r = 2 \\
e^2 (1-e)^1 & \text{ si } r = 3
\end{cases}
\end{equation}

Cuando ambos sacan el mismo resultado, la distribución Beta aumenta en un sólo punto, pues recibe como producto $(1-e)$ y $e$ respectivamente.
En el caso $r = (0,0,1)$ recibe como producto $(\frac{1}{3} + \frac{1}{3}e )$, y en el caso $r = (0,1,1)$ recibe como producto $(\frac{2}{3} - \frac{1}{3}e )$.
Veamos el primero,

\begin{equation}
\begin{split}
p(e_1) & \propto \frac{1}{3} e^1 (1-e)^2 + \frac{2}{3} e^2 (1-e)^1  \\
& =  e^1 (1-e)^1 ( \frac{1}{3} (1-e) +  \frac{2}{3} e)  \\
& =  \underbrace{e^1 (1-e)^1}_{\text{prior}} \underbrace{(1/3 + (1/3)e )}_{\text{fitness}}
\end{split}
\end{equation}

Es decir, el primer posterior simplemente es el producto del prior por una de esas pendientes.
Y esa pendiente es compartida por todos los cooperadores.
Por lo que el segundo posterior va a ser nuevamente agregar como producto una de esas pendientes.
Es decir, en cada paso se van acumulando las pendietes.
Los posibles fitness de los cooperadores son entonces las pendientes, distribuidas como una Binomial. 


\begin{equation}
p(e_t) \propto \prod_i^t \text{pendiente}_i \propto (1-e)^{n_0} (\frac{1}{3} + \frac{1}{3}e )^{n_1} (\frac{2}{3} - \frac{1}{3}e )^{n_2} e^{n_3}
\end{equation}

Donde $n_i$ representa la cantidad de veces que el conjunto de cooperadores recibieron como pago cada una de las pendientes, con $\sum n_i = t$.
En el límite todas las trayectorias individuales estarán gobernadas por el mismo factor $\overline{f}$.

\begin{equation}
\begin{split}
\lim_{T \rightarrow \infty} p(e_T) & \propto \overline{f}^T \\
(\lim_{T \rightarrow \infty} p(e_T) )^{1/T} & \propto \overline{f} \\
(1-e)^{p(r=0)} (\frac{1}{3} + \frac{1}{3}e )^{p(r=1)} (\frac{2}{3} - \frac{1}{3}e )^{p(r=2)} (e)^{p(r=3)} & 
\end{split}
\end{equation}


\vspace{0.3cm}

\paragraph{Hipótesis inductiva} Generalización del fitness de los cooperadores sin desertores.
Sea $n$ el tamaño de la población, $r$ la cantidad de éxitos y $n-r$ la cantidad de fracasos, el fitness de los cooperadores es
\begin{equation}
f_c(r|e,n) = \frac{r}{n} e + \frac{n-r}{n}(1-e)
\end{equation}

La distribución de probabilidad del fitness de los cooperadores es
\begin{equation}
p(f_c(r|e,n)) = \text{Binomial}(r,n)
\end{equation}

y la probabilidad de las estrategias es
\begin{equation}
\overline{f_c}(e,n) = \prod_{r=1}^n \left( \frac{r}{n} e + \frac{n-r}{n}(1-e) \right)^{p(f_c(r|e,n))}
\end{equation}



\subsection{Tasa de crecimiento poblaciones mixtas}

Generalización del fitness de los cooperadores cuando existen desertores.

\paragraph{Hipótesis inductiva} Sea $N$ el tamaño de la población, sea $n$ la cantidad de cooperadores y $N-n$ la cantidad de desertores, sea  $r$ la cantidad de éxitos obtenidos por la población cooperadora.

\begin{equation}
f_c(r|e,n,N) =  \frac{r}{N} e + \frac{n-r}{N}(1-e) = \frac{n}{N} \left( \frac{r}{n} e + \frac{n-r}{n}(1-e) \right)
\end{equation}

La distribución de probabilidad del fitness de los cooperadores es
\begin{equation}
p(f_c(r|e,n,N)) = \text{Binomial}(r,n)
\end{equation}

y la probabilidad de las estrategias es
\begin{equation}
\overline{f_c}(e,n,N) = \prod_{r=1}^n \left( \frac{r}{N} e + \frac{n-r}{N}(1-e) \right)^{p(f_c(r|e,n,N))}
\end{equation}


¿Cuál es la tasa de crecimiento de los desertores?
Supongamos que primero juega y después recibe la parte del fondo común.
\begin{equation}
w(D|e,d,a,t) =
\begin{cases}
 f_d(1) & \ \  t=1 \\
 (w(D|e,d,a,t-1) + w(C|e,d,a,t-1)) f_d(t) & \ \  t>1
\end{cases}
\end{equation}
En $t=1$ actualiza sus recursos con el pago recibido $f_d(1)$.
En $t>1$ incorpora la recursos recibidos del fondo común generado en el tiempo anterior, $w(C|e,d,a,t-1)$, y finalmente actualiza sus recursos totales (la suma) con el pago $r(t)$.


Veamos la forma que tiene el tamaño de la población en el tiempo
\begin{align}
w(D|t=2) & = r(2) (w(D|t=1) + w(C|t=1)) \\
& = r(2)r(1) + r(2)w(C|t=1))
\end{align}
Para simplificar la notación, hacemos implicitos los valores de $e$, $d$ y $a$.
Además, como el tamaño de los recursos del fondo común no dependen del tiempo

\begin{align}
w(D|t=3) & = r(3) (w(D|t=2)+w(C|t=2)) \\
& = r(3) (r(2)r(1) + r(2)w(C|t=1) + w(C|t=1) ) \\
& = r(3)r(2)r(1) + r(3)r(2)w(C|t=1) + r(3)w(C|t=2) 
\end{align}

Acá ya podemos ver un patrón a partir del cual proponemos la siguiente hipótesis inductiva (HI).
\begin{equation} \label{eq:HI} \tag{HI} 
w(D|t) = \prod^t_{i=1} r(i) + \sum^{t-1}_{j=1} w(C|t=j)
\prod^t_{j<k} r(k)
\end{equation}

\paragraph{Caso Base.} Queremos ver que vale la \ref{eq:HI} en el tiempo $t=1$.
\begin{equation} 
w(D|t=1) \overset{\ref{eq:HI}}{=} \prod^1_{i=1} r(i) + \underbrace{\sum^{0}_{j=1} \Delta w(C)^j \prod^1_{j<k} r(k) }_{\text{vale $0$ por rango}} = r(1)
\end{equation}
Luego, vale el caso base.

\paragraph{Paso inductivo.} Dado que vale $w(D|t)$ quiero ver que vale $w(D|t+1)$: $w(D|t) \Rightarrow w(D|t+1)$

\begin{equation}
\begin{split}
w(D|t+1) &:= (w(D|t) + w(C|t)) r(t)  \\
& \overset{\ref{eq:HI}}{=} \bigg( \big(\prod^t_{i=1} r(i) + \sum^{t-1}_{j=1} w(C|t) \prod^t_{j<k} r(k) \big)+ w(C|t) \bigg) r(t+1) \\
& = r(t+1) \prod^t_{i=1} r(i) + \sum^{t-1}_{j=1} \Delta w(C)^j r(t+1) \prod^t_{j<k} r(k) + w(C|t)  r(t+1) \\ 
& = \prod^{t+1}_{i=1} r(i) + \sum^{t-1}_{j=1} \Delta w(C)^j  \prod^{t+1}_{j<k} r(k) + w(C|t)  r(t+1)  \\
& = \prod^{t+1}_{i=1} r(i) + \sum^t_{j=1} \Delta w(C)^j  \prod^{t+1}_{j<k} r(k) 
\end{split}
\end{equation}
Luego, vale el paso inductivo. 




\subsection{Tasa de crecimiento en poblaciones mixtas infinitas}

Las estrtegias que en una población infinita representan una proporción mayor a $0$ son a su vez infinitas.
Sea $d$ la proporción de desertores y $1-d$ la proporción de cooperadores.

% 

La población cooperadora, al ser infinita, siempre crece como
\begin{equation}
\Delta w(C|e,d,a) = (1-d) \underbrace{(e\cdot a + (1-e)\cdot(1-a))}_{\text{Media de estados}}
\end{equation}
En cada paso las apuestas del conjunto de cooperadores crece como la media de estados, la cual se divide en partes iguales.
Como el cambio en los recursos no depende de $t$, en t pasos la población tiene un tamaño de
\begin{equation}
w(C|e,d,a,t) = \Delta w(C|e,d,a)^t 
\end{equation}


\begin{figure}[H]
    \centering
    \begin{subfigure}[b]{0.66\textwidth}
    \includegraphics[width=\linewidth]{figures/pdf/cpr_analyticTimeAverage_frequency-based-fitness.pdf}
    \end{subfigure}
    \caption{
    }
    \label{fig:cpr_analyticTimeAverage_frequency-based-fitness}
\end{figure}


\begin{figure}[H]
    \centering
    \begin{subfigure}[b]{0.66\textwidth}
    \includegraphics[width=\linewidth]{figures/pdf/multilevel-selection-5.pdf}
    \end{subfigure}
    \caption{
    }
    \label{fig:multilevel-selection-5}
\end{figure}




\subsection{Selección multinivel}

Primero definimos la probabilidad de los grupos mediante un prior uniforme.
Esto significa que no tenemos preferencia sobre ningún modelo.
Los modelos compiten bajos mismas condiciones iniciales.
Vamos a suponer la existencia de cuatro grupos.
Un grupo de 2 cooperadores, un grupo de 1 cooperador y 1 desertor, un grupos de desertores, y un grupo adicional con estrategias individuales bien adaptadas.

\begin{align}
\centering
P(r_{t}|e_{t}) \propto \begin{tabular}{|c|c|c|c|c|c|c|}
        \hline
        & $r_t=(0,0)$ & $r_t=(0,1)$ & $r_t=(1,0)$ &  $r_t=(1,1)$  \\ \hline
       $e_{t}=1$ & $0.6$ & $1.5$ & $0.6$ & $1.5$ \\ \hline
       $e_{t}=2$ & $0.6$ & $0.6$ & $1.5$ & $1.5$  \\ \hline
       $e_{t}=3$ & $0.6$ & $1.5$ & $0.6$ & $1.5$  \\ \hline
       $e_{t}=4$ & $0.6$ & $0.6$ & $1.5$ & $1.5$ \\ \hline
       $e_{t}=5$ & $0.6$ & $1.5$ & $0.6$ & $1.5$ \\ \hline
       $e_{t}=6$ & $1.05$ & $1.05$ & $1.05$ & $1.05$  \\ \hline
\end{tabular}
\end{align}





\begin{align}
\centering
P(e_{t}|e_{t-1}) = \begin{tabular}{|c|c|c|c|c|c|c|}
        \hline
        & $e_t=1$ & $e_t=2$ & $e_t=3$ &  $e_t=4$ & $e_t=5$ & $e_0=6$ \\ \hline
       $e_{t-1}=1$ & $0.5$ & $0.5$ & $0$ &  $0$ & $0$ & $0$ \\ \hline
       $e_{t-1}=2$ & $0.5$ & $0.5$ & $0$ & $0$ & $0$ & $0$ \\ \hline \hline
       $e_{t-1}=3$ & $0$ & $0$ & $0.5$ & $0.5$ & $0$ & $0$ \\ \hline
       $e_{t-1}=4$ & $0$ & $0$ & $0$ & $1.0$ & $0$ & $0$ \\ \hline \hline
       $e_{t-1}=5$ & $0$ & $0$ & $0$ & $0$ & $1.0$ & $0$ \\ \hline \hline
       $e_{t-1}=6$ & $0$ & $0$ & $0$ & $0$ & $0$ & $1.0$ \\ \hline
\end{tabular}
\end{align}




\begin{align}
\centering
P(g) = \begin{tabular}{|c|c|c|c|}
        \hline
        $g=1$ & $g=2$ & $g=3$  & $g=4$\\ \hline
        $1/4$ & $1/4$ & $1/4$ & $1/4$ \\ \hline
\end{tabular}
\end{align}

\begin{align}
\centering
P(e_0|g) = \begin{tabular}{|c|c|c|c|c|c|c|}
        \hline
        & $e_0=1$ & $e_0=2$ & $e_0=3$ &  $e_0=4$ & $e_0=5$ & $e_0=6$ \\ \hline
       $g=1$ & $0.5$ & $0.5$ & $0$ &  $0$ & $0$ & $0$ \\ \hline
       $g=2$ & $0$ & $0$ & $0.5$ & $0.5$ & $0$ & $0$ \\ \hline
       $g=3$ & $0$ & $0$ & $0$ & $0$ & $1.0$ & $0$ \\ \hline
       $g=4$ & $0$ & $0$ & $0$ & $0$ & $0$ & $1.0$ \\ \hline
\end{tabular}
\end{align}





\begin{figure}[H]
\centering
\tikz{
    \node[latent] (m) {$G$};

    \node[latent, right=of m] (e0) {$e_0$};
    
    \node[latent, right=of e0] (e1) {$e_1$};
    \node[latent, below=of e1] (r1) {$r_1$};
    
    \node[latent, right=of e1] (e2) {$e_2$};
    \node[latent, below=of e2] (r2) {$r_2$};
    
    \node[latent, right=of e2] (e3) {$e_3$};
    
    
    \edge {m} {e0};
    \edge {e0} {e1};
    \edge {e1} {r1,e2};
    \edge {e2} {r2,e3};
}
\caption{}
\label{fig:modelo_grafico}
\end{figure}




\end{document}
