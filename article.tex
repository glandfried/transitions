\documentclass[a4paper,10pt]{article}
\usepackage[utf8]{inputenc}
\input{auxiliar/tex/encabezado.tex}
\input{auxiliar/tex/tikzlibrarybayesnet.code.tex}
\newif\ifen
\newif\ifes
\newcommand{\en}[1]{\ifen#1\fi}
\newcommand{\es}[1]{\ifes#1\fi}
\entrue

\newcommand{\E}{\en{S}\es{E}}
\newcommand{\A}{\en{E}\es{A}}
\newcommand{\Ee}{\en{s}\es{e}}
\newcommand{\Aa}{\en{e}\es{a}}



\newtheorem{conclution}{\en{Conclution}\es{Conclusión}}%[section]
\newtheorem{objective}{\en{Objective}\es{Objetivo}}%[section]



%opening
\title{Multilevel selection in causal models: the non-ergodicity of evolutionary and probabilistic theories as the general driver of the emergence of cooperation and specialization. }
\author{Gustavo Landfried}

\begin{document}

\maketitle

\begin{abstract}
\en{In the last third of the universe's history, a simple self-replicating organization of matter emerged on earth. }%
\es{En el último tercio de la historia del universo surgió en la tierra una organización de la materia simple capaz de autoreplicarse. }%
%
\en{The growth of these lineages followed multiplicative and noisy processes: sequences of survival and reproduction rates. }%
\es{El crecimiento de estos linajes siguieron procesos multiplicativos y ruidosos: secuencias de tasas de supervivencia y reproducción. }%
%
% \en{The errors produced during replication diversified the life forms, and the growth rates of the different strategies favored those better adapted to the environment. }%
% \es{Los errores producidos durante la replicación diversificaron las formas de vida, y las tasas de crecimiento de las diferentes estrategias favorecieron a aquellas mejor adaptadas al ambiente. }%
% %
\en{The current complexity of life is the consequence of a series of evolutionary transitions in which entities capable of self-replication after the transition become part of higher level cooperative units. }%
\es{La complejidad actual de la vida es consecuencia de una serie de transiciones evolutivas en las que entidades capaces de autoreplicación luego de la transición pasan a formar parte de unidades cooperativas de nivel superior. }%
%
% \en{How to explain this permanent tendency of life in favor of cooperative aggregation and specialization? }%
% \es{¿Cómo se explica esta tendencia permanente de la vida en favor de la agregación cooperativa y la especialización? }%
\en{To explain evolutionary transitions, it is necessary to demonstrate the evolutionary advantage of cooperation, but also the advantage of specialization. }%
\es{Para explicar las transiciones evolutivas es necesario demostrar la ventaja evolutiva de la cooperación, pero también la ventaja de la especiliazación. }%

% Parrafo

\en{Recently it has been shown that, as a consequence of the non-ergodicity of multiplicative processes, fluctuations have a negative effect on individual growth rates, which can be reduced through mutual cooperation, leading to an increase in the long-term growth rates of all individuals. }%
\es{Recientemente se ha mostrado que, como consecuencia de la no-ergodicidad de los procesos multiplicativos, las fluctuaciones tienen un efecto negativo en las tasas de crecimiento individuales, las cuales pueden ser reducidas a través de la mutua cooperación, produciendo un aumento de la tasas de crecimiento a largo plazo de todos los individuos. }%
%
% \en{Ole Peters believs that the increase in the growth rate is sufficient argument to demonstrate the evolutionary advantage of cooperation, which he proposes as the main explanation for evolutionary transitions. }%
% \es{Ole Peters considera que el aumento de la tasa de crecimiento es argumento suficiente para demostrar la ventaja evolutiva de la cooperación, lo que propone como principal explicación de las transiciones evolutivas. }%
% %
% \en{However, he does not consider the problem of defection, who says ``our cooperators are unable to break the cooperative pact''. }%
% \es{Sin embargo no considera el problema de la deserción, quien dice ``our cooperators are unable to break the cooperative pact''. }%
% %
% \en{Yaari suggests a group selection argument but he did not provide a formal demonstration. }
% \es{Yaari sugiere un argumento de selección de grupos pero no prové una demostración formal. }
\en{However, these papers do not provide a formal proof of the evolutionary advantage of cooperation. }%
\es{Sin embargo estos trabajos no porveen una demostración formal que justifique la ventaja evolutiva de la cooperación. }%
% Parrafo
%
\en{For this purpose it is necessary to consider selection at both the individual and group level. }%
\es{Para este propósito es necesario considerar selección tanto a nivel individual como a nivel grupal. }%
%
\en{The co-author of the concept of evolutionary transitions (Szathmáry) recently proposed to analyze the evolution of populations subject to multilevel selection by means of Bayesian hierarchical models, making use of the isomorphism between evolutionary theory and Bayesian inference. }%
\es{El co-autor del concepto de transiciones evolutivas (Szathmary) propuso recientemente analizar la evolución de las poblaciones sujetas a selección multinivel mediante modelos jerárquicos bayesianos, haciendo uso del isomorfismo entre las teoría de la evolución y la inferencia bayesiana. }%
%
\en{However, this work does not provide any model that performs multilevel selection, so the proposal remains open. }%
\es{Sin embargo, este trabajo no proveé ningun modelo que realice selección multinivel, por lo que la propuesta sigue abierta. }%

% Parrafo

\en{In this paper we specify a probabilistic causal model, in which individuals are affected by the environment and by the social behaviors of cooperation and defection of their context. }%
\es{En este trabajo especificamos un modelo causal probabilístico, en el que los individuos se ven afectados por el ambiente y por los comportamientos sociales de cooperación y deserción de su contexto. }%
%
\en{Under this minimal set of hypotheses, where we consider \emph{unconditionally} cooperative individuals who generate a common good that can be exploited by defecting individuals without receiving some kind of punishment in return (e.g. end of cooperation), the evolution of cooperation literature predicts defection as the only evolutionarily stable strategy. }%
\es{Bajo este conjunto mínimo de hipótesis, donde consideramos individuos \emph{incondicionalmente} cooperadores que generan un bien común que puede ser explotado por individuos desertores sin que reciban a cambio algún tipo de castigo (e.g fin de la cooperación), la literatura de evolución de la cooperación predice que la deserción es la única estrategia evolutivamente estable. }%
%
% \en{It is assumed that common goods, if not accompanied by special conditions (such as communication allowing coordination, memory allowing rewards or punishments, etc.), lead to their overexploitation, because even if the optimum is obtained through mutual cooperation there would be an individual incentive to defect. }%
% \es{Se supone que los bienes comunes, si no están acompañados de condiciones especiales (como la comunicación que permita la coordinación, la memoria que permita aplicar premios o castigos, etc), conducen a su sobreexplotación, porque aunque el óptimo se obtenga a través de la mutua cooperación habría una incentivo individual para desertar. }%
%
\en{However, probabilistic inference shows that cooperative individuals are favored by multilevel selection. }%
\es{Sin embargo, la inferencia probabilística muestra que los individuos cooperadoras se ven favorecidas mediante selección multinivel. }%
%
\en{While resource-avoidance strategies can invade entirely cooperative groups by natural selection, such behavior at the same time increases their own individual fluctuations, reducing their own long-term growth rate without the need to introduce penalties. }%
\es{Si bien las estrategias que evitan compartir recursos pueden invadir por selección natural grupos enteramente cooperadores, tal comportamiento aumenta sus propias fluctuaciones individuales, reduciendo su propia tasa de crecimiento a largo plazo sin necesidad de introducir castigos. }%
%
\en{That is, contrary to the belief established since the mid-20th century in economics, we show that the dynamics of common goods cannot be represented by a prisoner's dilemma payoff matrix. }%
\es{Es decir, en contra de la creencia establecida que en economía se tiene desde mediados del siglo 20, mostramos que los dinámicas de bienes comunes no puede representarse mediante una matriz de pagos del dilema del prisionero. }%
%
\en{Moreover, contrary to the belief that specialization is too complex a feature to produce a benefit in simple aggregations, we show that as soon as cooperation emerges, an advantage in favor of specialist strategies appears even in groups of size 2. }
\es{Además, en contra de la creencia de que la especialización es una caracterísitica demasiado compleja para que produzca un beneficio en agregaciones simples, mostramos que apenas surge la cooperación, aparece una ventaja a favor de las estrategias especialistas incluso en grupos de tamaño 2. }%

% Parrafo

\en{Even if our causal model does not assume any kind of process to update resources, probabilistic inference is proportional to the resources obtained through a multiplicative process. }%
\es{Incluso si nuestro modelo causal no presupone ningún tipo de proceso para actualizar los recursos, la inferencia probabilistica es proporcional a los recursos obtenidos a través de un proceso multiplicativo. }%
%
\en{The multiplicative updating of the probabilities of individuals, which arises naturally from applying the rules of probability to the causal model, is in line with the long-established idea in evolutionary theory that the growth of lineages follow multiplicative processes. }%
\es{La actualización multiplicativa de las probabilidades de los individuos, que surge naturalmente de aplicar las reglas de la probabilidad al modelo causal, está en línea con la idea largamente establecida en la teoría de la evolución de que el crecimiento de los linajes siguen procesos multiplicativos. }% 
%
%
\en{The reason why an advantage in favor of cooperation and specialization arises in our simple causal model is due to the multiplicative (non-ergodic) nature of probabilistic theory and its isomorphism with evolutionary theory. }%
\es{El motivo por el cual surge una ventaja a favor de la cooperación y la especialización en simple modelo causal se debe a la naturaleza multiplicativa (no-ergódica) de la teoría de la probabililidad y a su isomorfismo con la teoría de la evolución. }%
\end{abstract}


\section{Introducción}

\en{In the last third of the history of the Universe, sometime around 4 billion years ago, a simple form of matter organization capable of self-replication appeared on Earth. }%
\es{En el último tercio de la historia del Universo, en algún momento hace aproximadamente 4000 millones de años, apareció en la tierra una forma de organización de la materia capaz de auto-replicarse. }%
%
\en{The growth of these lineages followed multiplicative and noisy processes: sequences of survival and reproduction rates. }%
\es{El crecimiento de estos linajes siguieron procesos multiplicativos y ruidosos: secuencias de probabilidades de supervivencia y reproducción. }%
%
\en{The errors produced during replication diversified the life forms, and the growth rates of the different strategies favored those better adapted to the environment. }%
\es{Los errores producidos durante la replicación diversificaron las formas de vida, y las tasas de crecimiento de las diferentes estrategias favorecieron a aquellas mejor adaptadas al ambiente. }%
%
\en{From that moment until now, life has acquired an extraordinary complexity. }%
\es{Desde aquel momento hasta ahora la vida adquirió una extraordinaria complejidad. }%
%
\begin{figure}[H]
    \centering
    \begin{subfigure}[b]{0.65\textwidth}
    \includegraphics[width=\linewidth]{auxiliar/images/biomass.jpg}
    \end{subfigure}
    \caption{
    \en{Current distribution of biomass on Earth estimated by Bar-On et al..~\cite{barOn2018-biomass}. }
	\es{Distribución actual de la biomasa en la Tierra estimada por Bar-On et al.~\cite{barOn2018-biomass}. }%
    }
    \label{fig:biomass}
\end{figure}
%
\en{The current complexity of life is the consequence of a series of evolutionary transitions in which entities capable of self-replication after the transition become part of higher level cooperative units~\cite{maynardSmith1995-majorTransitions, szathmary1995-evolutionaryTransitions, szathmary2015-evolutionaryTransitions}. }%
\es{La complejidad actual de la vida es consecuencia de una serie de transiciones evolutivas en las que entidades capaces de autoreplicación luego de la transición pasan a formar parte de unidades cooperativas de nivel superior~\cite{maynardSmith1995-majorTransitions, szathmary1995-evolutionaryTransitions, szathmary2015-evolutionaryTransitions}. }%
%
\en{Some of the paradigmatic transitions are: from replicating molecules to protocells; from prokaryotic to eukaryotic cells; and from protists to animals, plants and fungi (cell differentiation). }%
\es{Algunas de las transiciones paradigmáticas son: de las moléculas replicantes a las protocélulas; de las celulas procariotas a las eucariotas; y de los protistas a los animales, plantas y hongos (diferenciación celular). }%
%
\en{How to explain this permanent tendency of life in favor of cooperative aggregation and specialization? }%
\es{¿Cómo se explica esta tendencia permanente de la vida en favor de la agregación cooperativa y la especialización? }%
%``The transition must be explained in terms of inmmediate selection advantage to individual replicators'' szathmary1995-evolutionaryTransitions

% Parrafo

\en{In evolution, the growth of a lineage over time, $\omega(t)$, is governed by a stochastic sequence of survival and reproduction rates $f(\cdot)$ dependent on a random environment $a$, }
\es{En evolución, el crecimiento de un linaje en el tiempo, $\omega(t)$, esta gobernado por una secuencias estocástica de tasas de supervivencia y reproducción $f(\cdot)$ dependientes de un ambiente aleatorio $a$, }%
%
\begin{equation} \label{eq:modelo_exponencial}
\omega(T) = \prod_t^T f(a(t)) \approx g^T
\end{equation}
%
\en{where $a(t)$ represents the state of the environment at time $t$ and $g$ represents the characteristic growth rate when $T$ is sufficiently large. }%
\es{donde $a(t)$ representa el estado del ambiente en el tiempo $t$ y $g$ representa la tasa de crecimiento caracterísitica cuando $T$ es suficientemente grande. }%
%
\en{For example, suppose nature flips a coin, if it comes up heads the population reproduces 50\% and if it comes up tails it survives 60\%. }%
\es{Por ejemplo, supongamos que la naturaleza lanza una moneda, si sale cara la población se reproduce 50\% y si sale seca sobrevive 60\%. }%
\begin{equation} \label{eq:estrategia_base}
f(\Aa) =
\begin{cases}
 1.5 & \Aa = \text{ \en{Head}\es{Cara} } \\
 0.6 & \Aa = \text{ \en{Tail}\es{Sello} }
\end{cases}
\end{equation}
%
\en{A similar example was proposed by Lewontin and Cohen (1969)~\cite{lewontin1969-randomlyVaryingEnvironment}. }%
\es{Un ejemplo similar fue propuesto por Lewontin y Cohen (1969)~\cite{lewontin1969-randomlyVaryingEnvironment}. }%
%
\en{Other strategies $\Ee$ will have other functions $f(\Ee,\Aa)$. }%
\es{Otras estrategias $\Ee$ tendrán otras funciones $f(\Ee,\Aa)$. }%
%
\en{According to the standard model of evolution, known as \emph{replicator dynamic} \cite{taylor1978-replicatorDynamic}, the change in the proportion of a strategy in the population, $x_\Ee$, is determined by its characteristic growth rate $g_\Ee$, }%
\es{Según el modelo estándar de evolución, conocido como \emph{replicator dynamic} \cite{taylor1978-replicatorDynamic}, el cambio de la proporción de una estrategia en la población, $x_\Ee$, está determinado por su tasa de crecimiento caracterísitica $g_\Ee$, }%
% schuster1983-replicatorDynamics, hofbauer2003-evolutionaryGameDynamics
%
\begin{equation} \label{eq:replicator_dynamic}  \tag{Replicator dynamic}
\hspace{3cm} x_\Ee^\prime = \frac{x_\Ee g_\Ee}{\sum_i x_i g_i}
\end{equation}
%
\en{where the denominator acts as a normalization constant. }%
\es{donde el denominador actúa como constante de normalización. }%
%
\en{But, what is the characteristic growth rate $g$? }%
\es{¿Cuál es la tasa de crecimiento característica $g$? }%
%
\en{Much of the evolutionary literature bases its analysis on populations of infinite size and considers that the correct estimate is obtained by the expected value of the resources over time, $g^t = \langle \omega \rangle_t$. }%
\es{Buena parte de la literatura en evolución basa su análisis en poblaciones de tamaño infinito y considera que la estimación correcta se obtiene mediante el valor esperado de los recursos en el tiempo, $g^t = \langle \omega \rangle_t$. }%
%
\begin{equation}
\langle \omega \rangle_t = \sum_{\omega \in \Omega_t} \omega \cdot  P(\omega)
\end{equation}
%
\en{Where $\Omega_t$ is the set of all possible resource trajectories at time $t$, and $P(\omega)$ is the probability that the $\omega$ resource state occurs. }%
\es{Donde $\Omega_t$ es el conjunto de todas las posibles trayectorias de los recursos en el tiempo $t$, y $P(\omega)$ es la la probabilidad de que ocurra el estado de los recursos $\omega$. }%
% 
\en{In the coin example, the expected value in the first two time steps is, }%
\es{En el ejemplo de la moneda, el valor esperado en los dos primeros pasos temporales es, }%
%
\begin{equation}
\begin{split}
\langle \omega_e \rangle_1 & = 1.5 \cdot \frac{1}{2} + 0.6 \cdot  \frac{1}{2} = 1.05 \\ 
\langle \omega_e \rangle_2 &=  1.5^2 \cdot \frac{1}{4} + 2 (0.6 \cdot 1.5 \cdot \frac{1}{4} ) + 0.6^2 \cdot \frac{1}{4}= 1.05^2
\end{split}
\end{equation}
%
\en{That is, the estimated growth rate according to the expected value is $5\%$ for each time step, $\langle \omega \rangle_t = 1.05^t$. }%
\es{Es decir, la tasa de crecimiento estimada según el valor esperado es de $5\%$ por cada paso temporal, $\langle \omega \rangle_t = 1.05^t$. }%
%
\en{And indeed that is what happens with the average of the individual trajectories, $\omega(t)$, when the population is sufficiently large, }%
\es{Y efectivamente eso es lo que ocurre con el promedio de las trayectoria individuales, $\omega(t)$, cuando la población es suficientemente grande, }%
%
\begin{figure}[H]
    \centering
    \begin{subfigure}[b]{0.45\textwidth}
    \includegraphics[width=\linewidth]{figures/pdf/ergodicity_expectedValue.pdf}
    \end{subfigure}
    \caption{
    \en{Average of individual resources over time for different population sizes, in logarithmic scale. }%
    \es{Promedio de los recursos individuales en el tiempo para diferentes tamaños de la población, en escala logarítimica. }%
    %
    \en{As we increase the size of the population, the average approaches the expected value of $1.05^t$. }%
    \es{A medida que aumentamos el tamaño de la población, el promedio se acerca al valor esperado $\langle \omega \rangle_t = 1.05^t$. }%
    }
    \label{fig:ergodicity_expectedValue}
\end{figure}
%
\en{However, the expected value does not represent what happens to the agents over time. }%
\es{Sin embargo, el valor esperado no representa lo que le ocurre a los agentes en el tiempo. }%
%
\en{Individually, all the trajectories lose in the long term at a rate close to 5\%. }%
\es{Individualmente, todas las trayectorias pierden a largo plazo a una tasa cercana al 5\%. }%
%
\en{The trajectories observed in figure \ref{fig:ergodicity_individual_trayectories} are variable, but the longer we observe the system the smoother these lines become (figure \ref{fig:ergodicity_individual_trayectories_longrun}). }%
\es{Las trayectorias observadas en la figura \ref{fig:ergodicity_individual_trayectories} son variables, pero cuanto más tiempo observemos el sistema más suave se vuelven esas líneas (figura \ref{fig:ergodicity_individual_trayectories_longrun}). }%
%
\begin{figure}[H]
    \centering
    \begin{subfigure}[b]{0.45\textwidth}
    \includegraphics[width=\linewidth]{figures/pdf/ergodicity_individual_trayectories.pdf}
    \caption{}
    \label{fig:ergodicity_individual_trayectories}
    \end{subfigure}
    \begin{subfigure}[b]{0.45\textwidth}
    \includegraphics[width=\linewidth]{figures/pdf/ergodicity_individual_trayectories_longrun.pdf}
    \caption{}
    \label{fig:ergodicity_individual_trayectories_longrun}
    \end{subfigure}
    \caption{
    \en{The black line represents the expected value. }%
    \es{La recta negra representan el valor esperado. }%
    %
    \en{Figure \ref{fig:ergodicity_individual_trayectories}: size of individual resources over time, $ \log(\omega(t))$. }%
    \es{Figura \ref{fig:ergodicity_individual_trayectories}: tamaño de los recursos individuales en el tiempo, $ \log(\omega(t))$. }%
    %
    \en{Figure \ref{fig:ergodicity_individual_trayectories_longrun}: given enough time, all individual trajectories stick to the blue line. }% 
    \es{Figura \ref{fig:ergodicity_individual_trayectories_longrun}: con suficiente tiempo todas las trayectorias individuales se pegan a la recta azul. }% 
    }
    \label{fig:cpr_individual}
\end{figure}
%La relación entre el valor esperado y lo que le ocurre a los agentes individuales en el tiempo es un problema bien conocido en mecánica estadística.
\en{When the individual trajectories can be described by the expected value of the system states, then the process is said to be ergodic~\cite{peters2019-ergodicityEconomics}. }%
\es{Cuando lo que le ocurre a los agentes individuales en el tiempo puede describirse mediante el valor esperado de los estados del sistema, luego se dice que el proceso es ergódico~\cite{peters2019-ergodicityEconomics}. }%
%
\en{However, the conditions are very restrictive, and are not fulfilled in the case of multiplicative processes. }%
\es{Sin embargo, las condiciones para que esto se cumpla son muy restrictivas y no se satisfacen para el caso de los procesos multiplicativos. }%
%
\en{To calculate the characteristic growth rate $g$, we first express the product as follows, }%
\es{Para calcular la tasa de crecimiento caracterísitica $g$, primero expresaramos la productoria de la siguiente manera, }%
%
\begin{equation}
\omega(T) = \prod^T_{t=1} f(a(t)) = f(\text{\en{head}\es{cara}})^{n_1} f(\text{\en{tail}\es{sello}})^{n_2}
\end{equation}
%
\en{where $n_1$ and $n_2$ represents the number of occurrences of $f(\text{\en{head}\es{cara}})$ and $f(\text{\en{tail}\es{sello}})$, with $n_1 + n_2 = T$. }%
\es{donde $n_1$ y $n_2$ representa la cantidad de ocurrencias de $f(\text{cara})$ y $f(\text{seca})$, con $n_1 + n_2 = T$. }%
%
\en{In the limit, $T \rightarrow \infty$ all individual trajectories will be determined by the same characteristic growth rate $g$. }%
\es{En el límite, $T \rightarrow \infty$ todas las trayectorias individuales estarán determinadas por la misma tasa de crecimiento caracterísitica $g$. }%
%
\begin{equation} \label{eq:geometric_mean}
\begin{split}
\lim_{T \rightarrow \infty} \omega_e(T) & = {g}^T \\
\left( \lim_{T \rightarrow \infty} \omega_e(T) \right)^{1/T} & =  {g} \\
\lim_{T \rightarrow \infty} f(\text{cara})^{n_1/T} f(\text{seca})^{n_2/T} & 
 \end{split}
\end{equation}
%
\en{Where the frequencies $\frac{n_1}{T}$ and $\frac{n_2}{T}$ in the limit $T \rightarrow \infty$ are equal to the probabilities of occurrence of the system states. }%
\es{Donde las frecuencias $\frac{n_1}{T}$ y $\frac{n_2}{T}$ en el límite $T \rightarrow \infty$ son iguales a las probabilidades de ocurrencia de los estados del sistema. }%
%
\en{Therefore, the growth rate is, }%
\es{Por lo tanto, la tasa de crecimiento es, }%
%
\begin{equation}
g = (1.5 \cdot 0.6)^{1/2} \approx 0.95
\end{equation}
%
\en{This formula, which allows computing the long-term growth rate of individual trajectories, has previously been used in the evolution literature under the name \emph{geometric mean}. }%
\es{Esta fórmula, que permite computar la tasa de crecimiento a largo plazo de las trayectorias individuales, ha sido usada previamente en la literatura de evolución bajo el nombre de \emph{media geométrica}~\cite{dempster1955-geometricMean}. }%
%
\en{An important property of the geometric mean is that its value is always less than the expected value (or arithmetic mean). }%
\es{Una propiedad importante de la media geométrica es que su valor siempre es menor al valor esperado (o media aritmética). }%
%
\en{This is because in multiplicative processes the physical impacts of losses are usually stronger than those of gains. }%
\es{Esto se debe a que en los procesos multiplicativos los impactos físicos de las pérdidas suelen ser más fuertes que los de las ganancias. }%
%
\en{In an extreme case, a single zero in the product is enough to generate an extinction. }%
\es{En un caso extremo, un único cero en la productoria alcanza para generar su extinción. }%

% Decimos que un proceso es ergódico si se cumple que,
% \begin{equation}
%  \underbrace{\lim_{T \mapsto \infty} \frac{1}{T} \sum_{t=1}^T \omega(t)}_{\text{Media temporal}}  = \underbrace{\sum_{\omega} \omega \cdot p(\omega)}_{\text{Media de estados}}
% \end{equation}
% 

\subsection{Cooperacion}

\en{As a consequence of the non-ergodicity of multiplicative processes, fluctuations have a negative effect on individual growth rates, which can be reduced through mutual cooperation~\cite{yaari2010-cooperationEvolution, peters-cooperation2019.03.04}. }%
\es{Como consecuencia de la no-ergodicidad de los procesos multiplicativos, las fluctuaciones tienen un efecto negativo en las tasas de crecimiento individuales, las cuales pueden ser reducidas a través de la mutua cooperación~\cite{yaari2010-cooperationEvolution, peters-cooperation2019.03.04}. }%
%
%Parafraseando a Den Boer~\cite{denBoer1968-spreadingRisk}, la  supervivencia de una población depende de la distribución del riesgo dentro de la población y entre las poblaciones de diferentes especies.
\en{Ole Peters~\cite{peters-cooperation2019.03.04} considers the following cooperative strategy and analyzes the consequences it has on the growth rate of the agents. }%
\es{Ole Peters~\cite{peters-cooperation2019.03.04} considera la siguiente estrategia cooperativa y analiza la consecuencias que tiene sobre la tasa de crecimiento de los agentes. }%
%
\begin{figure}[H]
\centering
\scalebox{0.75}{
\tikz{

    \node[latent, minimum size=2cm ] (x1_0) {$\omega_1(t)$} ;
    \node[latent, below=of x1_0, minimum size=2cm ] (x2_0) {$\omega_2(t)$} ;

    \node[latent, right=of x1_0, minimum size=3cm ] (x1_0g) {$ \omega_1(t)\cdot f(\Aa_1(t))$} ;
    \node[latent, right=of x2_0, minimum size=1.8cm, xshift=0.6cm , align=left] (x2_0g) {$\omega_2(t)\cdot$\\$f(\Aa_2(t))$} ;
    
    \node[latent, right=of x1_0g, minimum size=3.8cm, yshift=-1.33cm, align=right] (x_0) {$\omega_1(t)\cdot f(\Aa_1(t))$\\$+\omega_2(t)\cdot f(\Aa_2(t))$ } ;
    
    \node[const, above=of x_0] (nx_0) {$\overbrace{\text{Pool}\hspace{2.5cm}\text{Share}}^{\text{\normalsize Cooperaci\'on}}$} ;
    
    \node[latent, right=of x1_0g, minimum size=2.5cm,  xshift=4.5cm] (x1_1) {$\omega_1(t+1)$ } ;
    \node[latent, below=of x1_1, minimum size=2.5cm, yshift=0.7cm] (x2_1) {$\omega_2(t+1)$ } ;
    
    \edge {x1_0} {x1_0g};
    \edge {x2_0} {x2_0g};
    \edge {x1_0g,x2_0g} {x_0};
    \edge {x_0} {x1_1,x2_1};
    
}
}
\caption{
\en{Agents start with the same initial resources. They then grow independently according to the equation \ref{eq:estrategia_base}. They then cooperate by pooling and sharing their resources. }%
\es{Los agentes comienzan con los mismos recursos iniciales. Luego crecen independientemente de acuerdo con la ecuaci\'on \ref{eq:estrategia_base}. Luego cooperan poniendo sus recursos en un fondo común que dividen en partes iguales. }%
}
\label{fig:protocolo}
\end{figure}
%
\en{Fully cooperative populations reduce their fluctuations, which generates an increase in the growth rate of all their members. }%
\es{Las poblaciones enteramente cooperadoras reducen sus fluctuaciones, lo que genera una aumento en la tasa de crecimiento de todos sus miembros. }%
%
\en{In Figure \ref{fig:ergodicity_cooperation} we show the trajectory of an agent in a cooperating population of size 33. }%
\es{En la figura \ref{fig:ergodicity_cooperation} mostramos la trayectoria de un agente en una población cooperadora de tamaño 33. }%
%
\begin{figure}[H]
    \centering
    \begin{subfigure}[b]{0.45\textwidth}
    \includegraphics[width=\linewidth]{figures/pdf/ergodicity_cooperation.pdf}
    \end{subfigure}
    \caption{
    \en{The resources of an individual from a population of 33 agents who share their wealth after each iteration (green line), approaches the expected value (black line). As a visual reference, we show the characteristic growth rate of the individuals (blue line). }%
    \es{Los recursos de un individuo de una población de 33 agentes que comparten su riqueza luego de cada iteración (recta verde), se pega al valor esperado (recta negra).
    Como referencia visual, dejamos la tasa de crecimiento caracterísitica de los individuos (recta azul). }%
    }
    \label{fig:ergodicity_cooperation}
\end{figure}
%
% \paragraph{Conclusión Ole Peters} (La ventaja de la cooperación)\textbf{.}
\en{Individuals in fully cooperative populations achieve growth rates equivalent to the average of system states, which in non-ergodic systems is always higher than individual growth rates. }%
\es{Los individuos de las poblaciones enteramente cooperadoras logran acceder a tasas de crecimiento equivalentes al promedio de estados del sistema, que en los sistemas no-ergódicos es siempre superior que la tasas de crecimiento individual. }%
%
\en{Ole Peters believs that the increase in the growth rate is sufficient argument to demonstrate the evolutionary advantage of cooperation, which he proposes as the main explanation for evolutionary transitions. }%
\es{Ole Peters considera que el aumento de la tasa de crecimiento es argumento suficiente para demostrar la ventaja evolutiva de la cooperación, lo que propone como principal explicación de las transiciones evolutivas. }%
%
\en{However, he does not consider the problem of defection, who says ``our cooperators are unable to break the cooperative pact''. }%
\es{Sin embargo no considera el problema de la deserción, quien dice ``our cooperators are unable to break the cooperative pact''. }%
%
\en{This does not seem to be a minor problem, considering the temptation to stop contributing to the common fund while continuing to receive its benefits. }%
\es{No parece ser un problema menor, teniendo en cuenta la tentación de dejar de aportar al fondo común mientras se siguen recibiendo sus beneficios. }%

\subsection{Multilevel selection}

\en{To demonstrate the evolutionary advantage of cooperation in the presence of defection it is necessary to consider selection at both the individual and group level. }%
\es{Para demostrar la ventaja evolutiva de la cooperción en presencia de desertores es necesario considerar selección tanto a nivel individual como a nivel grupal. }%
%
\en{Yaari and Solomon~\cite{yaari2010-cooperationEvolution} already suggest a verbal argument for group selection: while there would be a temptation to defect, fully cooperative groups will do better. }%
\es{Yaari y Solomon~\cite{yaari2010-cooperationEvolution} sugieren ya un argumento verbal de la selección de grupo: si bien habría una tentación para desertar, a los grupos enteramente cooperadores le irá mejor. }%
%
\en{However, they do not provide this argument with a formal demonstration. }%
\es{Sin embargo no acompañan este argumento con una demostración formal. }%

% Parrafo

\en{The co-author of the concept of evolutionary transitions (Szathmary~\cite{szathmary1995-evolutionaryTransitions, szathmary2015-evolutionaryTransitions}) recently proposed to analyze the evolution of populations subject to multilevel selection by means of Bayesian hierarchical models~\cite{czegel2019-bayesianEvolution}, making use of the isomorphism between evolutionary theory and Bayesian inference~\cite{harper2009-replicatorAsInference,shalizi2009-replicatorAsInference}. }%
\es{El co-autor del concepto de transiciones evolutivas (Szathmary~\cite{szathmary1995-evolutionaryTransitions, szathmary2015-evolutionaryTransitions}) propuso recientemente analizar la evolución de las poblaciones sujetas a selección multinivel mediante modelos jerárquicos bayesianos~\cite{czegel2019-bayesianEvolution}, haciendo uso del isomorfismo entre las teoría de la evolución y la inferencia bayesiana~\cite{harper2009-replicatorAsInference,shalizi2009-replicatorAsInference}. }%
%
\en{However, beyond the proposal, the authors of this article (Czegel et al~\cite{czegel2019-bayesianEvolution}) fail to offer a model that represents multilevel selection, their examples are only pictorial. }%
\es{Sin embargo, más allá de la propuesta, los autores de este artículo (Czegel et al~\cite{czegel2019-bayesianEvolution}) no logran ofrecer un modelo que represente selección multinivel, sus ejemplos son sólo pictóricos. }%

% Parrafo

\en{In this paper we demonstrate the evolutionary advantage of cooperation in the presence of defection using a Bayesian hierarchical model representing evolution under multilevel selection. }%
\es{En este trabajo demostramos la ventaja evolutiva de la cooperación en presencia de deserción mediante un modelo jerárquico bayesiano que represente evolución bajo selección multinivel. }%
%
\en{To the best of our knowledge, our work would be the first to develop a Bayesian hierarchical model to solve an evolution problem under multilevel selection. }%
\es{Hasta donde sabemos, nuestro trabajo sería el primero en desarrollar un modelo jerárquico bayesiano para resolver un problema de evolución bajo selección multinivel. }%
%
% \en{With this model we demonstrate, as Ole Peters intended, that the advantage of cooperation (now in the presence of defection) is a consequence of the most basic and fundamental assumption of evolutionary theory: that evolutionary processes are multiplicative and noisy. }%
% \es{Con este modelo demostramos, como pretendía Ole Peters, de que la ventaja de la cooperación (ahora en presencia de deserción) es consecuencia del supuesto más básico y fundamental de la teoría de la evolución: que los procesos evolutivos son multiplicativos y ruidosos. }%

\subsection{Especialización}

% \en{According to Ole Peters, an important evolutionary phenomenon on which his analysis can shed new light is the transition to multicellularity. }%
% \es{Según Ole Peters, un importante fenómeno evolutivo sobre el que su análisis puede arrojar nueva luz es la transición a la multicelularidad. }%
%
\en{The evolutionary transition from unicellularity to multicellularity does not only involve cooperation, it also involves a specialization of cells: from protists to plants, fungi and animals. }%
\es{La transición evolutiva de la unicelularidad a la multiclelularidad no involucra sólo cooperación, involucra también una especialización de las células: de los los protistas a las plantas, hongos y animales. }%
%
\en{Then, to explain evolutionary transitions, it is necessary to demonstrate the evolutionary advantage of cooperation in the presence of defection, but also the advantage of specialization. }%
\es{Luego, para explicar las transiciones evolutivas es necesario demostrar la ventaja evolutiva de la cooperación en presencia de deserción, pero también la ventaja de la especiliazación. }%

% Parrafo

\en{However, Ole Peters believes that the cause of the cooperation advantage (the assumption of non-ergodicity of noisy multiplicative processes) does not allow explaining the specialization advantage. }%
\es{Sin embargo, Ole Peters cree que la causa de la ventaja de la cooperación (el supuesto de la no-ergodicidad de los procesos multiplicativos ruidosos) no permite explicar la ventaja de la especialización. }%
%
\en{Ole Peters believes that specialization is a too complex property to produce a benefit in simple aggregations of, for example, two agents. }
\es{Ole Peters considera a la especialización como una caracterísitica demasiado compleja para que produzca un beneficio en agregaciones simples de por ejemplo dos agentes. }%
%
\en{He says the ``[advantage of cooperation] it may explain the transition from single cells to bicellular organisms, too small and simple to benefit from new function or specialization.'' }
\es{Dice que la ventaja de la cooperación: ``it may explain the transition from single cells to bicellular organisms, too small and simple to benefit from new function or specialization.'' }%
%
\en{He believes that other explanations of evolutionary transitions, such as the emergence of specialization, are implausible. }%
\es{El cree que otras explicaciones de las transiciones evolutivas, como la emergencia de la especialización, son implausibles. }%

% Parrafo

\en{In this paper we show that the same minimal assumptions used by Ole Peters to explain the evolutionary advantage of cooperation also allow us to explain the evolutionary advantage of specialization, even for the simplest aggregations of two agents. }%
\es{En este trabajo demostramos que los mismos supuestos mínimos que utiliza Ole Peters para explicar la ventaja evolutiva de la cooperación, permiten también explicar la ventaja evolutiva de especialización, incluso para las agregaciones más simples de dos agentes. }%
%
\en{We show that, once cooperation emerges, individually maladapted strategies (specialists) manage to outperform, even organized in small groups, both individually well-adapted strategies (generalists) and their cooperative groups of infinite size. }%
\es{Mostramos que, una vez que la cooperación emerge, las estrategias individualmente mal adaptadas (especialistas) consigen superar, incluso organizadas en grupos pequeños, tanto a las estrategias bien adaptas individualmemte (generalistas) como a sus grupos cooperativos de tamaño infinito. }%

% Parrafo


\en{It is extraordinary that such a simple assumption has such fundamental conclusions for understanding the complexity of life. }%
\es{Es extraordinario que un supuesto tan simple tenga conclusiones tan fudamentales para entender la complejidad de la vida. }%
%
\en{Because the evolutionary advantage of cooperation and specialization is a direct consequence of the non-ergodicity of the multiplicative processes to which life is subject, we support Ole Peters' idea of considering it the first cause of major evolutionary transitions. }%
\es{Debido a que la ventaja evolutiva de la cooperación y la especialización es consecuencia directa de la no-ergodicidad de los procesos multiplicativos a los que está sujeto la vida, apoyamos la idea de Ole Peters de consideradala la primera causa de las transiciones evolutivas mayores. }%

\section{Metodología}

\en{In this section we present the ismorphism between probability theory and evolutionary theory. }%
\es{En esta sección presentamos el ismorfismo entre la teoría de la probabilidad y la teoría de la evolución. }%
%
\en{In the Results section we will demonstrate the evolutionary advantage of cooperation and specialization using a Bayesian hierarchical model representing evolution under multilevel selection. }%
\es{En la sección Reseultados demostraremos la ventaja evolutiva de la cooperación en presencia de deserción mediante un modelo jerárquico bayesiano que represente evolución bajo selección multinivel. }%
%

\subsection{Teoría de la probabilidad y los modelos causales}

\en{Probability theory is currently the most widely used approach to computing uncertainty. }%
\es{La teoría de la probabilidad es el enfoque más utilizado en la actualidad para computar la incertidumbre. }%
%
\en{Its rules have been formally derived from conceptually distinct and mutually independent axiomatic systems~\cite{halpern2017-RAU2}, which is one of the strong points in its favor. }%
\es{Sus reglas han sido derivadas formalmente a partir de sistemas axiomáticos conceptualmente distintos e independientes entre sí~\cite{halpern2017-RAU2}, lo cual es uno de los punto fuertes a su favor. }%
%
\en{But perhaps more importantly, its strict application maximizes uncertainty given empirical and formal information (data and causal models)~\cite{jaynes2003-bookProbabilityTheory}, the source of validation for empirical sciences. }%
\es{Pero quizás más importante sea que su aplicación estricta maximiza la incertidumbre dada la información empírica y formal (datos y modelos causales)~\cite{jaynes2003-bookProbabilityTheory}, fuente de validación de las ciencias empíricas. }%

% Parrafo

\en{The whole theory of probability can be summarized in two rules: the \ref{eq:sum_rule} and the ~\ref{eq:product_rule}. }%
\es{Toda la teoría de la probabilidad puede resumirse en dos reglas: la~\ref{eq:sum_rule} y la~\ref{eq:product_rule}. }%
%
\en{The \ref{eq:sum_rule} states that any marginal distribution can be obtained by integrating or summing the joint distribution. }%
\es{La \ref{eq:sum_rule} afirma que cualquier distribución marginal se puede obtener integrando o sumando la distribución conjunta. }%
%
\begin{equation} \label{eq:sum_rule}
 \tag{\en{sum rule}\es{regla de la suma}}
 P(x) = \sum_{y} P(x,y) \ \ \ \ \ \text{or} \ \ \ \ \ p(x) = \int p(x,y) \, dy
\end{equation}
%
\en{Where $P(\cdot)$ and $p(\cdot)$ represent discrete and continuous probability distributions respectively. }%
\es{Donde $P(\cdot)$ y $p(\cdot)$ representan distribuciones de probabilidad discretas y continuas respectivamente. }%
%
\en{On the other hand, the \ref{eq:product_rule} states that any joint distribution can be expressed as the product of one-dimensional conditional distributions. }%
\es{Por su parte, la \ref{eq:product_rule} se\~nala que cualquier distribuci\'on conjunta puede ser expresada como el producto de distribuciones condicionales uni-dimensionles. }%
%
\begin{equation}\label{eq:product_rule}
\tag{\en{product rule}\es{regla del producto}}
 P(x,y,z) = P(x|y,z) P(y|z) P(z)
\end{equation}
%
\en{This expression can be simplified when we eliminate from the conditional the variables that do not have any effect on the function. }%
\es{Esta expresión se puede simplificar cuando eliminamos del condicional las variables que no afectan a la función. }%
%
\en{The minimum set of variables that appear on the right side of the conditional are said to be the ``parents'' of the variable that appears on the left side of the conditional. }%
\es{Al conjunto mínimo de variables que aparecen del lado derecho del condicional decimos que son los ``padres'' de la variable que aparece del lado izquierdo del condicional. }%

% Parrafo

\en{Another way to define joint distributions is by using graphical models, in which there is a node for each variable, and an arrow for each ``parent$\rightarrow$child'' pair. }%
\es{Otra forma de definir distribuciones conjuntas es mediante el uso de modelos gráficos, en los que hay un nodo por cada variable, y una flecha por cada par ``padre$\rightarrow$hijo''. }% 
%
\en{Graphical models are a well-established tool in probability because they allow to: 1. provide an intuitive language to unambiguously express all hypotheses, 2. reduce the dimensionality of the joint probability distribution, 3. and perform inference efficiently through the sum-product algorithm~\cite{kschischang2001-factorGraphsAndTheSumProductAlgorithm}. }%
\es{Los modelos gráficos son una herramienta bien establecida en probabilidad debido a que permiten: 1s. ofrecen un lenguaje intuitivo para expresar sin ambiguedad todas las hipótesis; 2. reducir la dimensionalidad de la distribución de probabilidad conjunta; 3. y realizar inferencia de forma eficiente a través del algoritmos suma-producto~\cite{kschischang2001-factorGraphsAndTheSumProductAlgorithm}. }%
\en{To use the graphical models no causal interpretation is required, in fact the conditional probabilities can be expressed in any order. }%
\es{Para utilizar los modelos gráficos no se requiere una interpretación causal, de hecho las probabilidades condicionales pueden expresar en cualquier orden. }%
 
% Parrafo

\en{There are several advantages when conditional probabilities are defined on the basis of causal interpretations~\cite{pearl2009-causality}. }%
\es{Son varias las ventajas que se consiguen al definir las probabilidades condicionales en base a interpretaciones causales~\cite{pearl2009-causality}. }%
%
\en{Science elaborates its theories on the basis of causal stories: stable and autonomous mechanisms that induce conditional probabilities between causes and effects. }%
\es{La ciencia elabora sus teorías en base a historias causales: mecanismos estables y autónomos que inducen probabilidades condicionales entre causas y efectos. }%
%
\en{In this sense, justifying conditional probabilities on the basis of a causal story is a more natural way of expressing what we know or believe about the world. }%
\es{En este sentido, justificar las probabilidades condicionales en base a una historia causal es una forma más natural de expresar lo que sabemos o creemos sobre el mundo. }%
%
\en{In addition to facilitating intuition in the research process, causal interpretation also modularizes conditional probabilities. }%
\es{Además de facilitar la intuición en el proceso de investigación, la interpretación causal también modulariza las probabilidades condicionales. }%
%
\en{Eventual changes in any of the causal mechanisms locally affect the topology of the Bayesian network, allowing the effect of external interventions to be predicted with a minimum of additional information. }%
\es{Eventuales cambios en alguno de los mecanismos causales afectan localmente la topología de la red bayesiana, permitiendo predecir el efecto de las intervenciones externas con un mínimo de información adicional. }%
%
\en{For these reasons, in this paper we will express the graphical models in terms of causal relationships. }%
\es{Por estos motivos, en este trabajo expresaremos los modelos gráficos en términos de relaciones causales. }%

\subsection{Isomorfismo entre las teorías de la evolución y la probabilidad.}

%
\en{From the product rule, we immediately obtain the Bayes' theorem, with which we can compute the uncertainty about the hidden hypothesis space given the data and the model: }%
\es{De la regla del producto obtenemos inmediatamente el teorema de Bayes con el que podemos computrar la incertidumbre sobre el espacio de hipótesis ocultas dados los datos y el modelo: }%
%
\begin{equation}\label{eq:bayes_theorem}
 \underbrace{P(\overbrace{\text{\en{Hypothesis}\es{Hipótesis}$_i$}}^{\text{\en{Hidden}\es{Oculta}}}|\overbrace{\text{Dat\en{a}\es{os}}}^{\text{Observ\en{ed}\es{ado}}})}_{\text{Posterior}} = \frac{\overbrace{P(\,\text{Dat\en{a}\es{os}}\,|\,\text{\en{Hypothesis}\es{Hipótesis}$_i$})}^{\text{\en{Likelihood}\es{Verosimilitud}}}\overbrace{P(\text{\en{Hypothesis}\es{Hipótesis}$_i$})}^{\text{Prior}}}{\underbrace{P(\text{Dat\en{a}\es{os}})}_{\text{Evidenc\en{e}\es{ia} o\en{r}\es{ predicci\'on a} prior \en{prediction}}}}
\end{equation}

\en{where the only free variable is the skill hypothesis $i$, the data and the model are fixed. }%
\es{Donde la \'unica variable libre es la hip\'otesis $i$, los datos y el modelo están fijos. }%
%
\en{The likelihood and the evidence are both probabilities of the data , so they can be seen as predictions. }%
\es{La verosimilitud y la evidencia son ambas probabilidades de los datos, por lo que pueden ser vistas como predicciones. }%
%
\en{When the data is a discrete variable, predictions always take values between $0$ and $1$. }%
\es{Cuando el dato es una variable discreta, las predicciones siempre toman valores entre $0$ y $1$. }%
%
\en{Unlike the likelihood, which makes a different prediction for each hypothesis, the evidence averages all the predictions, weighted by the prior probability of the hypotheses, }%
\es{A diferencia de la verosimilitud, que realiza una predicción distinta por cada hipótesis, la evidencia realiza un promedio de todas las prediciones, pesado por la probabilidad a priori de las hipótesis, }%
%
\begin{equation}
P(\text{Dat\en{a}\es{os}}) = \sum_i P(\,\text{Dat\en{a}\es{os}}\,|\,\text{\en{Hypothesis}\es{Hipótesis}$_i$}) P(\text{\en{Hypothesis}\es{Hipótesis}$_i$})
\end{equation}
%
\en{The evidence then functions as a normalization constant. }%
\es{La evidencia funciona entonces como una constante de normalización. }%
%
\en{Therefore, the likelihood is the only factor that updates the probability distribution of the hypothesis: the posterior is just the prior probability that is not filtered by the likelihood. }%
\es{Luego, la verosimilitud es lo único que actualiza la distribuión de probabilidad de la hipótesis: el posterior no es más que la probabilidad del prior no filtrada por la verosimilitud. }%

% Parrafo

\en{There is a immediate isomorphism between the fundamental equations of evolutionary theory (replicator dynamic) and probability theory (Bayes theorem) when we identify that the probability distribution of the hypotheses as the distribution of the strategies, and the fitness as the likelihood~\cite{shalizi2009-replicatorAsInference,harper2009-replicatorAsInference}, }%
\es{El isomorfismo entre las ecuaciones fundamentales de la teoría de la evolución (replicator dynamic) y la teoría de la probabilidad (teorema de bayes) es inmediato cuando identificamos que la distribución probabilidad de las hipótesis es la distribución de las estrategias, y la aptitud es la verosimilitud~\cite{shalizi2009-replicatorAsInference,harper2009-replicatorAsInference}, }%
%
\begin{align*}
\centering
 \begin{tabular}{l|l}
  \en{Bayes theorem}\es{Teorema de Bayes} & Replicator dynamic  \\ \hline
  Prior $P(H)$ & \en{Old distribution of strategies}\es{Distribución de estrategias} $x_\Ee$ \\ \hline
  \en{Likelihood}\es{Verosimilitud} $P(D|H)$ & \en{Fitness}\es{Aptitud} $f(\Ee, \Aa)$ \\ \hline
  Posterior $P(H|D)$ & \en{New distribution of strategies}\es{Nueva distribución de estrategias} $x_\Ee^\prime$ \\ \hline
  Evidenc\en{e}\es{ia} $P(D)$ & \en{Population mean fitness}\es{Aptitud media de la población} $\sum x_\Ee f(\Ee,\Aa) $ \\ \hline
 \end{tabular}
\end{align*}
%
\en{Based on this isomorphism, Czégel, Zachar and Szathmáry recently proposed to analyze the evolution of a population subject to multilevel selection through hierarchical Bayesian modeling~\cite{czegel2019-bayesianEvolution}. }
\es{En base a este isomorfismo, Czégel, Zachar y Szathmáry propusieron recientemente analizar la evolución de una población sujeta a selección multinivel a través de modelo bayesianos jerárquicos~\cite{czegel2019-bayesianEvolution}. }%
%
\en{To do so, they display a series of graphical models without specifying the conditional probabilities. }%
\es{Para ello exhiben una serie de modelos gráficos sin especificar las probabilidades condicionales. }%
%
\begin{figure}[H]
\centering
\tikz{
    \node[latent] (c2) {$C^2$};
    \node[const, above=of c2, yshift=0.1cm] (pc2) {$P(c^2)$};
    \node[const, above=of pc2, yshift=0.1cm] (nc2) {Col\en{l}ectiv\en{e}\es{o} 2};
    \node[const, above=of nc2] (fc2) {\includegraphics[width=0.06\linewidth]{static/collective2A.png} \ \includegraphics[width=0.045\linewidth]{static/collective2B.png}};
    
    
    \node[latent, right=of c2, xshift=1cm] (c1) {$C^1$};
    \node[const, above=of c1, yshift=0.1cm] (pc1) {$P(c^1|c^2)$};
    \node[const, above=of pc1, yshift=0.1cm] (nc1) {Col\en{l}ectiv\en{e}\es{o} 1};
    \node[const, above=of nc1] (fc1) {\includegraphics[width=0.04\linewidth]{static/collective1A.png} \ \includegraphics[width=0.04\linewidth]{static/collective1B.png}};
    
    \node[latent, right=of c1, xshift=1cm] (i) {$I$};
    \node[const, above=of i, yshift=0.1cm] (pi) {$P(i|c^1,c^2)$};
    \node[const, above=of pi, yshift=0.1cm] (ni) {Individu\en{al}\es{o}:};
    \node[const, above=of ni] (fi) {\includegraphics[width=0.09\linewidth]{static/individual.png}\ };
    
    \node[latent, right=of i, xshift=1cm] (a) { $\A$};
    \node[const, above=of a, yshift=0.1cm] (pa) {$P(\Aa|i,c^1,c^2)$};
    \node[const, above=of pa, yshift=0.1cm] (na) {\en{Environment}\es{Ambiente}};
    \node[const, above=of na] (fa) {\includegraphics[width=0.03\linewidth]{static/water.png}\includegraphics[width=0.04\linewidth]{static/wind.png}\includegraphics[width=0.025\linewidth]{static/fire.png} \ };
    
    \edge {c2} {c1};
    \edge {c1} {i};
    \edge {i} {a};
    \path[draw, ->, fill=black,sloped] (c2) edge[bend right,draw=black] node[midway,above,color=black] {} (i);
    \path[draw, ->, fill=black,sloped] (c2) edge[bend right,draw=black] node[midway,above,color=black] {} (a);
    \path[draw, ->, fill=black,sloped] (c1) edge[bend right,draw=black] node[midway,above,color=black] {} (a);
    
    \plate {aa} {(a)} {}; 
    }
\caption{
\en{Model proposed by Czégel, Zachar and Szathmáry~\cite{czegel2019-bayesianEvolution} to represent multilevel selection. }%
\es{Modelo propuesto por Czégel, Zachar y Szathmáry~\cite{czegel2019-bayesianEvolution} para representar selección multinivel. }%
%
\en{The box represents repetition of the variable. }%
\es{La caja representa repetición de la variable. }%
%
\en{The uppercase letters are variable names, and the lowercase letters are specific values. }%
\es{Las mayusculas son nombres de variables, y las minúsculas son sus valores específicos. }%
}

\label{fig:czegel_et_el}
\end{figure}
%
\en{The validity of this model is trivial as the product rule always allows the following decomposition, }%
\es{La validez de este modelo es trivial en tanto la regla del producto siempre permite la siguente descomposición, }%
%
\begin{equation}
P(\Aa,i,c^1,c^2)= P(\Aa|i,c^1,c^2)P(i|c^1,c^2)P(c^1|c^2)P(c^2)
\end{equation}
%
\en{However, the model does not express the causal relationship between the variables. }%
\es{Sin embargo el modelo no expresa la relación causal entre las variables. }%
%
\en{This is evident in all conditional probability distributions. }%
\es{Esto se constata en todas las distribuciones de probabilidad condicional. }%
%
\en{At one extreme, the level 2 collective, which is composed of both the level 1 collective and individuals, appears in the model as an independent variable. }%
\es{En un extremo, el colectivo de nivel 2, que está compuesta tanto por el colectivo de nivel 1 y por los individuos, aparece en el modelo como una variable independiente. }%
%
\en{At the other extreme, the environment, which is usually considered in causal models as an independent random variable, appears in this model as dependent on individuals and collectives. }%
\es{En el otro extremo, el ambiente, que suele considerarse en los modelos causales como una variable aleatoria independiente, aparece en este modelo como dependiente de los individuos y los colectivos. }%
%
\en{Conversely, all collectives and individuals appear independent of the environment. }%
\es{Y a la inversa, todas los colectivos e individuos aparecen independientes del ambiente. }%
%
\en{This perhaps explains why the authors avoided specifying conditional probability distributions, and instead provided a pictorial example of the joint probability distribution. }%
\es{Esto quizás explique porque los autores evitaron especificar las distribuciones de probabilidad condicional, y ofrecieron en cambio un ejemplo pictórico de la distribución de probabilidad conjunta. }%

\subsection{Ejemplo de modelo causal}

\en{To define conditional probability distributions it is convenient to express the models in causal terms. }%
\es{Para definir las distribuciones de probabilidad condicional es conveniente expresar los modelos en términos causales. }%
%
\en{In the Lewontin-Cohen and Peters models, the binary states of the environment depend on a probability $p$. }%
\es{En los modelos de Lewontin-Cohen y de Peters, los estados binarios del ambiente depende de un probabilidad $p$. }%
%
\begin{equation}
P(\Aa) = p^{\Aa} (1-p)^{(1-\Aa)}
\end{equation}
%
\en{To define the probability of the strategies, we need a space of alternative strategies, not present in the Lewontin-Cohen and Peters models. }%
\es{Para definir la probabilidad de las estrategia, necesitamos un espacio de estrategias alternativas, no presente en los modelo de Lewontin-Cohen y de Peters. }%
%
\en{While we are free to propose any set of strategies, we will consider only those that allocate a limited resource $R$ among the binary states of the environment, }%
\es{Si bien tenemos la libertad de proponer cualquier conjunto de estrategias, vamos a considerar sólo a aquellas que reparten un recurso limitado $R$ entre los estados binarios del ambiente, }%
%
\begin{equation}\label{eq:familia_de_aptitudes}
f_R(\Ee, \Aa) = \begin{cases}
 \Ee & \Aa = 1 \\
 R-\Ee & \Aa = 0
  \end{cases}
\end{equation}
%
\en{with $ 0 \leq \Ee \leq R$. }%
\es{con $ 0 \leq \Ee \leq R$. }%
%
\en{The strategy proposed by Lewontin-Cohen is $\Ee=1.7$ with $R=2.2$, and the strategy proposed by Peters is $\Ee=1.5$ with $R=2.1$. }%
\es{La estrategia propuesta por Lewontin-Cohen es $\Ee=1.7$ con $R=2.2$, y la de Peters es $\Ee = 1.5$ con $R=2.1$. }%
%
\en{To express the strategies with a single parameter, we will work only with the family of fitnesses in which $R=1$, $f_1(\Ee, \Aa)$. }%
\es{Para expresar las estrategias con un sólo parámetro, vamos a trabajar solamente con la familia de aptitudes en la que $R=1$, $f_1(\Ee, \Aa)$. }%
%
\en{Now, all strategies lie between 0 and 1: the strategy $\Ee = 1.5/2.1 \approx 0.71$ is the one proposed by Ole Peters, and the strategy $\Ee 1.7/2.2 \approx 0.77$ is the one proposed by Lewontin and Cohen. }%
\es{Ahora, todas las estrategias están entre 0 y 1: la estrategia $\Ee = 1.5/2.1 \approx 0.71$ es la propuesta por Ole Peters, y la estrategia $\Ee 1.7/2.2 \approx 0.77$ es la propuesta por Lewontin y Cohen. }%
%
\en{To define the probability of the strategies given the environment, $P(\Ee|\Aa)$, we need the set of strategies to integrate 1. }%
\es{Para definir la probabilidad de las estrategias dado el ambiente, $P(\Ee|\Aa)$, necesitamos que el conjunto de estrategias integren 1. }%
%
\en{Let's say, for now, that the probability of the strategy is proportional to its fitness, }%
\es{Digamos, por ahora, que la probabilidad de la estrategia es proporcional a su aptitud, }%
%
\begin{equation}\label{eq:probabilidad_propro_aptitud}
P(\Ee | \Aa) \propto  \Ee^{\Aa}(1-\Ee)^{1-\Aa} = f_1(\Ee,\Aa)
\end{equation}
%

\en{This allows us to partially define a graphical model with a causal interpretation, }%
\es{Esto nos permite definir parcialmente un modelo gráfico con una interpretación causal, }%
%
\begin{figure}[H]
\centering
\tikz{
    \node[latent] (e) {$\A_t$};
    \node[const, right=of e] (en) {\ $P(\Aa) = p^{\Aa} (1-p)^{(1-\Aa)}$};
    \node[const, left=of e] (ne) {\en{Environment}\es{Ambiente}: \ \ \ };
    
    
    \node[latent, below=of e] (r) {$\E$};
    \node[const, right=of r] (rn) {$P(\Ee|\Aa) \propto f_1(\Ee,\Aa)$};
    \node[const, left=of r] (nr) {\en{Strategy}\es{Estrategia}: \ \ \ };
    
    \edge {e} {r};
    \plate {ee} {(e)} {$t$}; 
    }
\caption{Modelo causal Lewontin-Peters}
\label{fig:model-lewontin-peters}
\end{figure}
%
\en{in which the environment is an independent random variable and strategy depends on the observed states of the environment. }%
\es{en el que el ambiente es una variable aleatoria independiete y las estrategias dependen de los estados observados del ambiente. }%
%
\en{How can we calculate the probability of $P(\Ee|\Aa)$? }%
\es{¿Cómo podemos calcular la probabilidad de $P(\Ee|\Aa)$? }%

\subsection{Sum-product algorithm}

\en{The \emph{sum-product algorithm} takes advantage of the factoriztion of the joint probability distribution, induced by the causal model, to efficiently apply the rules of probability. }%
\es{El \emph{sum-product algorithm} aprovecha la factorización de la distribuci\'on de probabilidad conjunta, inducida por el modelo causal, para aplicar eficientemente las reglas de la probabilidad. }%
%
\en{It is a way of computing the rules of probability (the sum and product rule) by message passing between probability distributions and their variables. }%
\es{Es una forma de computar las reglas de la probabilidad (la regla de la suma y el producto) mediante pasaje de mensajes entre las distribuciones de probabilidad y sus variables. }%
%
\en{For this purpose, it is convenient to represent the graphical model with a \emph{factor graph}, a bipartite graph with variable nodes (white circles) and function nodes (black squares). }%
\es{Para ello es conveniente representar el modelo mediante un \emph{factor graph}, un gráfos con nodos variables (círculos blancos), y nodos funciones (cuadrados negros). }%
%
\en{The edge between node variables and node functions represents the mathematical relationship ``the variable $v$ is an argument of the function $f$''. }%
\es{Los ejes entre nodos variables y nodos funciones representan la relaci\'on matem\'atica ``la variable $v$ es argumento de la funci\'on $f$''. }%
%
El siguiente factor graph representa el modelo de la figura \ref{fig:model-lewontin-peters}.
%
\begin{figure}[H]
\centering
\tikz{

    \node[factor] (fa) {};
    \node[const, above=of fa] (nfa) {$P(\Aa)$};
    
    \node[latent, right=of fa] (a) {$\A$};
    
    \node[factor, right=of a] (fe) {};
    \node[const, above=of fe] (nfe) {$P(\Ee|\Aa) \propto f_1(\Ee,\Aa)$};
    
    \node[latent, right=of fe, xshift=0.3cm] (e) {$\E$};
    

     \plate {hola} {(fa)(nfa)(a)(fe)(nfe)} {}; 

    \edge[-] {fa} {a};
    \edge[-] {a} {fe};
    \edge[-] {fe} {e};
    }
\caption{
 \en{Graphical way of representing the factorization of joint distribution induced by the basic causal model (Fig.~\ref{fig:model-lewontin-peters}). }%
 \es{Forma gráfica de representar la factorizaci\'on de la distribución conjunta inducida por el modelo causal básico (figura~\ref{fig:model-lewontin-peters}). }%
 }
\label{fig:factor_graph}
\end{figure}
%
\en{There are two types of messages: those sent by variable-type nodes to their function-type neighbors ($m_{V \rightarrow F}(v)$) and the ones that function-type nodes send to their variable-type neighbors ($m_{F \rightarrow V}(v)$). }%
\es{Hay dos tipos de mensajes: los mensajes que envian los nodos variables a sus funciones vecinas ($m_{V \rightarrow F}(v)$); y los mensajes que env\'ian los nodos funciones a sus variables vecinas ($m_{F \rightarrow V}(v)$). }%
%
\en{The former partially performs the product rule. }%
\es{El primero codifica una porci\'on de la regla del producto. }%
%
\begin{equation*}\label{eq:m_v_f} \tag{\text{\en{product step}\es{paso del producto}}}
m_{V \rightarrow F}(v) = \prod_{H}^{n(V) \setminus \{F\} } m_{H \rightarrow V}(v)
\end{equation*}
%
\en{where $n(V) \setminus \{F\}$ represents the set of neighbor nodes to $V$ except $F$. }%
\es{donde $n(V) \setminus \{F\}$ representa el conjunto de vecinos del nodo $V$ salvo $F$. }%
%
\en{In a brief, the messages sent by the variable-type node $V$ are simply the product of the messages that $V$ receives from the rest of their neighbors $H \in n(V)$ except $F$. }%
\es{En pocas palabras, los mensajes que env\'ia una variables $V$ es simplemente la multiplicaci\'on de los mensajes que recibi\'o del resto de sus vecinos $H \in n(V)$ salvo $F$. }%
%
\en{The messages sent by the function-type nodes encode a portion of the sum rule. }%
\es{Los mensajes que env\'ian los nodos funciones codifican una parte de la regla de la suma. }%
%
\begin{equation*}\label{eq:m_f_v}  \tag{\text{\en{sum step}\es{paso de la suma}}}
m_{F \rightarrow V}(v) = \sum_{\bm{u}} \Big( F(\bm{u},v) \prod_{U}^{\bm{U}} m_{U \rightarrow F}(u) \Big)
\end{equation*}
%
\en{where $\bm{U} = n(F)\setminus \{V\}$ is the set of all neighbors to $F$ except $V$, $\bm{i}$ a set of specific values with $u \in \bm{u}$, and $F(\bm{u},v)$ represents the function $F$, evaluated in all its arguments. }%
\es{donde $\bm{U} = n(F)\setminus \{V\}$ es el conjunto de todos los vecinos de $F$ salvo $V$, $\bm{u}$ un conjunto de valores específicos con $u \in \bm{u}$, y $F(\bm{u},V)$ representa la funci\'on $F$, evaluada en todos sus argumentos. }%
%
\en{The messages sent by the function-type node $F$ to a neighboring variable-type node $V$ is simply the sum (or integration) over $\bm{U}$ of the product of itself and all the messages that $F$ receives from the rest of its neighbors $\bm{U}$ except $V$. }%
\es{En pocas palabras, los mensajes que envía una funci\'on $F$ a una variable vecina $V$ es simplemente la suma (o integraci\'on) sobre $\bm{U}$ del producto de sí mismo con todos los mensajes que $F$ recibe del resto de sus vecinos $\bm{U}$ salvo $V$. }%
%
\en{Finally, the marginal probability distribution of a variable $V$ is simply the product of the messages that $V$ receives from all its neighbors. }%
\es{Finalmente, la distribuci\'on de probabilidad marginal de una variable $V$ es simplemente la multiplicaci\'on de los mensajes que $V$ recibe de todos sus vecinos. }%
%
\begin{equation*}\label{eq:marginal}  \tag{\text{\en{marginal probability}\es{probabilidad marginal}}}
P(V=v) = \prod_{H \in n(V)} m_{H \rightarrow V}(v)
\end{equation*}
%
\en{When we observe some variable $u$, we exclude it from the summations, which allows us to compute the marginal probability over two variables, $P(v,u)$. }%
\es{Cuando observamos alguna variable $u$, la excluímos de las sumatorias, lo que nos permite computar la probabilidad marginal sobre dos variables, $P(v,u)$. }%
%
\en{This algorithm encodes the minimum number of steps required to calculate any marginal probability distribution. }%
\es{Este algoritmo codifica la m\'inima cantidad de pasos que se requieren para calcular cualquier distribuci\'on de probabilidad marginal. }%

\subsection{Computo de marginales}

\en{As we will later use the sum-product algorithm to solve a more complex causal model, we show its use in this simple causal model (figure~\ref{fig:model-lewontin-peters}) to gain some intuition. }%
\es{Como más adelante utilizaremos el sum-product algorithm para resolver un modelo causal más complejo, mostrtamos su uso en este modelo causal simple (figure~\ref{fig:model-lewontin-peters}) para adquirir un poco de intuición.  }%
%
\en{Using this procedure we will normalize the probability of the strategies given the environment $P(\Ee|\Aa)$, and obtain the posterior of the strategies given a vector of observations $P(\Ee,\vec{\Aa}\,)$. }%
\es{Mediante este procedimiento normalizaremos la probabilidad de las estrategias dado el ambiente $P(\Ee|\Aa)$, y obtendremos el posterior de las estrategias dado un vector de observaciones $P(\Ee,\vec{\Aa}\,)$. }%
%
\en{These conclusions will be of interest in the results section. }%
\es{Estas conclusiones serán de interés en la sección resultados. }%

% Parrafo 

% \en{Since we want $P(\Ee|\Aa)$ we need the marginals $P(\Ee,\Aa)/P(\Aa)$. }%
% \es{Como queremos calcular $P(\Ee|\Aa)$ necesitamos las marginales $P(\Ee,\Aa)/P(\Aa)$. }%
% %
% \en{Since one of the probability distributions is defined by a proportional term, its marginals will also be proportional. }%
% \es{Notar que una de las distribuciones de probabilidad está definida de forma proporcional, sus marginales también serán proporcionales. }%
% %
\en{To compute $P(\Ee,\Aa)$, we will consider $\Aa^*$ observable. }%
\es{Para computar $P(\Ee,\Aa)$, consideraremos $\Aa^*$ observable. }%
%
\en{The message from the environment factor $P(\Aa)$ to the environment variable $A$ is, }%
\es{El mensaje del factor del ambiente $P(\Aa)$ a la variable ambiente $\A$ es, }%
%
\begin{equation}
m_{P(\A) \rightarrow \A }(\Aa^*) = P(\Aa^*) = m_{\A \rightarrow P(\E|\A)}(\Aa^*)
\end{equation}
%
\en{Which is the same message that the variable $\A$ sends to the strategy factor $P(\E|\A)$. }%
\es{Que es el mismo mensaje que envía la variable $\A$ al factor de estrategias $P(\E|\A)$. }%
%
\en{Then, the message sent by the strategies factor $P(\E|\A)$ to the strategies variable is, }%
\es{Luego, el mensaje que envía el factor de estrategias $P(\E|\A)$ a la variable estrategias es, }%
%
\begin{equation}
m_{P(\E|\A) \rightarrow \E }(\Ee) \propto P(\Aa^*)f_1(\Ee,\Aa^*)
\end{equation}
%
\en{Note that we do not integrate in the values of the environment because we consider that the variable $\Aa^*$ is observable and therefore constant. }%
\es{Notar que no integramos en los valores del ambiente porque consideramos que la variable $\Aa^*$ es observable y por lo tanto constante. }%
%
\en{Then, the marginal of the strategies and a vector of environmental states $\vec{\Aa}$ is, }%
\es{Luego, la marginal de las estrategias y un vector de estados del ambiente $\vec{\Aa}$ es, }%
%
\begin{equation}
P(\Ee,\vec{\Aa}) \propto \prod_{\Aa}^{\vec{\Aa}} P(\Aa)f_1(\Ee,\Aa)
\end{equation}
%
% Integrando sobre las estrategia $\E$ obtenemos la marginal de las observaciones,
% %
% \begin{align}
% P(\vec{\Aa}) & \propto \sum_{\Ee}^\E \prod_{\Aa}^{\vec{\Aa}} P(\Aa)f(\Ee,\Aa) \\
% & = \Big(\prod_{\Aa}^{\vec{\Aa}} P(\Aa) \Big)  \Big( \sum_{\Ee} \prod_{\Aa}^{\vec{\Aa}}f(\Ee,\Aa) \Big) 
% \end{align}
% %
\en{Normalizing it we obtain the posterior, }%
\es{Normalizando obtenemos el posterior, }%
%
\begin{equation}\label{eq:replicator_dynamic_posterior}
P(\Ee|\vec{\Aa}) = \frac{ \prod_{\Aa}^{\vec{\Aa}}f_1(\Ee,\Aa)}{\int_{\Ee} \prod_{\Aa}^{\vec{\Aa}}f_1(\Ee,\Aa)}
\end{equation}
%
\en{which again has the same structure as the replicator dynamic, this time with an integral instead of a sum because the set of strategies is infinite. }%
\es{que nuevamente tiene la misma estructura del replicator dynamic, esta vez con un integral en vez de una suma debido a que el espacio de estrategias es infinito. }%
%
%Es decir, el replicator dynamic surgió de considerar el modelo causal en el que el ambiente causa las estrategias y en el que el tamaño de las estrategias depende de la adaptabilidad de esa estrategia al ambiente.
%
\en{Then, the posterior belongs to the Beta distribution (note that the denominator of the equation \ref{eq:replicator_dynamic_posterior} is the Beta function or the Euler integral).}
\es{Luego, el posterior pertenece a la distribución Beta (notar que el denominador de la ecuación \ref{eq:replicator_dynamic_posterior} es la función Beta o la integral de Euler). }%
%
\begin{equation}
P(\Ee|\vec{\Aa}) = \text{Beta}(\texttt{sum}(\vec{\Aa}), \, \texttt{length}(\vec{\Aa}) - \texttt{sum}(\vec{\Aa})) =  \frac{\Ee^{\texttt{sum}(\vec{\Aa})} \cdot (1-\Ee)^{\texttt{length}(\vec{\Aa}) - \texttt{sum}(\vec{\Aa})}}{B(\texttt{sum}(\vec{\Aa}), \, \texttt{length}(\vec{\Aa}) - \texttt{sum}(\vec{\Aa}))} 
%
\end{equation}
%
\en{with $B(\cdot,\cdot)$ the Beta function. }%
\es{con $B(\cdot,\cdot)$ la función Beta. }%
%
\en{The result of the posterior will indicate the evolutionary stability of the strategies. }%
\es{El resultado del posterior nos indicará la estabilidad evolutiva de las estrategias. }%

% Parrafo

\en{Suppose that the environmental states are generated with probabilities $P(\A = 1) = 0.71$ and $P(\A = 0) = 0.29$. }%
\es{Supongamos que los estados del ambiente se generan con probabilidades $P(\A = 1) = 0.71$ y $P(\A = 0) = 0.29$. }%
%
\en{In Figure~\ref{fig:estrategias_individuales} we show how the posterior of the strategies changes as we add observations to the model. }%
\es{En la figura~\ref{fig:estrategias_individuales} mostramos cómo cambia el posterior de las estrategias a medida que agregamos observaciones al modelo. }%
%
\begin{figure}[H]
    \centering
    \begin{subfigure}[b]{0.32\textwidth}
    \includegraphics[width=\linewidth]{figures/coin1.pdf}
    \caption{$T = 1$}
    \end{subfigure}
    \begin{subfigure}[b]{0.32\textwidth}
    \includegraphics[width=\linewidth]{figures/coin2.pdf}
    \caption{$T = 10$}
    \end{subfigure}
    \begin{subfigure}[b]{0.32\textwidth}
    \includegraphics[width=\linewidth]{figures/coin3.pdf}
    \caption{$T = 10^5$}
    \end{subfigure}
    \caption{
    \en{Posterior probability of strategies as time progresses ($T=1, \, T=10, \, T=10^5$). }%
    \es{Probabilidad posterior de las estrategias a medida que avanza el tiempo ($T=1, \, T=10, \, T=10^5$). }%
    }
    \label{fig:estrategias_individuales}
\end{figure}
%
\en{The evolutionary process selects the strategy that allocates resources in the same proportion as the states of the environment are generated. }%
\es{El proceso evolutivo selecciona la estrategia que reparte los recursos en la misma proporción que se generan los estados del ambiente. }%
%
\en{When the environment generates the states with probability $p=0.71$, then the best adapted strategy is $\Ee = 0.71$. }%
\es{Cuando el ambientes genera los estados con una probabilidad $p=0.71$, entonces la estrategia mejor adaptada es $\Ee = 0.71$. }%

% Parrafo

\en{This selection of individual strategies, perhaps intuitive, will be different, and perhaps counter-intuitive, when we incorporate the possibility of cooperation and defection into the model. }
\es{Esta selección de estrategias individuales, quizás intuitiva, será diferente, y quizás contra-intuitiva, cuando incorporemos al modelo la posibilidad de cooperación y deserción. }

\section{Resultados}

\en{In this section we will analyze whether there is indeed any evolutionary advantage of cooperation and specialization in the presence of defection. }%
\es{En esta sección analizaremos si efectivamente existe alguna ventaja evolutiva de la cooperación y la especialización en presencia de deserción. }%
%
\en{To do so, we will extend the causal model proposed in the methodology section, by simply incorporating the cooperative behavior presented in the introduction section, including a defector behavior (who receives the benefit of cooperation without contributing to it). }%
\es{Para ello, extenderemos el modelo causal propuesto en la sección metodología, incorporando simplemente el comportamiento cooperativo presentado en la sección introducción, incluyendo un comportamiento desertor (que recibe el beneficio de la cooperación sin aportar al mismo). }%
%
\en{This modification alone has a fundamental evolutionary consequence: the formation of groups and thus the emergence of a new evolutionary entity. }%
\es{Esta sola modificación tiene una consecuencia evolutiva fundamental: la formación de grupos y por lo tanto la aparición de una nueva entidad evolutiva. }
%
\en{Because, while defecting individuals may have a relative advantage over cooperating individuals, groups may be favored by mutual cooperation. }%
\es{Porque, si bien los individuos desertores puede tener una ventaja relativa sobre los individuos cooperadores, los grupos pueden verse favorecidos por la mututa cooperación. }%
%
\en{Darwin has already discussed this topic. }
\es{Darwin ya discutió sobre este tema. }
%
\begin{quotation}
It must not be forgotten that although a high standard of morality gives but a slight or no advantage to each individual man and his children over the other men of the same tribe, yet that an increase in the number of well-endowed men and an advancement in the standard of morality will certainly given an immense advantage to one tribe over another.
\end{quotation}
%
\en{The incorporation of cooperation in the probabilistic causal model induces the formation of groups and therefore a type of group-level selection that acts in parallel to individual-level selection. }%
\es{La incorporación de la cooperación en el modelo causal probabilísitico induce la formación de grupos y por lo tanto un tipo de selección a nivel grupal que actúa de forma paralela a la selección de nivel individual. }%
%
\en{The joint result of both selection processes, known as multilevel selection, will be the property of interest. }%
\es{El resultado de ambos procesos de selección, conocido como selección multinivel, será la propiedad de interés. }%

\subsection{Modelo causal probabilísitico.}

\en{Suppose we have regions of $N$ individuals in which $n$ are cooperators and $N-n$ defectors. }%
\es{Supongamos que tenemos regiones de $N$ individuos en el que $n$ son cooperadores y $N-n$ desertores. }%
%
\en{With $N$ individuals, there are $N+1$ types of possible regions, from $n=0$ (all defectors) to $n=N$ (all cooperators). }%
\es{Con $N$ individuos, hay $N+1$ tipos de regiones posibles, desde $n=0$ (todos desertores) hasta $n=N$ (todos cooperadores). }%
%
\en{Then, if each individual belongs to a single region we need $M = N(N+1)$ total individuals, $i \in \{1, \dots, M\}$. }%
\es{Luego, si cada individuo pertenece a una única región necesitamos $M = N(N+1)$ individuos totales, $i \in \{1, \dots, M\}$. }%
%
\en{Individuals $i$ are characterized by three attributes: }%
\es{Los individuos $i$ están caracterizados por tres atributos: }%
%
\en{the region to which it belongs, $\texttt{region}(i)=i \ \texttt{div} \ N$; }%
\es{la región a la que pertenece, $\texttt{region}(i)=i \ \texttt{div} \ N$; }%
%
\en{if their social behavior is cooperative, $\texttt{coop}(i) =  i \ \texttt{mod} \ N < \texttt{region}(i)$ (the first $n$ individuals in the region are cooperators and the rest defectors); }%
\es{si su comportamiento social es cooperador, $\texttt{coop}(i) =  i \ \texttt{mod} \ N < \texttt{region}(i)$ (los primeros $n$ individuos de la región son cooperadores y el resto desertores); }%
%
\en{and the strategy $\Ee_i$ they use to allocate resources $f_1(\Ee,\Aa)$ (equation \ref{eq:familia_de_aptitudes}). }%
\es{y la estrategia $\Ee_i$ que utilizan para alocar recursos $f_1(\Ee,\Aa)$ (ecuación~\ref{eq:familia_de_aptitudes}). }%
%
\en{A priori, there is no preference for defector or cooperative social behavior. }%
\es{A priori, no hay preferencia por el comportamiento social desertor o cooperador. }%
%
\en{Then, by combinatorics, it will be more likely a priori that individuals will inhabit mixed regions. }%
\es{Entonces, por simple combinatoria, será más probable a priori que los individuos habiten regiones mixtas. }%
%
\begin{equation}
P(i) = \frac{1}{N} \text{Binomial}(\texttt{region}(i),N,0.5) 
\end{equation}
%
\en{where the binomial distribution gives more weight to individuals from mixed regions, the parameter $0.5$ indicates a uniform prior between cooperating and defecting social behaviors, and the factor $\frac{1}{N}$ indicates that all individuals in the same region start with the same initial resources. }%
\es{donde la distribución binomial otorga más peso a los individuos de las regiones mixtas, el parámetro $0.5$ indica un prior uniforme entre los comportamientos sociales cooperador y desertor, y el factor $\frac{1}{N}$ indica que todos los individuos de la misma región empiezan con los mismos recursos iniciales. }%
%
\en{As we have seen in the methodology section, the probability of the environmental states is, }%
\es{Como hemos visto en la sección metodología, la probabilidad de los estados ambientales es, }%
%
\begin{equation}
P(\Aa) = p^\Aa (1-p)^{1-\Aa}
\end{equation}
%
\en{but now at each time $t$ we have a vector $\vec{\Aa}$, in which the $i-$th element influences the $i-$th individual, }%
\es{pero ahora en cada tiempo $t$ tenemos un vector $\vec{\Aa}$, en el que el $i-$ésimo elemento influencia al $i-$ésimo individuo, }%
%
\begin{equation}
P(i|\vec{\Aa}) = \frac{\Ee_i^{\Aa_i} (1-\Ee_i)^{1-\Aa_i}}{\sum_j \Ee_j^{\Aa_j} (1-\Ee_j)^{1-\Aa_j}} \propto \Ee_i^{\Aa_i} (1-\Ee_i)^{1-\Aa_i}
\end{equation}
%
\en{Now, however, the agents are also influenced by the resources of the other agents at the previous time, }%
\es{Ahora sin embargo, los agentes también están influenciados por los recursos del los otros agentes en el tiempo anterior, }%
%
\begin{equation}
P(i^{t+1}|i^t) = 
\begin{cases}
1/N & \texttt{coop}(i^t) \wedge (\texttt{region}(i^{t+1}) = \texttt{region}(i^{t})) \\
1 & \neg \texttt{coop}(i^t) \wedge i^{t+1} = i^t \\
0 & \texttt{else} \\
\end{cases}
\end{equation}
%
\en{Cooperating individuals divide the wealth equally with members of the same region, and defecting individuals retain all their wealth. }%
\es{Los individuos cooperadores dividen la riqueza en partes iguales con los miembros de la misma región, y los individuos desertores se quedan con toda su riqueza. }%
%
\en{Finally, the groups are constituted by the members of each region, $P(g|i)=\mathbb{I}(\texttt{region}(i) = g)$, the indexing function which is $1$ when individuals belong to region $g$ and $0$ otherwise. }%
\es{Finalmente, los grupos están constituidos por los miembros de cada región, $P(g|i)=\mathbb{I}(\texttt{region}(i) = g)$, la función indiciadora que es $1$ cuando los individuos pertenecen a la región $g$ y $0$ en otro caso. }%
%
\begin{figure}[H]
\centering
 \begin{subfigure}[b]{0.4\textwidth}    
 \centering
 \tikz{
    
    \node[obs] (a1) {$\vec{\A}^{\,1}$};
    \node[obs, right=of a1] (a2) {$\vec{\A}^{\,2}$};
    
    \node[latent, below=of a1 ] (i1) {$I^1$};
    \node[latent, below=of a2 ] (i2) {$I^2$};
    \node[latent, right=of i2 ] (i3) {$I^3$};

    \node[latent, below=of i1 ] (c1) {$G^1$};
    \node[latent, below=of i2 ] (c2) {$G^2$};
    \node[latent , below=of i3 ] (c3) {$G^3$};
    
    \node[invisible, below=of c2, yshift=-0.5cm] (inv) {};
    
    
    \edge {a1} {i1};
    \edge {a2} {i2};
    \edge {i1} {i2,c1};
    \edge {i2} {i3,c2};
    \edge {i3} {c3};
    
    }
 \caption{\en{Bayesian network}\es{Red bayesiana}}
 \label{fig:red_bayesiana_multinivel}
 \end{subfigure}
\begin{subfigure}[b]{0.58\textwidth}    
 \centering
 \tikz{
    
    \node[factor] (fa1) {};
    \node[obs, yshift=0.5cm, below=of fa1] (a1) {$\vec{\A}^{\,1}$};
    
    \node[factor, yshift=0.5cm, below=of a1] (fia1) {};
    
    \node[latent, yshift=0.5cm, below=of fia1 ] (i1) {$I^1$};
    
    \node[factor, xshift=0.3cm, left=of i1 ] (fi0) {};
    
    \node[factor, xshift=-0.3cm, right=of i1] (fii1) {};
    \node[factor, yshift=0.5cm, below=of i1] (fg1) {};
    \node[latent, yshift=0.5cm, below=of fg1 ] (g1) {$G^1$};
    
    \node[latent, xshift=-0.3cm, right=of fii1 ] (i2) {$I^2$};
    
    \node[factor, yshift=-0.5cm, above=of i2] (fia2) {};
    \node[obs, yshift=-0.5cm, above=of fia2] (a2) {$\vec{\A}^{\,1}$};
    \node[factor, yshift=-0.5cm, above=of a2] (fa2) {};
    \node[factor, yshift=0.5cm, below=of i2] (fg2) {};
    \node[latent, yshift=0.5cm, below=of fg2 ] (g2) {$G^2$};
    
    \node[factor, xshift=-0.3cm, right=of i2] (fii2) {};
    
    \node[latent, xshift=-0.3cm, right=of fii2 ] (i3) {$I^3$};
    \node[factor, yshift=0.5cm, below=of i3] (fg3) {};
    \node[latent, yshift=0.5cm, below=of fg3 ] (g3) {$G^3$};
    
     \node[const, left=of fa1] (pa) {$P(\vec{\Aa})$};
     \node[const, left=of fia1] (pea) {$P(i|\vec{\Aa})$};
     \node[const, left=of fg1] (pg) {$P(g|i)$};
     \node[const, above=of fii2] (pee) {$P(i^{t+1}|i^t)$};
     \node[const, left=of fi0] (pa) {$P(i^1)$};
    
    
    \edge[-] {a1} {fa1, fia1};
    \edge[-] {a2} {fa2, fia2};
    \edge[-] {i1} {fii1,fg1,fia1,fi0};
    \edge[-] {i2} {fii1, fii2,fg2,fia2};
    \edge[-] {i3} {fii2, fg3};
    \edge[-] {g1} {fg1};
    \edge[-] {g2} {fg2};
    \edge[-] {g3} {fg3};
    }
 \caption{Factor graph}
 \label{fig:factor_graph_multinivel}
 \end{subfigure}
 \caption{
 \en{Hierarchical model. }%
 \es{Modelo jerárquico. }%
 %
 \en{Figure~\ref{fig:red_bayesiana_multinivel}: the probabilistic dependencies arising from the multilevel causal model. }%
 \es{Figura~\ref{fig:red_bayesiana_multinivel}: las dependencias probabilísiticas que surge del modelo causal multinivel. }%
 %
 \en{Figure~\ref{fig:factor_graph_multinivel}: the factor graph induced by the Bayesian network, which will be used to apply the sum-product algorithm. }%
 \es{Figura~\ref{fig:factor_graph_multinivel}: el grafo de factores inducido por la red bayesiana, que será utilizado para aplicar el algoritmo suma-producto. }%
 %
 \en{The gray variables are observed. }%
 \es{Las variables en gris se consideran observables. }%
 }
\label{fig:multilevel_model}
\end{figure}
%
\en{In short, at each time $t$ the environment influences individuals, and individuals influence the groups of their own time and the individuals of time $t+1$. }%
\es{En resumen, en cada tiempo $t$ el ambiente influencia a los individuos, y los individuos influencian a los grupos de su propio tiempo y a los individuos del tiempo $t+1$. }%
%
\en{This model has $3$ hyperparameters: the vector of strategies $\vec{\Ee}$, the probability of the environment $p$, and the size of the groups $N$. }%
\es{Este modelo tiene $3$ hiperparámetros: el vector de estrategias $\vec{\Ee}$, la probabilidad del ambiente $p$, y el tamaño de los grupos $N$. }%
%
%Este modelo se puede simplificar levemente, usando una única variable de grupo que dependa de la última variable individuo, gracias a que la margnial de los grupos es la misma en cualquier tiempo $t$.

% Parrafo

\en{The evolutionary problem we are interested in is the selection of cooperative individuals given the environment within each group (level 1), $P(\texttt{coop}(i^T)|\vec{\Aa}^{\,1}, \dots, \vec{\Aa}^{\,T-1}, g)$, the selection of groups given the environment (level 2), $P(g^T|\vec{\Aa}^{\,1}, \dots, \vec{\Aa}^{\,T-1})$,  and selection of cooperative individuals given the environment for all groups (multilevel), $P(\texttt{coop}(i^T)|\vec{\Aa}^{\,1}, \dots, \vec{\Aa}^{\,T-1})$. }%
\es{El problema evolutivo que nos interesa es la selección de los individuos cooperadores dado el ambiente dentro de cada grupo (nivel 1), $P(\texttt{coop}(i^T)|\vec{\Aa}^{\,1}, \dots, \vec{\Aa}^{\,T-1}, g)$, la selección de los grupos dado el ambiente (nivel 2), $P(g^T|\vec{\Aa}^{\,1}, \dots, \vec{\Aa}^{\,T-1})$, y la selección de los individuos cooperadores dado el ambiente integrando todos los grupos (multinivel), $P(\texttt{coop}(i^T)|\vec{\Aa}^{\,1}, \dots, \vec{\Aa}^{\,T-1})$. }%
%
\en{The multilevel selection is obtained by integrating the product of level 1 and 2 selections, }%
\es{La selección multinivel se obtiene integrando el producto de las selecciones de nivel 1 y 2, }%
%
\begin{equation}\label{eq:posterior_multinivel}
\underbrace{P(\texttt{coop}(i^T)|\vec{\Aa}^{\,1}, \dots, \vec{\Aa}^{\,T-1})}_{\text{\en{Multilevel selection}\es{Selección multinivel}}} = \sum_{g=0}^N \underbrace{P(\texttt{coop}(i^T)|\vec{\Aa}^{\,1}, \dots, \vec{\Aa}^{\,T-1}, g)}_{\text{\en{Level 1 selection}\es{Selección de nivel 1}}} \cdot \underbrace{P(g^T|\vec{\Aa}^{\,1}, \dots, \vec{\Aa}^{\,T-1})}_{\text{\en{Level 2 selection}\es{Selección de nivel 2}}}
\end{equation}
%
\en{where the posterior of level 1 selection is obtained by integrating the probability of all cooperating individuals that are part of the group $g$, }%
\es{donde el posterior de la selección de nivel 1 se obtiene integrando la probabilidad de todos los individuos cooperadores que forman parte del grupo $g$, }%
%
\begin{equation}\label{eq:posterior_nivel_1}
\begin{split}
P(\texttt{coop}(i^T)|\vec{\Aa}^{\,1}, \dots, \vec{\Aa}^{\,T-1}, g) &= \sum_{j=1}^M P(I^T=j|\vec{\Aa}^{\,1}, \dots, \vec{\Aa}^{\,T-1}, g)\mathbb{I}(\texttt{coop}(j))
\end{split}
\end{equation}

\subsection{Probabilidades marginales}

\en{To compute the posteriors it will be sufficient to compute the marginal probabilities, since both expressions are proportional. }%
\es{Para calcular los posteriors va a ser suficiente con calcular las probabilidades marginales, pues ambas expresiones son proporcionales. }%
%
\en{The marginal probabilities can be obtained by applying the sum-product algorithm in the model of figure~\ref{fig:multilevel_model}, taking as observed the variables that in the posterior appear in the conditional (see methodology section). }%
\es{Las probabilidades marginales se pueden obtener aplicando el sum-product algorithm en el modelo de la figura~\ref{fig:multilevel_model} pero que tiene como observables las variables que en el posterior aparecen en el condicional (ver sección metodología). }%
%
% Ya hemos visto en la sección metodología que la probabilidad marginal de una variable $v$ es el producto de los mensajes que recibe de los nodos vecinos del factor graph.
% %
% Las variables observables no deben integrarse cuando se aplica el sum-product algorithm, por lo que la marginal de $v$ será la probabilidad conjunta de $v$ y todas las variables observables del modelo.
%
\en{The marginal probability of the individuals at time $T$, when the environment variables $vec{\Aa}^{\,t}$ and the group variable $g$ are observed, is }%
\es{La probabilidad marginal de los individuos en el tiempo $T$ cuando todos los ambientes $\vec{\Aa}^{\,t}$ y el grupo $g$ son observados, }%
%
\begin{equation}\label{eq:posterior_individuos_dado_grupo}
P(i^{T},\vec{\Aa}^{\,1}, \dots, \vec{\Aa}^{\,T-1}, g) = m_{P(I^{T}|I^{T-1}) \rightarrow I^T}(i^T) \cdot m_{P(G^{T}|I^{T}) \rightarrow I^T}(i^T) 
\end{equation}
%
\en{the product of the messages that the variable $i^T$ receives from the social factor $P(I^{T}|I^{T-1})$ and from the group factor $P(G^{T}|I^{T})$, in the model that has the set $\{ \vec{\Aa}^{\,1}, \dots, \vec{\Aa}^{\,T-1}, g\}$ as observed variables. }%
\es{el producto de los mensajes que la variable $i^T$ recibe del factor social $P(I^{T}|I^{T-1})$ y del factor de grupo $P(G^{T}|I^{T})$, en el modelo que tiene al conjunto $\{ \vec{\Aa}^{\,1}, \dots, \vec{\Aa}^{\,T-1}, g\}$ como variables observables. }%
%
\en{The marginal probability of the groups at time $T$, when we observe all the environmental states $\vec{\Aa}^{\,t}$, is }%
\es{La probabilidad marginal de los grupos en el tiempo $T$, cuando observamos todos los estados ambientales $\vec{\Aa}^{\,t}$, es }%
%
\begin{equation}\label{eq:posterior_nivel_2}
P(g^{T},\vec{\Aa}^{\,1}, \dots, \vec{\Aa}^{\,T-1}) = m_{P(G^{T}|I^{T}) \rightarrow G^T}(g^T)
\end{equation}
%
\en{the message that the variable $G^T$ receives from the group factor $P(G^{T}|I^{T})$, in the model that has $\{ \vec{\Aa}^{\,1}, \dots, \vec{\Aa}^{\,T-1} \}$ as observed variables. }%
\es{el mensaje que la variable $G^T$ recibe del factor de grupo $P(G^{T}|I^{T})$, en el modelo que tiene a $\{ \vec{\Aa}^{\,1}, \dots, \vec{\Aa}^{\,T-1}\}$ como variables observadas. }%
%
\en{An alternative way to compute the multilevel selection marginal (Eq.~\ref{eq:posterior_multinivel}) is by integrating the probability of all cooperating individuals given the environments }%
\es{Una forma alternativa de calcular la marginal de la selección multinivel (Eq.~\ref{eq:posterior_multinivel}) es integrando la probabilidad de los individuos cooperadores dado los ambientes }%
%
\begin{equation}\label{eq:posterior_multinivel_alternativa}
\begin{split}
P(\texttt{coop}(i^T),\vec{\Aa}^{\,1}, \dots, \vec{\Aa}^{\,T-1}) &= \sum_{j=1}^M P(I^T=j,\vec{\Aa}^{\,1}, \dots, \vec{\Aa}^{\,T-1})\mathbb{I}(\texttt{coop}(j))
\end{split}
\end{equation}
%
\en{where the probability of individuals, when the environmental states $\vec{\Aa}^{\,t}$ are observed, is }%
\es{donde la probabilidad de los individuos cuando tenemos como observable todos los ambientes $\vec{\Aa}^{\,t}$ es, }%
%
\begin{equation}
P(i^{T},\vec{\Aa}^{\,1}, \dots, \vec{\Aa}^{\,T-1}) = m_{P(I^{T}|I^{T-1}) \rightarrow I^T}(i^T)
\end{equation}
%
\en{the message that the variable $i^T$ receives from the social factor $P(I^{T}|I^{T-1})$ in the model that has $\{ \vec{\Aa}^{\,1}, \dots, \vec{\Aa}^{\,T-1}\}$ as observed variables. }%
\es{el mensaje que la variable $i^T$ recibe del factor social $P(I^{T}|I^{T-1})$ en el modelo que tiene a $\{ \vec{\Aa}^{\,1}, \dots, \vec{\Aa}^{\,T-1}\}$ como observables. }%
%
\en{In summary, we need to compute only three messages. }%
\es{En resumen, necesitamos calcular solamente tres mensajes. }%
%
\begin{table}[H]
\centering
\begin{tabular}{cccc}
 \en{Messages}\es{Mensajes} & $m_{P(G^{T}|I^{T}) \rightarrow I^T}(i^T)$ & $m_{P(G^{T}|I^{T}) \rightarrow G^T}(g^T)$ & $m_{P(I^{T}|I^{T-1}) \rightarrow I^T}(i^T)$ \\[0.1cm]
 \en{Observed}\es{Observables} & $\{g\}$ &  $\{ \vec{\Aa}^{\,1}, \dots, \vec{\Aa}^{\,T-1}\}$ &   $\{ \vec{\Aa}^{\,1}, \dots, \vec{\Aa}^{\,T-1}\}$
\end{tabular}
\caption{
\en{The messages that comprise the marginals of the level 1, 2 and multilevel selection. }%
\es{Los mensajes que componen las marginales de la selección de nivel 1, 2 y multinivel. }%
}
\end{table}
%
\en{The first message, which is sent by the group factor to the individual variable when the group variable is observed, is }%
\es{El primer mensaje, que envía el factor de grupo a la variable individuo cuando la variable de grupo es observada, es }%
%
\begin{equation}
 m_{P(G^{T}|I^{T}) \rightarrow I^T}(i^T) = P(g|i) = \mathbb{I}(\texttt{region}(i) = g)
\end{equation}
%
\en{the indicator function which is $1$ for individuals in the region $r = g$ and $0$ for the rest. }%
\es{la función indiciadora que vale $1$ para los individuos que se encuentran en la región $r = g$ y $0$ para el resto. }%
%
\en{The second message, which the group factor sends to the group variable, is }%
\es{El segundo mensaje, que el factor grupo envía a la variable grupo, es }%
%
\begin{equation}
 m_{P(G^{T}|I^{T}) \rightarrow G^T}(g^T) = \sum_i P(g^T|i) \,  m_{P(I^{T}|I^{T-1}) \rightarrow I^T}(i) 
\end{equation}
%
\en{which is composed of the third message, which the social factor sends to the individuals' variable. }%
\es{el cual está compuesto por el tercer mensaje, que el factor social envía a la variable individuo. }%
%
\en{To compute this third and last message we must find the preceding messages that compose it. }%
\es{Para computar este tercer y último mensaje debemos calcular los mensajes precedentes que lo componen. }%
%
\en{Among them are the messages that the environmental factor sends to the environmental variable, }%
\es{Entre ellos están los mensajes que el factor ambiente envía a la variable ambiente, }%
%
\begin{equation}
m_{P(\vec{\A}^{\,t})\rightarrow \vec{\A}^{\,t}}(\vec{\Aa}^{\,t}) = P(\vec{\Aa}^{\,t}) = m_{\vec{\A}^{\,t} \rightarrow P(I^t|\vec{\A}^{\,t})} (\vec{\Aa}^{\,t})
\end{equation}
%
\en{which is the same message that the environment variable sends to the individuals factor. }
\es{que es el mismo mensaje que la variable ambiente envía al factor individuo. }%
%
\en{Then we can generate the message that the individuals factor sends to the individuals variable, }%
\es{Luego podemos generar el mensaje que envía el factor individuo a la variable individuo, }%
%
\begin{equation}
m_{P(I^t|\vec{\A}^{\,t}) \rightarrow I^t}(i^t) = P(\vec{\Aa}^{\,t}) \, P(i^t|\vec{\Aa}^{\,t})
\end{equation}
%
\en{that since the environmental variable is observed, the integration is not performed. }%
\es{que como la variable ambiente es observable, no incluye la integración que suelen hacer los mensajes que envían los factores. }%
%
\en{And finally, the message that the individual variable sends to the social factor, }%
\es{Y por último, el mensaje que la variable individuo envía al factor social, }%
%
\begin{equation}\label{eq:m_i_pii}
\begin{split}
m_{I^t \rightarrow P(I^{t+1}|I^t)}(i^t) =m_{P(I^t|\vec{\A}^{\,t}) \rightarrow I^t}(i^t) \, m_{P(I^t|I^{t-1}) \rightarrow I^t }(i^t) % & = P(\vec{\Aa}^{\,t}) \, P(i^t|\vec{\Aa}^{\,t}) \text{Prior}(i^t)
\end{split}
\end{equation}
%
% donde $\text{Prior}(i^t) = m_{P(i^t|i^{t-1}) \rightarrow i^t }(i^t)$ es el mensaje que la variable individuo recibe del factor social pasado.
% %
\en{We do not include in the product the message that the group factor sends to the individual variable, because as it is not observed, the message is automatically cancelled, $m_{P(g^t|i^{t}) \rightarrow i^t}(i^t) = \sum_g P(g|i^t) = 1$. }%
\es{No incluimos en la multiplicación el mensaje que el factor de grupo envía a la variable individuo porque al no ser observable se cancela automáticamente, $m_{P(g^t|i^{t}) \rightarrow i^t}(i^t) = \sum_g P(g|i^t) = 1$. }%
%
\en{Finally, the third target message (encoding multilevel selection) introduces the social factor over individual resources. }%
\es{Finalmente, el tercer mensaje objetivo (que codifica la selección multinivel) introduce el factor social sobre los recursos individuales. }%
%
\begin{equation}\label{eq:m_pii_i}
\begin{split}
m_{P(I^{t+1}|I^{t}) \rightarrow I^{t+1} }(i^{t+1}) & = \sum_{i^t} P(i^{t+1}|i^t) \, P(\vec{\Aa}^{\,t}) \, P(i^t|\vec{\Aa}^{\,t}) \,  m_{P(I^t|I^{t-1}) \rightarrow I^t }(i^t) 
\end{split}
\end{equation}
%
\en{This message is recursively defined. }%
\es{Este mensaje está definido recursivamente. }%
%
%
% En la introducción hemos derivado la tasa de crecimiento de esta marginal para los grupos enteramente desertoras y solo hemos presentado un resultado numérico para el grupo enteramente cooperador.
% %
% En la siguiente sección resolveremos analíticamente el posterior de la selección de grupos y multinivel y verificaremos la hipótesis de Ole Peters de la ventaja evolutiva de la cooperación.
% %
% Y la sección \ref{sec:especialization}, haremos uso de los resultado análitico para demostrar la ventaja evolutiva de la especilización, rechazando la hipótesis de Ole Peters de que la especialización es un mecanismo implausible para poblaciones chicas.
% 
% 
% En esta subsección resolveremos la marginal que necesitamos para computar el posterior de la la selección de nivel 1, de nivel 2 y multinivel,
% %
% \begin{equation}
% \begin{split}
% P(I^{T+1}=k, \vec{\Aa}^{\,1}, \dots, \vec{\Aa}^{\,T}) & = m_{P(I^{T+1}|I^{T}) \rightarrow I^{T+1} }(k)
% \end{split}
% \end{equation}
% %
% donde el mensaje está definido recursivamente en la ecuación~\ref{eq:m_pii_i}.
%
\en{In the appendix we solve it with mathematical induction by cases: first for individuals in the entirely defecting region, then for the entirely cooperating region, and finally for the mixed region. }%
\es{En el anexo lo resolvemos con inducción matemática por casos: primero para los individuos de la región enteramente desertora, luego para la región enteramente cooperadores, y finalmente para la región mixta. }%
%
\en{In all cases, the marginal is equal to the posterior times the probability of the observed environment, }%
\es{En todos los casos, la marginal es igual al posterior por la probabilidad del ambiente observado, }%
%
\begin{equation}
\begin{split}
P(I^{T+1}=k, \vec{\Aa}^{\,1}, \dots, \vec{\Aa}^{\,T}) & = P(k| \vec{\Aa}^{\,1}, \dots, \vec{\Aa}^{\,T}) \prod_{t=1}^T P(\vec{\Aa}^{\,t})
\end{split}
\end{equation}
%
\en{where}\es{donde}
%
\begin{equation}
P(k| \vec{\Aa}^{\,1}, \dots, \vec{\Aa}^T) = 
\begin{cases}
P(k)\prod_{t=1}^{T} P(k|\vec{\Aa}^{\,t}) &  r=0  \\
P(k)\prod_{t=1}^{T} \sum_j^{\texttt{\en{partners}\es{socios}}(r)} \frac{1}{N} P(j|\vec{\Aa}^{\,t}) & r  = N  \\
\Big(P(k)\prod_{t=1}^{T} P(k|\vec{\Aa}^{\,t}) \Big) + \Big(\sum_{t=1}^{T} P(c|\wedge_{q=1}^t\vec{\Aa}^{\,q})  \prod_{q=t+1}^T P(k|\vec{\Aa}^{\,q}) \Big) & 0 < r < N  
\end{cases}
\end{equation}
%
\en{where $r = \texttt{region}(k)$, and $c$ is a cooperative individual belonging the specific region, $c \in \texttt{partners}(r)$. }%s
\es{donde $r = \texttt{region}(k)$, y $c$ es un individuo cooperador que pertence a esa región, $c \in \texttt{\en{partners}\es{socios}}(r)$. }%

\subsection{Equivalencia con el modelo de Ole Peters}

\en{The proposed causal model does not require assuming that the resource updating process is multiplicative, as proposed by Ole Peters for his game. }%
\es{El modelo causal propuesto no requiere suponer que el proceso de actualización de los recursos sea multiplicativo, como propone Ole Peters para su juego. }%
%
\en{All the assumptions of our model are made explicit in the Bayesian network in Figure~\ref{fig:multilevel_model}. }
\es{Todos los supuestos de nuestro modelo se encuentran explicitados en la red bayesiana de la figura~\ref{fig:multilevel_model}. }%
%
\en{It does not assume any particular process, other than that individuals are affected by the environment, and that individuals can cooperate and defect. }%
\es{No supone ningún tipo de proceso particular, más allá de que los individuos se ven afectados por el ambiente, y que los individuos pueden cooperar y desertar. }%
%
\en{However, our probabilistic causal model and the game proposed by Ole Peters are equivalent. }%
\es{Sin embargo, nuestro modelo causal probabilístico y el juego propuesto por Ole Peters son equivalentes. }%
 
% Parrafo

\en{Without the need to presuppose a multiplicative process, the posteriors of individuals in our causal model are proportional to the trajectory of resources arising from the game proposed by Ole Peters. }%
\es{Sin necesidad de presuponer un proceso multiplicativo, los posteriors de los individuos en nuestro modelo causal son proporcionales a la trayectoria de los recursos que surgen del juego propuesto por Ole Peters. }%
%
\en{The posterior of individuals from the entirely deserting region is, }%
\es{El posterior de los individuos de la región enteramente desertora es, }%
%
\begin{equation}
\begin{split}
P(i^{T+1} | \vec{\Aa}^{\,1}, \dots, \vec{\Aa}^{\,T}) & \overset{r=0}{=}  P(i^{T+1})  \prod_{t=1}^{T} P(i^{T+1}|\vec{\Aa}^{\,t}) \\
& \propto P(i^{T+1}) \prod_{t=1}^{T} \Ee_k^{\Aa_k} (1-\Ee_k)^{1-\Aa_k}   \propto P(i^{T+1}) \prod_{t=1}^{T} 2.1 f_1(\Ee_i,\Aa_{i}^{t})
\end{split}
\end{equation}
%
\en{where $r = \texttt{region}(i^{T+1}) = 0$ denotes the region without cooperators and $f_i$ is the fitness function that we define in the equation \ref{eq:familia_de_aptitudes} from the game proposed by Ole Peters. }%
\es{donde $r = \texttt{region}(i^{T+1}) = 0$ indica la región sin cooperadores y $f$ es la familia de aptitudes que definimos en la ecuación \ref{eq:familia_de_aptitudes} a partir del juego propuesto por Ole Peters. }%
%
\en{The first proportional is valid because of the normalization constant of the probability distribution $P(i^t|\vec{\Aa}^{\,t})$. }%
\es{El primer proporcional vale por constante de normalización de la distribución de probabilidad $ P(i^t|\vec{\Aa}^{\,t})$. }%
%
\en{And the second proportional simply includes the factor $2.1$, so that the fitness function matches the one proposed by Ole Peters. }%
\es{Y el segundo proporcional simplemente incluye el factor $2.1$, para que la función de aptitudes coincida con la prouesta por Ole Peters. }%
%
\en{If we consider that the prior $P(i^{T+1})$ represents the initial resources $\omega(0)$, then the posterior of the individual defectors in entirely defecting groups is proportional to the growth of the individual resources studied by Ole Peters. }%
\es{Si consideramos que el prior $P(i^{T+1})$ representa los recursos inciales $\omega(0)$, luego el posterior de los individuos desertores en grupos enteramenete desertoras es proporcional al crecimiento de los recursos individuales estudiado pr Ole Peters. }%

% Parrafo

\en{The same is true for the posteriors of cooperating individuals. }%
\es{Lo mismo ocurre con los posteriors de los individuos cooperadores. }%
%
\en{Note that the message sent by the individual variables to the social factor (eq~\ref{eq:m_i_pii}) is proportional to the second node of the cooperative protocol (fig~\ref{fig:protocolo}), in which the previous resources $\omega_i(t)$ are updated by the product of the fitness function. }%
\es{Notar que el mensaje que envían las variables individuos al factor social (ecuación~\ref{eq:m_i_pii}) es proprocional al segundo nodo de del protocolo cooperativo (figura \ref{fig:protocolo}), en el que los recursos previos $\omega_i(t)$ mediante el producto de la función de fitness. }%
%
\begin{equation}
\begin{split}
m_{I^t \rightarrow P(I^{t+1}|I^t)}(i^t) & = m_{P(I^t|\vec{\A}^{\,t}) \rightarrow I^t}(i^t) \, m_{P(I^t|I^{t-1}) \rightarrow I^t }(i^t) \\
& = P(i^t| \vec{\Aa}^{\,t}) \, P(i^t, \vec{\Aa}^{\,t}, \dots, \vec{\Aa}^{\,t}) \\
& \propto 2.1\,f_1(\,\Ee_i,\Aa_{i}^{t}) \omega(t-1)
\end{split}
\end{equation}
%
%donde la marginal $P(i^t, \vec{\Aa}^{\,t}, \dots, \vec{\Aa}^{\,t})$ es proporcional al posterior $P(i^t| \vec{\Aa}^{\,t}, \dots, \vec{\Aa}^{\,t})$.
%
\en{Again, the proportionality holds by construction of the probability distribution $P(i^t|\vec{\Aa}^{\,t})$. }%
\es{Nuevamente, la prorporcionalidad vale por construcción de la distribución de probabilidad $P(i^t|\vec{\Aa}^{\,t})$. }%
%
\en{And the message that the social factor sends to the individual variable of the following time (equation~\ref{eq:m_pii_i}) is proprotional to the individual resources after pooling and sharing, }%
\es{Y el mensaje que envía el factor social a la variable individuo del siguiente tiempo (ecuación~\ref{eq:m_pii_i}) es proprocional a los recursos individuales luego de la redistribución del fondo común, }%
%
\begin{equation}
\begin{split}
m_{P(I^{t+1}|I^{t}) \rightarrow I^{t+1} }(i^{t+1}) & = \sum_{i^t} P(i^{t+1}|i^t) \, P(\vec{\Aa}^{\,t}) \, P(i^t|\vec{\Aa}^{\,t}) \,  m_{P(I^t|I^{t-1}) \rightarrow I^t }(i^t) \\
%& \propto \sum_{i^t} P(i^{t+1}|i^t) \, P(i^t|\vec{\Aa}^{\,t}) \,  \text{Prior}(i^t) \\
= P(I^{t+1}, \vec{\Aa}^{\,1} , \dots, \vec{\Aa}^{\,T} ) & \propto \sum_{i^t} P(i^{t+1}|i^t) \, f_{\Ee_i}(\Aa_{i}^{t}) \omega_i(t)  = \omega_i(t+1)
\end{split}
\end{equation}
%
\en{where $P(i^{t+1}|i^t) = \frac{1}{N}$ when individuals belong to the same region $\texttt{region}(i^{T+1}) = \texttt{region}(i^{T+1})$. }%
\es{donde $P(i^{t+1}|i^t) = \frac{1}{N}$ cuando los individuos pertenecen a la misma región $\texttt{region}(i^{T+1}) = \texttt{region}(i^{T+1})$. }%
%
\en{This message generalizes the last node of the cooperative protocol (figure~\ref{fig:protocolo}). }%
\es{Este mensaje generaliza el último nodo del protocolo cooperativo (figura~\ref{fig:protocolo}). }%

% Parrafo

\en{In general, the posterior of individuals is nothing more than the proportion of resources they manage. }%
\es{En general, el posterior de los individuos no es más que la proporción de recursos que maneja. }%
%
\begin{equation}
P(k| \vec{\Aa}^{\,1}, \dots,  \vec{\Aa}^{\,T}) = \frac{\omega_k(T)}{\sum_j \omega_j(T)}
\end{equation}
%
\en{The proportionality between the posteriority of individuals and the trajectories of resources proposed by Ole Peters allows us to work with both expressions interchangeably. }%
\es{La proporcionalidad entre el posterior de los individuos y las trayectorias de los recursos propuesto por Ole Peters nos permite trabajar indistintamanete con una u otra expresión. }%

\subsection{Dilema de los bienes comunes}

\en{In the last few decades evolutionary biology has begun to adopt the analogy of the ``tragedy of the commons''~\cite{rankin2007-tragedyCommonsBiology}. }%
\es{En las últimas décadas la biología evolutiva ha comenzado a adoptar la analogía de la ``tragedia de los comunes''~\cite{rankin2007-tragedyCommonsBiology}. }%
%
\en{This concept contains the idea that the commons has a payoff structure isomorphic to the N-player prisoner's dilemma~\cite{hardin1971-collectiveAsPrisionerDilema}. }%
\es{Este concepto contiene la idea de que los bienes comunes tienen una estructura de pagos isomorfa al dilema del prisionero de N jugadores~\cite{hardin1971-collectiveAsPrisionerDilema}. }%
%
\en{In a two-player prisoner's dilemma, cooperating implies a cost $c$ for the other person to receive a benefit $b$, with $b > c$, and defecting means refusing to cooperate and carries no cost. }%
\es{En un dilema del prisionero de dos jugadores, cooperar implica un coste $c$ para que la otra persona reciba un beneficio b, con $b > c$, y desertar significa negarse a cooperar y no conlleva ningún coste. }% 
%
\begin{equation}
  \bordermatrix{ & C & D \cr
      C & b-c & -c \cr
      D & b & 0 } 
\end{equation}
%
\en{Players gain more if they opt for mutual cooperation than for mutual defection, since $b - c > 0$. }%
\es{Los jugadores ganan más si optan por la cooperación mutua que por la deserción mutua, ya que $b - c > 0$. }%
%
\en{However, regardless of what the other player does, it is better not to cooperate: if my partner defects, it is better for me to defect than to cooperate, since $0 > -c$; if my partner cooperates, it is still better for me to defect than to cooperate, since $b > b - c$. }%
\es{Sin embargo, independientemente de lo que haga el otro jugador, es mejor no cooperar: si mi compañero deserta, es mejor para mí desertar que cooperar, ya que $0 > -c$; si mi compañero coopera, sigue siendo mejor para mí desertar que cooperar, ya que $b > b - c$. }%
%
\en{This creates a dilemma: although mutual cooperation is a preferable outcome, no individual has the incentive to cooperate. }%
\es{De ahí el dilema: aunque la cooperación mutua es un resultado preferible, ningún individuo tiene el incentivo de cooperar. }%

% Parrafo

%En vez de utilizar una matriz de pagos del dilema del prisionero para representar un proceso de bienes comunes, en nuestro trabajo definimos el proceso de bienes comunes a partir del cual podemos derivar la matriz de pagos asociada.
%
\en{If our causal model had a payoff structure isomorphic to the prisoner's dilemma, then we should observe that defectors have a higher growth rate than cooperators. }%
\es{Si nuestro modelo causal tuviera una estrucutra de pagos isomorfa al dilema del prisioner, entonces deberíamos observar que los desertores tienen una tasa de crecimiento mayor a lo cooperadores. }%
%
\en{What is found, however, is that the first defector from an entirely cooperative group obtains a lower growth rate than before defecting. }%
\es{Sin embargo, lo que se encuentra es que el primer desertor de un grupo enteramente cooperadora obtiene una tasa de crecimiento menor a la que tenía antes de desertar. }% 
%
\en{In the figure~\ref{fig:ergodicity_desertion} we can observe the trajectories of the resources, equivalent to the proportional posterior (see previous section), of the individuals that are in a group of size 100. }%
\es{En la figura~\ref{fig:ergodicity_desertion} podemos observar las trayectoria de los recursos, equivalente el posterior proporcional (ver sección anterior), de los individuos que están en un grupos de tamaño 100. }%
%
\begin{figure}[ht!]
    \centering
    \begin{subfigure}[b]{0.5\textwidth}
    \includegraphics[width=\linewidth]{figures/pdf/ergodicity_desertion.pdf}
    \end{subfigure}
    \caption{
    \en{The colors represent groups of size 100 with 0, 1 and 2 defectors. }%
    \es{Los colores representan los grupos de tamaño 100 con 0, 1 y 2 desertores. }%
    %
    \en{The curves of the individual defectors are those at the top of each of the groups. }%
    \es{Las curvas de los individuos desertores son las que están arriba en cada uno de los grupos. }%
    }
    \label{fig:ergodicity_desertion}
\end{figure}
%
\en{The resources of the first individual defector (blue group with 1 defector), is below the resources of the individuals in the fully cooperative group (green group with 0 defectors). }%
\es{Los recursos del primer individuo desertor (grupo azul con 1 desertor), está por debajo de los recursos de los individuos del grupo enteramente cooperador (grupo verde con 0 desertores). }%
%
\en{The reduction in resources occurs even for the second individual who changes from cooperative to a defective behavior: the resources of the defector individuals in the group with 2 defectors are below the resources of the cooperative individuals in the group with 1 defector. }%
\es{La reducción de recursos le ocurre incluso al segundo individuo que cambia de comportamiento cooperador a desertor: los recursos de los individuos desertores del grupo con 2 desertores está por debajo de los recursos de los individuos cooperadores del grupo con 1 desertor. }%

% Parrafo

\en{Defecting, instead of increasing the growth rate of the defective sindividuals, reduces it. }%
\es{Desertar, en vez de aumentar la tasa de crecimiento de los individuos desertores, la reduce. }%
%
\en{In other words, the commons does not have the structure of the prisoner's dilemma, as is usually claimed in the literature. }%
\es{Es decir, los bienes comunes no tiene la estructura del dilema del prisionero como habitualmente se afirma en la literatura. }%
%
\en{Let us calculate the payoff matrix arising from the cooperative causal model. }%
\es{Calculemos la matriz de pagos que surge del modelo causal cooperativo. }%
%
\en{We want to estimate the temporal growth rate of the posteriors, }%
\es{Queremos estimar la tasa de crecimiento temporal de los posteriors, }%
%
\begin{equation}
\frac{P(k|\vec{\Aa}^{\,1}, \dots , \vec{\Aa}^{\,T})}{P(k)} \approx g(k|\vec{\Ee},p,N)^T
\end{equation}
%
\en{where the approximation is an equality when time tends to infinity, $\lim_{T \rightarrow \infty}$. }%
\es{donde la aproximación es una igualdad cuando el tiempo tiende a infinito, $\lim_{T \rightarrow \infty}$. }%
%
\en{The growth rate depends on the hyperparameters of the model, $\vec{\Ee}$, $p$ and $N$. }%
\es{La tasa de crecimiento depende de los hyperparámtros del modelo, $\vec{\Ee}$, $p$ y $N$. }%
%
\en{Here we will only consider models in which all individuals have the same strategy $\Ee$. }
\es{Aquí vamos a considerar únicamente los modelos en el que todos los individuos tienen la misma estrategia $\Ee$. }%
%
\en{We will solve this problem by cases, depending on whether the behavior is cooperative, $g_C$, or defective, $g_D$, for different social contexts $g_{[\cdot]}^n(k|\cdot)$, where $n$ represents the number of cooperators with which the individual $k$ interacts (excluding $k$). }%
\es{Resolveremos este problema por casos, según el comportamiento sea cooperativo, $g_C$, o desertivos, $g_D$, para diferente contextos sociales $g_{[\cdot]}^n(k|\cdot)$, donde $n$ representa la cantidad de cooperadores con lo que interactúa el individuo $k$ (excluyendo a $k$). }%
% 
\en{The growth rate of mutual defection is, }%
\es{La tasa de crecimiento de la deserción mutua es, }%
%
\begin{equation}
\begin{split}
\lim_{T \rightarrow \infty} g_D^0(k|\Ee,p,N)^T &= \prod_{t=1}^T P(k|\vec{\Aa}^{\,T+1}) \propto \prod_{t=1}^T \Ee^{\Aa^t_k} (1-\Ee)^{1-\Aa^t_k} \\
\lim_{T \rightarrow \infty} g_D^0(k|\Ee,p,N) &\propto \Big( \prod_{t=1}^T \Ee^{\Aa^t_k} (1-\Ee)^{1-\Aa^t_k} \Big) ^{1/T} = \Ee^{p} (1-\Ee) ^{1-p}
\end{split}
\end{equation}
%
\en{which is independent of the size $N$. }%
\es{que es independiete del tamaño $N$. }%
%
\en{The proportional is valid because of the normalization constant of the probability distribution $P(i^t|\vec{\Aa}^{\,t})$. }%
\es{El proporcional vale por constante de normalización de la distribución de probabilidad $ P(i^t|\vec{\Aa}^{\,t})$. }%
%
\en{The growth rate of a strategy $\Ee= 0.71 \approx 1.5/2.1$ in an environment with $p=0.5$ is $ 0.71^{1/2}\cdot0.29^{1/2} \approx 0.452$. }%
\es{La tasa de crecimiento de un estrategia $\Ee= 0.71 \approx 1.5/2.1$ en una ambiente con $p=0.5$ es $ 0.71^{1/2}\cdot0.29^{1/2} \approx 0.452$. }%

% Parrafo

\en{The growth rate of the cooperators can also be computed using the geometic average, }
\es{La tasa de crecimiento de los cooperadores también la podemos calcular utilizando la media geomética, }%
%
\begin{equation}
\begin{split}
\lim_{T \rightarrow \infty} g_C^n(k|\Ee,p,N)^T &= \prod_{t=1}^{T} \sum_j^{\texttt{\en{partners}\es{socios}}(k)} \frac{1}{N} P(j|\vec{\Aa}^{\,t}) \propto \prod_{t=1}^{T}  \sum_j^{\texttt{\en{partners}\es{socios}}(k)} \frac{1}{N} \Ee^{\Aa^t_j} (1-\Ee)^{1-\Aa^t_j} \\
\lim_{T \rightarrow \infty} g_C^n(k|\Ee,p,N) & \propto \Big( \prod_{t=1}^{T}  \sum_j^{\texttt{\en{partners}\es{socios}}(k)} \frac{1}{N} \Ee^{\Aa^t_j} (1-\Ee)^{1-\Aa^t_j} \Big)^{1/T} = \prod_{x=0}^{n+1} \Big(\frac{x}{N} \Ee + \frac{n+1-x}{N}(1-\Ee)\Big)^{\text{B}(x|n+1,p)}
\end{split}
\end{equation}
%
\en{where $x$ represents the number of successes within the cooperating group, and $\text{B}(x|n+1,p)$ is the binomial probability of obtaining $x$ successes in a sample of size $n+1$. }%
\es{donde $x$ representa la cantidad de éxitos dentro del grupo cooperador, y $\text{B}(x|n+1,p)$ es la probabilidad binomial de obtener $x$ éxitos en una muestra de tamaño $n+1$. }%
%
% %
% \begin{equation}
% f_C(\Ee,\vec{\Aa}\,) =
% \begin{cases}
% (1-\Ee) & \text{ si } \texttt{sum}(\vec{\Aa}\,) = 0 \\
% \frac{1}{3} \Ee + \frac{2}{3} (1-\Ee)  & \text{ si } \texttt{sum}(\vec{\Aa}\,) = 1 \\
% \frac{2}{3} \Ee + \frac{1}{3} (1-\Ee)    & \text{ si } \texttt{sum}(\vec{\Aa}\,) = 2 \\
% \Ee & \text{ si } \texttt{sum}(\vec{\Aa}\,) = 3
% \end{cases}
% \end{equation}
% %
% Y en general, 
% \begin{equation}\label{eq:fitness_cooperador}
% f_{C}(\Ee, \vec{\Aa}\,,N) = \frac{\texttt{sum}(\vec{\Aa}\,)}{N} \Ee + \frac{N-\texttt{sum}(\vec{\Aa}\,)}{N}(1-\Ee)
% \end{equation}
% %
% \en{where $f_C(\Ee, \vec{\Aa}\,,N) \sim \text{Binomial}(\texttt{sum}(\vec{\Aa}\,)|N,p)$, with $p$ the probability of the environment. }%
% \es{donde $f_C(\Ee, \vec{\Aa}\,,N) \sim \text{Binomial}(\texttt{sum}(\vec{\Aa}\,)|N,p)$, con $p$ la probabilidad del ambiente. }%
% %
% \en{Then, the growth rate of the mutual cooperation can be computed using the geometic average~\ref{eq:geometric_mean}, }
% \es{Luego, la tasa de crecimiento de la mutua cooperación la podemos calcular utilizando la media geomética~\ref{eq:geometric_mean}, }%
% %
% \begin{equation}
% g_{C}(\Ee,N) = \prod_{x=0}^N \Big(\frac{x}{N} \Ee + \frac{N-x}{N}(1-\Ee)\Big)^{\text{Binomial}(x|N,p)}
% \end{equation}
%
\en{Then, the cooperative growth in a group of size 2 are $g_C^2(k|\Ee=0.71,p=0.5,N=2) \propto 0.71^{1/4}\cdot 0.5^{1/2} \cdot 0.29^{1/4} \approx 0.475$ and $g_C^1(k|\Ee=0.71,p=0.5,N=2) \propto (\frac{0.71}{2})^{1/4}\cdot (\frac{0.29}{2})^{1/2}  \approx 0.226$ }%
\es{Luego, la tasa de crecimiento cooperativa en grupos de tamaño $2$ son $g_C^2(k|\Ee=0.71,p=0.5,N=2) \propto 0.71^{1/4}\cdot 0.5^{1/2} \cdot 0.29^{1/4} \approx 0.475$ y $g_C^1(k|\Ee=0.71,p=0.5,N=2) \propto (\frac{0.71}{2})^{1/4}\cdot (\frac{0.29}{2})^{1/2}  \approx 0.226$ . }%
%
\en{When the group is very large, $N\rightarrow \infty$, }%
\es{Cuando el grupo es muy grande, $N\rightarrow \infty$, }%
%
\begin{equation}
\lim_{N \rightarrow \infty} g_C^n(k|\Ee,p,N) \propto \Big( p \Ee + (1-p)(1-\Ee) \Big)  \frac{n+1}{N}
\end{equation}
%
\en{the growth rate is just the arithmetic mean, weighed by the proportion of cooperators. }%
\es{la tasa de crecimiento es la media aritmética pesada por la proporción de cooperadores. }%
% %
% \en{The cooperative growth rate of a strategy $\Ee=0.71$ in a group of infinite size in an environment with probability $p=0.5$ is $0.5\cdot 0.71 + 0.5 \cdot 0.29 = 0.5$. }%
% \es{La tasa de crecimiento cooperativa de una estrategia $\Ee=0.71$ en un grupo de tamaño infinito en un ambiente con probabilidad $p=0.5$, es $0.5\cdot 0.71 + 0.5 \cdot 0.29 = 0.5$. }%

% Parrafo

\en{The growth rate of the defectos in goups with at least one cooperator is, }
\es{La tasa de crecimiento de desertores en grupos con al menos un cooperador es, }%
%
\begin{equation}
\begin{split}
\lim_{T \rightarrow \infty} g_D^n(k|\Ee,p,N)^T &= \Big(\prod_{t=1}^{T} P(k|\vec{\Aa}^{\,t}) \Big) + \Big(\sum_{t=1}^{T} \prod_{q=1}^{t} \sum_j^{\texttt{\en{partners}\es{socios}}(k)} \frac{1}{N} P(j|\vec{\Aa}^{\,q})  \prod_{q=t+1}^T P(k|\vec{\Aa}^{\,q}) \Big) \\
& \approx  g_D^0(k|\Ee,p,N)^T + \sum^{T}_{t=1}  g_C^{n-1}(k|\Ee,p,N)^t g_D^0(k|\Ee,p,N)^{T-t} \\
& \propto (\Ee^{p} (1-\Ee) ^{1-p})^T + \sum^{T}_{t=1} \Big(\prod_{x=0}^n(\frac{x}{N} \Ee + \frac{n-x}{N}(1-\Ee))^{\text{B}(x|n,p)}\Big)^t (\Ee^{p} (1-\Ee) ^{1-p})^{T-t}
\end{split}
\end{equation}
%
\en{the sum of individual growth plus a moving average of the growth of cooperators weighted by individual growth,  which we approximate using the growth rates $g_D^0$ and $g_C^n$. }%
\es{la suma del crecimiento individual más una media movil del crecimiento de los cooperadores pesado por crecimienti individual, que aproximamos usando las tasas de crecimiento $g_D^0$ y $g_C^n$. }%
% %
% \en{Because we did not find a simpler analytical solution, we performed the growth rate estimation as the difference between time steps, }%
% \es{Debido a que no encontramos una solución analítica más simple, realizamos la estimación de la tasa de crecimiento como la diferencia entre pasos temporales, }%
% %
% \begin{equation}
% g_{D}(\Ee,n,N) = 
% \end{equation}
%
\en{This growth rate is not constant, but quickly stabilizes at the higher growth rate, $g_D^n(k|\Ee,p,N)$ $\approx$ $\texttt{max}(g_D^0(k|\Ee,p,N),$ $g_C^n(k|\Ee,p,N))$. }%
\es{Esta tasa de crecimiento no es constante, pero se estabiliza rápidamente de pasos temporales en la tasa de crecimiento que sea mayor, $g_D^n(k|\Ee,p,N)$ $\approx$ $\texttt{max}(g_D^0(k|\Ee,p,N),$ $g_C^n(k|\Ee,p,N))$. }%
%
\en{In figure \ref{fig:multilevel-selection-7} we show the proprotional growth rate as a function of the number of defectors in a group of size $1000$. }%
\es{En la figura \ref{fig:multilevel-selection-7} mostramos el proporcional de la tasa de crecimiento en función del número de desertores totales en una grupo de tamaño $1000$. }%
%
\en{In Figure~\ref{fig:multilevel-selection-5} we rescale the proportional growth rate by a factor $2.1$, and we see that it overlaps over the trajectories of resource presented in Figure~\ref{fig:ergodicity_desertion}. }
\es{En la figura~\ref{fig:multilevel-selection-5} reesclamos la tasa de crecimiento proporcional por un factor $2.1$, y vemos que se solapa sobre las trayectorias de los recursos presentado en la figura \ref{fig:ergodicity_desertion}. }%
%
\begin{figure}[H]
    \centering
    \begin{subfigure}[b]{0.48\textwidth}
    \includegraphics[width=\linewidth]{figures/pdf/multilevel-selection-7.pdf}
    \caption{}
    \label{fig:multilevel-selection-7}
    \end{subfigure}
    \begin{subfigure}[b]{0.48\textwidth}
    \includegraphics[width=\linewidth]{figures/pdf/multilevel-selection-5.pdf}
    \caption{}
    \label{fig:multilevel-selection-5}
    \end{subfigure}
    \caption{
    \en{Figure \ref{fig:multilevel-selection-7}: estimation of proportional growth rate in mixed groups of size 100. }%
    \es{Figura \ref{fig:multilevel-selection-7}: estimación de la tasa de crecimiento proporcional en grupos mixtos de tamaño 100. }%
    %
    \en{Figure \ref{fig:multilevel-selection-5}: trajectory of the resources of a defector (blue) and cooperator (green) in a group of size 100 with a single defector (see figure~\ref{fig:ergodicity_desertion}), the black curves are the estimates growth rates, and the dotted black curve are the resources of cooperators in regions without defectors. }%
    \es{Figura \ref{fig:multilevel-selection-5}: trayectoria de los recursos de un desertor (azul) y cooperador (verde) en un grupo de tamaño 100 con un único desertor (ver figura~\ref{fig:ergodicity_desertion}), las curvas negras son las estimaciones de las tasas de crecimiento, y la curva negra punteadas son los recursos de mutua cooperación. }%
    }
    \label{fig:growth_rate_defector_mixed}
\end{figure}
% 
% \en{In Figure \ref{fig:multilevel-selection-5} we see that from the growth rates of cooperators $g_C(\Ee=1.5/2.1,n=99,N=100)$ and defectors $g_D(\Ee=1.5/2.1)$, we can correctly estimate the resource trajectory of a defector individual in a mixed population. }%
% \es{En la figura \ref{fig:multilevel-selection-5} vemos que a partir de las tasas de crecimiento de los individuos cooperdores $g_C(\Ee=1.5/2.1,n=99,N=100)$ y desertores $g_D(\Ee=1.5/2.1)$, podemos estimar correctamente la trayectoria de los recursos de un individuo desertor en una población mixta. }%
%
\en{Note that the growth rate of the individual defector is higher than that of the cooperators only in the first few time steps, which places the defectors in a better relative position. }%
\es{Notar que la tasa de crecimiento del individuo desertor es mayor a la de los cooperadores sólo en las primeros pasos temporales, lo que lo coloca a los desertores en una posición relativa mejor. }%
%
\en{But regardless of the size of the group, mutual cooperation always offers the highest growth rate, and the first defection always produces a drop in all growth rate, reducing even the growth rate of the individual defector. }%
\es{Pero, no importa el tamaño del grupo, siempre la mutua cooperación ofrece la tasa de crecimiento más alta, y la primera deserción produce una baja de la tasa de crecimiento incluso del individuo desertor. }%
%
\en{The following matrices summarize the growth rates of cooperators and defectors (rows), for different numbers of defectors in the social context (columns), in groups of size 2 (left) and size 16 (right). }%
\es{Las siguientes matrices resumen las tasas de crecimiento para individuos cooperadores y desertores (filas), para diferente cantidad de desertores en el contexto social (columnas), en grupos de tamaño 2 (izquierda) y tamaño 16 (derecha). }%
%
\begin{equation*}
\begin{split}
\ \ \ \ \ \ \ g_{[\cdot]}^n(k|\Ee=0.71, \, p=0.5, \, N=2)  & \ \ \ \ \ \ \ \  \ \ \ \ \ \ \ \ \ \ \ \ \ \ \ g_{[\cdot]}^n(k|\Ee=0.71, \, p=0.5, \, N=16) \\[0.1cm] 
 \propto \bordermatrix{ & n=1 & n=0 \cr
      C & 0.475 & 0.226 \cr
      D & 0.452 &  0.452 }\ \ \ \ \ \ \ \  &  \ \ \ \ \ \ \ \ \ \ \  \propto \bordermatrix{ & n=15 & n=14 & n=13 & n=12 \cr
      C & 0.497 & 0.466 & 0.435 & 0.403 \cr
      D & 0.466 & 0.452 & 0.452 & 0.452}       
\end{split}
\end{equation*}
%
\en{The first agent that unilaterally ``decides'' to defect will reduce its own growth rate. }%
\es{El primer agente que unilateralmente ``decida'' desertar va a reducir su propia tasa de crecimiento. }%
%
\en{Similarly, when all (or many) of the group members are defectors, the first agent who unilaterally ``decides'' to cooperate will also reduce his own growth rate. }%
\es{De modo similar, cuando todos (o muchos) de los miembros del grupo son desertores, el primer agente que unilateralmente ``decida'' cooperar , también reducirá su propia tasa de crecimiento. }%
%
\en{This means that the payoff matrix is not isomorphic to a prisoner's dilemma. }%
\es{Esto significa que la matriz de pagos no son isomorfas al dilema del prisionero. }%
%
\en{In fact, the payoff structure of the left-hand matrix is known as stag-hunt. }%
\es{De hecho, la estructura de pagos de la matriz izquierda se conoce como stag-hunt. }%
%
\begin{conclution}[\en{Commons are not prisoner's dilemmas}\es{Los bienes comunes no son dilemas del prisionero}]
\en{Without penalties, defector strategies negatively affect their own long-term growth rate because their own behavior increases the fluctuations of the random variable on which they depend. }%
\es{Sin castigos, las estrategias desertoras afectan negativamente su propia tasa de crecimiento a largo plazo debido a que su propio comportamiento aumenta las fluctuaciones de la variable aleatoria de la que dependen.}
\end{conclution}
%
\en{In all cases, the highest growth rate is obtained by mutual cooperation. }%
\es{En todos los casos, la tasa de crecimiento más alta se obtiene por mutua cooperación. }%
% %
% \en{For this reason, multilevel selection will favors cooperators over defectors, even though the presence of defectors negatively affects their growth rate. }%
% \es{Por este motivo, la selección multinivel favorecerá a los individuos cooperadores sobre los desertores, a pesar de que la presencia de destores afecte negativa su tasa de crecimiento. }%


\subsection{Selección de nivel 1, nivel 2 y multinivel}

\en{In the previous section we have seen that unilateral defection reduces the growth rate of the defectors themselves compared to what they could have through mutual cooperation. }%
\es{En la sección anterior hemos visto que desertar unilateralmente reduce la tasa de crecimiento de los proprios desertores respecto de la que podrían tener a través de la mutua cooperación. }%
%
\en{However, the growth rate of cooperators is reduced to a greater extent. This means that the defectors always have a better relative position than the cooperators in their own group. }%
\es{Sin embargo, la tasa de crecimiento de los cooperadores se reduce en mayor medida. Esto hace que los desertores siempre tengan una posición relativa mejor que los cooperadores de su propio grupo. }%
%
\en{Therefore, evolution will favor defector behaviors through individual selection (level 1). }%
\es{Por lo tanto, la evolución favorecerá los comportamientos desertores a través de la selección individual (nivel 1). }%
%
\en{In Figure~\ref{fig:multilevel-selection-level-1-posterior} we can see the posterior of cooperating and deserting individuals in regions with 1 defector in groups of size 2 and 16. }%
\es{En la figura~\ref{fig:multilevel-selection-level-1-posterior} podemos ver el posterior los individuos cooperadores y desertores en regiones con 1 destertor en grupos de tamaño 2 y 16. }%
%
\begin{figure}[H]
    \centering
    \begin{subfigure}[b]{0.48\textwidth}
    \includegraphics[width=\linewidth]{figures/pdf/multilevel-selection-level-1-posterior.pdf}
    \caption{$N=2$}
    \end{subfigure}
    \begin{subfigure}[b]{0.48\textwidth}
    \includegraphics[width=\linewidth]{figures/pdf/multilevel-selection-level-1-posterior-N16.pdf}
    \caption{$N=16$}
    \end{subfigure}
    \caption{
    \en{Posterior of cooperator/defector social behaviors within a region with 1 defector in groups of size $2$ and $16$. }%
    \es{Posterior de los comportamientos sociales cooperador/desertor dentro de una región con 1 desertor en grupos de tamaño $2$ y $16$. }%
    }
    \label{fig:multilevel-selection-level-1-posterior}
\end{figure}
%
\en{Defector behaviors can invade within groups of size 2, as the posterior of the defectors quickly stabilizes at 1. }%
\es{Los comportamientos desertores pueden invadir al interior de los grupos de tamaño 2, pues el posterior de los desertores rápidamente se estabiliza en 1. }%
%
\en{However, defector behaviors cannot invade within groups of size 16, as the posterior of both behaviors never stabilizes at 0 and 1. }%
\es{Sin embargo, los comportamientos desertores no pueden invadir al interior de grupos de tamaño 16, pues el posterior de ambos comportamientos nunca se estabiliza en 0 y 1. }%

% Parrafo


\en{Although defectors may invade size 2 groups, regions that remain fully cooperative will have a great advantage over mixed regions because the growth rate of mutual cooperation is always the highest one. }%
\es{A pesar de que los desertores puedan invadir grupos de tamaño 2, las regiones que persistan enteramente cooperadoras tendrán una gran ventaja sobre las regiones mixtas gracias a que la tasa de crecimiento de la mutua cooperación es siempre mayor que el resto. }%
%
\en{Therefore, evolution will favor fully cooperative groups through group selection (level 2). }%
\es{Por lo tanto, la evolución favorecerá a los grupos enteramente cooperadores a través de la selección de grupos (nivel 2). }%
%
\en{In figure~\ref{fig:posterior_level_2} we see the posterior of the groups, of size 2 and 16. }%
\es{En la figura~\ref{fig:posterior_level_2} vemos el posterior de los grupos de tamaño 2 y 16. }%
% %
% \begin{equation}
% P(g|\vec{\Aa}^{\,1}, \dots, \vec{\Aa}^T) = \sum_i \mathbb{I}(\texttt{region}(i)=g) P(i|\vec{\Aa}^{\,1}, \dots, \vec{\Aa}^T)
% \end{equation}
% %
\begin{figure}[H]
    \centering
    \begin{subfigure}[b]{0.48\textwidth}
    \includegraphics[width=\linewidth]{figures/pdf/multilevel-selection-6.pdf}
    \caption{$N=2$}
    \label{fig:multilevel-selection-6}
    \end{subfigure}
    \begin{subfigure}[b]{0.48\textwidth}
    \includegraphics[width=\linewidth]{figures/pdf/multilevel-selection-level-2-N16.pdf}
    \caption{$N=16$}
    \label{fig:multilevel-selection-level-2-N16}
    \end{subfigure}
    \caption{
    \en{Group selection (level 2) of size $2$ and of size $16$. }%
    \es{Selección de grupos (nivel 2) entre grupos de tamaño $2$ y tamaño $16$. }%
    %
    \en{The gray lines represent the posterior of mixed groups. }%
    \es{Las rectas grises representan el posterior de grupos mixtos. }%
    }
    \label{fig:posterior_level_2}
\end{figure}
%
\en{Note that the prior of the entirely cooperative group is $0.25$ in groups of size 2 and approximately $ \text{B}(0|N=16,0.5) \approx 0$ for groups of size $16$. }
\es{Notar que el prior del grupo enteramente cooperador es $0.25$ en grupos de tamaño 2 y aproximadamente $ \text{Binomial}(0|N=16,0.5) \approx 0$ para grupos de tamaño $16$. }%
%
\en{The choice of a prior that rejects homogeneous populations causes the advantage of the cooperating group to take time to stabilize. }%
\es{La elección de un prior que rechaza poblaciones homogenes hace que la ventaja del grupo cooperador necesite un tiempo hasta estabilizarse. }%
%
\en{With a uniform prior, the advantage of the fully cooperative group would be seen immediately. }%
\es{Si el prior fuera uniforme, la ventaja del grupo enteramente cooperador se vería inmediatamente. }%
%
\en{In any case, since the prior of homogeneous populations is never zero and the growth rate of the fully cooperative population is higher than the rest, there is always a time $t$ in which the posterior of the fully cooperative group will be higher than that of the rest of the groups. }%
\es{En cualquier caso, debido a que el prior de las poblaciones homogeneas nunca es cero y que la tasa de crecimiento de la población enteramenete cooperadora sea superior al resto, siempre existe un tiempo $t$ en el cual el posterior del grupo enteramente cooperador será mayor al del resto de los grupos. }%
%
\en{The larger $N$ is, the closer the cooperators are to the arithmetic mean, but the more weight individuals from mixed regions receive. }%
\es{Cuanto más grande es $N$, más cerca están los cooperadores de la media aritmética, pero más peso reciben los individuos de regiones mixtas. }%
%
\en{Given a maximum time, the optimal group size will always be finite, which is reasonable in evolutionary terms. }
\es{Dada un tiempo máximo, el tamaño óptimo del grupo será siempre finito, lo que es razonable en términos evolutivos. }%

% Parrafo

% Si la y de los comportamientos cooperadores (multinivel),
% %
% \begin{equation}
% P(\texttt{coop}|\vec{\Aa}^{\,1}, \dots, \vec{\Aa}^T) = \sum_i P(i|\vec{\Aa}^{\,1}, \dots, \vec{\Aa}^T)\mathbb{I}(\texttt{coop}(i))
% \end{equation}
% %
\en{When there is level 2 selection in favor of fully cooperative groups, there is also multilevel selection in favor of cooperative individuals. }%
\es{Cuando se produce una selección de nivel 2 a favor de los grupos enteramenete cooperadores, se produce también una selección multinivel a favor de los individuos cooperadores. }%
%
\en{In Figure~\ref{fig:multilevel-selection-multilevel-posterior} we see the posterior of the cooperative individuals, integrating all groups. }%
\es{En la figura~\ref{fig:multilevel-selection-multilevel-posterior} vemos el posterior de los individuos cooperadores, integrando todos los grupos. }%
%
\begin{figure}[H]
    \centering
    \begin{subfigure}[b]{0.48\textwidth}
    \includegraphics[width=\linewidth]{figures/pdf/multilevel-selection-multilevel-posterior.pdf}
    \caption{$N=2$}
    \end{subfigure}
    \begin{subfigure}[b]{0.48\textwidth}
    \includegraphics[width=\linewidth]{figures/pdf/multilevel-selection-multilevel-posterior-N16.pdf}
    \caption{$N=16$}
    \end{subfigure}
    \caption{
    \en{Multilevel selection of cooperative individuals in groups of size $2$ and $16$. }%
    \es{Selección multinivel del los individuos cooperadores en grupos de tamaño $2$ y $16$. }%
    }
    \label{fig:multilevel-selection-multilevel-posterior}
\end{figure}
%
\en{The probability of cooperative individuals starts at $0.5$ due to the symmetry of the binomial prior between regions and the uniform prior between cooperative and defective behaviors. }%
\es{La probabilidad de los individuos cooperadores comienza a $0.5$ debido a la simetría del prior binomial entre regiones y el prior uniforme entre comportamientos cooperador y desertor. }%
%
\en{Because in most mixed regions the posterior of cooperative individuals drops abruptly, we see at the beginning a drop in the multilevel posterior. }%s
\es{Debido a que en la mayoría de las regiones mixtas el posterior de los individuos cooperdores cae abtruptamente, vemos al principio una baja en el posterior multinivel. }%
%
\en{But since the advantage of the fully cooperative group is imposed after a certain time (delay produced by the binomial prior), we finally observe that cooperative behaviors can invade populations with defectors as the posterior multilevel stabilizes at 1. }%
\es{Pero como la ventaja del grupo enteramente cooperador se impone luego de cierto tiempo (demora producida por el priori binomial), finalmente observamos que los comportamientos cooperadores pueden invadir poblaciones con desertores pues el posterior multinivel se estabiliza en 1. }%
%
\begin{conclution}[\en{The evolutionary advantage of cooperation}\es{La ventaja evolutiva de la cooperación}]
\en{Multilevel selection favors cooperative strategies even with groups of minimum size (two). }%
\es{La selección multinivel favorece a las estrategias cooperativas incluso con grupos de tamaño mínimo (dos). }%
\end{conclution}

\subsection{La ventaja de la especialización}

\en{To explain evolutionary transitions, it is necessary to demonstrate the evolutionary advantage of cooperation in the presence of defection, but also the advantage of specialization. }%
\es{Para explicar las transiciones evolutivas es necesario demostrar la ventaja evolutiva de la cooperación en presencia de deserción, pero también la ventaja de la especiliazación. }%
%
\en{Generalist strategies are those that achieve similar returns in each of the environmental states. }%
\es{Las estrategias generalistas son las que logran retornos similares en cada uno de los estados ambientales. }%
%
\en{The most extreme case is the strategy $\Ee = 0.5$, which has the same individual growth rate $g(k|\Ee=0.5,p,N=1) \propto 0.5$ irrespective of the type of environment $p$. }%
\es{El caso más extremo es la estrategia $\Ee = 0.5$, que tiene la misma tasa de crecimiento individual $g(k|\Ee=0.5,p,N=1) \propto 0.5$ indistintamente del tipo de ambiente $p$. }%
%
\en{Specialist strategies are those that have high returns in one of the environmental states and high losses in the other. }%
\es{Las estrategias especialistas son las que tienen altos retornos en uno de los estados ambientales y altas pérdidas en el otro. }%
%
\en{The extreme case is the strategy $\Ee = 1.0$, unfeasible in stochastic environments because its individual growth rate is $g(k|\Ee=1.0,p,N=1) = 0$ when $p\neq0$. }%
\es{El caso extremos es la estrategia $\Ee = 1.0$, inviable en ambientes estocásticos debido a que su tasa de crecimiento individual es $g(k|\Ee=1.0,p,N=1) = 0$ cuando $p\neq0$. }%

% Parrafo

\en{We have seen in the methodological section that the individual strategy best adapted to the environment $p=0.71$ was $\Ee^*=0.71$. }%
\es{Hemos visto en la sección metodológica que la estrategia individual mejor adaptada al ambiente $p=0.71$ fue $\Ee^*=0.71$. }%
%
\en{In general the best adapted strategy is $\Ee^*=p$. }%
\es{En general la estrategia individual mejor adaptada es $\Ee^*=p$. }%
%
\en{Now that we know that there is an evolutionary advantage in favor of cooperation, is there another strategy that is better adapted to the environment? }%
\es{Ahora que sabemos que existe una ventaja evolutiva a favor de la cooperación, ¿hay una estrategia mejor adaptada al ambiente? }%
%
\en{If there were an advantage in favor of specialization we would expect to see that if the probability of the environment is biased toward one of the states, then the optimal strategy is biased even more, $\Ee^* > p > 0.5$ or $\Ee^* < p < 0.5$. }%
\es{Si hubiera una ventaja a favor de la especialización esperaríamos ver que si la probabilidad del ambiente está sesgada hacia uno de los estados, entonces la estrategia óptima esté sesgada aún más, $\Ee^* > p > 0.5$ o $\Ee^* < p < 0.5$. }%

% Parrafo

\en{In figure~\ref{fig:tasa-temporal-0} we compute the individual (solid lines) and cooperative (dashed line) growth rate of the strategies $\Ee \in \{0.5, 0.71, 0.99\}$ at all possible $p$ values of the environment. }%
\es{En la figura~\ref{fig:tasa-temporal-0} calculamos la tasa de crecimiento individual (líneas continuas) y cooperativa (línea punteada) de las estrategias $\Ee \in \{0.5, 0.71, 0.99\}$ para todos los posibles valores $p$ del ambiente. }%
%
\en{Note that the individual growth rates of the strategies are proportional to the geometric mean of their bets and that the cooperative growth rates are equivalent to the arithmetic mean of their bets (in groups of infinite size), and that both means are equal for the strategy $\Ee = 0.5$. }%
\es{Notar que las tasas de crecimiento individual de las estrategias son proporcionales a la media geométrica de sus apuestas y que las tasas de crecimiento cooperativo son equivalentes a la media aritmética de sus apuestas (en grupos de tamaño infinito), y que ambas medias son iguales para la estrategia $\Ee = 0.5$. }%
%
\begin{figure}[H]
    \centering
    \begin{subfigure}[b]{0.5\textwidth}
    \includegraphics[width=\linewidth]{figures/pdf/tasa-temporal-0.pdf}
    \end{subfigure}
    \caption{
    \en{Proportional of the individual and cooperative growth rates (continuous and dotted lines), of three strategies ($\Ee \in \{0.5, 0.71, 0.99\}$) in different environment $p$. }%
    \es{Proporcionales de las tasas de crecimient individual y cooperativa (líneas continuas y punteadas), de tres estrategias ($e \in \{0.5, 0.71, 0.99\}$) en diferentes ambiente $p$. }%
    }
    \label{fig:tasa-temporal-0}
\end{figure}
%
\en{The arrow represents Ole Peters' conclusion presented in the introduction: in an environment that generates states with probability $p = 0.5$, the strategy $\Ee=1.5/2.1 \approx 0.71$ can increase its individual growth rate, equal to the geometric mean, to a cooperative growth rate equal to its arithmetic mean. }%
\es{La flecha representa la conclusión de Ole Peters presentada en la introducción: en un ambiente que genera los estados con una probabilidad de $p = 0.5$, la estretgia $\Ee=1.5/2.1 \approx 0.71 $ puede aumentar su tasa de crecimiento individual, equivalente a la media geométrica, a una tasa de crecimiento cooperativa equvialente a su media aritmética. }%
%
\en{The red dot represents the conclusion we made in the methodology section, that in an environment with $p=0.71$ the best adapted individual strategy is $\Ee^*=0.71$. }%
\es{El punto rojo representa la conclusión que sacamos en la sección metodología, que en un ambiente con $p=0.71$ la estrategia individual mejor adaptada es $\Ee=0.71$. }%

% Parrafo 

\en{With the figure~\ref{fig:tasa-temporal-0} we can derive some new conclusions. }%
\es{Con la figura~\ref{fig:tasa-temporal-0} podemos sacar algunas conclusiones nuevas. }%
%
\en{Note that above the red dot is the cooperative growth rate of the specialist strategy $\Ee=0.99$. }%
\es{Notar que arriba del punto rojo se encuentra la tasa de crecimiento cooperativa de la estrategia especialista $\Ee=0.99$. }%
%
\es{This suggests that a strategy that is individually poorly adapted to the environment, as is the case of the specialist strategy $g(k|\Ee=0.99,p=0.71,N=1) < g(k|\Ee=0.71,p=0.71,N=1)$, achieves in cooperative groups a growth rate that is higher than the growth rate that the individually well adapted strategy achieves through cooperative groups, $g_C^{N-1}(k|\Ee=0.99,p=0.71,N=\infty) > g_C^{N-1}(k|\Ee=0.71,p=0.71,N=\infty)$. }%
\es{Esto sugiere que una estrategia que individualmente están mal adaptadas al ambiente, como es el caso de la estrategia especialista $g(k|\Ee=0.99,p=0.71,N=1) < g(k|\Ee=0.71,p=0.71,N=1)$, logra en grupos cooperativos una tasa de crecimiento que es mayor que las tasa de crecmiento que la estrategia individalmente bien adaptada logra a través de grupos cooperativos, $g_C^{N-1}(k|\Ee=0.99,p=0.71,N=\infty) > g_C^{N-1}(k|\Ee=0.71,p=0.71,N=\infty)$. }%

% Parrafo

\en{We have computed the cooperative growth rate for groups of infinite size. }%
\es{La tasa de crecimiento cooperativa la hemos calculado para grupos de tamaño infinito. }%
%
\en{To be an interesting conclusion for evolutionary theory we need this result to occur also in small groups. }%
\es{Para que sea una conclusión interesante en términos evolutivos, necesitamos que este mismo resultado ocurra en grupos finitos, particularmente pequeños. }%
%
\en{In Figure~\ref{fig:tasa-temporal-1} we plot the growth rates of the specialist strategy $\Ee=0.99$ for cooperative groups of size 1 to 5. }%
\es{En la figura~\ref{fig:tasa-temporal-1} graficamos las tasas de crecimiento de la estrategia especialista $\Ee=0.99$ para grupos cooperativos de tamaño 1 a 5. }%
%
\begin{figure}[H]
    \centering
    \begin{subfigure}[b]{0.5\textwidth}
    \includegraphics[width=\linewidth]{figures/pdf/tasa-temporal-1.pdf}
    \end{subfigure}
    \caption{
    \en{Proportional growth rate of the specialist strategy ($\Ee=0.99$) as a function of the probability of environment $p$, for cooperative groups of size 1 to 5. }%
    \es{Tasa de crecimiento proporcional de la estrategia especialista ($\Ee=0.99$) en función de la probabilidad del ambiente $p$, para grupos cooperativos de tamaño 1 a 5. }%
    %
    \en{The black dotted line represents the proportional growth rate of an infinitely large cooperative group. }%
    \es{La línea punteada negra represeta la tasa de crecimiento proporcional de un grupo cooperativo inifinitamente grande. }%
    %
    \en{The gray lines are visual references to the strategies $\Ee \in \{0.5, 0.71\}$ discussed in the previous figure. }%
    \es{Las rectas grises son referencias visuales de las estrategias $e \in \{0.5, 0.71\}$ analizadas en la figura anterior. }%
    }
    \label{fig:tasa-temporal-1}
\end{figure}
%
\en{Note that in an environment $p=0.71$ the specialist strategy $\Ee=0.99$ achieves in cooperative groups of size 3 a growth rate that is above the growth rate of the strategy individually well adapted to the environment $\Ee=0.71$, outperforming even the growth rate that the individually well-adapted strategy obtains in cooperative groups of infinite size, $g_C^{2}(k|\Ee=0.99,p=0.71,N=3) > g_C^{N-1}(k|\Ee=0.71,p=0.71,N=\infty)$! }%
\es{Notar que en un ambiente $p=0.71$ la estrategia especilista $\Ee=0.99$ logra en grupos cooperativos de tamaño 3 una tasa de crecimiento que está por arriba de la tasa de crecimiento de la estrategia individualmente bien adapatada al ambiente $\Ee = 0.71$, superando incluso la tasa de crecimiento que la estrategia individulamente bien adaptada obtiene en grupo cooperativos de tamaño infinito, $g_C^{2}(k|\Ee=0.99,p=0.71,N=3) > g_C^{N-1}(k|\Ee=0.71,p=0.71,N=\infty)$! }%
%
\en{It is extraordinary that the same basic assumption that offers an evolutionary advantage in favor of cooperation also offers an evolutionary advantage in favor of specialization, even in small groups. }%
\es{Es extraordinario que el mismo supuesto básico que ofrece una ventaja evolutiva a favor de la cooperación, ofrezca también una ventaja evolutiva a favor de la especilización incuso en grupos pequeños. }%
%
\en{The emergence of cooperation immediately produces an evolutionary advantage in favor of specialization. }%
\es{La emergencia de la cooperación produce inmediatamente una ventaja evolutiva a favor de la especialización. }%

% Parrafo

\en{The optimal level of specialization depends on the size of the groups. }%
\es{El nivel de especialización óptimo depende del tamaño de los grupos. }%
%
\en{In Figure~\ref{fig:tasa-temporal-2} we set the environment at $p = 0.71$ and analyze how the cooperative growth rate of all possible strategies varies in groups of sizes 1 to 5. }%
\es{En la figura~\ref{fig:tasa-temporal-2} fijamos el ambiente en $p = 0.71$ y analizamos como varía la tasa de crecimiento cooperativa de todas las posibles estrategias en grupos de tamaños 1 a 5. }%
%
\begin{figure}[H]
    \centering
    \begin{subfigure}[b]{0.5\textwidth}
    \includegraphics[width=\linewidth]{figures/pdf/tasa-temporal-2.pdf}
    \end{subfigure}
    \caption{
    \en{The proportional growth rate of all possible strategies for entirely cooperative groups of size 1 to 5, in an environment $p=0.71$. }%
    \es{La tasa de crecimiento proprocional de todas las posibles estrategias para grupos enteramenete cooperativos de tamaño 1 a 5, en un ambiente $p=0.71$. }%
    %
    \en{The dots indicate the optimum strategy in each of the sizes. }%
    \es{Los puntos indican la estretgia óptima en cada uno de los tamaños. }%
    %
    }
    \label{fig:tasa-temporal-2}
\end{figure}
%
\en{When the group has size 1, the best adapted strategy is the one that divides the bet in the same proportion as the probability of the environment $\Ee^*=p$. }%
\es{Cuando el grupo tiene tamaño 1, la estrategia mejor adaptada es la que divide la apuesta en la misma proproción que la probabilidad de el ambiente $\Ee^*=p$. }%
%
\en{But as soon as cooperation arises, an advantage in favor of the specialist strategies $\Ee^* > p = 0.71$ appears. }%
\es{Pero apenas surge la cooperación aparece una ventaja a favor de las estrategias especialistas $\Ee^* > p = 0.71$. }%
%
\en{The larger the groups, the more specialist the optimal strategy becomes. }%
\es{Cuanto más grande son los grupos, más especialista es vuelve la estrategia óptima. }%
%
\en{At the extremes (cooperative groups of infinite size) the optimal strategy is reached at the maximum level of specialization, in which all individuals bet all resources on the most frequent environmental state. }%
\es{En el extremos (grupos cooperativos de tamaño inifinito) la estrategia óptima se alcanza con el nivel de especialización máximo, en el que todos los individuos apuestan todos los recursos al estado ambiental más frecuente. }%

\begin{conclution}[\en{The advantage of specialization}\es{La ventaja de la especialización}]
\en{Cooperation offers an advantage in favor of specialist strategies in groups of minimum size (two). }%
\es{La cooperación ofrece una ventaja a favor de las estrategias especialistas en grupos de tamaño mínimo (dos). }%
%
\en{Strategies that individually are poorly adapted to the environment, cooperatively achieve better results than those obtained by cooperative groups of individually well-adapted strategies. }
\es{Estrategias que individualmente están mal adaptadas al ambiente, cooperando logran mejorar los resultados que obtienen grupos cooperativos de estrategias individualmemte bien adaptada al ambiente. }%
\end{conclution}

% 
% \en{Here we show that strategies that are individually poorly adapted to the environment (specialists), once cooperation emerges, manage to outperform both individually well-adapted strategies (generalists), as well as their cooperative groups of infinite size. }%
% \es{Aquí mostramos que, una vez que la cooperación emerge, las estrategias individualmente mal adaptadas al ambiente (especialistas) consigen superar tanto a las estrategias bien adaptas individualmemte (generalistas), como a sus grupos cooperativos de tamaño infinito. }%
% %

\section{Discusiones}

\en{In this paper we specify a probabilistic causal model, in which individuals are affected just by the environment and by the social behaviors of cooperation and defection of their context. }%
\es{En este trabajo definimos un modelo causal en el que los individuos se ven afectados solamente por el ambiente y por los comportamientos sociales de cooperación y deserción de su contexto. }%
%
\en{Under this minimal set of hypotheses, where we consider \emph{unconditionally} cooperative individuals who generate a common good that can be exploited by defecting individuals without receiving some kind of punishment in return (e.g. end of cooperation), the evolution of cooperation literature predicts defection as the only evolutionarily stable strategy. }%
\es{Bajo este conjunto mínimo de hipótesis, donde consideramos individuos \emph{incondicionalmente} cooperadores que generan un bien común que puede ser explotado por individuos desertores sin que reciban a cambio algún tipo de castigo (e.g fin de la cooperación), la literatura de evolución de la cooperación predice que la deserción es la única estrategia evolutivamente estable. }%
%
\en{It is assumed that common goods, if not accompanied by special conditions (such as communication allowing coordination, memory allowing rewards or punishments, etc.), lead to their overexploitation, because even if the optimum is obtained through mutual cooperation there would be an individual incentive to defect. }%
\es{Se supone que los bienes comunes, si no están acompañados de condiciones especiales (como la comunicación que permita la coordinación, la memoria que permita aplicar premios o castigos, etc), conducen a su sobreexplotación, porque aunque el óptimo se obtenga a través de la mutua cooperación habría una incentivo individual para desertar. }%
%
\en{This type of scenario is known in the social sciences as the ``tragedy of the commons''~\cite{hardin1971-collectiveAsPrisionerDilema}, an analogy that evolutionary biology has begun to adopt in recent decades~\cite{rankin2007-tragedyCommonsBiology}. }%
\es{Este tipo de escenarios se los conoce en ciencias sociales como ``tragedia de los comunes''~\cite{hardin1971-collectiveAsPrisionerDilema}, una analogía que la biología evolutiva ha comenzado a adoptar en las últimas décadas~\cite{rankin2007-tragedyCommonsBiology}. }%

% Parrafo

\en{This theoretical paradigm is forced to explain why cooperation is systematically observed in nature. }%
\es{El paradigma teórico de la ``tragedia de los comunes'' se ve obligado a explicar por qué la cooperación se observa sistemáticamente en las ciencias de la vida. }%
%
\en{In the last third of the universe's history, a simple self-replicating organization of matter emerged on earth. }%
\es{En el último tercio de la historia del universo surgió en la tierra una organización de la materia simple capaz de autoreplicarse. }%
%
\en{The errors produced during replication diversified the life forms, and the growth rates of the different strategies favored those better adapted to the environment. }%
\es{Los errores producidos durante la replicación diversificaron las formas de vida, y las tasas de crecimiento de las diferentes estrategias favorecieron a aquellas mejor adaptadas al ambiente. }%
%
\en{The current complexity of life is the consequence of a series of evolutionary transitions in which entities capable of self-replication after the transition become part of higher level cooperative units. }%
\es{La complejidad actual de la vida es consecuencia de una serie de transiciones evolutivas en las que entidades capaces de autoreplicación luego de la transición pasan a formar parte de unidades cooperativas de nivel superior. }%
%
\en{The theoretical paradigm of the ``tragedy of the commons'' makes it necessary to identify special conditions in each case to explain the observed tendency of life in favor of cooperative aggregation (and specialization). }%
\es{El paradigma teórico de la ``tragedia de los comunes'' obliga a identificar condiciones especiales en cada caso que expliquen la tendencia observada de la vida en favor de la agregación cooperativa (y la especialización). }%

% Parrafo

\en{However, the result of the probabilistic inference that emerges from the proposed causal model, without including any of these special conditions, reveals an advantage in favor of cooperation and specialization after individuals interact for a certain period of time. }%
\es{Sin embargo, el resultado de la inferencia probabilística que surge del modelo causal propuesto, sin incluir ninguna de estas condiciones especiales, revela una ventaja a favor de la cooperación y la especialización luego de que los individuos interactúan durante un cierto tiempo. }% 
%
\en{How can this counter-intuitive result emerge from such a simple model? }%
\es{¿Cómo se explica que emerga este resultado contra-intuitivo de un modelo tan simple? }%
%
\en{While our model only specifies that environments affect individuals proportionally to certain values, the rules of probability theory update the posterior of individuals through the product rule. }%
\es{Si bien nuestro modelo solamente especifica que los ambientes afectan a los individuos de forma proporcional ciertos valores, las reglas de la teoría de la probabilidad actualizan el posterior de los individuos a través de la regla del producto. }%
%
\en{Because in multiplicative processes the impacts of losses are greater than those of gains, fluctuations produce a negative effect on growth rates, which can be reduced through mutual cooperation. }%
\es{Debido a que en los procesos multiplicativos los impactos de las pérdidas son mayores a los de las ganancias, las fluctuaciones producen un efecto negativo en las tasas de crecimiento, las cuales pueden ser reducidas a través de la mutua cooperación. }%
%
\en{And then, as soon as cooperation emerges, an advantage in favor of specialist strategies appears, because it is no longer necessary for individuals to reduce fluctuations through generalist strategies that avoid bad outcomes in all possible states, and individuals can devote themselves to take advantage of the most frequent environmental state. }%
\es{Y luego, apenas surge la cooperación, aparece una ventaja en favor de las estrategias especialistas, porque deja de ser necesario para los individuos reduicir la fluctuaciones través de estrategias generalistas que evitan malos resultados en todos los posibles estados, y los indviduos pueden dedicarse a sacar provecho del estado ambiental más frecuente. }%

% Parrafo

\en{Even if our causal model does not assume any kind of process to update resources, the result of the inference is proportional to the resources obtained through a multiplicative process. }%
\es{Incluso si nuestro modelo causal no presupone ningún tipo de proceso para actualizar los recursos, el resultado de la inferencia es proporcional a los recursos obtenidos a través de un proceso multiplicativo. }%
%
\en{The multiplicative updating of the probabilities of individuals, which arises naturally from applying the rules of probability to the causal model, is in line with the long-established idea in evolutionary theory that the growth of lineages follow multiplicative processes~\cite{dempster1955-geometricMean, denBoer1968-spreadingRisk}. }%
\es{La actualización multiplicativa de las probabilidades de los individuos, que surge naturalmente de aplicar las reglas de la probabilidad al modelo causal, está en línea con la idea largamente establecida en la teoría de la evolución de que el crecimiento de los linajes siguen procesos multiplicativos~\cite{dempster1955-geometricMean, denBoer1968-spreadingRisk}. }% 
%
\en{This coincidence is an additional support to the hypothesis of isomorphism between evolutionary and probabilistic theory, previously identified between the fundamental equations of both theories: the Bayes theorem and the replicator dynamic~\cite{harper2009-replicatorAsInference,shalizi2009-replicatorAsInference}. }%
\es{Esta coincidecia es un apoyo adicional a la hipótesis de isomorfimo entre la teoría de la evolutivas y probabilisticas, previamente identificado entre las ecuaciones fundamentales de ambas teorías: el teorema de bayes y el replicator dynamic~\cite{harper2009-replicatorAsInference,shalizi2009-replicatorAsInference}. }%
%
%
\en{Based on this isomorphism, the co-author of the concept of evolutionary transitions (Szathmary~\cite{szathmary1995-evolutionaryTransitions, szathmary2015-evolutionaryTransitions}) recently proposed to analyze the evolution of populations subject to multilevel selection by means of Bayesian hierarchical models~\cite{czegel2019-bayesianEvolution}. }%
\es{Basados en este isomorfismo, el co-autor del concepto de transiciones evolutivas (Szathmary~\cite{szathmary1995-evolutionaryTransitions, szathmary2015-evolutionaryTransitions}) propuso recientemente analizar la evolución de las poblaciones sujetas a selección multinivel mediante modelos jerárquicos bayesianos~\cite{czegel2019-bayesianEvolution}. }%

% Parrafo

\en{Our work, to the best of our knowledge, would be the first to develop a Bayesian hierarchical model to solve an evolution problem under multilevel selection. }%
\es{Nuestro trabajo, hasta donde sabemos, sería el primero en desarrollar un modelo jerárquico bayesiano para resolver un problema de evolución bajo selección multinivel. }%
%
\en{There we were able to identify a ``multilevel posterior'' (i.e. the probability of individuals integrating all groups) as the average of the ``level 1 posteriors'' (i.e. probability of individuals within groups) weighted by the ``level 2 posteriors'' (i.e. probability of groups). }%
\es{Allí pudimos identificar un ``posterior multinivel'' (i.e. la probabilidad de los individuos integrando todos los grupos) como el promedio del ``posteriors de nivel 1'' (i.e. probabilidad de los individuos al interior de los grupos) pesado por el ``posterior de nivel 2'' (i.e la probabilidad de los grupos). }%
%
\en{The reason why an advantage in favor of cooperation and specialization arises in our simple causal model is due to the multiplicative (non-ergodic) nature of probabilistic theory and its isomorphism with evolutionary theory. }%
\es{El motivo por el cual surge una ventaja a favor de la cooperación y la especialización en simple modelo causal se debe a la naturaleza multiplicativa (no-ergódica) de la teoría de la probabililidad y a su isomorfismo con la teoría de la evolución. }%
%
\en{That is, contrary to the belief established since the mid-20th century in economics, we show that the dynamics of common goods cannot be represented by a prisoner's dilemma payoff matrix. }%
\es{Es decir, en contra de la creencia establecida que en economía se tiene desde mediados del siglo 20, mostramos que los dinámicas de bienes comunes no puede representarse mediante una matriz de pagos del dilema del prisionero. }%
%
\en{Moreover, contrary to the belief that specialization is too complex a feature to produce a benefit in simple aggregations, we show that as soon as cooperation emerges, an advantage in favor of specialist strategies appears even in groups of size 2. }
\es{Además, en contra de la creencia de que la especialización es una caracterísitica demasiado compleja para que produzca un beneficio en agregaciones simples, mostramos que apenas surge la cooperación, aparece una ventaja a favor de las estrategias especialistas incluso en grupos de tamaño 2. }%

% Parrafo

\en{It is extraordinary that such a simple system as the one analyzed has such fundamental conclusions to understand the complexity of life. }%
\es{Es realmente extraordinario que un sistema tan simple como el que hemos analizando tenga conclusiones tan fudamentales para entender la complejidad de la vida. }%
%
\en{Cooperation and specialization are the two main characteristics of the major evolutionary transitions, through which life acquired increasing complexity. }%
\es{La cooperación y la especilización son las dos caracteristicas principales de las transiciones evolutivas mayores, a través de las cuales la vida fue adquiriendo una complejidad cada vez mayor. }%
%
\en{Using Czégel-Zachar-Szathmáry's methodological approach \cite{czegel2019-bayesianEvolution} (multilevel selection as hierarchical Bayesian inference) we formally demonstrate the evolutionary advantage of cooperation and specialization suggested by Yaari-Peters \cite{yaari2010-cooperationEvolution, peters-cooperation2019.03.04} (noisy multiplicative processes). }%
\es{Mediante la propuesta metodológica de Czégel \cite{czegel2019-bayesianEvolution} (la selección multinivel como inferencia bayesiana jerárquica) resolvimos formalmente la demostración evolutiva de la cooperación y la especialización que le faltaba al modelo Yaari-Peters \cite{yaari2010-cooperationEvolution, peters-cooperation2019.03.04} (procesos multiplicativos ruidosos). }%
%
\en{And in turn, with the Yaari-Peters model we provided the concrete example that was missing from the methodological proposal of Czegel et al. }%
\es{Y a su vez, con el modelo de Yaari-Peters proveímos el ejemplo concreto que le faltaba a la propuesta metodológica de Czegel et al. }%
%
\en{Both proposals combined offer a new solution to the problem of major evolutionary transitions, which is simpler than the previous ones (noisy multiplicative processes), based on well-founded mathematical principles (the strict application of the rules of probability). }%
\es{Ambas propuestas combinadas ofrecen una solución nueva al problema de las transiciones evolutivas mayores, que es más sencilla que las anterios (procesos multiplicativos ruidosos), basada en principios matemáticos bien fundados (la aplicación estrica de las reglas de la probabilidad). }%


% 
% La ventaja evolutiva de la cooperación y la especialización es consecuencia de la no-ergodicidad de los procesos multiplicativos a los que está sujeto la vida, por lo que ésta debe ser considerada la primera causa de las transiciones evolutivas mayores.
% 


%
% \en{We show that while resource-avoidance strategies can invade entirely cooperative groups by natural selection, such behavior at the same time increases their own individual fluctuations, reducing their own long-term growth rate without the need to introduce penalties. }%
% \es{Mostramos que, si bien las estrategias que evitan compartir recursos pueden invadir por selección natural grupos enteramente cooperadores, tal comportamiento aumenta sus propias fluctuaciones individuales, reduciendo su propia tasa de crecimiento a largo plazo sin necesidad de introducir castigos. }%
% %
% %
% %

% 
% 
% 
% %
% \en{However, Ole Peters believes that specialization is a too complex property to produce a benefit in simple aggregations of, for example, two agents. }
% \es{Sin embargo, Ole Peters considera a la especialización como una caracterísitica demasiado compleja para que produzca un beneficio en agregaciones simples de por ejemplo dos agentes. }%
% %
% \en{He says the ``[advantage of cooperation] it may explain the transition from single cells to bicellular organisms, too small and simple to benefit from new function or specialization.'' }
% \es{Dice que la ventaja de la cooperación: ``it may explain the transition from single cells to bicellular organisms, too small and simple to benefit from new function or specialization.'' }%


%% 
% Según Czegel \cite{czegel2019-bayesianEvolution} el isomorfismo entre los procesos evolutivos y la inferencia bayesiana multinivel,  ``support a learning theory-oriented narrative of evolutionary complexification: the complexity and depth of the hierarchical structure of individuality mirror the amount and complexity of data that have been integrated about the environment through the course of evolutionary history.''
% Esta especulación es rechazada por el modelo de Ole Peters, el cual muestra que un simple proceso multiplicativo ruidoso favorece la selección multinivel de grupos cooperativos por sobre grupos con desertores.
% 
% Por otra parte, según \cite{peters-cooperation2019.03.04} su modelo ``paints a picture of cooperation driven by self-interest, not altruism, with cooperators outgrowing similar non-cooperators''.
% Esta especulación es rechazada por la metodología de Czegel \cite{czegel2019-bayesianEvolution}, la cual muestra que si bien las estrategias cooperativas no son evolutivamente estables al interior de los grupos, estos se ven favorecidos gracias a la selección multinivel.
% 
% Además Czegel \cite{czegel2019-bayesianEvolution} dice que, ``This isomorphism allows for a natural interpretation of evolutionary transitions in individuality as \emph{learning the structure}''.
% En este trabajo mostramos que lo que se aprende no es la estructura, sino que \emph{aprenden la dinámica}, en particular la ventaja que la no-ergódico de los procesos multiplicativos ofrece a favor de la cooperación y la especialización.
% 
% %La posibilidad de supervivencia y reproducción de una población depende no sólo del sistema de reciprocidad para la propagación del riesgo dentro de las poblaciones y entre las poblaciones de diferentes especies.
% 
% 
% %Con la idea de analizar los efetos de la selección multinivel, traduicimos el modelo de Ole Peters a una distirbución de probabilidad jerárquica, la cual describiremos en términos gráficos y analizaremos utilizando solamente las reglas de la probabilidad.


{\footnotesize
\bibliographystyle{auxiliar/biblio/plos2015.bst}
\bibliography{auxiliar/biblio/biblio_notUrl.bib}
}

\section{Apéndice}



\subsection{Selección de nivel 1, nivel 2 y multinivel}

En esta sección resolveremos por casos la marginal que necesitamos para computar el posterior de la la selección de nivel 1, de nivel 2 y multinivel,
%
\begin{equation}
\begin{split}
P(i^{T+1}, \vec{\Aa}^{\,1}, \dots, \vec{\Aa}^{\,T}) & = m_{P(I^{T+1}|I^{T}) \rightarrow I^{T+1} }(i^{T+1})
\end{split}
\end{equation}
%
donde el mensaje está definido en la ecuación~\ref{eq:m_pii_i}.
%
Debido a que ese mensaje está definido de forma recursiva, realizaremos demostraciones por inducción, separada por casos: primero para las regiones enteramente desertoras, luego para las regiones enteramente cooperadoras, y finalmente para las regiones mixtas.

\subsubsection{Regiones enteramente desertoras}

Debido a que ya hemos visto en la introducción, que la tasa de crecimiento de los individuos se puede calcular como la media geométrica, proponemos la siguiente hipótesis inductiva para la región enteramente desertora ($\text{HI}_d(T)$)
%
\begin{equation}
 m_{P(I^{T+1}|I^{T}) \rightarrow I^{T+1} }(i^{T+1}) \overset{\text{HI}_d}{=} P(i^{T+1}) \prod_{t=1}^{T} P(\vec{\Aa}^{\,t}) \, P(i^{T+1}|\vec{\Aa}^{\,t})
\end{equation}
%
con $\texttt{region}(k)=0$.
%
\paragraph{Caso Base}
Esta hipótesis vale en el caso base, $T=1$, 
pues $P(i^{t+1}|i^t) = \mathbb{I}(i^{t+1} = i^t)$,
%
\begin{equation}
\begin{split}
m_{P(I^{2}|I^{1}) \rightarrow I^{2} }(i^2) & \overset{\hfrac{eq}{\ref{eq:m_pii_i}}}{=}  \sum_{i^1} P(i^2|i^1) \, P(\vec{\Aa}^{\,1}) \, P(i^1|\vec{\Aa}^{\,1}) \,  P(i^1) \\
&= P(\vec{\Aa}^{\,1}) \, P(i^{2}|\vec{\Aa}^{\,1}) P(i^2)
\end{split}
\end{equation}
%
Notar que se produjo un cambio de variable debido a que el único elemento de la sumatoria que sobrevive es el que $i^1 = i^2$. 
%
Y dado que vale la hipótesis inductiva para el tiempo $T$, $\text{HI}_d(T)$, también vale para el tiempo $T+1$, pues
%
\begin{equation}
\begin{split}
m_{P(I^{T+1}|I^{T}) \rightarrow I^{T+1} }(i^{T+1}) & \overset{\hfrac{eq}{\ref{eq:m_pii_i}}}{=}  \sum_{i^T} P(i^{T+1}|i^T) \, P(\vec{\Aa}^{\,T}) \, P(i^T|\vec{\Aa}^{\,T}) \,  m_{P(I^T|I^{T-1}) \rightarrow I^T }(i^T) \\
&\overset{\hfrac{}{\text{caso}}}{=} P(\vec{\Aa}^{\,T}) \, P(i^{T+1}|\vec{\Aa}^{\,T}) \,  m_{P(I^T|I^{T-1}) \rightarrow I^T }(i^{T+1}) \\
&\overset{\text{HI}_d}{=} P(\vec{\Aa}^{\,T}) \, P(i^{T+1}|\vec{\Aa}^{\,T}) \, P(i^{T+1}) \, \prod_{t=1}^{T-1} P(\vec{\Aa}^{\,t}) \, P(i^{T+1}|\vec{\Aa}^{\,t}) \\[-0.3cm]
& =  P(i^{T+1}) \prod_{t=1}^{T} P(\vec{\Aa}^{\,t}) \, P(i^{T+1}|\vec{\Aa}^{\,t})
\end{split}
\end{equation}
%
Luego, el posterior de los individuos en la región enteramente desertora es,
%
\begin{equation}\label{eq:posterior_multinivel_desertor}
\begin{split}
P(i^{T+1} | \vec{\Aa}^{\,1}, \dots, \vec{\Aa}^{\,T}) \overset{\hfrac{\text{caso}}{1}}{=}  P(i^{T+1})  \prod_{t=1}^{T} P(i^{T+1}|\vec{\Aa}^{\,t}) & \propto  P(i^{T+1}) \prod_{t=1}^{T} f(\Ee_i,\Aa_{i}^{t})
\end{split}
\end{equation}
%
con $i = i^{T+1}$.

\subsubsection{Regiones enteramente cooperadoras}


Teniendo en cuenta que ya hemos visto en la introducción que los individuos cooperadores dividen los recursos en partes iguales luego de cada paso, proponemos la siguiente hipótesis inductiva para el grupo enteramente cooperador ($\text{HI}_c(T)$)
%
\begin{equation}
 m_{P(I^{T+1}|I^{T}) \rightarrow I^{T+1} }(i^{T+1}) =  P(i^{T+1})\prod_{t=1}^{T}  P(\vec{\Aa}^{\,t})  \sum_j^{\texttt{\en{partners}\es{socios}}(r)} \frac{1}{N} P(j|\vec{\Aa}^{\,t}) 
\end{equation}
%
donde $r = \texttt{region}(i^{T+1})$, y $\texttt{\en{partners}\es{socios}}(r)$ es el conjunto de todos los miembros cooperadores que pertenecen a la región $r$.
%
Esta hipótesis vale en el caso $T=1$.
%
\begin{equation}
\begin{split}
m_{P(I^{2}|I^{1}) \rightarrow I^{2} }(i^{2}) & \overset{\hfrac{eq}{\ref{eq:m_pii_i}}}{=}  \sum_{i^1} P(i^{2}|i^1) \, P(\vec{\Aa}^{\,1}) \, P(i^1|\vec{\Aa}^{\,1}) \,   P(i^1) \\[-0.3cm]
&= P(\vec{\Aa}^{\,1}) \sum_j^{\texttt{\en{partners}\es{socios}}(r)} \frac{1}{N} \, P(j|\vec{\Aa}^{\,1}) P(j) \\[-0.3cm]
& = P(i^2) P(\vec{\Aa}^{\,1}) \sum_j^{\texttt{\en{partners}\es{socios}}(r)} \frac{1}{N} \, P(j|\vec{\Aa}^{\,1})
\end{split}
\end{equation}
%
pues $P(i^{t+1}|i^t) = \frac{1}{N}\mathbb{I}(\texttt{region}(i^{t+1}) = \texttt{region}(i^t))$, y para todo $j \in \texttt{\en{partners}\es{socios}}(region(i^2))$ vale que $P(j) = P(i^2)$
%
Y dado que vale la hipótesis inductiva para el tiempo $T$, $\text{HI}_c(T)$, también vale para el tiempo $T+1$, pues
%
\begin{equation}
\begin{split}
m_{P(I^{T+1}|I^{T}) \rightarrow I^{T+1} }(i^{T+1}) & \overset{\hfrac{eq}{\ref{eq:m_pii_i}}}{=}  \sum_{i^T} P(i^{T+1}|i^T) \, P(\vec{\Aa}^{\,T}) \, P(i^T|\vec{\Aa}^{\,T}) \,  m_{P(I^T|I^{T-1}) \rightarrow I^T }(i^T) \\
&\overset{\hfrac{}{\text{caso}}}{=} \sum_j^{\texttt{\en{partners}\es{socios}}(r)} P(\vec{\Aa}^{\,T}) \frac{1}{N} \, P(j|\vec{\Aa}^{\,T})m_{P(I^T|I^{T-1}) \rightarrow I^T }(j) \\
&\overset{\hfrac{*}{\text{caso}}}{=} \Big(\sum_j^{\texttt{\en{partners}\es{socios}}(r)} P(\vec{\Aa}^{\,T}) \frac{1}{N} \, P(j|\vec{\Aa}^{\,T}) \Big) \Big(m_{P(I^T|I^{T-1}) \rightarrow I^T }(i^{T+1}) \Big) \\
& \overset{\text{HI}_c}{=} \Big(\sum_j^{\texttt{\en{partners}\es{socios}}(r)} P(\vec{\Aa}^{\,T}) \frac{1}{N} \, P(j|\vec{\Aa}^{\,T}) \Big) \Big( P(i^{T+1})\prod_{t=1}^{T-1}  \sum_j^{\texttt{\en{partners}\es{socios}}(r)} P(\vec{\Aa}^{\,t}) \frac{1}{N} P(j|\vec{\Aa}^{\,t})  \Big) \\
& = P(i^{T+1})\prod_{t=1}^{T}  P(\vec{\Aa}^{\,t})  \sum_j^{\texttt{\en{partners}\es{socios}}(r)} \frac{1}{N} P(j|\vec{\Aa}^{\,t})
\end{split}
\end{equation}
%
donde la igualdad $\overset{\hfrac{*}{\text{caso}}}{=}$ vale porque en los grupos enteramente cooperadores, los mensajes $m_{P(I^T|I^{T-1}) \rightarrow I^T }(j)$ son el mismo para todos los miembros del grupo $j$, lo que nos permite remplazar el índice $j$ por la variable $i^{T+1}$.
%
Luego, la marginal objetivo en el caso de la población enteramente cooperadora es 
\begin{equation}\label{eq:marginal_multinivel_cooperador}
P(i^{T+1}|\vec{\Aa}^{\,1},\dots,\vec{\Aa}^{\,T}) \overset{\hfrac{}{\text{caso}}}{=} P(i^{T+1}) \prod_{t=1}^{T} \sum_j^{\texttt{\en{partners}\es{socios}}(r)} \frac{1}{N} P(j|\vec{\Aa}^{\,t}) \propto P(i^{T+1})\prod_{t=1}^{T} \sum_j^{\texttt{\en{partners}\es{socios}}(r)} \frac{1}{N} f(\Ee_j,\Aa_{j}^{t})
\end{equation}
%

\subsubsection{Regiones mixtas, individuos cooperadores}

Los bienes comunes son generados por el conjunto de individuos cooperadores de la región $r$, \texttt{\en{partners}\es{socios}}($r$).
%
El tamaño de este conjunto depende de la cantidad de cooperadores que están en la misma región.
%
Sin embargo, la división del bien común se sigue dividiendo en partes iguales entre todos los miembros de la región $\frac{1}{2}$.
%
\begin{equation}
 P(i^{T+1}|\vec{\Aa}^{\,1},\dots,\vec{\Aa}^{\,T}) = P(i^{T+1}) \prod_{t=1}^{T}  \frac{1}{N} \sum_j^{\texttt{\en{partners}\es{socios}}(r)} P(j|\vec{\Aa}^{\,t}) 
\end{equation}
%
donde la demostración es equivalente al caso anterior. 

\subsubsection{Regiones mixtas, individuos desertores}

Antes de proponer una hipótesis inductiva, veamos qué ocurre con los primero mensajes de la recursión de modo de ganar intuición.
%
Por definición, 
%
\begin{equation}
m_{P(I^{2}|I^{1}) \rightarrow I^{2} }(k) = \sum_{i^1} P(k|i^1)  P(\vec{\Aa}^{\,1})P(i^1|\vec{\Aa}^{\,1}) P(i^1)
\end{equation}
%
En el caso de regiones mixtas, el factor social vale $1$ cuando $k=i^1$ y vale $1/N$ cuando $i^1 \in \texttt{\en{partners}\es{socios}}(r)$.
%
\begin{equation}
\begin{split}
m_{P(I^{2}|I^{1}) \rightarrow I^{2} }(k) = P(\vec{\Aa}^{\,1}) \Big(P(k) P(k|\vec{\Aa}^{\,1})\phantom{\Bigg|_j} + \underbrace{P(c)\sum_j^{\texttt{\en{partners}\es{socios}}(r)} \frac{1}{N} P(j|\vec{\Aa}^{\,1})}_{P(c|\vec{\Aa}^{\,1})} \Big) 
\end{split}
\end{equation}
%
donde $P(c|\vec{\Aa}^{\,1})$ es el posteriors de los individuos cooperadoras.
%
Por definición, el siguiente mensaje 
%
\begin{equation}
\begin{split}
m_{P(I^{3}|I^{2}) \rightarrow I^{3} }(k) &= \sum_{i^2} P(k|i^2)  P(\vec{\Aa}^{\,2}) P(i^2|\vec{\Aa}^{\,2}) m_{P(I^{2}|I^{1}) \rightarrow I^{2} }(i^2) \\
& =  P(\vec{\Aa}^{\,2}) \Big( P(k|\vec{\Aa}^{\,2}) \underbrace{m_{P(I^{2}|I^{1}) \rightarrow I^{2} }(k)}_{P(k)P(k|\vec{\Aa}^{\,1})P(\vec{\Aa}^{\,1})} + \sum_j^{\texttt{\en{partners}\es{socios}}(r)} \frac{1}{N} P(j|\vec{\Aa}^{\,2}) \underbrace{m_{P(I^{2}|I^{1}) \rightarrow I^{2} }(j)}_{P(c|\vec{\Aa}^{\,1})P(\vec{\Aa}^{\,1})}  \Big) \\
\end{split}
\end{equation}
%
donde todos los mensajes que reciben los individuos cooperadores $j$, $m_{P(I^{2}|I^{1}) \rightarrow I^{2} }(j)$, son iguales.
%
\begin{equation}
\begin{split}
m_{P(I^{3}|I^{2}) \rightarrow I^{3} }(k)
& = \Big(\prod_{t=1}^2 P(\vec{\Aa}^{\,t}) \Big) \Big( P(k|\vec{\Aa}^{\,2}) P(k|\vec{\Aa}^{\,1})P(k) + P(c|\vec{\Aa}^{\,1}) \sum_j^{\texttt{\en{partners}\es{socios}}(r)} \frac{1}{N} P(j|\vec{\Aa}^{\,2})   \Big) \\
& = \Big(\prod_{t=1}^2 P(\vec{\Aa}^{\,t}) \Big) \Big( \underbrace{P(k|\vec{\Aa}^{\,2}) P(k|\vec{\Aa}^{\,1}) P(k)+ P(c|\vec{\Aa}^{\,1},\vec{\Aa}^{\,2})}_{P(k|\vec{\Aa}^{\,1},\vec{\Aa}^{\,2})}   \Big) \\
\end{split}
\end{equation}
%
Abriendo la recursión nos encontramos con, 
%
\begin{equation}
P(k|\vec{\Aa}^{\,1},\vec{\Aa}^{\,2}) = P(k)P(k|\vec{\Aa}^{\,1})P(k|\vec{\Aa}^{\,2}) + P(k|\vec{\Aa}^{\,2})P(c|\vec{\Aa}^{\,1}) + P(c|\vec{\Aa}^{\,1},\vec{\Aa}^{\,2})
\end{equation}
%
Luego, la hipótesis inductiva $\text{HI}_M(T)$ para individuos desertores $k$ de regiones $r$ mixtas es,
%
\begin{equation}\label{eq:HI_M}
\begin{split}
P(k|\vec{\Aa}^{\,1}, \dots, \vec{\Aa}^{\,T}) &\overset{\text{HI}_M(T)}{=} \Big(P(k)\prod_{t=1}^{T} P(k|\vec{\Aa}^{\,t}) \Big) + \Big(\sum_{t=1}^{T} P(c|\wedge_{q=1}^t\vec{\Aa}^{\,q})  \prod_{q=t+1}^T P(k|\vec{\Aa}^{\,q}) \Big) \\[-0.2cm]
m_{P(I^{T+1}|I^{T}) \rightarrow I^{T+1} }(k) &\ \ \ = P(k|\vec{\Aa}^{\,1}, \dots, \vec{\Aa}^{\,T}) \prod_{t=1}^T P(\vec{\Aa}^{\,t})
\end{split}
\end{equation}
%
donde el posterior del individuo desertor en una región mixta $r$ es 
la suma del posterior de un individuo desertor en una región enteramente desertora, y un promedio movil exponencial del posterior de los individuos cooperadores.
%
\paragraph{Caso Base.} El caso base ya está demostrado por extensión.

\paragraph{Paso inductivo.} Dado que vale $\text{HI}_{M_D}(T)$, $\text{HI}_{M_C}(T)$ quiero ver que vale $\text{HI}_{M_D}(T+1)$.

\begin{equation}
\begin{split}
m_{P(I^{T+1}|I^{T}) \rightarrow I^{T+1} }(k) &= \sum_{i^T} P(k|i^T)  P(\vec{\Aa}^{\,T}) P(i^T|\vec{\Aa}^{\,T}) m_{P(I^{T}|I^{T-1}) \rightarrow I^{T} }(i^T) \\
& =  P(\vec{\Aa}^{\,T}) \Big( P(k|\vec{\Aa}^{T}) m_{P(I^{T}|I^{T-1}) \rightarrow I^{T} }(k) + \sum_j^{\texttt{\en{partners}\es{socios}}(r)} \frac{1}{N} P(j|\vec{\Aa}^{\,T}) m_{P(I^{T}|I^{T-1}) \rightarrow I^{T} }(j)  \Big) \\
&\overset{\text{HI}}{=} \Big(\prod_{t=1}^T P(\vec{\Aa}^{\,t}) \Big) \Big( P(k|\vec{\Aa}^{\,T}) P(k|\vec{\Aa}^{\,1},\dots, \vec{\Aa}^{\,T-1}) + P(c|\vec{\Aa}^{\,1}, \dots, \vec{\Aa}^{\,T})\Big) \\
&=\Big(\prod_{t=1}^T P(\vec{\Aa}^{\,t}) \Big) \Bigg( \Big(P(k)\prod_{t=1}^{T} P(k|\vec{\Aa}^{\,t}) \Big) + \Big(\sum_{t=1}^{T} P(c|\wedge_{q=1}^t\vec{\Aa}^{\,q})  \prod_{q=t+1}^T P(k|\vec{\Aa}^{\,q}) \Big) \Bigg)
\end{split}
\end{equation}
%
Luego, vale la hipótesis inductiva. 

% 
% \subsection{Tasa de supervivencia}
% % Parrafo 
% 
% Para determinar la tasa de crecimiento de las estrategia desertora en la población mixta, calculamos el cambio en el tamaño de la población luego de un paso temporal usando las tasas de crecimiento caracterísiticas 
% \begin{equation}
% \lim_{T \rightarrow \infty} \omega_C(T) = \overline{f}_c^T \ \ \ \  \ \ \lim_{T \rightarrow \infty} \prod^T_{t=1} r(t) = \overline{f}_d^T
% \end{equation}
% donde $\overline{f}_c$ y $\overline{f}_d$ están definidas en las ecuaciones \ref{eq:coop_temporal_average} y \ref{eq:des_temporal_average} respectivamente.
% Luego, el tamaño de la población desertora lo podemos aproximar como 
% \begin{equation}
% \omega_D(t) = \overline{f}(e,c=0)^t + \sum^{t-1}_{j=1} \overline{f}(e,n,N,c=1)^j \overline{f}(e,c=0)^{t-j}
% \end{equation}
% Si reescalamos las tasas de crecimiento por un factor de $2.1$, recuperamos el juego propuesto por Ole Peters.
% En la siguiente figura mostramos los recursos de los agentes desertor (azul) y cooperador (verde) en una población de tamaño 100 con un único desertor obtenida a partir del juego original propuesto por Ole Peters, y las curvas negras son las estimaciones temporales obtenidas con las tasas de crecimiento caracterísitica $\overline{f}_c$ y $\overline{f}_d$, de la estrategia $e=1.5/2.1$ reescladas por el factor $2.1$.
% \begin{figure}[H]
%     \centering
%     \begin{subfigure}[b]{0.66\textwidth}
%     \includegraphics[width=\linewidth]{figures/pdf/multilevel-selection-5.pdf}
%     \end{subfigure}
%     \caption{
%     }
%     \label{fig:multilevel-selection-5}
% \end{figure}
% Es interesante que la tasa de crecimiento del agente desertor es la misma que la tasa de crecimiento de los agentes cooperadores.
% La tasa de crecimiento de los desertores se mantiene igual hasta que la tasa de crecimiento de la población cooperdora cae por debajo de la tasa de crecimiento de los desertores.
% %
% \begin{figure}[H]
%     \centering
%     \begin{subfigure}[b]{0.66\textwidth}
%     \includegraphics[width=\linewidth]{figures/pdf/multilevel-selection-7.pdf}
%     \end{subfigure}
%     \caption{
%     }
%     \label{fig:multilevel-selection-7}
% \end{figure}
% %
% 
% % Parrafo
% 
% Finalmente, en la siguiente figura mostramos qué estrategias se seleccionan para una población de tamaño 9 enteramente cooperativa (figura~\ref{fig:multilevel-selection-1}), con un agente desertor (figura~\ref{fig:multilevel-selection-2})y con dos agentes desertores (figura~\ref{fig:multilevel-selection-3}).
% %
% \begin{figure}[H]
%     \centering
%     \begin{subfigure}[b]{0.32\textwidth}
%     \includegraphics[width=\linewidth]{figures/pdf/multilevel-selection-1.pdf}
%     \caption{9/9}
%     \label{fig:multilevel-selection-1}
%     \end{subfigure}
%     \begin{subfigure}[b]{0.32\textwidth}
%     \includegraphics[width=\linewidth]{figures/pdf/multilevel-selection-2.pdf}
%     \caption{8/9}
%     \label{fig:multilevel-selection-2}
%     \end{subfigure}
%     \begin{subfigure}[b]{0.32\textwidth}
%     \includegraphics[width=\linewidth]{figures/pdf/multilevel-selection-3.pdf}
%     \caption{7/9}
%     \label{fig:multilevel-selection-3}
%     \end{subfigure}
%     \caption{
%     Posterior de las estrategias individuales para una población de 9 agentes con 0, 1 y 2 desertores (\ref{fig:multilevel-selection-1}, \ref{fig:multilevel-selection-2} y \ref{fig:multilevel-selection-3} respectivamente).
%     Los valores negativo del eje x representa todo el rango de estrategias desertoras, y los valores positivos representan todo el rango de las estrategias cooperadoras.
%     }
%     \label{fig:multilevel-selection-123}
% \end{figure}
% %
% Cuando la población es enteramente cooperadora o tiene un desertor, la estrategia predominante es especialista.
% Cuando la población tiene más de una desertor, la estrategia predominante generalista.
% 
% % Parrafo
% 
% \paragraph{Posibles extensiones del modelo multinivel} Existen varias posibles extensiones al modelo probabilístico de selección multinivel.
% \begin{itemize}
% \item Agregar un prior con decaimiento exponencial al tamaño de la población para ver en que momento deja de convenir seguir agrandando la población.
% \item Agregar mutaciones.
% \item Considerar poblaciones heterogéneas, con diferente estrategias $e$ al interior de la población.
% \end{itemize}
% 


% 
% 
% \subsection{Tasa de crecimiento en poblaciones mixtas infinitas}
% 
% Las estrtegias que en una población infinita representan una proporción mayor a $0$ son a su vez infinitas.
% Sea $d$ la proporción de desertores y $1-d$ la proporción de cooperadores.
% 
% % 
% 
% La población cooperadora, al ser infinita, siempre crece como
% \begin{equation}
% \Delta w(C|e,d,a) = (1-d) \underbrace{(e\cdot a + (1-e)\cdot(1-a))}_{\text{Media de estados}}
% \end{equation}
% En cada paso las apuestas del conjunto de cooperadores crece como la media de estados, la cual se divide en partes iguales.
% Como el cambio en los recursos no depende de $t$, en t pasos la población tiene un tamaño de
% \begin{equation}
% w(C|e,d,a,t) = \Delta w(C|e,d,a)^t 
% \end{equation}
% 
% 
% 
% \begin{figure}[H]
%     \centering
%     \begin{subfigure}[b]{0.66\textwidth}
%     \includegraphics[width=\linewidth]{figures/pdf/multilevel-selection-4.pdf}
%     \end{subfigure}
%     \caption{
%     }
%     \label{fig:multilevel-selection-4}
% \end{figure}

% 
% 
% 
% 
% 
% 
% \begin{quotation}
% 
% The prisoner’s dilemma is an extreme example of a social dilemma, because (D, D) is the only Nash equilibrium.
% The stug-hunt may be more realistic models of the underlying social dilemma.
% 
% \begin{equation}
%   \bordermatrix{ & C & D \cr
%       C & s/2 & 0 \cr
%       D & h & h } 
% \end{equation}
% 
% con $s=kb/2$, $k>2$ y $h=b$. 
% In this case, both (C, C) and (D, D) are Nash equilibria: once playing one of them, no player would be willing to unilaterally change his mind.
% The cooperative equilibrium is Pareto optimal, which means that no further Pareto improvements (changes in the strategy profile making at least one player better off without making the other player worse off) are possible.
% The defective equilibrium, however, is risk-dominant; no matter what my co-player does, if I decide to hunt hare I will get h; if I decide to hunt stag instead and my co-player abandons me, I will
% get nothing. In other words, stag hunters depend on each other whereas harehunters are independent. This makes the defective equilibrium salient for risk-averse individuals, even if it is Pareto inferior. Pessimists will invariably prefer
% to play D.
% \end{quotation}
% 
% 


\end{document}
