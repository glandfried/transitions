\documentclass[a4paper,10pt]{article}
\usepackage[utf8]{inputenc}
\input{auxiliar/tex/encabezado.tex}

\newif\ifen
\newif\ifes
\newcommand{\en}[1]{\ifen#1\fi}
\newcommand{\es}[1]{\ifes#1\fi}
\estrue

%opening
\title{Ventaja evolutiva a favor de la cooperación.}
\author{Gustavo Landfried}

\begin{document}

\maketitle

\begin{abstract}
La complejidad actual de la vida es consecuencia de una serie de transiciones evolutivas en las que las entidades, que antes eran capaces de replicarse de forma independiente, luego de la transición sólo pueden replicarse como partes de una unidad mayor.
Las entidades que cooperan ceden algo de valor a la otras para formar esta unidad de mayor nivel.
Los enfoques clásicos presuponen la existencia de un dilema que modelas mediante una estructura de pagos que si bien tiene un óptimo en la mutua cooperación, ofrece incentivos individuales para desertar.
Esta formulación del problema abrió una importante línea de investigación dedicada a descrubir las condiciones que requieren las estrategias cooperadoras para ser evolutivamente estable primero y para invadir después.

En este artículo proponemos una solución simple que no requiere tales suposiciones.
%basada en dos conceptos recientemente incorporados a la teoría de la evolución, que permite mostrar que existe una ventaja física en favor de la agregación cooperativa de las entidades individuales.
La ventaja evolutiva en favor de la agregación cooperativa de las entidades se debe fundamentalente a que los procesos multiplicativos a los que está sujeto la vida, i.e. la secuencia de probabilidades de superviencia y reproducción, son no-ergódicos.
Esto hace que las fluctuaciones tengan un efecto negativo en la tasa de crecimiento individual a largo plazo, pero no en la tasa de crecimiento del valor esperado.
Al compartir recursos se reducen las fluctuaciones, aumentando la tasa de crecimiento a largo plazo de los cooperantes.
Cualquier agente que evite compartir recursos reducirá su propia tasa de crecimiento a largo plazo debido al inmediato aumento de las fluctuaciones, haciendo que las estrategias incondicionalmente cooperadoras sean evolutivamente estables sin la necesidad de introducir castigos.
Si bien las estrategias desertoras son también evolutivamente estables, los grupos cooperadores se ven favorecidos por la selección multinivel debido a su mayor tasa de crecimiento.
Finalmente, en este trabajo traducimos el proceso multiplicativo como un modelo

Este resultado se verifica cuando analizamos la selec La evolución de una población jerárquica bajo selección multinivel es equivalente a la inferencia bayesiana en modelos jerárquicos bayesianos.



\end{abstract}

En el último tercio de la historia del Universo, en algún momento hace aproximadamente 4500 millones de años, apareció en la tierra una forma de organización de la materia capaz de auto-replicarse.
El crecimiento de estos linajes siguieron procesos multiplicativos y ruidosos: secuencias de probabilidades de supervivencia y reproducción.
Las mutaciones producidas durante la replicación diversificaron las formas de organización de la materia, y las diferentes tazas de superviencia favorecieron a aquellas mejor adaptadas al ambiente.

Las diversas formas que fue adquiriendo la vida representa información que es capaz de interectuar con el medio y sobrevivir.

A crucial preliminary condition is the alignment of interests: to undergo an evolutionary transition in individuality, organisms must exhibit cooperation, originating from genetic relatedness and/or synergistic fitness interactions [4].

{\footnotesize
\bibliographystyle{auxiliar/biblio/plos2015.bst}
\bibliography{auxiliar/biblio/biblio_notUrl.bib}
}

\end{document}
