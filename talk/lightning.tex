\documentclass[shownotes,aspectratio=169]{beamer}

\input{../auxiliar/tex/diapo_encabezado.tex}
\input{../auxiliar/tex/tikzlibrarybayesnet.code.tex}
 \mode<presentation>
 {
 %   \usetheme{Madrid}      % or try Darmstadt, Madrid, Warsaw, ...
 %   \usecolortheme{default} % or try albatross, beaver, crane, ...
 %   \usefonttheme{serif}  % or try serif, structurebold, ...
  \usetheme{Antibes}
  \setbeamertemplate{navigation symbols}{}
 }
 
\usepackage{todonotes}
\setbeameroption{show notes}

\newcommand{\E}{\en{S}\es{E}}
\newcommand{\A}{\en{E}\es{A}}
\newcommand{\Ee}{\en{s}\es{e}}
\newcommand{\Aa}{\en{e}\es{a}}


\newif\ifen
\newif\ifes
\newcommand{\en}[1]{\ifen#1\fi}
\newcommand{\es}[1]{\ifes#1\fi}
\entrue

%\title[Bayes del Sur]{}

\begin{document}

\color{black!85}
\large

\begin{frame}[plain,noframenumbering]
 
 \begin{textblock}{120}(03,04)
 \huge \textcolor{black!66}{Multilevel selection \\[-0.1cm] in nonergodic systems}
\end{textblock}

 %\vspace{2cm}brown
%\maketitle
\Wider[2cm]{
\includegraphics[width=1\textwidth]{../auxiliar/images/peligro_predador}
}

 \begin{textblock}{47}(123,68)
\centering \Large  \textcolor{white!55}{Cooperation \\[-0.2cm] and specialization \hfill }
\end{textblock}

\end{frame}




\begin{frame}[plain]
\begin{textblock}{160}(0,4)
 \centering \LARGE
\en{The complexity of life}\es{La complejidad de la vida}
\end{textblock}
\vspace{0.75cm}

\centering
\includegraphics[width=0.8\linewidth]{../auxiliar/images/biomass.jpg}\\[0.2cm]
\footnotesize{Bar-On et al (2018)}

\end{frame}


\begin{frame}[plain]
\begin{textblock}{160}(0,4)
 \centering \LARGE
\en{Major evolutionary transitions}\es{Transiciones evolutivas mayores}
\end{textblock}
\vspace{0.75cm}

\centering
\includegraphics[width=0.6\linewidth]{../auxiliar/images/transition-west2015.jpg}\\[0.2cm]
\footnotesize{West et al (2015) \emph{Major evolutionary transitions in individuality}}

\end{frame}


\begin{frame}[plain]
\begin{textblock}{160}(0,4)
 \centering \LARGE
\en{Lineage growth}\es{Crecimiento de los linajes}
\end{textblock}
\vspace{0.75cm}

\begin{equation*}
\omega(T) = \prod_t^T f(a(t))
\end{equation*}

\vspace{0.5cm}

\begin{equation*}
f(a) =
\begin{cases}
 1.5 & \text{ Environment } = \text{ \en{Head}\es{Cara} } \\
 0.6 & \text{ Environment } = \text{  \en{Tail}\es{Seca} }
\end{cases}
\end{equation*}

\end{frame}


\begin{frame}[plain]
\begin{textblock}{160}(0,4)
 \centering \LARGE
\en{Advantage of mutual cooperation}\es{Ventaja de la mutua cooperación}
\end{textblock}


\begin{textblock}{160}(0,20)
\centering
\includegraphics[width=0.45\linewidth]{../figures/pdf/ergodicity_individual_trayectories.pdf}
\end{textblock}

\begin{textblock}{160}(118,20)
Cooperative
\end{textblock}

\begin{textblock}{160}(118,53)
Individuals
\end{textblock}


\only<2>{
\begin{textblock}{140}(10,73)
$\bullet$ But is unconditional cooperation an evolutionary stable strategy? \\[0.1cm]
$\bullet$ And what about specialization?
\end{textblock}
}

\end{frame}

\begin{frame}[plain]
\begin{textblock}{160}(0,4)
 \centering \LARGE
\en{The Evolutionary-Probability isomorphism}\es{El isomorfismo Evolución-Probabilidad }
\end{textblock}
\vspace{0.75cm}

\centering
\includegraphics[width=0.9\linewidth]{../static/title.png}


\end{frame}

\begin{frame}[plain]
\begin{textblock}{160}(0,4)
 \centering \LARGE
\en{Basic model}\es{Modelo básico}
\end{textblock}
\vspace{0.75cm}
\centering

\tikz{
    \node[latent] (e) {$\A_t$};
    \node[const, right=of e] (en) {\ $P(\Aa) = p^{\Aa} (1-p)^{(1-\Aa)}$};
    \node[const, left=of e] (ne) {\en{Environment}\es{Ambiente}: \ \ \ };
    
    
    \node[latent, below=of e] (r) {$\E$};
    \node[const, right=of r] (rn) {$p(\Ee|\Aa) \propto f(\Ee,\Aa) = \begin{cases}
 \Ee & \Aa = 1 \\
 1-\Ee & \Aa = 0
  \end{cases} $};
    \node[const, left=of r] (nr) {\en{Strategy}\es{Estrategia}: \ \ \ };
    
    \edge {e} {r};
    \plate {ee} {(e)} {$t$}; 
    }


\end{frame}



\begin{frame}[plain]
\begin{textblock}{160}(0,4)
 \centering \LARGE
\en{Basic model}\es{Modelo básico}
\end{textblock}
\vspace{0.75cm}
\centering

$P(\Aa) = 0.71$

\begin{figure}[H]
    \centering
    \begin{subfigure}[b]{0.3\textwidth}
    \includegraphics[width=\linewidth]{../figures/coin1.pdf}
    \caption*{$T = 1$}
    \end{subfigure}
    \begin{subfigure}[b]{0.3\textwidth}
    \includegraphics[width=\linewidth]{../figures/coin2.pdf}
    \caption*{$T = 10$}
    \end{subfigure}
    \begin{subfigure}[b]{0.3\textwidth}
    \includegraphics[width=\linewidth]{../figures/coin3.pdf}
    \caption*{$T = 10^5$}
    \end{subfigure}
\end{figure}

\pause

$\bullet$ Always the optimal individual strategy is the generalist one $s^* = P(\Aa)$

\end{frame}


\begin{frame}[plain]
\begin{textblock}{160}(0,4)
 \centering \LARGE
\en{Extended model}\es{Modelo extendido} \\
\large Cooperation
\end{textblock}
\vspace{0.75cm}
\centering

\begin{textblock}{140}(10,24)

\centering
\scalebox{0.75}{
\tikz{

    \node[latent, minimum size=2cm ] (x1_0) {$\omega_1(t)$} ;
    \node[latent, below=of x1_0, minimum size=2cm ] (x2_0) {$\omega_2(t)$} ;

    \node[latent, right=of x1_0, minimum size=3cm ] (x1_0g) {$ \omega_1(t)\cdot f(a_1(t))$} ;
    \node[latent, right=of x2_0, minimum size=1.8cm, xshift=0.6cm , align=left] (x2_0g) {$\omega_2(t)\cdot$\\$f(a_2(t))$} ;
    
    \node[latent, right=of x1_0g, minimum size=3.8cm, yshift=-1.33cm, align=right] (x_0) {$\omega_1(t)\cdot f(a_1(t))$\\$+\omega_2(t)\cdot f(a_2(t))$ } ;
    
    \node[const, above=of x_0] (nx_0) {$\overbrace{\text{Pool}\hspace{2.5cm}\text{Share}}^{\text{\normalsize Cooperation}}$} ;
    
    \node[latent, right=of x1_0g, minimum size=2.5cm,  xshift=4.5cm] (x1_1) {$\omega_1(t+1)$ } ;
    \node[latent, below=of x1_1, minimum size=2.5cm, yshift=0.7cm] (x2_1) {$\omega_2(t+1)$ } ;
    
    
    \node[invisible, above=of x1_0g, yshift=0.1cm] (i1) {} ;
    \node[invisible, above=of x1_0, xshift=13.5cm] (i1) {} ;
    
    \edge {x1_0} {x1_0g};
    \edge {x2_0} {x2_0g};
    \edge {x1_0g,x2_0g} {x_0};
    \edge {x_0} {x1_1,x2_1};
    
}
}
\end{textblock}


\end{frame}

\begin{frame}[plain]
\begin{textblock}{160}(0,4)
 \centering \LARGE
\en{Extended model}\es{Modelo extendido} \\
\large Defection
\end{textblock}
\vspace{0.75cm}
\centering

\begin{textblock}{140}(10,24)
\centering
\scalebox{0.75}{
\tikz{

    \node[latent, minimum size=2cm ] (x1_0) {$\omega_1(t)$} ;
    \node[latent, below=of x1_0, minimum size=2cm ] (x2_0) {$\omega_2(t)$} ;

    \node[latent, right=of x1_0, minimum size=3cm ] (x1_0g) {$ \omega_1(t)\cdot f(a_1(t))$} ;
    \node[latent, right=of x2_0, minimum size=1.8cm, xshift=0.6cm , align=left] (x2_0g) {$\omega_2(t)\cdot$\\$f(a_2(t))$} ;
    
    \node[latent, right=of x1_0g, minimum size=3cm, yshift=-1.33cm, align=right] (x_0) {$\omega_1(t)\cdot f(a_1(t))$ } ;
    
    \node[const, above=of x_0] (nx_0) {$\overbrace{\text{Pool}\hspace{2.5cm}\text{Share}}^{\text{\normalsize Cooperation}}$} ;
    
    \node[latent, right=of x1_0g, minimum size=2.2cm,  xshift=4.5cm] (x1_1) {$\omega_1(t+1)$ } ;
    \node[latent, below=of x1_1, minimum size=3cm, yshift=0.7cm, align=left] (x2_1) {$\omega_1(t+1) +$ \\$ \omega_2(t)\cdot f(a_2(t))$ } ;
    
    \node[invisible, above=of x1_0g, yshift=0.1cm] (i1) {} ;
    \node[invisible, above=of x1_0, xshift=13.5cm] (i1) {} ;
    
    
    
    \edge {x1_0} {x1_0g};
    \edge {x2_0} {x2_0g};
    \edge {x1_0g} {x_0};
    \edge {x_0} {x1_1,x2_1};
    \edge {x2_0g} {x2_1};
    
}
}
\end{textblock}

\end{frame}

\begin{frame}[plain]
\begin{textblock}{160}(0,4)
 \centering \LARGE
Evolutionary advantage of cooperation
\end{textblock}
\vspace{1cm}
\centering


\begin{equation*}\label{eq:posterior_multinivel}
\underbrace{P(\texttt{coop}(i^T)|\vec{\Aa}^{\,1}, \dots, \vec{\Aa}^{\,T-1})}_{\text{\en{\small Multilevel selection}\es{Selección multinivel}}} = \sum_{g=0}^N \underbrace{P(\texttt{coop}(i^T)|\vec{\Aa}^{\,1}, \dots, \vec{\Aa}^{\,T-1}, g)}_{\text{\en{\small Individual selection}\es{Individual selección}}} \cdot \underbrace{P(g^T|\vec{\Aa}^{\,1}, \dots, \vec{\Aa}^{\,T-1})}_{\text{\en{\small Group selection}\es{Selección de nivel 2}}}
\end{equation*}
%

\begin{figure}[H]
    \centering
    \begin{subfigure}[b]{0.48\textwidth}
    \centering
    \includegraphics[width=\linewidth]{../figures/pdf/multilevel-selection-multilevel-posterior.pdf}
    \caption*{Multilevel selection}
    \end{subfigure}
    \begin{subfigure}[b]{0.48\textwidth}
    \includegraphics[width=\linewidth]{../figures/pdf/multilevel-selection-6.pdf}
    \caption*{Group selection}
    \end{subfigure}
\end{figure}



\end{frame}


\begin{frame}[plain]
\begin{textblock}{160}(0,4)
 \centering \LARGE
Evolutionary advantage of specialization
\end{textblock}
\vspace{1cm}
\centering


\begin{figure}[H]
    \centering
    \begin{subfigure}[b]{0.6\textwidth}
    \includegraphics[width=\linewidth]{../figures/pdf/tasa-temporal-2.pdf}
    \end{subfigure}
\end{figure}

$\bullet$ In groups, the optimal strategy is specialist $s^* > P(\Aa)= 0.71$


\end{frame}


\begin{frame}[plain]
\begin{textblock}{160}(0,4)
 \centering \LARGE
Irreversibility of evolutionary transitions
\end{textblock}
\vspace{1cm}
\centering

\en{Specialist individuals cannot leave groups  \\ without a reduction of their evolutionary viability. }%

\end{frame}

\begin{frame}[plain]

\centering
  \includegraphics[width=0.4\textwidth]{../auxiliar/images/pachacuteckoricancha.jpg}

\end{frame}


\end{document}



