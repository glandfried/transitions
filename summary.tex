\documentclass[a4paper,10pt]{article}
\usepackage[utf8]{inputenc}
\input{auxiliar/tex/encabezado.tex}
\input{auxiliar/tex/tikzlibrarybayesnet.code.tex}
\newif\ifen
\newif\ifes
\newcommand{\en}[1]{\ifen#1\fi}
\newcommand{\es}[1]{\ifes#1\fi}
\entrue

\newcommand{\E}{\en{S}\es{E}}
\newcommand{\A}{\en{E}\es{A}}
\newcommand{\Ee}{\en{s}\es{e}}
\newcommand{\Aa}{\en{e}\es{a}}


\newtheorem{conclution}{\en{Conclution}\es{Conclusión}}%[section]
\newtheorem{objective}{\en{Objective}\es{Objetivo}}%[section]

%opening
\title{\vspace{-2.2cm}Multilevel selection in causal models: the multiplicative nature of evolutionary and probabilistic selection processes as the general driver for the irreversibility emergence of cooperation and specialization (Summary paper).\footnote{Add stable link to this summary file} }
\author{Gustavo Landfried$^1$}
\affil{\small 1. Universidad de Buenos Aires. Facultad de Ciencias Exactas y Naturales. Departamento de Computaci\'on. Buenos Aires, Argentina}
\affil{\en{Mail}\es{Correo:} \url{glandfried@dc.uba.ar}}

\begin{document}

\maketitle

\begin{abstract}

\en{To explain major evolutionary transitions, it is necessary to demonstrate the advantage of cooperation, specialization and their irreversibility. }%
\es{Para explicar las grandes transiciones evolutivas, es necesario demostrar la ventaja de la cooperación, la especialización y la irreversibilidad de las mismas. }%
%
\en{For this purpose it is necessary to consider selection at both the individual and group level. }%
\es{Para este propósito es necesario considerar selección tanto a nivel individual como a nivel grupal. }%
%
\en{The co-author of the concept of evolutionary transitions (Szathmáry) recently proposed to analyze the evolution of populations subject to multilevel selection by means of Bayesian hierarchical models, making use of the isomorphism between evolutionary theory and Bayesian inference. }%
\es{El co-autor del concepto de transiciones evolutivas (Szathmary) propuso recientemente analizar la evolución de las poblaciones sujetas a selección multinivel mediante modelos jerárquicos bayesianos, haciendo uso del isomorfismo entre las teoría de la evolución y la inferencia bayesiana. }%
%
\en{However, the proposal remains open. }%
\es{Sin embargo, la propuesta sigue abierta. }%
%
% \en{Moreover, the models do not express a causal relationship between the variables. }%
% \es{Además, los modelos no expresan una relación causal entre las variables. }%

% Parrafo

\en{In this paper we specify a probabilistic causal model, in which individuals are affected by the environment and by the social behaviors of cooperation and defection of their context. }%
\es{En este trabajo especificamos un modelo causal probabilístico, en el que los individuos se ven afectados por el ambiente y por los comportamientos sociales de cooperación y deserción de su contexto. }%
%
\en{Under this minimal set of hypotheses, where we consider \emph{unconditionally} cooperative individuals who generate a common good that can be exploited by defecting individuals without receiving some kind of punishment in return (e.g. end of cooperation), probabilistic inference shows that cooperative individuals are favored by multilevel selection. }%
\es{Bajo este conjunto mínimo de hipótesis, donde consideramos individuos \emph{incondicionalmente} cooperadores que generan un bien común que puede ser explotado por individuos desertores sin que reciban a cambio algún tipo de castigo (e.g fin de la cooperación), la inferencia probabilística muestra que los individuos cooperadoras se ven favorecidas mediante selección multinivel. }%
%
\en{In addition, we show that as soon as cooperation emerges, an advantage in favor of specialist strategies appears. }%
\es{Además, mostramos que apenas surge la cooperación, aparece una ventaja a favor de las estrategias especialistas. }%
%
\en{Since the specialist strategies are individually poorly adapted to the environment, an irreversibility of the evolutionary transition is created. }%
\es{Como la estrategias especialistas están individualmente mal adaptadas al ambiente, se crea una irreversbilidad de la transici evolutivas. }%

% Parrafo

\en{The reason why an advantage in favor of cooperation and specialization arises in our simple causal model is due to the multiplicative (non-ergodic) nature of probability theory and its isomorphism with evolutionary theory. }%
\es{El motivo por el cual surge una ventaja a favor de la cooperación y la especialización en simple modelo causal se debe a la naturaleza multiplicativa (no-ergódica) de la teoría de la probabililidad y a su isomorfismo con la teoría de la evolución. }%
% %
% \en{The multiplicative updating of the probabilities of individuals, which arises naturally from applying the rules of probability to the causal model, is in line with the long-established idea in evolutionary theory that the growth of lineages follow multiplicative processes. }%
% \es{La actualización multiplicativa de las probabilidades de los individuos, que surge naturalmente de aplicar las reglas de la probabilidad al modelo causal, está en línea con la idea largamente establecida en la teoría de la evolución de que el crecimiento de los linajes siguen procesos multiplicativos. }% 

\end{abstract}




\section{\en{Introduction}\es{Introducción}} \label{sec:introduction}

\en{In the last third of the history of the Universe, sometime around 4 billion years ago, a simple form of matter organization capable of self-replication appeared on Earth. }%
\es{En el último tercio de la historia del Universo, en algún momento hace aproximadamente 4000 millones de años, apareció en la tierra una forma de organización de la materia capaz de auto-replicarse. }%
%
\en{The growth of these lineages followed multiplicative and noisy processes: sequences of survival and reproduction rates. }%
\es{El crecimiento de estos linajes siguieron procesos multiplicativos y ruidosos: secuencias de probabilidades de supervivencia y reproducción. }%
%
\en{The errors produced during replication diversified the life forms, and the growth rates of the different strategies favored those better adapted to the environment. }%
\es{Los errores producidos durante la replicación diversificaron las formas de vida, y las tasas de crecimiento de las diferentes estrategias favorecieron a aquellas mejor adaptadas al ambiente. }%
%
\en{From that moment until now, life has acquired an extraordinary complexity, both in terms of cooperation and specialization. }%
\es{Desde aquel momento hasta ahora la vida adquirió una extraordinaria complejidad, tanto a nivel de cooperación como de especialización. }%

% Parrafo

\begin{figure}[ht!]
    \centering
    \begin{subfigure}[b]{0.65\textwidth}
    \includegraphics[width=\linewidth]{auxiliar/images/biomass.jpg}
    \end{subfigure}
    \caption{
    \en{Distribution of biomass on Earth estimated by Bar-On et al.~\cite{barOn2018-biomass}. }
	\es{Distribución de la biomasa en la Tierra estimada por Bar-On et al.~\cite{barOn2018-biomass}. }%
    }
    \label{fig:biomass}
\end{figure}
%
\en{The current complexity of life is the consequence of a series of evolutionary transitions in which entities capable of self-replication after the transition become part of higher level cooperative units~\cite{maynardSmith1995-majorTransitions, szathmary1995-evolutionaryTransitions, szathmary2015-evolutionaryTransitions}. }%
\es{La complejidad actual de la vida es consecuencia de una serie de transiciones evolutivas en las que entidades capaces de autoreplicación luego de la transición pasan a formar parte de unidades cooperativas de nivel superior~\cite{maynardSmith1995-majorTransitions, szathmary1995-evolutionaryTransitions, szathmary2015-evolutionaryTransitions}. }%
%
\en{Some of the paradigmatic transitions are: from prokaryotic to eukaryotic cells; from protozoa to animals, plants and fungi (cell differentiation and emergence of multicellular organisms); and from solitary individuals to societies. }%
\es{Algunas de las transiciones paradigmáticas son: de las células procariotas a las eucariotas; de los protozoa a los animales, plantas y hongos (diferenciación celular y emergencia de organiismos multicelulares); y de los individuos solitarios a las sociedades. }%
%
\en{How to explain this permanent tendency of life in favor of cooperative and specialization? }%
\es{¿Cómo se explica esta tendencia permanente de la vida en favor de la cooperación y la especialización? }%
%``The transition must be explained in terms of inmmediate selection advantage to individual replicators'' szathmary1995-evolutionaryTransitions

% Parrafo

\en{In evolution it is said that the growth of a lineage over time, $\omega(t)$, is governed by a stochastic sequence of survival and reproduction rates $f(\cdot)$ dependent on a random environment $\Aa$, }
\es{En evolución se dice que el crecimiento de un linaje en el tiempo, $\omega(t)$, esta gobernado por una secuencias estocástica de tasas de supervivencia y reproducción $f(\cdot)$ dependientes de un ambiente aleatorio $\Aa$, }%
%
\begin{equation} \label{eq:modelo_exponencial}
\omega(T) = \prod_t^T f(\Aa(t)) \approx g^T
\end{equation}
%
\en{where $\Aa(t)$ represents the state of the environment at time $t$ and $g$ represents the characteristic growth rate when $T$ is sufficiently large. }%
\es{donde $\Aa(t)$ representa el estado del ambiente en el tiempo $t$ y $g$ representa la tasa de crecimiento caracterísitica cuando $T$ es suficientemente grande. }%
%
\en{For example, suppose nature flips a coin, if it comes up heads the population reproduces 50\% and if it comes up tails it survives 60\%. }%
\es{Por ejemplo, supongamos que la naturaleza lanza una moneda, si sale cara la población se reproduce 50\% y si sale seca sobrevive 60\%. }%
\begin{equation} \label{eq:estrategia_base}
f(\Aa) =
\begin{cases}
 1.5 & \Aa = \text{ \en{Head}\es{Cara} } \\
 0.6 & \Aa = \text{ \en{Tail}\es{Sello} }
\end{cases}
\end{equation}
%
\en{A similar example was proposed by Lewontin and Cohen (1969)~\cite{lewontin1969-randomlyVaryingEnvironment},  in which the population reproduces 70\% or survives 50\%. }%
\es{Un ejemplo similar fue anlizado por Lewontin y Cohen (1969)~\cite{lewontin1969-randomlyVaryingEnvironment}, en el que la población se reproduce 70\% o sobrevive 50\%. }%
%
\en{Different strategies $\Ee$ can be described with different functions $f_\Ee(\Aa)$. }%
\es{Diferentes estrategias $\Ee$ las podemos describr con diferentes funciones $f_\Ee(\Aa)$. }%
%
\en{According to the standard model of evolution, known as \emph{replicator dynamic} \cite{taylor1978-replicatorDynamic}, the change in the proportion of a strategy in the population, $x_\Ee$, is determined by its characteristic growth rate $g_\Ee$, }%
\es{Según el modelo estándar de evolución, conocido como \emph{replicator dynamic} \cite{taylor1978-replicatorDynamic}, el cambio de la proporción de una estrategia en la población, $x_\Ee$, está determinado por su tasa de crecimiento caracterísitica $g_\Ee$, }%
% schuster1983-replicatorDynamics, hofbauer2003-evolutionaryGameDynamics
%
\begin{equation} \label{eq:replicator_dynamic}  \tag{Replicator dynamic}
\hspace{3cm} x_\Ee^\prime = \frac{x_\Ee g_\Ee}{\sum_i x_i g_i}
\end{equation}
%
\en{where the denominator acts as a normalization constant. }%
\es{donde el denominador actúa como constante de normalización. }%
%
\en{But, what is the characteristic growth rate $g$? }%
\es{¿Cuál es la tasa de crecimiento característica $g$? }%
%
\en{Much of the evolutionary literature bases its analysis on populations of infinite size and considers that the correct estimate is obtained by the expected value, $g^t = \langle \omega \rangle_t$. }%
\es{Buena parte de la literatura en evolución basa su análisis en poblaciones de tamaño infinito y considera que la estimación correcta se obtiene mediante el valor esperado, $g^t = \langle \omega \rangle_t$. }%
%
\begin{equation}
\langle \omega \rangle_t = \sum_{\omega \in \Omega_t} \omega \cdot  P(\omega)
\end{equation}
%
\en{Where $\Omega_t$ is the set of all possible resource trajectories at time $t$, and $P(\omega)$ is the probability that the resource state $\omega$ occurs. }%
\es{Donde $\Omega_t$ es el conjunto de todas las posibles trayectorias de los recursos en el tiempo $t$, y $P(\omega)$ es la la probabilidad de que ocurra el estado de los recursos $\omega$. }%
% 
\en{In the coin example, the expected value in the first two time steps is, }%
\es{En el ejemplo de la moneda, el valor esperado en los dos primeros pasos temporales es, }%
%
\begin{equation}
\begin{split}
\langle \omega_e \rangle_1 & = 1.5 \cdot \frac{1}{2} + 0.6 \cdot  \frac{1}{2} = 1.05 \\ 
\langle \omega_e \rangle_2 &=  1.5^2 \cdot \frac{1}{4} + 2 (0.6 \cdot 1.5 \cdot \frac{1}{4} ) + 0.6^2 \cdot \frac{1}{4}= 1.05^2
\end{split}
\end{equation}
%
\en{That is, the estimated growth rate according to the expected value is $5\%$ for each time step, $\langle \omega \rangle_t = 1.05^t$. }%
\es{Es decir, la tasa de crecimiento estimada según el valor esperado es de $5\%$ por cada paso temporal, $\langle \omega \rangle_t = 1.05^t$. }%
%
\en{And indeed that is what happens with the average of several individual trajectories, $\omega(t)$. }%
\es{Y efectivamente eso es lo que ocurre con el promedio de varias trayectoria individuales, $\omega(tThis theoretical paradigm is forced)$. }%
%
\begin{figure}[H]
    \centering
    \begin{subfigure}[b]{0.45\textwidth}
    \includegraphics[width=\linewidth]{figures/pdf/ergodicity_expectedValue.pdf}
    \end{subfigure}
    \caption{
    \en{Average of individual resources over time for different population sizes, in log scale. }%
    \es{Promedio de los recursos individuales en el tiempo para diferentes tamaños de la población, en escala logarítimica. }%
    %
    \en{As we increase the size of the population, the average approaches the expected value of $1.05^t$. }%
    \es{A medida que aumentamos el tamaño de la población, el promedio se acerca al valor esperado $\langle \omega \rangle_t = 1.05^t$. }%
    }
    \label{fig:ergodicity_expectedValue}
\end{figure}
%
\en{However, the expected value does not represent what happens to the agents over time. }%
\es{Sin embargo, el valor esperado no representa lo que le ocurre a los agentes en el tiempo. }%
%
\en{Individually, all the trajectories lose in the long term at a rate close to 5\%. }%
\es{Individualmente, todas las trayectorias pierden a largo plazo a una tasa cercana al 5\%. }%
%
\en{The trajectories observed in figure \ref{fig:ergodicity_individual_trayectories} are variable, but the longer we observe the system the smoother these lines become (figure \ref{fig:ergodicity_individual_trayectories_longrun}). }%
\es{Las trayectorias observadas en la figura \ref{fig:ergodicity_individual_trayectories} son variables, pero cuanto más tiempo observemos el sistema más suave se vuelven esas líneas (figura \ref{fig:ergodicity_individual_trayectories_longrun}). }%
%
\begin{figure}[H]
    \centering
    \begin{subfigure}[b]{0.45\textwidth}
    \includegraphics[width=\linewidth]{figures/pdf/ergodicity_individual_trayectories.pdf}
    \caption{}
    \label{fig:ergodicity_individual_trayectories}
    \end{subfigure}
    \begin{subfigure}[b]{0.45\textwidth}
    \includegraphics[width=\linewidth]{figures/pdf/ergodicity_individual_trayectories_longrun.pdf}
    \caption{}
    \label{fig:ergodicity_individual_trayectories_longrun}
    \end{subfigure}
    \caption{
    \en{The black line represents the expected value. }%
    \es{La recta negra representan el valor esperado. }%
    %
    \en{Figure \ref{fig:ergodicity_individual_trayectories}: size of individual resources over time, $ \log(\omega(t))$. }%
    \es{Figura \ref{fig:ergodicity_individual_trayectories}: tamaño de los recursos individuales en el tiempo, $ \log(\omega(t))$. }%
    %
    \en{Figure \ref{fig:ergodicity_individual_trayectories_longrun}: given enough time, all individual trajectories stick to the blue line. }% 
    \es{Figura \ref{fig:ergodicity_individual_trayectories_longrun}: con suficiente tiempo todas las trayectorias individuales se pegan a la recta azul. }% 
    }
    \label{fig:cpr_individual}
\end{figure}
%La relación entre el valor esperado y lo que le ocurre a los agentes individuales en el tiempo es un problema bien conocido en mecánica estadística.
\en{When the individual trajectories can be described by the expected value of the system states, then the process is said to be ergodic~\cite{peters2019-ergodicityEconomics}. }%
\es{Cuando lo que le ocurre a los agentes individuales en el tiempo puede describirse mediante el valor esperado de los estados del sistema, luego se dice que el proceso es ergódico~\cite{peters2019-ergodicityEconomics}. }%
%
\en{However, the conditions are very restrictive, and are not fulfilled in the case of multiplicative processes. }%
\es{Sin embargo, las condiciones para que esto se cumpla son muy restrictivas y no se satisfacen en el caso de los procesos multiplicativos. }%
%
\en{To compute the growth rate $g$, we first express the product as follows, }%
\es{Para calcular la tasa de crecimiento $g$, primero expresaramos la productoria de la siguiente manera, }%
%
\begin{equation}
\omega(T) = \prod^T_{t=1} f(\Aa(t)) = f(\text{\en{head}\es{cara}})^{n_1} f(\text{\en{tail}\es{sello}})^{n_2}
\end{equation}
%
\en{where $n_1$ and $n_2$ represents the number of occurrences of $f(\text{\en{head}\es{cara}})$ and $f(\text{\en{tail}\es{sello}})$, with $n_1 + n_2 = T$. }%
\es{donde $n_1$ y $n_2$ representa la cantidad de ocurrencias de $f(\text{cara})$ y $f(\text{seca})$, con $n_1 + n_2 = T$. }%
%
\en{In the limit, $T \rightarrow \infty$ all individual trajectories will be determined by the same growth rate $g$. }%
\es{En el límite, $T \rightarrow \infty$ todas las trayectorias individuales estarán determinadas por la misma tasa de crecimiento $g$. }%
%
\begin{equation} \label{eq:geometric_mean}
\begin{split}
\lim_{T \rightarrow \infty} \omega_e(T) & = {g}^T \\
\left( \lim_{T \rightarrow \infty} \omega_e(T) \right)^{1/T} & =  {g} \\
\lim_{T \rightarrow \infty} f(\text{cara})^{n_1/T} f(\text{seca})^{n_2/T} & 
 \end{split}
\end{equation}
%
\en{Where the frequencies $\frac{n_1}{T}$ and $\frac{n_2}{T}$ in the limit $T \rightarrow \infty$ are equal to the probabilities of the environmental states $p$. }%
\es{Donde las frecuencias $\frac{n_1}{T}$ y $\frac{n_2}{T}$ en el límite $T \rightarrow \infty$ son iguales a las probabilidades de los estados ambientales $p$. }%
%
\en{Therefore, the growth rate is, }%
\es{Por lo tanto, la tasa de crecimiento es, }%
%
\begin{equation}
g = (1.5 \cdot 0.6)^{1/2} \approx 0.95
\end{equation}
%
\en{This formula, which allows computing the long-term growth rate of individual trajectories, has previously been used in the evolution literature under the name \emph{geometric mean}~\cite{dempster1955-geometricMean}. }%
\es{Esta fórmula, que permite computar la tasa de crecimiento a largo plazo de las trayectorias individuales, ha sido usada previamente en la literatura de evolución bajo el nombre de \emph{media geométrica}~\cite{dempster1955-geometricMean}. }%
%
\en{The geometric mean is always less than the arithmetic mean (or expected value). }%
\es{La media geométrica es siempre es menor a la media aritmética (o valor esperado). }%
%
\en{This is because in multiplicative processes the physical impacts of losses are usually stronger than those of gains. }%
\es{Esto se debe a que en los procesos multiplicativos los impactos físicos de las pérdidas suelen ser más fuertes que los de las ganancias. }%
%
\en{In an extreme case, a single zero in the product is enough to generate an extinction. }%
\es{En un caso extremo, un único cero en la productoria alcanza para generar su extinción. }%

% Decimos que un proceso es ergódico si se cumple que,
% \begin{equation}
%  \underbrace{\lim_{T \mapsto \infty} \frac{1}{T} \sum_{t=1}^T \omega(t)}_{\text{Media temporal}}  = \underbrace{\sum_{\omega} \omega \cdot p(\omega)}_{\text{Media de estados}}
% \end{equation}
% 

\subsection{Coopera\en{tion}\es{ción}}

\en{As a consequence of the non-ergodicity of multiplicative processes, fluctuations have a negative effect on individual growth rates. }%
\es{Como consecuencia de la no-ergodicidad de los procesos multiplicativos, las fluctuaciones tienen un efecto negativo en las tasas de crecimiento individuales. }%
%
%Parafraseando a Den Boer~\cite{denBoer1968-spreadingRisk}, la  supervivencia de una población depende de la distribución del riesgo dentro de la población y entre las poblaciones de diferentes especies.
\en{Yaari-Solomon~\cite{yaari2010-cooperationEvolution} and Peters-Adamou~\cite{peters-cooperation2019.03.04} (hereinafter Yaari and Peters) study the consequences that the following cooperative strategy has on the agents' growth rate. }%
\es{Yaari-Solomon~\cite{yaari2010-cooperationEvolution} and Peters-Adamou~\cite{peters-cooperation2019.03.04} (de aquí en adelante Yaari y Peters) estudian las consecuencias que la siguiente estrategia cooperativa tiene sobre la tasa de crecimiento de los agentes. }%
%
\begin{figure}[H]
\centering
\scalebox{0.75}{
\tikz{

    \node[latent, minimum size=2cm ] (x1_0) {$\omega_1(t)$} ;
    \node[latent, below=of x1_0, minimum size=2cm ] (x2_0) {$\omega_2(t)$} ;

    \node[latent, right=of x1_0, minimum size=3cm ] (x1_0g) {$ \omega_1(t)\cdot f(\Aa_1(t))$} ;
    \node[latent, right=of x2_0, minimum size=1.8cm, xshift=0.6cm , align=left] (x2_0g) {$\omega_2(t)\cdot$\\$f(\Aa_2(t))$} ;
    
    \node[latent, right=of x1_0g, minimum size=3.8cm, yshift=-1.33cm, align=right] (x_0) {$\omega_1(t)\cdot f(\Aa_1(t))$\\$+\omega_2(t)\cdot f(\Aa_2(t))$ } ;
    
    \node[const, above=of x_0] (nx_0) {$\overbrace{\text{Pool}\hspace{2.5cm}\text{Share}}^{\text{\normalsize Coopera\en{tion}\es{ci\'on}}}$} ;
    
    \node[latent, right=of x1_0g, minimum size=2.5cm,  xshift=4.5cm] (x1_1) {$\omega_1(t+1)$ } ;
    \node[latent, below=of x1_1, minimum size=2.5cm, yshift=0.7cm] (x2_1) {$\omega_2(t+1)$ } ;
    
    \edge {x1_0} {x1_0g};
    \edge {x2_0} {x2_0g};
    \edge {x1_0g,x2_0g} {x_0};
    \edge {x_0} {x1_1,x2_1};
    
}
}
\caption{
\en{The agents start with the same initial resources, update them independently according to the equation \ref{eq:estrategia_base}, and finally redistribute it in equal parts. }%
\es{Los agentes comienzan con los mismos recursos iniciales, los actualizan independientemente de acuerdo con la ecuación \ref{eq:estrategia_base}, y finalmente lo redistribuyen en partes iguales. }%
}
\label{fig:protocolo}
\end{figure}
%
\en{In Figure \ref{fig:ergodicity_cooperation} we show the trajectory of an agent in a cooperating group of size 33. }%
\es{En la figura \ref{fig:ergodicity_cooperation} mostramos la trayectoria de un agente en un grupo cooperador de tamaño 33. }%
%
\begin{figure}[ht!]
    \centering
    \begin{subfigure}[b]{0.45\textwidth}
    \includegraphics[width=\linewidth]{figures/pdf/ergodicity_cooperation.pdf}
    \end{subfigure}
    \caption{
    \en{The resources of an individual in a cooperative group of size 33 (green line) approaches the arithmetic mean (the black line below the green line).
    As a visual reference we show the geometric mean that drives the growth rate of individuals (blue line). }%
    \es{Los recursos de un individuo en un grupo cooperativo de tamaño 33 (recta verde) se aproxima a la media aritmética (la recta negra debajo de la recta verde).
    Como referencia visual mostramos la media geométrica que gobierna la tasa de crecimiento de los individuos (recta azul). }%
    }
    \label{fig:ergodicity_cooperation}
\end{figure}
%
\en{Fully cooperative groups reduce their fluctuations, generating an increase in the growth rate of all their members. }%
\es{Grupos enteramente cooperadoras reducen sus fluctuaciones, generando un aumento en la tasa de crecimiento de todos sus miembros. }%
% 

\paragraph{\en{Multilevel selection}\es{Selección multinivel}}

\en{To demonstrate the evolutionary advantage of cooperation in the presence of defection it is necessary to consider selection at both the individual and group level. }%
\es{Para demostrar la ventaja evolutiva de la cooperción en presencia de desertores es necesario considerar selección tanto a nivel individual como a nivel grupal. }%
%
\en{The co-author of the concept of evolutionary transitions (Szathmáry~\cite{szathmary1995-evolutionaryTransitions, szathmary2015-evolutionaryTransitions}) recently proposed to analyze the evolution of populations subject to multilevel selection by means of Bayesian hierarchical models~\cite{czegel2019-bayesianEvolution}, making use of the isomorphism between evolutionary theory and Bayesian inference~\cite{harper2009-replicatorAsInference,shalizi2009-replicatorAsInference}. }%
\es{El co-autor del concepto de transiciones evolutivas (Szathmáry~\cite{szathmary1995-evolutionaryTransitions, szathmary2015-evolutionaryTransitions}) propuso recientemente analizar la evolución de las poblaciones sujetas a selección multinivel mediante modelos jerárquicos bayesianos~\cite{czegel2019-bayesianEvolution}, haciendo uso del isomorfismo entre las teoría de la evolución y la inferencia bayesiana~\cite{harper2009-replicatorAsInference,shalizi2009-replicatorAsInference}. }%
%
\en{However, this work does not provide any model that performs multilevel selection, so the proposal remains open. }%
\es{Sin embargo, este trabajo no proveé ningun modelo que realice selección multinivel, por lo que la propuesta sigue abierta. }%
% 
\en{In this paper we demonstrate the evolutionary advantage of cooperation in the presence of defection using a probabilistic causal model representing evolution under multilevel selection. }%
\es{En este trabajo demostramos la ventaja evolutiva de la cooperación en presencia de deserción mediante un modelo causal probabilistico que represente evolución bajo selección multinivel. }%
%
\en{To the best of our knowledge, our work would be the first to develop a Bayesian hierarchical model to solve an evolution problem under multilevel selection. }%
\es{Hasta donde sabemos, nuestro trabajo sería el primero en desarrollar un modelo jerárquico bayesiano para resolver un problema de evolución bajo selección multinivel. }%
%
% \en{With this model we demonstrate, as Ole Peters intended, that the advantage of cooperation (now in the presence of defection) is a consequence of the most basic and fundamental assumption of evolutionary theory: that evolutionary processes are multiplicative and noisy. }%
% \es{Con este modelo demostramos, como pretendía Ole Peters, de que la ventaja de la cooperación (ahora en presencia de deserción) es consecuencia del supuesto más básico y fundamental de la teoría de la evolución: que los procesos evolutivos son multiplicativos y ruidosos. }%

\section{\en{Methodology}\es{Metodología}}\label{sec:methodology}

\en{In this section we present the isomorphism between probabilitic and evolutionary theory. }%
\es{En esta sección presentamos el isomorfismo entre las teorías de la probabilidad y de la evolución. }%
%
\en{In section \nameref{sec:results} we will prove the evolutionary advantage of cooperation, specialization and their irreversibility using a Bayesian hierarchical model with a causal interpretation representing evolution under multilevel selection. }%
\es{En la sección \nameref{sec:results} demostraremos la ventaja evolutiva de la cooperación, la especialización y su irreversbilidad mediante un modelo jerárquico bayesiano con una interpretación causal que representa evolución bajo selección multinivel. }%
%

\subsection{\en{Probability theory and causal models}\es{Teoría de la probabilidad y modelos causales}}

\en{Probability theory is currently the most widely used approach for computing uncertainty. }%
\es{La teoría de la probabilidad es el enfoque más utilizado en la actualidad para computar la incertidumbre. }%
%
\en{Its rules have been derived from several axiomatic systems, conceptually distinct and independent of each other~\cite{halpern2017-RAU2}, which is one of the strong points in its favor. }%
\es{Sus reglas han sido derivadas a partir de varios sistemas axiomáticos, conceptualmente distintos e independientes entre sí,~\cite{halpern2017-RAU2}, lo cual es uno de los punto fuertes a su favor. }%
%
\en{But perhaps more importantly, its strict application ensures maximization of uncertainty given empirical and formal information (data and causal models)~\cite{jaynes2003-bookProbabilityTheory}, source of validation for empirical propositions. }%
\es{Pero quizás más importante es que su aplicación estricta garatiza la maximización de la incertidumbre dada la información empírica y formal (datos y modelos causales)~\cite{jaynes2003-bookProbabilityTheory}, fuente de validación para las proposiciones empíricas. }%

% Parrafo

\en{Probability theory can be summarized in two rules: the sum rule and the product rule. }%
\es{La teoría de la probabilidad puede resumirse en dos reglas: la regal de la suma y la regla del producto. }%
%
\en{The product rule allows describing multidimensional models through causal mechanisms: belief distributions on each effect, given the hidden values of its causes (conditional probabilities). }%
\es{La regla del producto permite describir modelos multidimensionales a través de mecanismos causales: distribuciones de creencias sobre cada uno de los efectos, dados los valores ocultos de sus causas (probabilidades condicionales). }%
%
\en{The sum rule, on the other hand, predicts the behavior of a variable (marginal probabilities) with the contribution of all the hypotheses of the multidimensional causal model. }%
\es{La regla de la suma, por su parte, predice el comportamiento de una variable (probabilidades marginales) con la contribución de todas las hipótesis contradictorias del modelo causal multidimensional. }%
% Parrafo
\en{Graphical models are a well-established tool in probability because they: 1. provide an intuitive language to unambiguously express all hypotheses, 2. reduce the dimensionality of the joint probability distribution, 3. and allows to perform inference efficiently through the sum-product algorithm~\cite{kschischang2001-factorGraphsAndTheSumProductAlgorithm}. }%
\es{Los modelos gráficos son una herramienta bien establecida en probabilidad debido a que: 1. ofrecen un lenguaje intuitivo para expresar sin ambigüedad todas las hipótesis; 2. reducen la dimensionalidad de la distribución de probabilidad conjunta; 3. y permiten realizar inferencia de forma eficiente a través del algoritmos suma-producto~\cite{kschischang2001-factorGraphsAndTheSumProductAlgorithm}. }%
\en{To use the graphical models no causal interpretation is required, in fact the conditional probabilities can be expressed in any order. }%
\es{Para utilizar los modelos gráficos no se requiere una interpretación causal, de hecho las probabilidades condicionales pueden expresar en cualquier orden. }%
 
% Parrafo

\en{There are several advantages when conditional probabilities are defined on the basis of causal interpretations~\cite{pearl2009-causality}. }%
\es{Son varias las ventajas que se consiguen al definir las probabilidades condicionales en base a interpretaciones causales~\cite{pearl2009-causality}. }%
%
\en{Science elaborates its theories on the basis of causal stories: stable and autonomous mechanisms that induce conditional probabilities between causes and effects. }%
\es{La ciencia elabora sus teorías en base a historias causales: mecanismos estables y autónomos que inducen probabilidades condicionales entre causas y efectos. }%
%
\en{In this sense, justifying conditional probabilities on the basis of a causal story is a more natural way of expressing what we know or believe about the world. }%
\es{En este sentido, justificar las probabilidades condicionales en base a una historia causal es una forma más natural de expresar lo que sabemos o creemos sobre el mundo. }%
%
\en{In addition, causal interpretation also modularizes conditional probabilities: eventual changes in any of the causal mechanisms locally affect the topology of the Bayesian network, allowing the effect of external interventions to be predicted with a minimum of additional information. }%
\es{Además, la interpretación causal también modulariza las probabilidades condicionales: eventuales cambios en alguno de los mecanismos causales afectan localmente la topología de la red bayesiana, permitiendo predecir el efecto de las intervenciones externas con un mínimo de información adicional. }%
%
\en{For these reasons, in this paper we will express the graphical models in terms of causal relationships. }%
\es{Por estos motivos, en este trabajo expresaremos los modelos gráficos en términos de relaciones causales. }%

\subsection{\en{Isomorphism between evolutionary and probability theories}\es{Isomorfismo entre las teorías de la evolución y la probabilidad}}

\en{From the product rule, we immediately obtain the Bayes' theorem, with which we can compute the uncertainty about the hidden hypothesis space given the data and the model: }%
\es{De la regla del producto obtenemos inmediatamente el teorema de Bayes con el que podemos computrar la incertidumbre sobre el espacio de hipótesis ocultas dados los datos y el modelo: }%
%
\begin{equation}\label{eq:bayes_theorem}
 \underbrace{P(\overbrace{\text{\en{Hypothesis}\es{Hipótesis}$_i$}}^{\text{\en{Hidden}\es{Oculta}}}|\overbrace{\text{Dat\en{a}\es{os}}}^{\text{Observ\en{ed}\es{ado}}})}_{\text{Posterior}} = \frac{\overbrace{P(\,\text{Dat\en{a}\es{os}}\,|\,\text{\en{Hypothesis}\es{Hipótesis}$_i$})}^{\text{\en{Likelihood}\es{Verosimilitud}}}\overbrace{P(\text{\en{Hypothesis}\es{Hipótesis}$_i$})}^{\text{Prior}}}{\underbrace{P(\text{Dat\en{a}\es{os}})}_{\text{Evidenc\en{e}\es{ia} o\en{r}\es{ predicci\'on a} prior \en{prediction}}}}
\end{equation}
%
\en{where the only free variable is the hypothesis $i$. The data and the model are fixed. }%
\es{donde la única variable libre es la hipótesis $i$. Los datos y el modelo están fijos. }%
%
\en{The likelihood and the evidence are both probabilities of the data , so they can be seen as predictions. }%
\es{La verosimilitud y la evidencia son ambas probabilidades de los datos, por lo que pueden ser vistas como predicciones. }%
%
\en{When the data is a discrete variable, predictions always take values between $0$ and $1$. }%
\es{Cuando el dato es una variable discreta, las predicciones siempre toman valores entre $0$ y $1$. }%
%
\en{Unlike the likelihood, which makes a different prediction for each hypothesis, the evidence averages all the predictions, weighted by the prior probability of the hypotheses, }%
\es{A diferencia de la verosimilitud, que realiza una predicción distinta por cada hipótesis, la evidencia realiza un promedio de todas las prediciones, pesado por la probabilidad a priori de las hipótesis, }%
%
\begin{equation}
P(\text{Dat\en{a}\es{os}}) = \sum_i P(\,\text{Dat\en{a}\es{os}}\,|\,\text{\en{Hypothesis}\es{Hipótesis}$_i$}) P(\text{\en{Hypothesis}\es{Hipótesis}$_i$})
\end{equation}
%
\en{The evidence then functions as a normalization constant. }%
\es{La evidencia funciona entonces como una constante de normalización. }%
%
\en{Therefore, the likelihood is the only factor that updates the probability distribution of the hypothesis, so the posterior is just the prior probability that is not filtered by the likelihood. }%
\es{Luego, la verosimilitud es lo único que actualiza la distribuión de probabilidad de la hipótesis, por lo que el posterior no es más que la probabilidad del prior no filtrada por la verosimilitud. }%

% Parrafo

\en{Recently, an isomorphism has been identified between the fundamental equations of evolutionary theory (replicator dynamic) and probability theory (Bayes theorem)~\cite{shalizi2009-replicatorAsInference,harper2009-replicatorAsInference}. }
\es{Recientememte se ha identificado un isomorfismo entre las ecuaciones fundamentales de la teoría de la evolución (replicator dynamic) y la teoría de la probabilidad (teorema de bayes)~\cite{shalizi2009-replicatorAsInference,harper2009-replicatorAsInference}. }%
%
\en{The isomorphism is obvious when we link the probability of the hypotheses to the proportion of the strategies in the population and the likelihood to their fitnesses, }%
\es{El isomorfismo es obvio cuando asociamos la probabilidad de las hipótesis con la proproción de las estrategias en la población y la verosimilitud con sus aptitudes, }%
%
\begin{align*}
\centering
 \begin{tabular}{l|l}
  \en{Bayes theorem}\es{Teorema de Bayes} & Replicator dynamic  \\ \hline
  Prior $P(H)$ & \en{Old distribution of strategies}\es{Distribución de estrategias} $x_\Ee$ \\ \hline
  \en{Likelihood}\es{Verosimilitud} $P(D|H)$ & \en{Fitness}\es{Aptitud} $f_\Ee(\Aa)$ \\ \hline
  Posterior $P(H|D)$ & \en{New distribution of strategies}\es{Nueva distribución de estrategias} $x_\Ee^\prime$ \\ \hline
  Evidenc\en{e}\es{ia} $P(D)$ & \en{Population mean fitness}\es{Aptitud media de la población} $\sum_\Ee x_\Ee f_\Ee(\Aa) $ \\ \hline
 \end{tabular}
\end{align*}
%
\en{Based on this isomorphism, Czégel, Zachar and Szathmáry recently proposed to analyze the evolution of a population subject to multilevel selection through hierarchical Bayesian modeling~\cite{czegel2019-bayesianEvolution}. }
\es{En base a este isomorfismo, Czégel, Zachar y Szathmáry propusieron recientemente analizar la evolución de una población sujeta a selección multinivel a través de modelo bayesianos jerárquicos~\cite{czegel2019-bayesianEvolution}. }%
%
\en{To do so, they display a series of graphical models without specifying the conditional probabilities. }%
\es{Para ello exhiben una serie de modelos gráficos sin especificar las probabilidades condicionales. }%
%
\begin{figure}[H]
\centering
\tikz{
    \node[latent] (c2) {$C^2$};
    \node[const, above=of c2, yshift=0.1cm] (pc2) {$P(c^2)$};
    \node[const, above=of pc2, yshift=0.1cm] (nc2) {Col\en{l}ectiv\en{e}\es{o} 2};
    \node[const, above=of nc2] (fc2) {\includegraphics[width=0.06\linewidth]{static/collective2A.png} \ \includegraphics[width=0.045\linewidth]{static/collective2B.png}};
    
    
    \node[latent, right=of c2, xshift=1cm] (c1) {$C^1$};
    \node[const, above=of c1, yshift=0.1cm] (pc1) {$P(c^1|c^2)$};
    \node[const, above=of pc1, yshift=0.1cm] (nc1) {Col\en{l}ectiv\en{e}\es{o} 1};
    \node[const, above=of nc1] (fc1) {\includegraphics[width=0.04\linewidth]{static/collective1A.png} \ \includegraphics[width=0.04\linewidth]{static/collective1B.png}};
    
    \node[latent, right=of c1, xshift=1cm] (i) {$I$};
    \node[const, above=of i, yshift=0.1cm] (pi) {$P(i|c^1,c^2)$};
    \node[const, above=of pi, yshift=0.1cm] (ni) {Individu\en{al}\es{o}};
    \node[const, above=of ni] (fi) {\includegraphics[width=0.09\linewidth]{static/individual.png}\ };
    
    \node[latent, right=of i, xshift=1cm] (a) { $\A$};
    \node[const, above=of a, yshift=0.1cm] (pa) {$P(\Aa|i,c^1,c^2)$};
    \node[const, above=of pa, yshift=0.1cm] (na) {\en{Environment}\es{Ambiente}};
    \node[const, above=of na] (fa) {\includegraphics[width=0.03\linewidth]{static/water.png}\includegraphics[width=0.04\linewidth]{static/wind.png}\includegraphics[width=0.025\linewidth]{static/fire.png} \ };
    
    \edge {c2} {c1};
    \edge {c1} {i};
    \edge {i} {a};
    \path[draw, ->, fill=black,sloped] (c2) edge[bend right,draw=black] node[midway,above,color=black] {} (i);
    \path[draw, ->, fill=black,sloped] (c2) edge[bend right,draw=black] node[midway,above,color=black] {} (a);
    \path[draw, ->, fill=black,sloped] (c1) edge[bend right,draw=black] node[midway,above,color=black] {} (a);
    
    \plate {aa} {(a)} {}; 
    }
\caption{
\en{Model proposed by Czégel, Zachar and Szathmáry~\cite{czegel2019-bayesianEvolution} to represent multilevel selection. }%
\es{Modelo propuesto por Czégel, Zachar y Szathmáry~\cite{czegel2019-bayesianEvolution} para representar selección multinivel. }%
%
\en{The box represents repetition of the variable. }%
\es{La caja representa repetición de la variable. }%
%
\en{Usually uppercase letters are variable names, and the lowercase letters are specific values. }%
\es{Usualmente las mayúsculas son nombres de variables, y las minúsculas son sus valores específicos. }%
}

\label{fig:czegel_et_el}
\end{figure}
%
\en{The validity of this model is trivial as the product rule always allows the following decomposition, }%
\es{La validez de este modelo es trivial en tanto la regla del producto siempre permite la siguente descomposición, }%
%
\begin{equation}
P(\Aa,i,c^1,c^2)= P(\Aa|i,c^1,c^2)P(i|c^1,c^2)P(c^1|c^2)P(c^2)
\end{equation}
%
\en{However, the model does not express the causal relationship between the variables. }%
\es{Sin embargo el modelo no expresa la relación causal entre las variables. }%
%
\en{This is evident in all conditional probability distributions. }%
\es{Esto se constata en todas las distribuciones de probabilidad condicional. }%
%
\en{At one extreme, the level 2 collective $C^2$, which is composed of both the level 1 collective and individuals, appears in the model as an independent variable. }%
\es{En un extremo, el colectivo de nivel 2 $C^2$, que está compuesta tanto por el colectivo de nivel 1 y por los individuos, aparece en el modelo como una variable independiente. }%
%
\en{At the other extreme, the environment $\A$, which is usually considered in evolutionary causal models as an independent random variable, appears in this model as dependent on individuals and collectives. }%
\es{En el otro extremo, el ambiente $\A$, que suele considerarse en los modelos causales evolutivos como una variable aleatoria independiente, aparece en este modelo como dependiente de los individuos y los colectivos. }%
%
\en{Conversely, all collectives and individuals appear independent of the environment. }%
\es{Y a la inversa, todas los colectivos e individuos aparecen independientes del ambiente. }%
%
% \en{This perhaps explains why the authors avoided specifying conditional probability distributions, and instead provided a pictorial example of the joint probability distribution. }%
% \es{Esto quizás explique porque los autores evitaron especificar las distribuciones de probabilidad condicional, y ofrecieron en cambio un ejemplo pictórico de la distribución de probabilidad conjunta. }%

\subsection{\en{A basic causal model: advantage for generalist individuals}\es{Un modelo causal básico: ventaja para los individuos generalistas}}

\en{The basic causal model we present here serves to introduce the methodology we will use in section \nameref{sec:results}, and the conclusion we will draw from it will be of interest later on. }%
\es{El modelo causal básico que presentamos aquí sirve para introducir la metodología que utilizaremos en la sección \nameref{sec:results}, y la conclusión que de él saquemos será de interés más adelante. }%
%
\en{In them, the binary environmental states depend on a probability $p$ (for example $p$ is the probability of the coin coming up heads). }%
\es{En ellos, los estados ambientales binarios dependen de un probabilidad $p$ (por ejemplo $p$ es la probabilidad de la moneda de salir cara). }%
%
\begin{equation}
P(\Aa) = p^{\Aa} (1-p)^{(1-\Aa)}
\end{equation}
%
\en{While we are free to propose any set of strategies, we will consider only those that allocate a limited resource $R$ among the binary states of the environment, with $ 0 \leq \Ee \leq R$. }%
\es{Si bien tenemos la libertad de proponer cualquier conjunto de estrategias, vamos a considerar sólo aquellas que reparten un recurso limitado $R$ entre los estados binarios del ambiente, con $ 0 \leq \Ee \leq R$. }%
%
\en{The strategy proposed by Lewontin-Cohen is $\Ee=1.7$ with $R=2.2$, and the strategy proposed by Peters is $\Ee=1.5$ with $R=2.1$. }%
\es{La estrategia propuesta por Lewontin-Cohen es $\Ee=1.7$ con $R=2.2$, y la de Peters es $\Ee = 1.5$ con $R=2.1$. }%
%
\en{To express all this strategies with a single parameter, we will work only with the family of fitnesses in which $R=1$, $f(\Ee, \Aa)$. }%
\es{Para expresar las estrategias con un sólo parámetro, vamos a trabajar solamente con la familia de aptitudes en la que $R=1$, $f(\Ee, \Aa)$. }%
%
\begin{equation}\label{eq:familia_de_aptitudes}
f(\Ee, \Aa) = \begin{cases}
 \Ee & \Aa = 1 \\
 1-\Ee & \Aa = 0
  \end{cases}
\end{equation}
%
\en{Now, all strategies lie between 0 and 1: the strategy $\Ee = 1.5/2.1 \approx 0.71$ is the one proposed by Peters, and the strategy $\Ee = 1.7/2.2 \approx 0.77$ is the one proposed by Lewontin-Cohen. }%
\es{Ahora, todas las estrategias están entre 0 y 1: la estrategia $\Ee = 1.5/2.1 \approx 0.71$ es la propuesta por Peters, y la estrategia $\Ee = 1.7/2.2 \approx 0.77$ es la propuesta por Lewontin-Cohen. }%
%
\en{For other ways of defining this same function, see Yaari-Solomon equation 11~\cite{yaari2010-cooperationEvolution}. }%
\es{Para ver otras formas de definir esta misma función, ver la ecuación 11 de Yaari-Solomon~\cite{yaari2010-cooperationEvolution}. }%
%
\en{To define the probability of the strategies given the environment, $P(\Ee|\Aa)$, we need the set of strategies to integrate 1. }%
\es{Para definir la probabilidad de las estrategias dado el ambiente, $P(\Ee|\Aa)$, necesitamos que el conjunto de estrategias integren 1. }%
%
\en{Let's say, for now, that the probability of the strategy is proportional to its fitness, }%
\es{Digamos, por ahora, que la probabilidad de la estrategia es proporcional a su aptitud, }%
%
\begin{equation}\label{eq:probabilidad_propro_aptitud}
P(\Ee | \Aa) \propto  \Ee^{\Aa}(1-\Ee)^{1-\Aa} = f(\Ee,\Aa)
\end{equation}
%
\en{This allows us to partially define a graphical model with a causal interpretation, }%
\es{Esto nos permite definir parcialmente un modelo gráfico con una interpretación causal, }%
%
\begin{figure}[ht!]
\centering
\tikz{
    \node[latent] (e) {$\A_t$};
    \node[const, right=of e] (en) {\ $P(\Aa) = p^{\Aa} (1-p)^{(1-\Aa)}$};
    \node[const, left=of e] (ne) {\en{Environment}\es{Ambiente}: \ \ \ };
    
    
    \node[latent, below=of e] (r) {$\E$};
    \node[const, right=of r] (rn) {$P(\Ee|\Aa) \propto f(\Ee,\Aa)$};
    \node[const, left=of r] (nr) {\en{Strategy}\es{Estrategia}: \ \ \ };
    
    \edge {e} {r};
    \plate {ee} {(e)} {$t$}; 
    }
\caption{
 \en{Causal model in which the environment affects individuals. }%
 \es{Modelo causal en el que el ambiente afecta a los individuos. }%
 }
\label{fig:model-lewontin-peters}
\end{figure}
%
\en{in which the environment is an independent random variable and the strategies depends on the observed states of the environment. }%
\es{en el que el ambiente es una variable aleatoria independiete y las estrategias dependen de los estados observados del ambiente. }%
%
\en{The result of the posterior will indicate the evolutionary stability of the strategies. }%
\es{El resultado del posterior nos indicará la estabilidad evolutiva de las estrategias. }%

% Parrafo

\en{Suppose that the environmental states are generated with probabilities $P(\A = 1) = 0.71$ and $P(\A = 0) = 0.29$. }%
\es{Supongamos que los estados del ambiente se generan con probabilidades $P(\A = 1) = 0.71$ y $P(\A = 0) = 0.29$. }%
%
\en{In Figure~\ref{fig:estrategias_individuales} we show how the posterior of the strategies changes as we add observations to the model. }%
\es{En la figura~\ref{fig:estrategias_individuales} mostramos cómo cambia el posterior de las estrategias a medida que agregamos observaciones al modelo. }%
%
\begin{figure}[H]
    \centering
    \begin{subfigure}[b]{0.32\textwidth}
    \includegraphics[width=\linewidth]{figures/coin1.pdf}
    \caption{$T = 1$}
    \end{subfigure}
    \begin{subfigure}[b]{0.32\textwidth}
    \includegraphics[width=\linewidth]{figures/coin2.pdf}
    \caption{$T = 10$}
    \end{subfigure}
    \begin{subfigure}[b]{0.32\textwidth}
    \includegraphics[width=\linewidth]{figures/coin3.pdf}
    \caption{$T = 10^5$}
    \end{subfigure}
    \caption{
    \en{Posterior probability of strategies as time progresses ($T=1, \, T=10, \, T=10^5$). }%
    \es{Probabilidad posterior de las estrategias a medida que avanza el tiempo ($T=1, \, T=10, \, T=10^5$). }%
    }
    \label{fig:estrategias_individuales}
\end{figure}
%
\en{The evolutionary process selects the most generalist individual strategy of all, the one that distributes resources in the same proportion as the states of the environment are generated. }%
\es{El proceso evolutivo selecciona la estrategia individual más generalista de todas, la que reparte los recursos en la misma proporción que se generan los estados del ambiente. }%
%
\en{When the environment generates the states with probability $p=0.71$, then the best adapted strategy is $\Ee = 0.71$. }%
\es{Cuando el ambientes genera los estados con una probabilidad $p=0.71$, entonces la estrategia mejor adaptada es $\Ee = 0.71$. }%
%
\en{The selection of individual strategies, now intuitive, will be different and perhaps counter-intuitive when we incorporate the possibility of cooperation and defection into the model. }%
\es{La selección de estrategias individuales, ahora intuitiva, será diferente y quizás contra-intuitiva cuando incorporemos al modelo la posibilidad de cooperación y deserción. }%

\section{\en{Results}\es{Resultados}}\label{sec:results}

% \en{In this section we will analyze whether there is indeed any evolutionary advantage of cooperation and specialization in the presence of defection. }%
% \es{En esta sección analizaremos si efectivamente existe alguna ventaja evolutiva de la cooperación y la especialización en presencia de deserción. }%
% %
\en{In this section we will extend the causal model proposed in the \nameref{sec:methodology} section by incorporating unconditionally cooperative behaviors, such as the one presented in the \nameref{sec:introduction} section, including defecting behaviors (which receive the benefit of cooperation without contributing to it). }%
\es{En esta sección extenderemos el modelo causal propuesto en la sección \nameref{sec:methodology} incorporando comportamientos incondicionalmente cooperativos, como el presentado en la sección \nameref{sec:introduction}, incluyendo comportamientos desertores (que reciben el beneficio de la cooperación sin aportar al mismo). }%


\subsection{\en{Extended causal model}\es{Modelo causal extendido}}

\en{Individuals are spatially distributed and can interact only with members of the same region. }%
\es{Los individuos están distribuidos en el espacio y pueden interactuar solamente con los miembros de la misma región. }%
%
\en{Suppose we have regions of $N$ individuals in which $n$ are cooperators and $N-n$ defectors. }%
\es{Supongamos que tenemos regiones de $N$ individuos en el que $n$ son cooperadores y $N-n$ desertores. }%
%
\en{Then, there are $N+1$ types of possible regions, from $n=0$ (all defectors) to $n=N$ (all cooperators). }%
\es{Luego, hay $N+1$ tipos de regiones posibles, desde $n=0$ (todos desertores) hasta $n=N$ (todos cooperadores). }%
%
\en{Then, if each individual belongs to a single region we need $M = N(N+1)$ total individuals, $i \in \{1, \dots, M\}$. }%
\es{Luego, si cada individuo pertenece a una única región necesitamos $M = N(N+1)$ individuos totales, $i \in \{1, \dots, M\}$. }%
%
\en{Individuals $i$ are characterized by three attributes: }%
\es{Los individuos $i$ están caracterizados por tres atributos: }%
%
\en{the region to which it belongs, $\texttt{region}(i)=i \ \texttt{div} \ N$; }%
\es{la región a la que pertenece, $\texttt{region}(i)=i \ \texttt{div} \ N$; }%
%
\en{and if their social behavior is cooperative, $\texttt{coop}(i) =  i \ \texttt{mod} \ N < \texttt{region}(i)$ (the first $n$ individuals in the region are cooperators and the rest defectors); }%
\es{y si su comportamiento social es cooperador, $\texttt{coop}(i) =  i \ \texttt{mod} \ N < \texttt{region}(i)$ (los primeros $n$ individuos de la región son cooperadores y el resto desertores); }%
%
\en{and the strategy $\Ee_i$ they use to allocate resources $f(\Ee,\Aa)$ (equation \ref{eq:familia_de_aptitudes}). }%
\es{y la estrategia $\Ee_i$ que utilizan para alocar recursos $f(\Ee,\Aa)$ (ecuación~\ref{eq:familia_de_aptitudes}). }%
%
\en{By combinatorics, it will be more likely a priori that individuals will inhabit mixed regions. }%
\es{Por combinatoria, será más probable a priori que los individuos habiten regiones mixtas. }%
%
\begin{equation}
P(i) = \frac{1}{N} \mathcal{B}(\texttt{region}(i)|N,0.5) 
\end{equation}
%
\en{where the binomial distribution $\mathcal{B}$ gives more weight to individuals from mixed regions, the parameter $0.5$ indicates a uniform prior between cooperating and defecting social behaviors, and the factor $\frac{1}{N}$ indicates that all individuals in the same region start with the same initial resources. }%
\es{donde la distribución binomial $\mathcal{B}$ otorga más peso a los individuos de las regiones mixtas, el parámetro $0.5$ indica un prior uniforme entre los comportamientos sociales cooperador y desertor, y el factor $\frac{1}{N}$ indica que todos los individuos de la misma región empiezan con los mismos recursos iniciales. }%
%
\en{As we have seen in the section \ref{sec:methodology}, the probability of the environmental states is, }%
\es{Como hemos visto en la sección \ref{sec:methodology}, la probabilidad de los estados ambientales es, }%
%
\begin{equation}
P(\Aa) = p^\Aa (1-p)^{1-\Aa}
\end{equation}
%
\en{but now at each time $t$ we have a vector $\vec{\Aa}$, in which the $i-$th environmental state influences the $i-$th individual, }%
\es{pero ahora en cada tiempo $t$ tenemos un vector $\vec{\Aa}$, en el que el $i-$ésimo estado ambiental influencia al $i-$ésimo individuo, }%
%
\begin{equation}
P(i|\vec{\Aa}) = \frac{\Ee_i^{\Aa_i} (1-\Ee_i)^{1-\Aa_i}}{\sum_j \Ee_j^{\Aa_j} (1-\Ee_j)^{1-\Aa_j}} \propto \Ee_i^{\Aa_i} (1-\Ee_i)^{1-\Aa_i}
\end{equation}
%
\en{Now, however, the agents are also influenced by the resources of the other agents at the previous time, }%
\es{Ahora sin embargo, los agentes también están influenciados por los recursos del los otros agentes en el tiempo anterior, }%
%
\begin{equation}
P(i^{t+1}|i^t) = 
\begin{cases}
1/N & \texttt{coop}(i^t) \wedge (\texttt{region}(i^{t+1}) = \texttt{region}(i^{t})) \\
1 & \neg \texttt{coop}(i^t) \wedge i^{t+1} = i^t \\
0 & \texttt{else} \\
\end{cases}
\end{equation}
%
\en{Cooperating individuals divide the wealth equally with members of the same region, and defecting individuals retain all their wealth. }%
\es{Los individuos cooperadores dividen la riqueza en partes iguales con los miembros de la misma región, y los individuos desertores se quedan con toda su riqueza. }%
%
\en{Finally, the groups are made up of members from each region, $P(g|i)=\mathbb{I}(\texttt{region}(i) = g)$, the indexing function which is $1$ when individuals belong to region $g$ and $0$ otherwise. }%
\es{Finalmente, los grupos están constituidos por los miembros de cada región, $P(g|i)=\mathbb{I}(\texttt{region}(i) = g)$, la función indiciadora que es $1$ cuando los individuos pertenecen a la región $g$ y $0$ en otro caso. }%
%
\begin{figure}[ht!]
\centering
 \begin{subfigure}[b]{0.4\textwidth}    
 \centering
 \tikz{
    
    \node[obs] (a1) {$\vec{\A}^{\,1}$};
    \node[obs, right=of a1] (a2) {$\vec{\A}^{\,2}$};
    
    \node[latent, below=of a1 ] (i1) {$I^1$};
    \node[latent, below=of a2 ] (i2) {$I^2$};
    \node[latent, right=of i2 ] (i3) {$I^3$};

    \node[latent, below=of i1 ] (c1) {$G^1$};
    \node[latent, below=of i2 ] (c2) {$G^2$};
    \node[latent , below=of i3 ] (c3) {$G^3$};
    
    \node[invisible, below=of c2, yshift=-0.5cm] (inv) {};
    
    
    \edge {a1} {i1};
    \edge {a2} {i2};
    \edge {i1} {i2,c1};
    \edge {i2} {i3,c2};
    \edge {i3} {c3};
    
    }
 \caption{\en{Bayesian network}\es{Red bayesiana}}
 \label{fig:red_bayesiana_multinivel}
 \end{subfigure}
\begin{subfigure}[b]{0.58\textwidth}    
 \centering
 \tikz{
    
    \node[factor] (fa1) {};
    \node[obs, yshift=0.5cm, below=of fa1] (a1) {$\vec{\A}^{\,1}$};
    
    \node[factor, yshift=0.5cm, below=of a1] (fia1) {};
    
    \node[latent, yshift=0.5cm, below=of fia1 ] (i1) {$I^1$};
    
    \node[factor, xshift=0.3cm, left=of i1 ] (fi0) {};
    
    \node[factor, xshift=-0.3cm, right=of i1] (fii1) {};
    \node[factor, yshift=0.5cm, below=of i1] (fg1) {};
    \node[latent, yshift=0.5cm, below=of fg1 ] (g1) {$G^1$};
    
    \node[latent, xshift=-0.3cm, right=of fii1 ] (i2) {$I^2$};
    
    \node[factor, yshift=-0.5cm, above=of i2] (fia2) {};
    \node[obs, yshift=-0.5cm, above=of fia2] (a2) {$\vec{\A}^{\,1}$};
    \node[factor, yshift=-0.5cm, above=of a2] (fa2) {};
    \node[factor, yshift=0.5cm, below=of i2] (fg2) {};
    \node[latent, yshift=0.5cm, below=of fg2 ] (g2) {$G^2$};
    
    \node[factor, xshift=-0.3cm, right=of i2] (fii2) {};
    
    \node[latent, xshift=-0.3cm, right=of fii2 ] (i3) {$I^3$};
    \node[factor, yshift=0.5cm, below=of i3] (fg3) {};
    \node[latent, yshift=0.5cm, below=of fg3 ] (g3) {$G^3$};
    
     \node[const, left=of fa1] (pa) {$P(\vec{\Aa})$};
     \node[const, left=of fia1] (pea) {$P(i|\vec{\Aa})$};
     \node[const, left=of fg1] (pg) {$P(g|i)$};
     \node[const, above=of fii2] (pee) {$P(i^{t+1}|i^t)$};
     \node[const, left=of fi0] (pa) {$P(i^1)$};
    
    
    \edge[-] {a1} {fa1, fia1};
    \edge[-] {a2} {fa2, fia2};
    \edge[-] {i1} {fii1,fg1,fia1,fi0};
    \edge[-] {i2} {fii1, fii2,fg2,fia2};
    \edge[-] {i3} {fii2, fg3};
    \edge[-] {g1} {fg1};
    \edge[-] {g2} {fg2};
    \edge[-] {g3} {fg3};
    }
 \caption{Factor graph}
 \label{fig:factor_graph_multinivel}
 \end{subfigure}
 \caption{
 \en{Hierarchical model. }%
 \es{Modelo jerárquico. }%
 %
 \en{Figure~\ref{fig:red_bayesiana_multinivel}: the probabilistic dependencies arising from the multilevel causal model. }%
 \es{Figura~\ref{fig:red_bayesiana_multinivel}: las dependencias probabilísiticas que surge del modelo causal multinivel. }%
 %
 \en{Figure~\ref{fig:factor_graph_multinivel}: the factor graph induced by the Bayesian network, which will be used to apply the sum-product algorithm. }%
 \es{Figura~\ref{fig:factor_graph_multinivel}: el grafo de factores inducido por la red bayesiana, que será utilizado para aplicar el algoritmo suma-producto. }%
 %
 \en{The gray variables are observed. }%
 \es{Las variables en gris se consideran observables. }%
 }
\label{fig:multilevel_model}
\end{figure}
%
\en{In short, at each time $t$ the environment influences individuals, and individuals influence the groups of their own time and the individuals of time $t+1$. }%
\es{En resumen, en cada tiempo $t$ el ambiente influencia a los individuos, y los individuos influencian a los grupos de su propio tiempo y a los individuos del tiempo $t+1$. }%
%
\en{This model has $3$ hyperparameters: the vector of strategies $\vec{\Ee}$, the probability of the environment $p$, and the size of the groups $N$. }%
\es{Este modelo tiene $3$ hiperparámetros: el vector de estrategias $\vec{\Ee}$, la probabilidad del ambiente $p$, y el tamaño de los grupos $N$. }%
%
%Este modelo se puede simplificar levemente, usando una única variable de grupo que dependa de la última variable individuo, gracias a que la margnial de los grupos es la misma en cualquier tiempo $t$.

% Parrafo

\en{The evolutionary problem we are interested in is the selection of cooperative individuals given the environment within each group (level 1), $P(\texttt{coop}(i^T)|\vec{\Aa}^{\,1}, \dots, \vec{\Aa}^{\,T-1}, g)$, the selection of groups given the environment (level 2), $P(g^T|\vec{\Aa}^{\,1}, \dots, \vec{\Aa}^{\,T-1})$,  and selection of cooperative individuals given the environment for all groups (multilevel), $P(\texttt{coop}(i^T)|\vec{\Aa}^{\,1}, \dots, \vec{\Aa}^{\,T-1})$. }%
\es{El problema evolutivo que nos interesa es la selección de los individuos cooperadores dado el ambiente dentro de cada grupo (nivel 1), $P(\texttt{coop}(i^T)|\vec{\Aa}^{\,1}, \dots, \vec{\Aa}^{\,T-1}, g)$, la selección de los grupos dado el ambiente (nivel 2), $P(g^T|\vec{\Aa}^{\,1}, \dots, \vec{\Aa}^{\,T-1})$, y la selección de los individuos cooperadores dado el ambiente integrando todos los grupos (multinivel), $P(\texttt{coop}(i^T)|\vec{\Aa}^{\,1}, \dots, \vec{\Aa}^{\,T-1})$. }%
%
\en{The multilevel selection is obtained by integrating the product of level 1 and 2 selections, }%
\es{La selección multinivel se obtiene integrando el producto de las selecciones de nivel 1 y 2, }%
%
\begin{equation}\label{eq:posterior_multinivel}
\underbrace{P(\texttt{coop}(i^T)|\vec{\Aa}^{\,1}, \dots, \vec{\Aa}^{\,T-1})}_{\text{\en{Multilevel selection}\es{Selección multinivel}}} = \sum_{g=0}^N \underbrace{P(\texttt{coop}(i^T)|\vec{\Aa}^{\,1}, \dots, \vec{\Aa}^{\,T-1}, g)}_{\text{\en{Level 1 selection}\es{Selección de nivel 1}}} \cdot \underbrace{P(g^T|\vec{\Aa}^{\,1}, \dots, \vec{\Aa}^{\,T-1})}_{\text{\en{Level 2 selection}\es{Selección de nivel 2}}}
\end{equation}
%
\en{An alternative way to compute the multilevel selection marginal (Eq.~\ref{eq:posterior_multinivel}) is by integrating the probability of all cooperating individuals given the environments }%
\es{Una forma alternativa de calcular la marginal de la selección multinivel (Eq.~\ref{eq:posterior_multinivel}) es integrando la probabilidad de los individuos cooperadores dado los ambientes }%
%
\begin{equation}\label{eq:posterior_multinivel_alternativa}
\begin{split}
P(\texttt{coop}(i^T)|\vec{\Aa}^{\,1}, \dots, \vec{\Aa}^{\,T-1}) &= \sum_{j=1}^M P(I^T=j|\vec{\Aa}^{\,1}, \dots, \vec{\Aa}^{\,T-1})\mathbb{I}(\texttt{coop}(j))
\end{split}
\end{equation}
%
%
\en{where $\mathbb{I}(\cdot)$ is the indicator function. }%
\es{donde $\mathbb{I}(\cdot)$ es la función indiciadora. }%
%
\en{Then}\es{Luego} (\en{all steps are described in the long article}
\es{en artículo largo mostramos todos los pasos})
%
\begin{equation}
P(k| \vec{\Aa}^{\,1}, \dots, \vec{\Aa}^T) = 
\begin{cases}
P(k)\prod_{t=1}^{T} P(k|\vec{\Aa}^{\,t}) &  r=0  \\
P(k)\prod_{t=1}^{T} \sum_j^{\texttt{\en{partners}\es{socios}}(r)} \frac{1}{N} P(j|\vec{\Aa}^{\,t}) & r  = N  \\
\Big(P(k)\prod_{t=1}^{T} P(k|\vec{\Aa}^{\,t}) \Big) + \Big(\sum_{t=1}^{T} P(c|\wedge_{q=1}^t\vec{\Aa}^{\,q})  \prod_{q=t+1}^T P(k|\vec{\Aa}^{\,q}) \Big) & 0 < r < N  
\end{cases}
\end{equation}
%
\en{where $r = \texttt{region}(k)$ is the region to which individual $k$ belongs, $\texttt{partners}(r)$ is the set of cooperating individuals in region $r$, and $c$ is a cooperative individual belonging to that specific region, $c \in \texttt{partners}(r)$. }%s
\es{donde $r = \texttt{region}(k)$ es la región a la que pertenece el individuo $k$, $\texttt{\en{partners}\es{socios}}(r)$ es el conjunto de individuos cooperadores de la región $r$, y $c$ es un individuo cooperador de esa región, $c \in \texttt{\en{partners}\es{socios}}(r)$. }%

\subsection{\en{The multiplicative nature of evolutionary and probability theories}\es{La naturaleza multiplicativa de las teorías de la evolución y la probabilidad}}

\en{Even if our causal model does not assume any kind of process to update resources, the result of the inference is proportional to the resources obtained through a multiplicative process. }%
\es{Incluso si nuestro modelo causal no presupone ningún tipo de proceso para actualizar los recursos, el resultado de la inferencia es proporcional a los recursos obtenidos a través de un proceso multiplicativo. }%
% \en{The proposed causal model does not require assuming that the resource updating process is multiplicative, as proposed by Ole Peters for his game. }%
% \es{El modelo causal propuesto no requiere suponer que el proceso de actualización de los recursos sea multiplicativo, como propone Ole Peters para su juego. }%
% %
\en{All the assumptions of our model are made explicit in the Bayesian network in Figure~\ref{fig:multilevel_model}. }
\es{Todos los supuestos de nuestro modelo se encuentran explicitados en la red bayesiana de la figura~\ref{fig:multilevel_model}. }%
%
\en{Individuals are affected just by the environment and by the social behaviors of cooperation and defection of their context. }%
\es{Los individuos se ven afectados solamente por el ambiente y por los comportamientos sociales de cooperación y deserción de su contexto. }%
%
\en{However, our probabilistic causal model and the multiplicative process discussed in the introduction are equivalent. }%
\es{Sin embargo, nuestro modelo causal probabilístico y el proceso multiplicativo analizado en la introducción son equivalentes. }%
%
\en{In general, the posterior of individuals is no more than the proportion of resources they manage (all steps are described in the long article). }%
\es{En general, el posterior de los individuos no es más que la proporción de recursos que maneja (en artículo largo mostramos todos los pasos). }%
%
\begin{equation}
P(k| \vec{\Aa}^{\,1}, \dots,  \vec{\Aa}^{\,T}) = \frac{\omega_k(T)}{\sum_j \omega_j(T)}
\end{equation}
%
\en{The proportionality between the posterior probability of individuals and the trajectories of resources studied in the introduction section allows us to work with both expressions interchangeably. }%
\es{La proporcionalidad entre la probabilidad a posteriori de los individuos y las trayectorias de los recursos estudiada en la instroducción nos permite trabajar indistintamanete con una u otra expresión. }%

% Parrafo

\en{The multiplicative updating of the probabilities of individuals, which arises naturally from applying the rules of probability to the causal model, is in line with the long-established idea in evolutionary theory that the growth of lineages follow multiplicative processes~\cite{dempster1955-geometricMean, denBoer1968-spreadingRisk}. }%
\es{La actualización multiplicativa de las probabilidades de los individuos, que surge naturalmente de aplicar las reglas de la probabilidad al modelo causal, está en línea con la idea largamente establecida en la teoría de la evolución de que el crecimiento de los linajes siguen procesos multiplicativos~\cite{dempster1955-geometricMean, denBoer1968-spreadingRisk}. }% 
%
\en{This coincidence is an additional support to the hypothesis of isomorphism between evolutionary and probability theories, previously identified between their fundamental equations: the Bayes theorem and the replicator dynamic~\cite{harper2009-replicatorAsInference,shalizi2009-replicatorAsInference}. }%
\es{Esta coincidecia es un apoyo adicional a la hipótesis de isomorfimo entre las teorías de la evolutivas y probabilisticas, previamente identificado entre sus ecuaciones fundamentales: el teorema de bayes y el replicator dynamic~\cite{harper2009-replicatorAsInference,shalizi2009-replicatorAsInference}. }%

\subsection{\en{Commons dilemma}\es{Dilema de los bienes comunes}}

\en{In the last few decades evolutionary biology has begun to adopt the analogy of the ``tragedy of the commons''~\cite{rankin2007-tragedyCommonsBiology}. }%
\es{En las últimas décadas la biología evolutiva ha comenzado a adoptar la analogía de la ``tragedia de los comunes''~\cite{rankin2007-tragedyCommonsBiology}. }%
%
\en{This concept contains the idea that the commons has a payoff structure isomorphic to the N-player prisoner's dilemma~\cite{hardin1971-collectiveAsPrisionerDilema}. }%
\es{Este concepto contiene la idea de que los bienes comunes tienen una estructura de pagos isomorfa al dilema del prisionero de N jugadores~\cite{hardin1971-collectiveAsPrisionerDilema}. }%
%
\en{In a two-player prisoner's dilemma, cooperating implies a cost $c$ for the other person to receive a benefit $b$, with $b > c$, and defecting means refusing to cooperate and carries no cost. }%
\es{En un dilema del prisionero de dos jugadores, cooperar implica un coste $c$ para que la otra persona reciba un beneficio b, con $b > c$, y desertar significa negarse a cooperar y no conlleva ningún coste. }% 
%
\begin{equation}
  \bordermatrix{ & C & D \cr
      C & b-c & -c \cr
      D & b & 0 } 
\end{equation}
%
\en{Players gain more if they opt for mutual cooperation than for mutual defection, since $b - c > 0$. }%
\es{Los jugadores ganan más si optan por la cooperación mutua que por la deserción mutua, ya que $b - c > 0$. }%
%
\en{However, regardless of what the other player does, it is better not to cooperate: if my partner defects, it is better for me to defect than to cooperate, since $0 > -c$; if my partner cooperates, it is still better for me to defect than to cooperate, since $b > b - c$. }%
\es{Sin embargo, independientemente de lo que haga el otro jugador, es mejor no cooperar: si mi compañero deserta, es mejor para mí desertar que cooperar, ya que $0 > -c$; si mi compañero coopera, sigue siendo mejor para mí desertar que cooperar, ya que $b > b - c$. }%
%
\en{This creates a dilemma: although mutual cooperation is a preferable outcome, no individual has the incentive to cooperate. }%
\es{De ahí el dilema: aunque la cooperación mutua es un resultado preferible, ningún individuo tiene el incentivo de cooperar. }%

% Parrafo

%En vez de utilizar una matriz de pagos del dilema del prisionero para representar un proceso de bienes comunes, en nuestro trabajo definimos el proceso de bienes comunes a partir del cual podemos derivar la matriz de pagos asociada.
%
\en{If our causal model had a payoff structure isomorphic to the prisoner's dilemma, then defectors would have a higher growth rate than cooperators. }%
\es{Si nuestro modelo causal tuviera una estrucutra de pagos isomorfa al dilema del prisioner, entonces los desertores tendrían una tasa de crecimiento mayor a lo cooperadores. }%
%
\en{However, the first defector from an entirely cooperative group obtains a lower growth rate than before defecting. }%
\es{Sin embargo, el primer desertor de un grupo enteramente cooperadora obtiene una tasa de crecimiento menor a la que tenía antes de desertar. }% 
%
\en{In the figure~\ref{fig:ergodicity_desertion} we can observe the trajectories of the resources, equivalent to the proportional posterior (see previous section), of the individuals that are in a group of size 100. }%
\es{En la figura~\ref{fig:ergodicity_desertion} podemos observar las trayectoria de los recursos, equivalente el posterior proporcional (ver sección anterior), de los individuos que están en un grupos de tamaño 100. }%
%
\begin{figure}[ht!]
    \centering
    \begin{subfigure}[b]{0.5\textwidth}
    \includegraphics[width=\linewidth]{figures/pdf/ergodicity_desertion.pdf}
    \end{subfigure}
    \caption{
    \en{The colors represent groups of size 100 with 0, 1 and 2 defectors. }%
    \es{Los colores representan los grupos de tamaño 100 con 0, 1 y 2 desertores. }%
    %
    \en{The curves of the individual defectors are those at the top of each of the groups. }%
    \es{Las curvas de los individuos desertores son las que están arriba en cada uno de los grupos. }%
    }
    \label{fig:ergodicity_desertion}
\end{figure}
%
\en{The resources of the first individual defector (blue group with 1 defector), is below the resources of the individuals in the fully cooperative group (green group with 0 defectors). }%
\es{Los recursos del primer individuo desertor (grupo azul con 1 desertor), está por debajo de los recursos de los individuos del grupo enteramente cooperador (grupo verde con 0 desertores). }%
%
\en{The reduction in resources occurs even for the second individual who changes from cooperative to a defective behavior. }%
\es{La reducción de recursos le ocurre incluso al segundo individuo que cambia de comportamiento cooperador a desertor. }%

% Parrafo

\en{Defecting, instead of increasing the growth rate of the defective individuals, reduces it. }%
\es{Desertar, en vez de aumentar la tasa de crecimiento de los individuos desertores, la reduce. }%
%
\en{In other words, the commons does not have the structure of the prisoner's dilemma, as is usually claimed in the literature. }%
\es{Es decir, los bienes comunes no tiene la estructura del dilema del prisionero como habitualmente se afirma en la literatura. }%
%
\en{Let us calculate the payoff matrix arising from the cooperative causal model. }%
\es{Calculemos la matriz de pagos que surge del modelo causal cooperativo. }%
%
\en{We want to estimate the temporal growth rate of the posteriors, }%
\es{Queremos estimar la tasa de crecimiento temporal de los posteriors, }%
%
\begin{equation}
\frac{P(k|\vec{\Aa}^{\,1}, \dots , \vec{\Aa}^{\,T})}{P(k)} \approx g(k|\vec{\Ee},p,N)^T
\end{equation}
%
\en{where the approximation is an equality when time tends to infinity, $\lim_{T \rightarrow \infty}$. }%
\es{donde la aproximación es una igualdad cuando el tiempo tiende a infinito, $\lim_{T \rightarrow \infty}$. }%
%
\en{In figure \ref{fig:multilevel-selection-7} we show the proprotional growth rate as a function of the number of defectors in a group of size $1000$. }%
\es{En la figura \ref{fig:multilevel-selection-7} mostramos el proporcional de la tasa de crecimiento en función del número de desertores totales en una grupo de tamaño $1000$. }%
%
\en{In Figure~\ref{fig:multilevel-selection-5} we rescale the proportional growth rate by a factor $R=2.1$, and we see that it overlaps over the trajectories of resource presented in Figure~\ref{fig:ergodicity_desertion}. }
\es{En la figura~\ref{fig:multilevel-selection-5} reesclamos la tasa de crecimiento proporcional por un factor $R=2.1$, y vemos que se solapa sobre las trayectorias de los recursos presentado en la figura \ref{fig:ergodicity_desertion}. }%
%
\begin{figure}[H]
    \centering
    \begin{subfigure}[b]{0.48\textwidth}
    \includegraphics[width=\linewidth]{figures/pdf/multilevel-selection-7.pdf}
    \caption{$N=1000$}
    \label{fig:multilevel-selection-7}
    \end{subfigure}
    \begin{subfigure}[b]{0.48\textwidth}
    \includegraphics[width=\linewidth]{figures/pdf/multilevel-selection-5.pdf}
    \caption{$N=100$}
    \label{fig:multilevel-selection-5}
    \end{subfigure}
    \caption{
    \en{Figure \ref{fig:multilevel-selection-7}: proportional growth rate in mixed groups of size 1000. }%
    \es{Figura \ref{fig:multilevel-selection-7}: tasa de crecimiento proporcional en grupos mixtos de tamaño 1000. }%
    %
    \en{Figure \ref{fig:multilevel-selection-5}: trajectory of the resources of a defector (blue) and cooperator (green) in a group of size 100 with a single defector (see figure~\ref{fig:ergodicity_desertion}), the black curves are the estimates growth, and the dotted black curve are the resources of cooperators in regions without defectors. }%
    \es{Figura \ref{fig:multilevel-selection-5}: trayectoria de los recursos de un desertor (azul) y cooperador (verde) en un grupo de tamaño 100 con un único desertor (ver figura~\ref{fig:ergodicity_desertion}), las curvas negras son las estimaciones del crecimiento, y la curva negra punteadas son los recursos de mutua cooperación. }%
    }
    \label{fig:growth_rate_defector_mixed}
\end{figure}
% 
% \en{In Figure \ref{fig:multilevel-selection-5} we see that from the growth rates of cooperators $g_C(\Ee=1.5/2.1,n=99,N=100)$ and defectors $g_D(\Ee=1.5/2.1)$, we can correctly estimate the resource trajectory of a defector individual in a mixed population. }%
% \es{En la figura \ref{fig:multilevel-selection-5} vemos que a partir de las tasas de crecimiento de los individuos cooperdores $g_C(\Ee=1.5/2.1,n=99,N=100)$ y desertores $g_D(\Ee=1.5/2.1)$, podemos estimar correctamente la trayectoria de los recursos de un individuo desertor en una población mixta. }%
%
\en{Note that the growth rate of the individual defector is higher than that of the cooperators only in the first few time steps, which places the defectors in a better relative position. }%
\es{Notar que la tasa de crecimiento del individuo desertor es mayor a la de los cooperadores sólo en las primeros pasos temporales, lo que lo coloca a los desertores en una posición relativa mejor. }%
%
\en{But regardless of the size of the group, mutual cooperation always offers the highest growth rate, and the first defection always produces a drop in all growth rate, reducing even the growth rate of the individual defector. }%
\es{Pero, no importa el tamaño del grupo, siempre la mutua cooperación ofrece la tasa de crecimiento más alta, y la primera deserción produce una baja de la tasa de crecimiento incluso del individuo desertor. }%
%
\en{The following matrices summarize the growth rates of cooperators and defectors (rows), for different numbers of defectors in the social context (columns), in groups of size 2 (left) and size 16 (right). }%
\es{Las siguientes matrices resumen las tasas de crecimiento para individuos cooperadores y desertores (filas), para diferente cantidad de desertores en el contexto social (columnas), en grupos de tamaño 2 (izquierda) y tamaño 16 (derecha). }%
%
\begin{equation*}
\begin{split}
\ \ \ \ \ \ \ g_{[\cdot]}^n(k|\Ee=0.71, \, p=0.5, \, N=2)  & \ \ \ \ \ \ \ \  \ \ \ \ \ \ \ \ \ \ \ \ \ \ \ g_{[\cdot]}^n(k|\Ee=0.71, \, p=0.5, \, N=16) \\[0.1cm] 
 \propto \bordermatrix{ & n=1 & n=0 \cr
      C & 0.475 & 0.226 \cr
      D & 0.452 &  0.452 }\ \ \ \ \ \ \ \  &  \ \ \ \ \ \ \ \ \ \ \  \propto \bordermatrix{ & n=15 & n=14 & n=13 & n=12 \cr
      C & 0.497 & 0.466 & 0.435 & 0.403 \cr
      D & 0.466 & 0.452 & 0.452 & 0.452}       
\end{split}
\end{equation*}
%
\en{The first agent that unilaterally ``decides'' to defect will reduce its own growth rate. }%
\es{El primer agente que unilateralmente ``decida'' desertar va a reducir su propia tasa de crecimiento. }%
%
\en{Similarly, when all (or many) of the group members are defectors, the first agent who unilaterally ``decides'' to cooperate will also reduce his own growth rate. }%
\es{De modo similar, cuando todos (o muchos) de los miembros del grupo son desertores, el primer agente que unilateralmente ``decida'' cooperar , también reducirá su propia tasa de crecimiento. }%
%
\en{This means that the payoff matrix is not isomorphic to a prisoner's dilemma. }%
\es{Esto significa que la matriz de pagos no son isomorfas al dilema del prisionero. }%
%
\en{In fact, the payoff structure of the left-hand matrix is known as stag-hunt. }%
\es{De hecho, la estructura de pagos de la matriz izquierda se conoce como stag-hunt. }%
%
\begin{conclution}[\en{Commons are not prisoner's dilemmas}\es{Los bienes comunes no son dilemas del prisionero}]
\en{Without penalties, defector strategies negatively affect their own long-term growth rate because their own behavior increases the fluctuations of the random variable on which they depend. }%
\es{Sin castigos, las estrategias desertoras afectan negativamente su propia tasa de crecimiento a largo plazo debido a que su propio comportamiento aumenta las fluctuaciones de la variable aleatoria de la que dependen.}
\end{conclution}
%
\en{In all cases, the highest growth rate is obtained by mutual cooperation. }%
\es{En todos los casos, la tasa de crecimiento más alta se obtiene por mutua cooperación. }%
% %
% \en{For this reason, multilevel selection will favors cooperators over defectors, even though the presence of defectors negatively affects their growth rate. }%
% \es{Por este motivo, la selección multinivel favorecerá a los individuos cooperadores sobre los desertores, a pesar de que la presencia de destores afecte negativa su tasa de crecimiento. }%


\subsection{\en{Selection of level 1, level 2 and multilevel}\es{Selección de nivel 1, nivel 2 y multinivel}}

\en{In the previous section we have seen that unilateral defection reduces the growth rate of the defectors themselves compared to what they could have through mutual cooperation. }%
\es{En la sección anterior hemos visto que desertar unilateralmente reduce la tasa de crecimiento de los proprios desertores respecto de la que podrían tener a través de la mutua cooperación. }%
%
\en{However, the growth rate of cooperators is reduced to a greater extent. This means that the defectors always have a better relative position than the cooperators in their own group. }%
\es{Sin embargo, la tasa de crecimiento de los cooperadores se reduce en mayor medida. Esto hace que los desertores siempre tengan una posición relativa mejor que los cooperadores de su propio grupo. }%
%
\en{Therefore, evolution will favor defector behaviors through individual selection (level 1). }%
\es{Por lo tanto, la evolución favorecerá los comportamientos desertores a través de la selección individual (nivel 1). }%
%
\en{In Figure~\ref{fig:multilevel-selection-level-1-posterior} we can see the posterior of cooperating and deserting individuals in regions with 1 defector in groups of size 2 and 16. }%
\es{En la figura~\ref{fig:multilevel-selection-level-1-posterior} podemos ver el posterior los individuos cooperadores y desertores en regiones con 1 destertor en grupos de tamaño 2 y 16. }%
%
\begin{figure}[H]
    \centering
    \begin{subfigure}[b]{0.48\textwidth}
    \includegraphics[width=\linewidth]{figures/pdf/multilevel-selection-level-1-posterior.pdf}
    \caption{$N=2$}
    \end{subfigure}
    \begin{subfigure}[b]{0.48\textwidth}
    \includegraphics[width=\linewidth]{figures/pdf/multilevel-selection-level-1-posterior-N16.pdf}
    \caption{$N=16$}
    \end{subfigure}
    \caption{
    \en{Posterior of cooperator/defector social behaviors within a region with 1 defector in groups of size $2$ and $16$. }%
    \es{Posterior de los comportamientos sociales cooperador/desertor dentro de una región con 1 desertor en grupos de tamaño $2$ y $16$. }%
    }
    \label{fig:multilevel-selection-level-1-posterior}
\end{figure}
%
\en{Defector behaviors can invade within groups of size 2, as the posterior of the defectors quickly stabilizes at 1. }%
\es{Los comportamientos desertores pueden invadir al interior de los grupos de tamaño 2, pues el posterior de los desertores rápidamente se estabiliza en 1. }%
%
\en{However, defector behaviors cannot invade within groups of size 16, as the posterior of both behaviors never stabilizes at 0 and 1. }%
\es{Sin embargo, los comportamientos desertores no pueden invadir al interior de grupos de tamaño 16, pues el posterior de ambos comportamientos nunca se estabiliza en 0 y 1. }%

% Parrafo


\en{Although defectors may invade groups of size 2, regions that remain fully cooperative will have a great advantage over mixed regions because the growth rate of mutual cooperation is always the highest one. }%
\es{A pesar de que los desertores puedan invadir grupos de tamaño 2, las regiones que persistan enteramente cooperadoras tendrán una gran ventaja sobre las regiones mixtas gracias a que la tasa de crecimiento de la mutua cooperación es siempre mayor que el resto. }%
%
\en{Therefore, evolution will favor fully cooperative groups through group selection (level 2). }%
\es{Por lo tanto, la evolución favorecerá a los grupos enteramente cooperadores a través de la selección de grupos (nivel 2). }%
%
\en{In figure~\ref{fig:posterior_level_2} we see the posterior of the groups, of size 2 and 16. }%
\es{En la figura~\ref{fig:posterior_level_2} vemos el posterior de los grupos de tamaño 2 y 16. }%
% %
% \begin{equation}
% P(g|\vec{\Aa}^{\,1}, \dots, \vec{\Aa}^T) = \sum_i \mathbb{I}(\texttt{region}(i)=g) P(i|\vec{\Aa}^{\,1}, \dots, \vec{\Aa}^T)
% \end{equation}
% %
\begin{figure}[H]
    \centering
    \begin{subfigure}[b]{0.48\textwidth}
    \includegraphics[width=\linewidth]{figures/pdf/multilevel-selection-6.pdf}
    \caption{$N=2$}
    \label{fig:multilevel-selection-6}
    \end{subfigure}
    \begin{subfigure}[b]{0.48\textwidth}
    \includegraphics[width=\linewidth]{figures/pdf/multilevel-selection-level-2-N16.pdf}
    \caption{$N=16$}
    \label{fig:multilevel-selection-level-2-N16}
    \end{subfigure}
    \caption{
    \en{Group selection (level 2) of size $2$ and of size $16$. }%
    \es{Selección de grupos (nivel 2) entre grupos de tamaño $2$ y tamaño $16$. }%
    %
    \en{The gray lines represent the posterior of mixed groups. }%
    \es{Las rectas grises representan el posterior de grupos mixtos. }%
    }
    \label{fig:posterior_level_2}
\end{figure}
%
\en{Note that the prior of the entirely cooperative group is $0.25$ in groups of size 2 and approximately $\mathcal{B}(0|N=16,0.5) \approx 0$ for groups of size $16$. }%
\es{Notar que el prior del grupo enteramente cooperador es $0.25$ en grupos de tamaño 2 y aproximadamente $\mathcal{Binomial}(0|N=16,0.5) \approx 0$ para grupos de tamaño $16$. }%
%
\en{The choice of a prior that rejects homogeneous populations causes the advantage of the cooperating group to take time to stabilize. }%
\es{La elección de un prior que rechaza poblaciones homogenes hace que la ventaja del grupo cooperador necesite un tiempo hasta estabilizarse. }%
%
\en{With a uniform prior, the advantage of the fully cooperative group would be seen immediately. }%
\es{Si el prior fuera uniforme, la ventaja del grupo enteramente cooperador se vería inmediatamente. }%
%
\en{In any case, since the prior of homogeneous populations is never zero and the growth rate of the fully cooperative population is higher than the rest, there is always a time $t$ at which the posterior of the fully cooperative group will be higher than the others. }%
\es{En cualquier caso, debido a que el prior de las poblaciones homogeneas nunca es cero y que la tasa de crecimiento de la población enteramenete cooperadora sea superior al resto, siempre existe un tiempo $t$ en el cual el posterior del grupo enteramente cooperador será mayor al del resto de los grupos. }%
%
\en{The larger $N$ is, the closer the cooperators are to the arithmetic mean, but the more weight individuals from mixed regions receive. }%
\es{Cuanto más grande es $N$, más cerca están los cooperadores de la media aritmética, pero más peso reciben los individuos de regiones mixtas. }%
%
\en{Given a maximum time, the optimal group size will always be finite, which is reasonable in evolutionary terms. }
\es{Dada un tiempo máximo, el tamaño óptimo del grupo será siempre finito, lo que es razonable en términos evolutivos. }%

% Parrafo

% Si la y de los comportamientos cooperadores (multinivel),
% %
% \begin{equation}
% P(\texttt{coop}|\vec{\Aa}^{\,1}, \dots, \vec{\Aa}^T) = \sum_i P(i|\vec{\Aa}^{\,1}, \dots, \vec{\Aa}^T)\mathbb{I}(\texttt{coop}(i))
% \end{equation}
% %
\en{When there is level 2 selection in favor of fully cooperative groups, there is also multilevel selection in favor of cooperative individuals. }%
\es{Cuando se produce una selección de nivel 2 a favor de los grupos enteramenete cooperadores, se produce también una selección multinivel a favor de los individuos cooperadores. }%
%
\en{In Figure~\ref{fig:multilevel-selection-multilevel-posterior} we see the posterior of the cooperative individuals, integrating all groups. }%
\es{En la figura~\ref{fig:multilevel-selection-multilevel-posterior} vemos el posterior de los individuos cooperadores, integrando todos los grupos. }%
%
\begin{figure}[H]
    \centering
    \begin{subfigure}[b]{0.48\textwidth}
    \includegraphics[width=\linewidth]{figures/pdf/multilevel-selection-multilevel-posterior.pdf}
    \caption{$N=2$}
    \end{subfigure}
    \begin{subfigure}[b]{0.48\textwidth}
    \includegraphics[width=\linewidth]{figures/pdf/multilevel-selection-multilevel-posterior-N16.pdf}
    \caption{$N=16$}
    \end{subfigure}
    \caption{
    \en{Multilevel selection of cooperative individuals when $N=2$ and $N=16$. }%
    \es{Selección multinivel del los individuos cooperadores cuando $N=2$ y $N=16$. }%
    }
    \label{fig:multilevel-selection-multilevel-posterior}
\end{figure}
%
\en{The probability of cooperative individuals starts at $0.5$ due to the symmetry of the binomial prior between regions and the uniform prior between cooperative and defective behaviors. }%
\es{La probabilidad de los individuos cooperadores comienza a $0.5$ debido a la simetría del prior binomial entre regiones y el prior uniforme entre comportamientos cooperador y desertor. }%
%
\en{Because in most mixed regions the posterior of cooperative individuals drops abruptly, we see at the beginning a drop in the multilevel posterior. }%s
\es{Debido a que en la mayoría de las regiones mixtas el posterior de los individuos cooperdores cae abtruptamente, vemos al principio una baja en el posterior multinivel. }%
%
\en{But since the advantage of the fully cooperative group is imposed after a certain time (delay produced by the binomial prior), we finally observe that cooperative behaviors can invade populations with defectors as the posterior multilevel stabilizes at 1. }%
\es{Pero como la ventaja del grupo enteramente cooperador se impone luego de cierto tiempo (demora producida por el priori binomial), finalmente observamos que los comportamientos cooperadores pueden invadir poblaciones con desertores pues el posterior multinivel se estabiliza en 1. }%
%
\begin{conclution}[\en{The evolutionary advantage of cooperation}\es{La ventaja evolutiva de la cooperación}]
\en{Multilevel selection favors cooperative strategies even with groups of minimum size (two). }%
\es{La selección multinivel favorece a las estrategias cooperativas incluso con grupos de tamaño mínimo (dos). }%
\end{conclution}

\subsection{\en{The advantage of specialization}\es{La ventaja de la especialización}}

\en{To explain evolutionary transitions, it is necessary to demonstrate the evolutionary advantage of cooperation in the presence of defection, but also the advantage of specialization. }%
\es{Para explicar las transiciones evolutivas es necesario demostrar la ventaja evolutiva de la cooperación en presencia de deserción, pero también la ventaja de la especiliazación. }%
%
\en{Generalist strategies are those that achieve similar returns in each of the environmental states. }%
\es{Las estrategias generalistas son las que logran retornos similares en cada uno de los estados ambientales. }%
%
\en{The most extreme case is the strategy $\Ee = 0.5$, which has the same individual growth rate $g(k|\Ee=0.5,p,N=1) \propto 0.5$ irrespective of the type of environment $p$. }%
\es{El caso más extremo es la estrategia $\Ee = 0.5$, que tiene la misma tasa de crecimiento individual $g(k|\Ee=0.5,p,N=1) \propto 0.5$ indistintamente del tipo de ambiente $p$. }%
%
\en{By contrast, specialist strategies have high returns in one of the environmental states and high losses in the other. }%
\es{Por el contrario, las estrategias especialistas tienen altos retornos en uno de los estados ambientales y altas pérdidas en el otro. }%
%
\en{The extreme case is the strategy $\Ee = 1.0$, unfeasible in stochastic environments because its individual growth rate is $g(k|\Ee=1.0,p,N=1) = 0$ when $p\neq0$. }%
\es{El caso extremos es la estrategia $\Ee = 1.0$, inviable en ambientes estocásticos debido a que su tasa de crecimiento individual es $g(k|\Ee=1.0,p,N=1) = 0$ cuando $p\neq0$. }%

% Parrafo

\en{We have seen, in the methodological section, that the individual strategy best adapted to the environment $p=0.71$ was $\Ee^*=0.71$. }%
\es{Hemos visto en la sección metodológica que la estrategia individual mejor adaptada al ambiente $p=0.71$ fue $\Ee^*=0.71$. }%
%
\en{In general the optimal individual strategy is $\Ee^*=p$. }%
\es{En general la estrategia individual óptima es $\Ee^*=p$. }%
%
\en{Now that we know that there is an evolutionary advantage in favor of cooperation, is there another strategy that is better adapted to the environment? }%
\es{Ahora que sabemos que existe una ventaja evolutiva a favor de la cooperación, ¿hay una estrategia mejor adaptada al ambiente? }%
%
\en{If there were an advantage in favor of specialization we would expect to see that if the probability of the environment is biased toward one of the states, then the optimal strategy is biased even more, $\Ee^* > p > 0.5$ or $\Ee^* < p < 0.5$. }%
\es{Si hubiera una ventaja a favor de la especialización esperaríamos ver que si la probabilidad del ambiente está sesgada hacia uno de los estados, entonces la estrategia óptima esté sesgada aún más, $\Ee^* > p > 0.5$ o $\Ee^* < p < 0.5$. }%

% Parrafo

\en{In figure~\ref{fig:tasa-temporal-0} we compute the individual (solid lines) and cooperative (dashed line) growth rate of the strategies $\Ee \in \{0.5, 0.71, 0.99\}$ at all possible $p$ values of the environment. }%
\es{En la figura~\ref{fig:tasa-temporal-0} calculamos la tasa de crecimiento individual (líneas continuas) y cooperativa (línea punteada) de las estrategias $\Ee \in \{0.5, 0.71, 0.99\}$ para todos los posibles valores $p$ del ambiente. }%
%
\en{Note that the individual growth rates are proportional to the geometric mean, and that the cooperative growth rates are proprotional to the arithmetic mean (groups of infinite size), and that both means are equal for the strategy $\Ee = 0.5$. }%
\es{Notar que las tasas de crecimiento individual son proporcionales a la media geométrica, que las tasas de crecimiento cooperativo son proporcionales a la media aritmética (grupos de tamaño infinito), y que ambas medias son iguales para la estrategia $\Ee = 0.5$. }%
%
\begin{figure}[H]
    \centering
    \begin{subfigure}[b]{0.6\textwidth}
    \includegraphics[width=\linewidth]{figures/pdf/tasa-temporal-0.pdf}
    \end{subfigure}
    \caption{
    \en{Individuals and cooperative growth rates (continuous and dotted lines) of three strategies ($\Ee \in \{0.5, 0.71, 0.99\}$) in different environment $p$. }%
    \es{Tasas de crecimiento individual y cooperativa (líneas continuas y punteadas) de tres estrategias ($e \in \{0.5, 0.71, 0.99\}$) en diferentes ambiente $p$. }%
    }
    \label{fig:tasa-temporal-0}
\end{figure}
%
\en{The arrow represents the Yaari-Peters conclusion discussed in the introduction: mutual cooperation can increase the individual growth rate, equal to the geometric mean, to a cooperative growth rate equal to its arithmetic mean. }%
\es{La flecha representa la conclusión Yaari-Peters discutida en la introducción: la mutua cooperación puede aumentar la tasa de crecimiento individual, equivalente a la media geométrica, a una tasa de crecimiento cooperativa equvialente a su media aritmética. }%
%
\en{The red dot represents the conclusion we made in the methodology section, that in an environment with $p=0.71$ the best adapted individual strategy is $\Ee^*=0.71$. }%
\es{El punto rojo representa la conclusión que sacamos en la sección metodología, que en un ambiente con $p=0.71$ la estrategia individual mejor adaptada es $\Ee=0.71$. }%

% Parrafo 

\en{With the figure~\ref{fig:tasa-temporal-0} we can derive some new conclusions. }%
\es{Con la figura~\ref{fig:tasa-temporal-0} podemos sacar algunas conclusiones nuevas. }%
%
\en{Note that above the red dot is the cooperative growth rate of the specialist strategy $\Ee=0.99$. }%
\es{Notar que arriba del punto rojo se encuentra la tasa de crecimiento cooperativa de la estrategia especialista $\Ee=0.99$. }%
%
\en{This suggests that a strategy that is individually poorly adapted to the environment, as is the case of the specialist strategy $g_D^0(k|\Ee=0.99,p=0.71,N=1) < g_D^0(k|\Ee=0.71,p=0.71,N=1)$, achieves in cooperative groups a growth rate that is higher than the growth rate that the individually well adapted strategy achieves through cooperative groups, $g_C^{N-1}(k|\Ee=0.99,p=0.71,N=\infty) > g_C^{N-1}(k|\Ee=0.71,p=0.71,N=\infty)$. }%
\es{Esto sugiere que una estrategia que individualmente están mal adaptadas al ambiente, como es el caso de la estrategia especialista $g(k|\Ee=0.99,p=0.71,N=1) < g(k|\Ee=0.71,p=0.71,N=1)$, logra en grupos cooperativos una tasa de crecimiento que es mayor que las tasa de crecmiento que la estrategia individalmente bien adaptada logra a través de grupos cooperativos, $g_C^{N-1}(k|\Ee=0.99,p=0.71,N=\infty) > g_C^{N-1}(k|\Ee=0.71,p=0.71,N=\infty)$. }%

% Parrafo

\en{We have computed the cooperative growth rate for groups of infinite size. }%
\es{La tasa de crecimiento cooperativa la hemos calculado para grupos de tamaño infinito. }%
%
\en{To be an interesting conclusion for evolutionary theory we need this result to emerge also in small groups. }%
\es{Para que sea una conclusión interesante en términos evolutivos, necesitamos que este mismo resultado ocurra en grupos finitos, particularmente pequeños. }%
%
\en{In Figure~\ref{fig:tasa-temporal-1} we plot the growth rates of the specialist strategy $\Ee=0.99$ for cooperative groups of size 1 to 5. }%
\es{En la figura~\ref{fig:tasa-temporal-1} graficamos las tasas de crecimiento de la estrategia especialista $\Ee=0.99$ para grupos cooperativos de tamaño 1 a 5. }%
%
\begin{figure}[H]
    \centering
    \begin{subfigure}[b]{0.6\textwidth}
    \includegraphics[width=\linewidth]{figures/pdf/tasa-temporal-1.pdf}
    \end{subfigure}
    \caption{
    \en{Growth rate of the specialist strategy ($\Ee=0.99$) as a function of the probability of environment $p$, for cooperative groups of size 1 to 5. }%
    \es{Tasa de crecimiento de la estrategia especialista ($\Ee=0.99$) en función de la probabilidad del ambiente $p$, para grupos cooperativos de tamaño 1 a 5. }%
    %
    \en{The black dotted line represents the growth rate of an infinitely large cooperative group. }%
    \es{La línea punteada negra represeta la tasa de crecimiento de un grupo cooperativo inifinitamente grande. }%
    %
    \en{The gray lines are visual references to the strategies $\Ee \in \{0.5, 0.71\}$ discussed in the previous figure. }%
    \es{Las rectas grises son referencias visuales de las estrategias $e \in \{0.5, 0.71\}$ analizadas en la figura anterior. }%
    }
    \label{fig:tasa-temporal-1}
\end{figure}
%
\en{Note that in an environment $p=0.71$ the specialist strategy $\Ee=0.99$ achieves in cooperative groups of size 3 a growth rate that is above the growth rate of the strategy individually well adapted to the environment $\Ee=0.71$, outperforming even the growth rate that the individually well-adapted strategy obtains in cooperative groups of infinite size, $g_C^{2}(k|\Ee=0.99,p=0.71,N=3) > \lim_{N \rightarrow \infty }g_C^{N-1}(k|\Ee=0.71,p=0.71,N)$! }%
\es{Notar que en un ambiente $p=0.71$ la estrategia especilista $\Ee=0.99$ logra en grupos cooperativos de tamaño 3 una tasa de crecimiento que está por arriba de la tasa de crecimiento de la estrategia individualmente bien adapatada al ambiente $\Ee = 0.71$, superando incluso la tasa de crecimiento que la estrategia individulamente bien adaptada obtiene en grupo cooperativos de tamaño infinito, $g_C^{2}(k|\Ee=0.99,p=0.71,N=3) > \lim_{N \rightarrow \infty }g_C^{N-1}(k|\Ee=0.71,p=0.71,N)$! }%
%
\en{It is extraordinary that the same basic assumption that offers an evolutionary advantage in favor of cooperation also offers an evolutionary advantage in favor of specialization, even in small groups. }%
\es{Es extraordinario que el mismo supuesto básico que ofrece una ventaja evolutiva a favor de la cooperación, ofrezca también una ventaja evolutiva a favor de la especilización incuso en grupos pequeños. }%
%
\en{The emergence of cooperation immediately produces an evolutionary advantage in favor of specialization. }%
\es{La emergencia de la cooperación produce inmediatamente una ventaja evolutiva a favor de la especialización. }%

% Parrafo

\en{The optimal level of specialization depends on the size of the groups. }%
\es{El nivel de especialización óptimo depende del tamaño de los grupos. }%
%
\en{In Figure~\ref{fig:tasa-temporal-2} we set the environment at $p = 0.71$ and analyze how the cooperative growth rate of all possible strategies varies in groups of sizes 1 to 5. }%
\es{En la figura~\ref{fig:tasa-temporal-2} fijamos el ambiente en $p = 0.71$ y analizamos como varía la tasa de crecimiento cooperativa de todas las posibles estrategias en grupos de tamaños 1 a 5. }%
%
\begin{figure}[ht!]
    \centering
    \begin{subfigure}[b]{0.6\textwidth}
    \includegraphics[width=\linewidth]{figures/pdf/tasa-temporal-2.pdf}
    \end{subfigure}
    \caption{
    \en{The proportional growth rate of all possible strategies for entirely cooperative groups of size 1 to 5, in an environment $p=0.71$. }%
    \es{La tasa de crecimiento proprocional de todas las posibles estrategias para grupos enteramenete cooperativos de tamaño 1 a 5, en un ambiente $p=0.71$. }%
    %
    \en{The dots indicate the optimum strategy in each of the sizes. }%
    \es{Los puntos indican la estretgia óptima en cada uno de los tamaños. }%
    %
    }
    \label{fig:tasa-temporal-2}
\end{figure}
%
\en{When the group has size 1, the optimal strategy is $\Ee^*=p$. }%
\es{Cuando el grupo tiene tamaño 1, la estrategia óptima es $\Ee^*=p$. }%
%
\en{But as soon as cooperation arises, an advantage in favor of the specialist strategies $\Ee^* > p = 0.71$ appears. }%
\es{Pero apenas surge la cooperación aparece una ventaja a favor de las estrategias especialistas $\Ee^* > p = 0.71$. }%
%
\en{The larger the groups, the more specialist the optimal strategy becomes. }%
\es{Cuanto más grande son los grupos, más especialista es vuelve la estrategia óptima. }%
%
\en{At the extremes (cooperative groups of infinite size) the optimal strategy is reached at the maximum level of specialization, $\Ee=1$. }%
\es{En el extremos (grupos cooperativos de tamaño inifinito) la estrategia óptima se alcanza con el nivel de especialización máximo, $\Ee=1$. }%

\begin{conclution}[\en{The advantage of specialization}\es{La ventaja de la especialización}]
\en{Cooperation offers an advantage in favor of specialist strategies in groups of minimum size (two). }%
\es{La cooperación ofrece una ventaja a favor de las estrategias especialistas en grupos de tamaño mínimo (dos). }%
%
\en{Strategies that individually are poorly adapted to the environment, cooperatively achieve better results than those obtained by cooperative groups of individually well-adapted strategies. }
\es{Estrategias que individualmente están mal adaptadas al ambiente, cooperando logran mejorar los resultados que obtienen grupos cooperativos de estrategias individualmemte bien adaptada al ambiente. }%
\end{conclution}

% 
% \en{Here we show that strategies that are individually poorly adapted to the environment (specialists), once cooperation emerges, manage to outperform both individually well-adapted strategies (generalists), as well as their cooperative groups of infinite size. }%
% \es{Aquí mostramos que, una vez que la cooperación emerge, las estrategias individualmente mal adaptadas al ambiente (especialistas) consigen superar tanto a las estrategias bien adaptas individualmemte (generalistas), como a sus grupos cooperativos de tamaño infinito. }%
% %


\section{\en{Irreversibility of evolutionary transitions}\es{Irreversibilidad de las transitions evolutivas}}

\en{The reason why an advantage in favor of cooperation and specialization arises in our simple causal model is due to the multiplicative (non-ergodic) nature of probability theory and its isomorphism with evolutionary theory. }%
\es{El motivo por el cual surge una ventaja a favor de la cooperación y la especialización en simple modelo causal se debe a la naturaleza multiplicativa (no-ergódica) de la teoría de la probabililidad y a su isomorfismo con la teoría de la evolución. }%
%
\en{If the population structure persists for a certain minimum time, the evolutionary advantage in favor of cooperation is immediately produced. }%
\es{Si la estructura población persiste durante cierto tiempo mínimo, inmediátamente se produce la ventaja evolutiva de favor de la cooperación. }%
%
\en{With the formation of cooperative groups, an evolutionary advantage emerges in favor of specialist strategies. }%
\es{Con la formación de grupos cooperativo emerge una ventaja evolutiva a favor de estrategias especialistas. }%
%
\en{Specialist strategies are able to improve, within cooperative groups, the performance that individually well-adapted strategies can achieve both individually and cooperatively. }%
\es{Las estrategia especialista logran mejorar, al interior de los grupos cooperativos, el desempeño que las estrategias individualmente bien adaptadas pueden obtener de forma individual como cooperativa. }%
%
\en{But a characteristic of specialist strategies is that they are individually maladapted to the environment. }%
\es{Pero una caracterísitica de las estrategias especialistas es que están individualmente mal adaptadas al ambiente. }
%
\en{This means that, once the individuals of the groups acquire a specialist strategy, they are obliged to remain within the group, since leaving it produces a considerable drop in their evolutionary viability. }%
\es{Esto significa que, una vez que los individuos de los grupos adquieren una estrategia especialista, se ven obligados a permanecer al interior del grupo, pues dejarlo le produce una considerarble baja en su viabilidad evolutiva. }%
%
\en{The evolutionary advantage of cooperation and specialization therefore produces the irreversibility of evolutionary transitions. }
\es{La ventaja evolutiva de la cooperación y de la especialización produce, por lo tanto, la irreversibilidad de las transiciones evolutivas. }%
%
\begin{conclution}[\en{Irreversibility of evolutionary transitions}\es{Irreversibilidad de las transitions evolutivas}]
\en{The evolutionary advantage of specialization produces the irreversibility of evolutionary transitions, since individuals cannot leave groups without a reduction of their evolutionary viability. }%
\es{La ventaja evolutiva de la especialización produce la irreversibilidad de las transiciones evolutivas, pues las individos no pueden dejar los grupos sin una baja de su viabilidad evolutiva. }%
\end{conclution}


\section{\en{Discussions}\es{Discusiones}}

\en{In this paper we specify a probabilistic causal model, in which individuals are affected just by the environment and by the social behaviors of cooperation and defection of their context. }%
\es{En este trabajo definimos un modelo causal en el que los individuos se ven afectados solamente por el ambiente y por los comportamientos sociales de cooperación y deserción de su contexto. }%
%
\en{Under this minimal set of hypotheses, where we consider \emph{unconditionally} cooperative individuals who generate a common good that can be exploited by defecting individuals without receiving some kind of punishment in return (e.g. end of cooperation), the evolution of cooperation literature predicts defection as the only evolutionarily stable strategy. }%
\es{Bajo este conjunto mínimo de hipótesis, donde consideramos individuos \emph{incondicionalmente} cooperadores que generan un bien común que puede ser explotado por individuos desertores sin que reciban a cambio algún tipo de castigo (e.g fin de la cooperación), la literatura de evolución de la cooperación predice que la deserción es la única estrategia evolutivamente estable. }%
%
\en{It is assumed that common goods, if not accompanied by special conditions (such as communication allowing coordination, memory allowing rewards or punishments, etc.), lead to their overexploitation, because even if the optimum is obtained through mutual cooperation there would be an individual incentive to defect. }%
\es{Se supone que los bienes comunes, si no están acompañados de condiciones especiales (como la comunicación que permita la coordinación, la memoria que permita aplicar premios o castigos, etc), conducen a su sobreexplotación, porque aunque el óptimo se obtenga a través de la mutua cooperación habría una incentivo individual para desertar. }%
%
\en{This type of scenario is known in the social sciences as the ``tragedy of the commons''~\cite{hardin1971-collectiveAsPrisionerDilema}, an analogy that evolutionary biology has begun to adopt in recent decades~\cite{rankin2007-tragedyCommonsBiology}. }%
\es{Este tipo de escenarios se los conoce en ciencias sociales como ``tragedia de los comunes''~\cite{hardin1971-collectiveAsPrisionerDilema}, una analogía que la biología evolutiva ha comenzado a adoptar en las últimas décadas~\cite{rankin2007-tragedyCommonsBiology}. }%

% Parrafo

\en{However, cooperation is consistently observed in the history of life. }%
\es{Sin embargo, la cooperación se observa sistemáticamente en la historia de la vida. }%
%
\en{In the last third of the universe's history, a simple self-replicating organization of matter emerged on earth. }%
\es{En el último tercio de la historia del universo surgió en la tierra una organización de la materia simple capaz de autoreplicarse. }%
%
\en{The errors produced during replication diversified the life forms, and the growth rates of the different strategies favored those better adapted to the environment. }%
\es{Los errores producidos durante la replicación diversificaron las formas de vida, y las tasas de crecimiento de las diferentes estrategias favorecieron a aquellas mejor adaptadas al ambiente. }%
%
\en{The current complexity of life is the consequence of a series of evolutionary transitions in which entities capable of self-replication after the transition become part of higher level cooperative units. }%
\es{La complejidad actual de la vida es consecuencia de una serie de transiciones evolutivas en las que entidades capaces de autoreplicación luego de la transición pasan a formar parte de unidades cooperativas de nivel superior. }%
%
\en{Therefore, the theoretical paradigm of the ``tragedy of the commons'' is obliged to identify in each case the special conditions that explain the tendency observed in nature in favor of cooperative aggregation (and specialization). }%
\es{Por lo tanto, el paradigma teórico de la ``tragedia de los comunes'' está obligado a identificar en cada caso las condiciones especiales que expliquan la tendencia observada en la naturaleza a favor de la agregación cooperativa (y la especialización). }%

% Parrafo

\en{Without including any of these special conditions, the result of the probabilistic inference that emerges from the proposed causal model reveals an advantage in favor of cooperation and specialization over time. }%
\es{Sin incluir ninguna de estas condiciones especiales, el resultado de la inferencia probabilística que surge del modelo causal propuesto revela una ventaja a favor de la cooperación y la especialización en el tiempo. }% 
%
\en{How can this counter-intuitive result emerge from such a simple model? }%
\es{¿Cómo se explica que emerga este resultado contra-intuitivo de un modelo tan simple? }%
%
\en{While our model only specifies that environments affect individuals proportionally to certain values, the rules of probability theory update the posterior of individuals through the product rule. }%
\es{Si bien nuestro modelo solamente especifica que los ambientes afectan a los individuos de forma proporcional ciertos valores, las reglas de la teoría de la probabilidad actualizan el posterior de los individuos a través de la regla del producto. }%
%
\en{Because in multiplicative processes the impacts of losses are greater than those of gains, fluctuations produce a negative effect on growth rates, which can be reduced through mutual cooperation. }%
\es{Debido a que en los procesos multiplicativos los impactos de las pérdidas son mayores a los de las ganancias, las fluctuaciones producen un efecto negativo en las tasas de crecimiento, las cuales pueden ser reducidas a través de la mutua cooperación. }%
%
\en{As soon as cooperation emerges, an advantage in favor of specialist strategies appears, because it is no longer necessary for individuals to reduce fluctuations through generalist strategies that avoid bad outcomes in all possible states, and can devote themselves together with other individuals to take advantage of the most frequent environmental state. }%
\es{Apenas surge la cooperación, aparece también una ventaja en favor de las estrategias especialistas porque deja de ser necesario para los individuos reduicir la fluctuaciones través de estrategias generalistas que evitan malos resultados en todos los posibles estados, y pueden dedicarse en conjunto con otros individuos a sacar provecho del estado ambiental más frecuente. }%

% Parrafo

\en{Even if our causal model does not assume any kind of process to update resources, the result of the inference is proportional to the resources obtained through a multiplicative process. }%
\es{Incluso si nuestro modelo causal no presupone ningún tipo de proceso para actualizar los recursos, el resultado de la inferencia es proporcional a los recursos obtenidos a través de un proceso multiplicativo. }%
%
\en{The multiplicative updating of the probabilities of individuals, which arises naturally from applying the rules of probability to the causal model, is in line with the long-established idea in evolutionary theory that the growth of lineages follow multiplicative processes~\cite{dempster1955-geometricMean, denBoer1968-spreadingRisk}. }%
\es{La actualización multiplicativa de las probabilidades de los individuos, que surge naturalmente de aplicar las reglas de la probabilidad al modelo causal, está en línea con la idea largamente establecida en la teoría de la evolución de que el crecimiento de los linajes siguen procesos multiplicativos~\cite{dempster1955-geometricMean, denBoer1968-spreadingRisk}. }% 
%
\en{This coincidence is an additional support to the hypothesis of isomorphism between evolutionary and probability theories, previously identified between their fundamental equations: the Bayes theorem and the replicator dynamic~\cite{harper2009-replicatorAsInference,shalizi2009-replicatorAsInference}. }%
\es{Esta coincidecia es un apoyo adicional a la hipótesis de isomorfimo entre las teorías de la evolutivas y probabilisticas, previamente identificado entre sus ecuaciones fundamentales: el teorema de bayes y el replicator dynamic~\cite{harper2009-replicatorAsInference,shalizi2009-replicatorAsInference}. }%
%
%
\en{Based on this isomorphism, the co-author of the concept of evolutionary transitions (Szathmary~\cite{szathmary1995-evolutionaryTransitions, szathmary2015-evolutionaryTransitions}) recently proposed to analyze the evolution of populations subject to multilevel selection by means of Bayesian hierarchical models~\cite{czegel2019-bayesianEvolution}. }%
\es{Basados en este isomorfismo, el co-autor del concepto de transiciones evolutivas (Szathmary~\cite{szathmary1995-evolutionaryTransitions, szathmary2015-evolutionaryTransitions}) propuso recientemente analizar la evolución de las poblaciones sujetas a selección multinivel mediante modelos jerárquicos bayesianos~\cite{czegel2019-bayesianEvolution}. }%

% Parrafo

\en{To the best of our knowledge, our work would be the first to develop a Bayesian hierarchical model to solve an evolution problem under multilevel selection. }%
\es{Hasta donde sabemos, nuestro trabajo sería el primero en desarrollar un modelo jerárquico bayesiano para resolver un problema de evolución bajo selección multinivel. }%
%
\en{We were able to identify in it a ``multilevel posterior'' (i.e. the probability of individuals integrating all groups) as the average of the ``level 1 posteriors'' (i.e. probability of individuals within groups) weighted by the ``level 2 posteriors'' (i.e. probability of groups). }%
\es{En él pudimos identificar un ``posterior multinivel'' (i.e. la probabilidad de los individuos integrando todos los grupos) como el promedio del ``posteriors de nivel 1'' (i.e. probabilidad de los individuos al interior de los grupos) pesado por el ``posterior de nivel 2'' (i.e la probabilidad de los grupos). }%
%
\en{The reason why an advantage in favor of cooperation and specialization arises in our simple causal model is due to the multiplicative (non-ergodic) nature of probability theory and its isomorphism with evolutionary theory. }%
\es{El motivo por el cual surge una ventaja a favor de la cooperación y la especialización en simple modelo causal se debe a la naturaleza multiplicativa (no-ergódica) de la teoría de la probabililidad y a su isomorfismo con la teoría de la evolución. }%
%
\en{That is, contrary to the belief established since the mid-20th century in economics, we show that the dynamics of common goods cannot be represented by a prisoner's dilemma payoff matrix. }%
\es{Es decir, en contra de la creencia establecida que en economía se tiene desde mediados del siglo 20, mostramos que los dinámicas de bienes comunes no puede representarse mediante una matriz de pagos del dilema del prisionero. }%
%
\en{Moreover, contrary to the belief that specialization is too complex a feature to produce a benefit in simple aggregations, we show that as soon as cooperation emerges, an advantage in favor of specialist strategies appears even in groups of size 2. }%
\es{Además, en contra de la creencia de que la especialización es una caracterísitica demasiado compleja para que produzca un beneficio en agregaciones simples, mostramos que apenas surge la cooperación, aparece una ventaja a favor de las estrategias especialistas incluso en grupos de tamaño 2. }%
%
\en{Since the specialist strategies are individually poorly adapted to the environment, an irreversibility of the evolutionary transition is created. }%
\es{Como la estrategias especialistas están individualmente mal adaptadas al ambiente, se crea una irreversbilidad de la transici evolutivas. }%

% Parrafo

\en{It is extraordinary that such a simple system as the one analyzed has such fundamental conclusions to understand the complexity of life. }%
\es{Es realmente extraordinario que un sistema tan simple como el que hemos analizando tenga conclusiones tan fudamentales para entender la complejidad de la vida. }%
%
\en{Cooperation and specialization are the two main characteristics of the major evolutionary transitions, through which life acquired increasing complexity. }%
\es{La cooperación y la especilización son las dos caracteristicas principales de las transiciones evolutivas mayores, a través de las cuales la vida fue adquiriendo una complejidad cada vez mayor. }%
%
\en{Using Czégel-Zachar-Szathmáry's methodological approach \cite{czegel2019-bayesianEvolution} (multilevel selection as hierarchical Bayesian inference) we formally demonstrate the evolutionary advantage of cooperation and specialization suggested by Yaari-Peters \cite{yaari2010-cooperationEvolution, peters-cooperation2019.03.04} (noisy multiplicative processes). }%
\es{Mediante la propuesta metodológica de Czégel \cite{czegel2019-bayesianEvolution} (la selección multinivel como inferencia bayesiana jerárquica) resolvimos formalmente la demostración evolutiva de la cooperación y la especialización que le faltaba al modelo Yaari-Peters \cite{yaari2010-cooperationEvolution, peters-cooperation2019.03.04} (procesos multiplicativos ruidosos). }%
%
\en{And in turn, with the Yaari-Peters model we provided the concrete example that was missing from the methodological proposal of Czegel et al. }%
\es{Y a su vez, con el modelo de Yaari-Peters proveímos el ejemplo concreto que le faltaba a la propuesta metodológica de Czegel et al. }%
%
\en{Both proposals combined offer a new solution to the problem of major evolutionary transitions, which is simpler than the previous ones (noisy multiplicative processes), based on well-founded mathematical principles (the strict application of the rules of probability). }%
\es{Ambas propuestas combinadas ofrecen una solución nueva al problema de las transiciones evolutivas mayores, que es más sencilla que las anterios (procesos multiplicativos ruidosos), basada en principios matemáticos bien fundados (la aplicación estrica de las reglas de la probabilidad). }%



{\footnotesize
\bibliographystyle{auxiliar/biblio/plos2015.bst}
\bibliography{auxiliar/biblio/biblio_notUrl.bib}
}

\end{document}
